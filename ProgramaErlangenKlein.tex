\documentclass[a4paper, 12pt]{article}

%%%%%%%%%%%%%%%%%%%%%%Paquetes
\usepackage[spanish]{babel}  
\usepackage[utf8]{inputenc}
\usepackage{tcolorbox}
\usepackage{cmbright}  %%%%%%% El tipo de letra
\usepackage{setspace}
\usepackage{endnotes}
\onehalfspacing  %%%%%%%%%%% Espacio y medio de interlineado
\parskip=1em  %%%%%%%%%%%% Separacion entre parrafos
%%%%%%%%%%%%%%%%%%%%%%



%%%%%%%%%%%%%%%%%
\title{Consideraciones Comparativas sobre las Investigaciones Geometricas Modernas}
\author{F. Klein}
\date{}
%%%%%%%%%%%%%%%%%

\begin{document}

\begin{tcolorbox}[colback=blue!5!white,colframe=blue!75!black]

\vspace{-1.8cm}
\textbf \maketitle

\end{tcolorbox}

\bigskip

Entre los trabajos efectuados desde hace cincuenta años [desde 1822] en el dominio de la \textit{Geometría}, el desarrollo de la \textit{Geometría proyectiva} ocupa el primer lugar (ver Nota I). Si, al comienzo, ha podido parecer que las relaciones denominadas \textit{métricas} no pueden resultarle accesibles, porque no son proyectivas, recientemente se ha aprendido a concebirlas igualmente desde el punto de vista proyectivo, de manera que el método proyectivo comprende ahora toda la Geometría. Las propiedades métricas ya no aparecen en ésta como propiedades \textit{intrínsecas} de los seres del \textit{espacio}, sino como relaciones de estos seres con un elemento fundamental [\textit{extrínseco}], el \textit{círculo imaginario del infinito}.

Si se comparan las nociones de la \textit{Geometría elemental} con esta manera, poco a poco aceptada, de considerar los seres del espacio, nos vemos conducidos a buscar un \textit{principio general} según el cual se puedan construir los dos métodos [sintético y analítico]. Esta cuestión parece tanto más importante cuanto que, junto a la Geometría elemental y a la Geometría proyectiva, pueden establecerse otros métodos, sin duda menos desarrollados, a los cuales hay que conceder el mismo derecho a una existencia propia. Así, la \textit{Geometría de los radios vectores recíprocos}, la \textit{Geometría de las transformaciones racionales}, etc.; geometrías que, en lo que sigue, también vamos a mencionar y exponer.

Emprendiendo aquí el establecimiento del principio mencionado, no desarrollamos, ciertamente, ningún pensamiento particularmente nuevo; no hacemos sino dar una expresión clara y precisa a lo que muchos ya han pensado de una forma más o menos precisa. Sin embargo, la publicación de consideraciones destinadas a establecer ese vínculo pareció tanto más justificada cuanto que la Geometría, aunque sea una esencialmente, se ha escindido demasiado, en razón del rápido desarrollo que ha tenido en estos últimos tiempos, en disciplinas casi separadas (ver nota II), de las cuales cada una continúa desarrollándose casi independientemente de las otras. Hemos tenido también la intención particular de exponer los métodos y los puntos de vista que Lie y yo hemos desarrollado en trabajos recientes. A pesar de la diversidad de sus objetos, estos trabajos han confluido en la manera general de considerar las cosas que exponemos aquí; a continuación, era de alguna manera necesario discutir igualmente esto, para caracterizarlas en cuanto a sus objetos y a sus tendencias.

Si hasta aquí sólo hemos hablado de investigaciones geométricas, es necesario comprender con éstas aquellas relativas a las \textit{variedades de cualquier de dimensión}, surgidas de la Geometría cuando se hace abstracción de las figuras que, desde un punto de vista puramente matemático, no son en modo alguno esenciales (ver notas III y IV). El estudio de las variedades comprende tantos géneros diferentes como el de la Geometría, y tiene sentido, como para ésta, poner en evidencia lo que tienen de común y de diferente investigaciones emprendidas independientemente una de la otra. Desde un punto de vista abstracto, no hubiera sido necesario, en lo que sigue, sino hablar de variedades de varias dimensiones; pero, ciñendo la exposición a las nociones más familiares del espacio, la misma resulta más simple y más inteligible. Partiendo de la consideración de los seres geométricos y desarrollando basándose en ellos, como ejemplo, las ideas generales, seguimos la vía que tomó la Ciencia en su desarrollo y que es la que se puede adoptar más provechosamente como base de nuestra exposición.

Una indicación preliminar del contenido de lo que sigue no es posible aquí, pues apenas puede ser reducido a una forma más concisa\endnote{ Esta concisión de la forma es un defecto en nuestra exposición, y nos tememos que haga su comprensión sensiblemente más penosa. No habría podido sin embargo remediarlo más que mediante una exposición mucho más extensa donde habrían sido desarrolladas en detalle las teorías particulares que aquí, dados los límites de esta exposición, no tocamos. }; los títulos de los parágrafos indicarán la marcha general de las ideas. He añadido al final una serie de notas, en las cuales desarrollo un poco más ciertos puntos particulares cuando eso me ha parecido útil en la exposición general del texto, o bien me he esforzado en separar el punto \enlargethispage{0.5 cm}de vista matemático abstracto, que es el que hemos adoptado para las consideraciones del texto, de puntos de vista que le están asociados. 

 \subsection*{1. Grupos de transformaciones del espacio. Grupo principal. Problema general.}

De las nociones necesarias para las consideraciones que van a seguir, la más esencial es la de \textit{grupo} de transformaciones del espacio.

La composición de un número cualquiera de transformaciones del espacio\endnote{ Entendemos que las transformaciones están siempre aplicadas a la totalidad de los elementos del espacio, y hablamos a continuación pura y simplemente de transformaciones del espacio. Las transformaciones, como, por ejemplo, las realizadas por dualidad, pueden introducir, en lugar de puntos, nuevos elementos. En el texto este caso no se distingue de los otros.} produce siempre otra transformación. Supongamos ahora que un conjunto dado de transformaciones tenga la propiedad de que toda transformación resultante de la composición de un número cualquiera de ellas pertenezca también al conjunto, ésto constituye lo que se llama un \textit{grupo de transformaciones}\endnote{ [Esta definición hace necesario un complemento que proporcionamos aquí: está implícitamente supuesto, en los grupos del texto, que cualquier operación que figura en ellos se acompaña de la operación inversa; pero, en el caso en que haya una infinidad de operaciones, esto no es en absoluto una consecuencia de la noción misma de grupo; es pues una hipótesis que debemos agregar explícitamente a la definición de grupo, tal como viene dada en el texto.] La noción y la denominación están tomadas de la teoría de las \textit{sustituciones} donde se trata, no de las \textit{transformaciones} de un campo continuo, sino de las permutaciones de un número finito de magnitudes discretas.}$.$

El conjunto de los desplazamientos (cada desplazamiento lo consideramos como una operación efectuada en la totalidad del espacio) nos ofrece ya el ejemplo [más sencillo] de un grupo de transformaciones. Un grupo que está contenido en este conjunto está formado, por ejemplo, por las rotaciones alrededor de un punto\endnote{Camille Jordan determinó todos los grupos contenidos en el grupo genérico de los desplazamientos: ``Sobre los grupos de los movimientos'', \textit{Annali di Matemática, }t. {\sc ii}.}, un grupo que, por el contrario, lo contiene está formado por el conjunto de las \textit{transformaciones homográficas}. En cambio, el conjunto de las \textit{transformaciones dualísticas} no forma grupo, pues dos de tales transformaciones vuelven a dar, cuando se las compone, una transformación homográfica; pero se obtiene de nuevo un grupo asociando las transformaciones dualísticas y las homográficas\endnote{ No es por otra parte en absoluto necesario, aunque sea siempre así para todos los grupos que tenemos que mencionar, que las transformaciones de un grupo se presenten en él en sucesiones continuas. Por ejemplo, los desplazamientos en número limitado que llevan a un cuerpo regular a recubrirse forman un grupo; de la misma manera aquellos, en número ilimitado, que llevan a una sinusoide a superponerse.}.

Hay transformaciones del espacio que no alteran en absoluto las propiedades geométricas de las figuras. Por naturaleza estas propiedades son, en efecto, independientes de la posición ocupada en el espacio por la figura considerada, de su magnitud absoluta y finalmente también del sentido\endnote{ Por \textit{sentido}, debemos entender aquí esta propiedad del orden por la cual una figura se distingue de su simétrica (imagen reflejada). Es así como puede distinguirse, por ejemplo, por el sentido, una hélice dextrógira de una hélice levógira.} en el cual sus partes están dispuestas. Los \textit{desplazamientos} en el espacio, las transformaciones por \textit{semejanza} y las transformaciones por \textit{simetría} no alteran las propiedades intrínsecas de las figuras, como tampoco lo hacen las transformaciones compuestas con estas. Llamaremos \textit{grupo principal} de transformaciones del espacio al conjunto de todas esas transformaciones\endnote{ Por definición, esas transformaciones forman necesariamente un grupo.}; así pues, \textit{las propiedades geométricas no son alteradas por las transformaciones del grupo principal. }La recíproca es igualmente verdadera: \textit{las propiedades geométricas se caracterizan como tales por su invariancia relativamente a las transformaciones del grupo principal}$.$ Si se considera, en efecto, un instante el espacio como no pudiendo desplazarse, etc. [hacer desplazamientos, semejanzas o simetrías en él], es decir, como una variedad fija, cada figura posee en él una individualidad propia; las propiedades que la figura posee como individuo [y que lo hacen reconocible como tal en su singularidad o individualidad], y sólo éstas, son las propiamente geométricas que las transformaciones del grupo principal no alteran. Esta prueba, que formulamos aquí un poco vagamente, se despejará más claramente en la continuación de la exposición.

Intentemos ahora hacer abstracción de la figura material concreta que, desde el punto de vista matemático, no es [lo] esencial, e intentemos ver el espacio solamente como una variedad [conjunto] de varias dimensiones, por ejemplo, ateniéndonos a la representación habitual del punto como elemento del espacio, una variedad de tres dimensiones. Por analogía con las transformaciones del [en el] espacio, podemos hablar de las transformaciones de la variedad. Ésas forman también \textit{grupos}. Pero no hay ya, como en el espacio, un grupo que se distinga de los otros por su significación [importancia]; un grupo cualquiera [desde el punto de vista matemático] no es ni más ni menos que cualquier otro. Como generalización de la geometría se plantea entonces la cuestión general siguiente:

\begin{quote}\it 

Dada una variedad [conjunto] y un grupo de transformaciones de esta variedad [de, ó en, este conjunto], estudiar sus seres desde el punto de vista de las propiedades que no son alteradas por las transformaciones del grupo.

\end{quote}

Si se adopta la manera actual de hablar de la que, ciertamente, no nos servimos más que para un grupo determinado, el de \textit{transformaciones lineales}, podemos también expresarnos así: 

\begin{quote}\it 

Se da una variedad y un grupo de transformaciones de esta variedad; desarrollar la teoría de los invariantes relativos a este grupo.

\end{quote} 

Tal es el problema general que abarca no solamente a la Geometría ordinaria, sino también los métodos geométricos modernos a los que tenemos que pasar revista y las diferentes maneras de estudiar las variedades de [en] un número cualquiera de dimensiones. Lo que es necesario, sobre todo, subrayar es la arbitrariedad que subsiste en la elección del grupo de transformaciones adjunto a la variedad y la facultad que se deduce de ella de aceptar igualmente todos los métodos de tratamiento desde el momento en que satisfacen la concepción general. 

 \subsection*{2.  Coordinación de los grupos de transformaciones que pueden contenerse [estar incluidos] uno en otro. Los diferentes tipos de investigaciones geométricas y sus relaciones mutuas.} 

 Dado que, como hemos señalado, las propiedades geométricas de los seres del espacio permanecen inalteradas en \textit{todas} las transformaciones del grupo principal, no tiene evidentemente ningún sentido buscar aquellas de estas propiedades que sólo son invariantes relativamente a una parte de estas transformaciones. Sin embargo, esta cuestión se legitima, desde el punto de vista, al menos, de las fórmulas, si se estudian las figuras del espacio en sus relaciones con elementos supuestamente fijos. Consideremos, por ejemplo, como en \textit{Trigonometría esférica}, los seres del espacio con distinción particular de un punto. La cuestión que se plantea en primer lugar es la siguiente: desarrollar las propiedades invariantes, relativamente al grupo principal, no ya de los seres mismos del espacio, sino del sistema [estructura] que forman con el punto dado. Pero podemos plantearla de manera diferente: estudiar los seres mismos del espacio desde el punto de vista de las propiedades que permanecen inalteradas por las transformaciones del grupo principal, que subsisten, cuando suponemos fijo el punto. En otros términos: es lo mismo estudiar, en el sentido del grupo principal, las figuras del espacio agregándoles el punto dado, o no agregando punto alguno, que reemplazar el grupo principal por el grupo, en él contenido, de las transformaciones que no cambian ese punto.

Es este un principio frecuentemente empleado en lo que sigue y que, a causa de eso, queremos enunciar, desde ahora, en toda su generalidad.

Dada una variedad [conjunto] y para realizar su estudio, uno de sus grupos de transformaciones. Se propone estudiar los seres de la variedad con respecto a uno de ellos. \textit{Podemos entonces, ya sea agregar éste al conjunto de los seres y buscar, en el sentido del grupo dado, las propiedades del sistema completo, o bien no agregar nada, sino limitar las transformaciones tomadas como base del estudio a aquellas del grupo dado que no alteran el ser considerado (y que forman necesariamente un grupo).}

Abordemos ahora la cuestión inversa de la que hemos planteado al comienzo del parágrafo y que se percibe inmediatamente. Se trata de encontrar las propiedades de los seres del espacio que permanecen inalteradas por las transformaciones de un grupo que contiene el grupo principal. Toda propiedad obtenida [que se obtenga] en esta búsqueda es una propiedad geométrica intrínseca del ser, pero la recíproca no es verdadera. Para esta recíproca, el principio que acabamos de establecer entra en vigor, y el grupo principal juega en ella el papel del grupo más restringido. Obtenemos así el siguiente teorema:

\begin{quote}\it 

Si se sustituye el grupo principal por un grupo más extenso, únicamente una parte de las propiedades geométricas se conservan. Las otras propiedades no aparecen ya como propiedades intrínsecas [invariantes por transformación extrínseca] de los seres geométricos, sino como propiedades del sistema obtenido agregándole un ser especial. Este ser especial, en tanto está, en general, determinado\endnote{Se engendra por ejemplo un ser tal cuando aplicamos las transformaciones del grupo principal a un elemento inicial cualquiera que no reproduce ninguna de las transformaciones del grupo dado\textit{, se define por la condición de que, suponiéndolo fijo, las únicas transformaciones, entre aquellas del grupo dado, que tengan todavía que aplicarse al espacio, sean las del grupo principal.}}

\end{quote} 

En este teorema se encuentra lo que caracteriza los \textit{métodos geométricos modernos} que tenemos que estudiar y lo que los vincula al \textit{método elemental}. Están, en efecto, caracterizadas por el hecho de que sus consideraciones en lugar de apoyarse en el grupo principal descansan sobre [se refieren a] grupos de transformaciones más extensos. Desde que sus grupos se contienen uno a otros, una ley análoga establece sus relaciones recíprocas. Esto se aplica también a las diferentes maneras de tratar las variedades de varias dimensiones que tenemos que considerar. Vamos a establecerlo ahora [en primer lugar] para cada método particular, y los teoremas, relativos al caso general, de este parágrafo y del anterior, van a encontrar así su esclarecimiento por su aplicación a objetos concretos. 



\subsection*{3. Geometría proyectiva} 



Cada transformación del espacio que no pertenece al grupo principal puede ser empleada para transferir a figuras nuevas propiedades de figuras conocidas. Así se utiliza la \textit{Geometría plana} para la \textit{Geometría de superficies} que son \textit{representables} en el plano; así, mucho antes del nacimiento de una verdadera \textit{Geometría proyectiva}, se concluía de las propiedades de una figura las propiedades de aquellas que se deducen de ella por proyección [mediante su comparación y estableciendo una serie de correspondencias]. Pero la Geometría proyectiva sólo nació [propiamente como tal] cuando se acostumbró a considerar como enteramente idénticas la figura primitiva y todas aquellas que pueden deducirse de la misma por proyección, y al enunciar las propiedades proyectivas de tal manera que se pusiera en evidencia su independencia frente a las modificaciones [concretas] aportadas por la proyección. Esto era tomar como base de las consideraciones, en el sentido del {\S}1, \textit{el grupo de las transformaciones proyectivas}, y podíamos de este modo establecer la diferencia entre las Geometrías proyectiva y ordinaria [o elemental].

Para cada especie de transformación del [en el] espacio, se puede imaginar un procedimiento de desarrollo semejante al que acabamos de describir; es un punto sobre el que volveremos de nuevo frecuentemente. Por lo que concierne a la Geometría proyectiva, este procedimiento se ha proseguido en dos direcciones. Un primer paso en la ampliación de las nociones se realizó admitiendo en el grupo fundamental de transformaciones las trasformaciones \textit{por vía de dualidad}. Desde el punto de vista moderno, hay que observar dos figuras correlativas ya no como dos figuras diferentes, sino como una única y la misma figura. Un segundo paso consiste en la extensión dada al grupo fundamental de transformaciones homográficas y dualisticas, admitiendo sus transformaciones imaginarias correspondientes. Exige que se haya ampliado primeramente el círculo de los elementos propios del espacio admitiendo en él sus elementos imaginarios, del mismo modo que la admisión de las transformaciones por dualidad en el grupo fundamental tiene como consecuencia la introducción simultánea del punto y del plano como elemento del espacio. No es este el lugar para extenderse sobre la utilidad de la introducción de elementos imaginarios, sólo mediante la cual es posible llegar a hacer corresponder exactamente la ciencia del espacio con el dominio, que ha sido adoptado como modelo, de las operaciones del álgebra; pero es necesario, por el contrario, insistir particularmente sobre el hecho de que es precisamente en la consideración de estas operaciones donde residen las razones de esta introducción y no en el grupo de las transformaciones proyectivas y dualísticas. De la misma manera que en éstas podemos limitarnos a las transformaciones reales, ya que las transformaciones homográficas y por dualidad que son reales forman un grupo, del mismo modo podemos introducir elementos imaginarios, incluso cuando no nos situamos en el punto de vista proyectivo, y debemos hacerlo en el caso en que tengamos sobre todo como objetivo el estudio de seres algebraicos.

El teorema general del parágrafo precedente muestra como deben concebirse las propiedades métricas desde el punto de vista proyectivo. Hay que considerarlas como relaciones proyectivas relativas a un elemento fundamental, el círculo imaginario del infinito\endnote{ Esta concepción debe considerarse como una de las más bellas invenciones [de la escuela francesa]; por sí sola da un sentido preciso a la distinción, que suele situarse al comienzo de la Geometría proyectiva, entre las propiedades métricas y las propiedades descriptivas.}, elemento que tiene la propiedad de ser transformado en sí mismo sólo por aquellas transformaciones del grupo proyectivo que son también transformaciones del grupo principal. Este teorema, que nos contentamos con enunciar, necesita todavía de un complemento indispensable que proviene del hecho de que se limiten las consideraciones habituales a los elementos reales del espacio (y a las transformaciones reales). Para estar de acuerdo completamente con este punto de vista, debemos aún agregar expresamente al círculo imaginario del infinito el sistema de los elementos reales (puntos) del espacio. Las propiedades en el sentido de la Geometría elemental que son proyectivas son: o bien propiedades intrínsecas de las figuras, o bien relaciones relativas a este sistema de elementos reales, o al círculo imaginario del infinito, o finalmente simultáneamente a los dos.

Podemos todavía recordar aquí como von Staud, en su \textit{Geometría de situación} construye la Geometría proyectiva, es decir, esta Geometría proyectiva cuyo grupo fundamental solo comprende las transformaciones reales proyectivas y por dualidad\endnote{ Solo será en las \textit{Beiträge zur geometrie der Lage} donde von Staud toma como base el grupo más extenso en el que figuran también transformaciones imaginarias.}.

Se sabe como, en esta obra, sólo toma del material de las consideraciones habituales lo que permanece inalterado por las transformaciones proyectivas. Si se quisiera ir así hasta la consideración de las propiedades métricas, sería necesario justamente introducirlas como relaciones relativas al círculo imaginario del infinito. La marcha de las ideas, así completada, es, para las consideraciones presentadas aquí, de una gran importancia, porque es posible construir de manera semejante la geometría en el sentido de cada uno de los métodos que nos queda por estudiar. 

 \subsection*{4. Correlación establecida por medio de una transformación de la variedad fundamental.}

Antes de pasar a la exposición de los métodos geométricos que se establecen junto a las Geometrías elemental y proyectiva, podemos desarrollar en general algunas consideraciones que se reproducirán constantemente a continuación, y para las cuales las teorías abordadas hasta aquí proporcionan ya un número suficiente de ejemplos. Les consagramos este parágrafo y el siguiente.

Supongamos que estudiamos una variedad $A$ tomando como base un grupo~$B$. Si, mediante una transformación cualquiera, se transforma $A$ en otra variedad $A'$, el grupo $B$ de transformaciones que reproducen $A$ se convierte en un grupo $B'$ cuyas transformaciones se relacionan con $A'$. Resulta desde ese momento un principio evidente que \textit{la manera de tratar $A$ tomando $B$ como base conduce a la de tratar $A'$ tomando $B'$ como base,} es decir que cada propiedad que posee, relativamente al grupo $B$, un ser de $A$, da una propiedad, relativamente al grupo $B'$, del ser correspondiente de $A'$.

Supongamos, por ejemplo, que $A$ sea una recta, y $B$ la triple infinidad de transformaciones lineales que la reproduce. El estudio de $A$ es entonces justamente lo que, en el álgebra moderna se llama \textit{la teoría de las formas binarias.} Ahora, podemos establecer una correspondencia entre los puntos de la recta y los de una cónica del plano, proyectando uno de los puntos de ésta. Se muestra fácilmente que las transformaciones lineales $B$ que reproducen la recta, se convierten en la transformaciones lineales $B'$ que reproducen la cónica, es decir las transformaciones de la cónica que corresponden a las transformaciones lineales del plano que reproducen la cónica.

Pero, según el principio del segundo parágrafo\endnote{ Si se quiere, este principio se aplica aquí de una forma un poco más general.}, corresponde al mismo estudiar la Geometría sobre una cónica suponiéndola fija y considerando únicamente las transformaciones lineales del plano que la reproducen, o estudiar la Geometría sobre la cónica considerando todas las transformaciones lineales del plano, y dejando que la cónica se modifique con ellas. Las propiedades que descubrimos en los sistemas de puntos de la cónica son entonces proyectivas en el sentido habitual de la palabra. Comparando esto con el resultado anterior, se ve que:

\begin{quote}\it 

La teoría de las formas binarias y la geometría proyectiva de los sistemas de puntos de una cónica son equivalentes, es decir que a cada teorema relativo a las formas binarias corresponde uno relativo a esos sistemas de puntos, y recíprocamente\endnote{ En lugar de una cónica del plano, podemos asimismo tomar una cúbica izquierda y, en general, proceder de manera semejante para el caso de $n$ dimensiones.}.

\end{quote}

He aquí otro ejemplo muy apropiado para esclarecer este tipo de consideraciones. Si se proyecta estereográficamente una cuádrica sobre un plano, se presenta entonces un punto fundamental sobre la superficie: el llamado punto de vista; dos se presentan sobre el plano: los trazos de las generatrices que pasan por el punto de vista. Ahora bien, se ve inmediatamente que las transformaciones lineales del plano, las que no alteran los dos puntos fundamentales, se convierten, por representación, en las de las transformaciones lineales de la cuádrica que las reproduce, sin cambiar, de todos modos, el centro de proyección. (Por transformaciones lineales que reproducen la superficie, hay que entender aquí las transformaciones que sufre la superficie cuando se efectúan transformaciones lineales del espacio que la llevan a recubrirse a sí misma). Así se hacen idénticos el estudio proyectivo de un plano con dos puntos fundamentales, y el de una cuádrica con un punto fundamental. Pero, si se emplean elementos imaginarios, la primera no es otra cosa que el estudio del plano en el sentido de la Geometría elemental. El grupo principal de transformaciones del plano, se compone, en efecto, precisamente de las transformaciones lineales que no alteran un par de puntos (los puntos cíclicos); de manera que, finalmente,

\begin{quote}\it 

La Geometría elemental del plano y el estudio proyectivo de una cuádrica con un punto fundamental son idénticos.

\end{quote}

Se pueden multiplicar a voluntad estos ejemplos\endnote{ Para otros ejemplos y también en particular, para la extensión a un mayor número de dimensiones, véase la exposición hecha en una de mis memorias: \textit{Ueber Linien-geometrie und metrische Geométrie }(\textit{Math. Annalen,} t. {\sc v}, 2); ver también los trabajos de Lie que vamos inmediatamente a citar.}. Hemos adoptado los dos que acaban de ser desarrollados, porque, a continuación, tendremos todavía la oportunidad de retomarlos. 



\subsection*{5. De lo arbitrario en la elección del elemento del espacio. Principio de correlación de Hesse. Geometría del espacio reglado.} 



Como elemento de la recta, del plano, del espacio, etc., y en general de una variedad a estudiar, se puede emplear, en lugar del punto, cualquier elemento que forme parte de la variedad: un grupo de puntos, en particular una curva, una superficie, etc. (ver nota IV). Como, \textit{a priori}, no hay nada determinado en el número de los parámetros arbitrarios de los que se harán depender este elemento, la línea, el plano, el espacio, etc. aparecen, según el elemento escogido, como provistos de un número cualquiera de dimensiones. Pero, \textit{en tanto que se toma como base del estudio geométrico el mismo grupo de transformaciones, nada se modifica en esta Geometría, }es decir, que toda proposición obtenida con un cierto elemento del espacio sigue siendo aún una proposición para cualquier otra elección de este elemento, lo único que cambia es el orden de los teoremas y sus conexiones.

Lo que es esencial, entonces, es el grupo de transformaciones; el número de dimensiones atribuida a la variedad aparece como algo secundario.

La comparación de esta observación con el principio del parágrafo anterior conduce a una serie de bellas aplicaciones, algunas de las cuales podemos desarrollar aquí. Más que cualquier análisis extenso, estos ejemplos, parecen en efecto, apropiados para explicar el sentido de las consideraciones generales.

Según el parágrafo anterior, la Geometría proyectiva sobre la recta (la teoría de las formas binarias) equivale a la Geometría proyectiva sobre una cónica. Podemos ahora sobre esta última considerar como elemento, en lugar del punto, el par de puntos. Pero puede establecerse una correspondencia entre el conjunto de los pares de puntos de la cónica y el conjunto de las rectas del plano, haciendo corresponder cada recta al par de puntos en los que ella encuentra a la cónica. Mediante esta representación, las transformaciones lineales que reproducen la cónica se convierten en las transformaciones lineales del plano (considerado como compuesto por rectas) que dejan la cónica inalterada. Ahora bien, según el {\S}2, considerar el grupo formado por estas últimas transformaciones, o partir de la totalidad de las transformaciones lineales del plano añadiendo siempre la cónica a la figura plana a estudiar, son cosas equivalentes. Resulta de todo esto que:

\begin{quote}\it 

La teoría de las formas binarias y la Geometría proyectiva del plano con una cónica fundamental son equivalentes.

\end{quote}

Finalmente, ya que, a causa de la identidad de los grupos, la Geometría proyectiva del plano con una cónica fundamental coincide con la Geometría métrica proyectiva que se puede establecer en el plano sobre una cónica (ver nota V), podemos también decir que:

\begin{quote}\it 

La teoría de las formas binarias y la Geometría métrica proyectiva general del plano son una sola y la misma Geometría.

\end{quote}

Podríamos, en el análisis anterior, sustituir la cónica del plano por una cúbica izquierda, etc.; pero podemos dispensarnos de estos desarrollos. La conexión que acabamos de exponer entre la Geometría del plano, después del espacio, o de una variedad de un número cualquiera de dimensiones, se armoniza esencialmente con el principio de correlación propuesto por Hesse (\textit{Journal de Borchardt, }t. LXVI).

La Geometría proyectiva del espacio, o, dicho de otra manera, la teoría de las formas cuaternarias, ofrece un ejemplo enteramente de la misma naturaleza. Tomemos la recta como elemento del espacio y, como en la geometría del espacio reglado, determinémosla por seis coordenadas homogéneas ligadas por una ecuación de segundo grado; las transformaciones lineales y por dualidad del espacio se ofrecen entonces como las de las transformaciones lineales de las seis variables supuestamente independientes, que transforman en sí misma la ecuación de conexión. Mediante una serie de deducciones como las que acabamos de desarrollar, obtenemos entonces el siguiente teorema:

\begin{quote}\it 

La teoría de las formas cuaternarias se armoniza con la determinación métrica proyectiva en la variedad engendrada por seis variables homogéneas.

\end{quote}

Para más detalle sobre estas nociones, remitiré a una Memoria publicada recientemente en los \textit{Math. Annalen }(t. VI): \textit{Ueber die sogenannte Nicht-Euclidische Geometrie (zweite Abhandlung)}, así como a una nota situada al final de este trabajo (ver nota VI).

Añadiremos aún dos observaciones a las consideraciones anteriores; la primera, es cierto, se encuentra ya implícitamente contenida en lo que hemos dicho, pero es necesario desarrollarla, porque el objeto al que se aplica está demasiado sujeto al malentendido.

Si se introducen seres cualesquiera como elementos del espacio, éste adquiere un número cualquiera de dimensiones. Pero si entonces nos situamos en el punto de vista habitual (elemental o proyectivo), el grupo que, por la variedad de [en] varias dimensiones, debemos tomar como base, viene dado \textit{a priori}: no es otro que el grupo principal o el grupo de transformaciones proyectivas. Si quisiéramos tomar como grupo fundamental otro grupo, deberíamos abandonar el punto de vista elemental o proyectivo. Así, en la medida en que es cierto que mediante una elección conveniente del elemento del espacio éste representa variedades de [en] un número cualquiera de dimensiones, en esa medida es importante añadir que \textit{con esta representación, es necesario, con vistas al estudio de la variedad, tomar como base un grupo determinado }a priori\textit{, o si no, que es necesario, para disponer a voluntad del grupo, adaptar a él convenientemente nuestras concepciones geométricas. }Si no se hiciera esta observación, se podría, por ejemplo, buscar una representación de la geometría del espacio regulada de la manera siguiente. En esta geometría, una recta tiene seis coordenadas; es también el número de los coeficientes de una cónica del plano. La reproducción de la geometría del espacio reglado sería así la geometría de un sistema de cónicas desprendido del conjunto de las cónicas por una relación cuadrática entre los coeficientes. Es exacto, si el grupo tomado como base de la Geometría plana es el grupo formado por el conjunto de las transformaciones representadas por las transformaciones lineales de los coeficientes de una cónica que reproducen la ecuación cuadrática como condición. Pero si conservamos la manera de ver elemental o proyectiva de la Geometría plana, no obtenemos \textit{absolutamente ninguna} representación.

La última observación se relaciona con la noción siguiente. Dado, para el espacio, un grupo cualquiera, por ejemplo el grupo principal. Elijamos una figura particular, como un punto, o una recta, o aún un elipsoide, etc., y efectuemos sobre ella todas las transformaciones del grupo fundamental. Obtenemos así un conjunto varias veces infinito en un número de dimensiones en general igual al número de los parámetros arbitrarios contenidos en el grupo. En ciertos casos particulares, este número es más pequeño, a saber, cuando la figura escogida en primer lugar tiene la propiedad de ser reproducida por un número infinito de transformaciones del grupo. Cada conjunto así engendrado se llama, relativamente al grupo generador, \textit{un cuerpo}\endnote{ Este nombre es elegido según Dedekind que, en la Teoría de los números, da a un conjunto de números el nombre de \textit{cuerpo} cuando resulta, por medio de operaciones dadas, de elementos dados (última edición de las \textit{Lecciones} de Dedekind).}$. $Si ahora, por una parte, queremos estudiar el espacio en el sentido del grupo, y, con esta finalidad, especificar como elemento del espacio algunas figuras determinadas; si, por otra parte, no queremos que algunas cosas equivalentes sean representadas de desigual manera, \textit{deberemos evidentemente escoger los elementos del espacio de tal manera que su conjunto forme un solo cuerpo o pueda descomponerse en cuerpos}\endnote{ [En el texto no está suficientemente subrayado que el grupo propuesto puede contener lo que se llaman subgrupos \textit{excepcionales}. Si una figura geométrica permanece inalterada por las operaciones de un subgrupo excepcional, sucede lo mismo con todas aquellas que se deducen de ella por las operaciones del grupo total, por consiguiente, de todos los elementos del cuerpo que resulta de ella. Ahora un cuerpo así formado es totalmente impropio para la representación de las operaciones del grupo. Sólo se deben tener en cuenta, pues, en el texto, cuerpos que resulten de elementos del espacio que no se conservan inalterados por ningún subgrupo excepcional del grupo propuesto.]}, haremos más tarde ({\S}9) una aplicación de esta observación evidente. La noción misma de cuerpo se representará una vez más, en el último parágrafo, asociada a nociones de la misma naturaleza. 

 \subsection*{6. Geometría de los radios vectores recíprocos. Interpretación de $x + iy$} 



Volvamos ahora a la discusión de las diferentes especies de investigaciones geométricas, empezada en los {\S}{\S}2,3. Desde varios puntos de vista, se puede considerar como análogo al género de consideraciones de la Geometría proyectiva una categoría de consideraciones geométricas donde se hace un constante uso de la transformación por radios vectores recíprocos: así las investigaciones relativas a lo que llamamos los cíclidos y las superficies analagmáticas, la teoría general de los sistemas ortogonales, y después algunas investigaciones sobre el potencial, etc. Si no se ha reunido todavía en una Geometría particular las consideraciones de estas teorías, como se ha hecho para las proyectivas, \textit{Geometría en la cual habría que tomar como grupo fundamental el conjunto de transformaciones obtenido reuniendo el grupo principal con su transformación por radios vectores recíprocos, }hay que atribuirlo a la circunstancia fortuita de que estas teorías no han sido todavía hasta ahora objeto de una exposición sistemática; los diferentes autores que han trabajado en este sentido no han estado alejados de semejante consideración metódica.

La analogía entre la Geometría de los radios vectores recíprocos y la Geometría proyectiva se ofrece por sí misma desde el momento en que uno se propone compararlas, y, a continuación, sin entrar en detalles, nos bastará atraer la atención sobre los puntos siguientes:

Las nociones elementales de la Geometría proyectiva son las del punto, la recta, el plano. La circunferencia y la esfera no son sino casos particulares de las secciones cónicas y de las superficies de segundo grado. El infinito se presenta en ellas como un plano; la figura fundamental que corresponde a la Geometría elemental es una sección cónica imaginaria del infinito.

Las nociones elementales de la Geometría de los radios vectores recíprocos son las del punto, la circunferencia, la esfera. La recta y el plano son casos particulares de estos dos últimos caracterizados por el hecho de que contienen un cierto punto, el punto al infinito, que, por lo demás, en el sentido del método, no es un punto más notable que los otros. Se obtiene la Geometría elemental desde el momento en que ese punto se supone fijo.

La Geometría de los radios vectores recíprocos puede ser presentada de manera tal que tome sitio junto a la teoría de las formas binarias y de la Geometría del espacio reglado, si de todos modos se trata a éstas como lo hemos indicado en los parágrafos anteriores. Podemos en primer lugar, para llegar a este resultado, limitarnos a la Geometría plana, y, a continuación, a la Geometría de los radios vectores recíprocos en el plano\endnote{ La Geometría de los radios vectores recíprocos sobre la recta equivale al estudio proyectivo de la recta, ya que las transformaciones son las mismas de una parte y de otra. En la Geometría de los radios vectores recíprocos, se puede entonces hablar también de la \textit{relación inarmónica} de cuatro puntos de una recta, y después de una circunferencia. }.

Estamos ya advertidos sobre la conexión que existe entre la Geometría plana elemental y la Geometría proyectiva de una superficie de segundo grado de la cual un punto es especificado particularmente. Si se hace abstracción de ese punto particular y se considera, por consiguiente, la Geometría proyectiva sobre la superficie misma, se tiene la representación de la Geometría plana de los radios vectores recíprocos. Es, en efecto, fácil convencerse\endnote{ Ver el trabajo ya citado: \textit{Ueber Liniengeometrie und metrische Geometrie }(\textit{Math Ann.,} t. {\sc v}).} de que al grupo de transformaciones por radios vectores recíprocos en el plano corresponde, por la representación de la superficie de segundo grado, el conjunto de las transformaciones lineales de ésta en ella misma. Por consiguiente:

\begin{quote}\it 

La Geometría de los radios vectores recíprocos en el plano y la Geometría proyectiva sobre una superficie de segundo grado son una sola y la misma cosa.

\end{quote}

Y de manera semejante: 

\begin{quote}\it 

La Geometría de los radios vectores recíprocos en el espacio es idéntica al estudio proyectivo de una variedad representada por una ecuación cuadrática entre cinco variables homogéneas.

\end{quote}

La Geometría en el espacio está así vinculada, por la Geometría de los radios vectores recíprocos, a una variedad de cuatro dimensiones, del mismo modo como lo está a una variedad de cinco dimensiones por la Geometría proyectiva del espacio reglado.

Considerando sólo las transformaciones reales, la Geometría de los radios vectores recíprocos nos da todavía, por otro lado, una representación y una aplicación interesantes. Si, en efecto, se representa de la manera habitual la variable compleja \textit{x + iy} sobre el plano, a sus transformaciones lineales corresponde el grupo de los radios vectores recíprocos limitado, tal como hemos dicho, a las transformaciones reales\endnote{ [La manera de decir del texto no es exacta. Todas las transformaciones lineales $$ z'=\frac{\alpha z+ \beta}{\gamma z+\delta} ( z' = x' + i y', z = x + i y) $$ corresponden a las únicas transformaciones del grupo de los radios vectores recíprocos que no invierten los ángulos (por las cuales los puntos cíclicos del plano no permutan entre sí). Para abrazar el grupo entero de los radios vectores recíprocos, es necesario agregar a las transformaciones anteriores también éstas (que no son menos importantes): $$ z'=\frac{\alpha \overline{z}+ \beta}{\gamma \overline{z}+\delta} $$ donde se tiene todavía $z' = x' + i y'$, pero donde $\overline{z} = x - i y$.]}. Pero el estudio de las funciones de una variable compleja que se supone sometido a transformaciones lineales cualesquiera no es otra cosa que lo que, con un método de exposición un poco diferente, se denomina la \textit{teoría de las formas binarias.} Así:

\begin{quote}\it 

La teoría de las formas binarias encuentra su representación en la Geometría de los radios vectores recíprocos y de tal manera que los valores complejos de las variables están también representados.

\end{quote}

Del plano podemos, para llegar al dominio de representación más habitual de las transformaciones proyectivas, pasar a la superficie de segundo grado. Ya que sólo consideramos elementos reales del plano, la elección de la superficie no es ya indiferente; es necesario evidentemente que ella no esté reglada. En particular, podemos, como por otra parte se hace también para la interpretación de una variable compleja, suponer que sea una esfera, y obtenemos así el siguiente teorema:

\begin{quote}\it 

La teoría de las formas binarias de variables complejas encuentra su representación en la Geometría proyectiva de una superficie esférica real.

\end{quote}

He creído tener también que mostrar en una nota (ver nota VII) hasta qué punto esta representación esclarece la teoría de las formas binarias y bicuadráticas. 



\subsection*{7. Generalización de lo anterior. Geometría de la esfera de Lie}

A la teoría de las formas binarias, a la Geometría de los radios vectores recíprocos y a la del espacio reglado, cuya coordinación acabamos de mostrar y que sólo parecen diferir por el número de las variables, se vinculan ciertas generalizaciones que ahora vamos a exponer. Servirán en primer lugar para esclarecer, mediante nuevos ejemplos, este pensamiento que el grupo que fija la manera de tratar un dominio dado puede ser generalizado a voluntad; pero, además, nuestro objetivo ha sido presentar, en sus relaciones con las ideas aquí expuestas, las consideraciones desarrolladas por Lie en una Memoria reciente\endnote{ \textit{Partielle Differentialgleichungen und Complexe }(\textit{Math. Annales}, t. {\sc v})}. La vía por la cual llegaremos a su Geometría de la esfera difiere de la que él ha adoptado en tanto que se vincula a nociones de la Geometría del espacio reglado; para adecuarnos mejor a la intuición geométrica ordinaria y para permanecer en conexión con lo que precede, nuestra exposición supondrá, en cambio, un número menor de variables. Como ya Lie lo ha puesto en evidencia (\textit{Göttinger Nachrichten, }1871, n$^{o}$ 7, 22), las consideraciones son independientes del número de las variables. Pertenecen al círculo extenso de investigaciones relativas al estudio proyectivo de las ecuaciones cuadráticas con un número cualquiera de variables, investigaciones que ya hemos tratado a menudo y que volveremos a encontrar aún en varias ocasiones (ver entre otros el {\S}10).

Parto de la correspondencia obtenida por proyección estereográfica entre el plano real y la esfera. En el {\S}5, haciendo corresponder a la recta del plano el par de puntos en que ella corta una cónica, ya hemos ligado la Geometría del plano con la Geometría sobre la cónica. Podemos, de la misma manera, establecer una correspondencia entre la Geometría del espacio y la Geometría sobre la esfera, haciendo corresponder a cada plano del espacio la circunferencia según la cual corta a la esfera. Si ahora, por proyección estereográfica, transportamos la Geometría establecida sobre la esfera de ésta al plano (y entonces cada circunferencia es transformada en una circunferencia), vemos que hay correspondencia entre: 

\begin{itemize}

\item La Geometría del espacio, que tiene como elemento el plano y como grupo las transformaciones lineales que transforman una esfera en ella misma. 

\item La Geometría plana, que tiene como elemento la circunferencia y como grupo el grupo de los radios vectores recíprocos.

\end{itemize}

Queremos ahora extender de dos maneras la primera de éstas Geometrías, tomando, en el lugar de su grupo, un grupo más general. La generalización que resulta de ello se transporta entonces inmediatamente, por medio de la representación, a la Geometría plana.

Hagamos la fácil modificación de escoger, en lugar de las transformaciones lineales del espacio, considerado como formado por planos, que transforman la esfera en ella misma, o bien el conjunto de las transformaciones lineales del espacio, o bien el conjunto de las transformaciones de planos del espacio que dejan [en un sentido que todavía tendremos que precisar] la esfera inalterada; en el primer caso, se hace abstracción de la esfera; en el segundo, del carácter de las transformaciones a emplear por ser lineales. La primera generalización se concibe inmediatamente; podemos pues examinarla en primer lugar y proseguir con sus consecuencias para la Geometría plana. Llegaremos a continuación a la segunda, donde tiene sentido determinar la transformación correspondiente más general.

Todas las transformaciones lineales del espacio transforman haces y gavillas de planos respectivamente en haces y gavillas de planos. Sobre la esfera, el haz de planos da un haz de circunferencias, es decir una serie simplemente infinita de circunferencias que se cortan en los mismos puntos; la gavilla de planos, una gavilla de circunferencias, es decir una serie doblemente infinita de circunferencias ortogonales a una circunferencia fija (la circunferencia de la cual el plano tiene como polo el punto por el cual pasan los planos). A las transformaciones lineales del espacio corresponden pues sobre la esfera y, por consiguiente, en el plano, las transformaciones circulares caracterizadas por la propiedad de que ellas transforman haces y gavillas de circunferencias en haces y gavillas de circunferencias\endnote{ Grassmann en su \textit{Ausdehnungslehre} considera fortuitamente estas transformaciones (p. 278 de la edición de 1862)}. \textit{La Geometría del plano obtenida adoptando ese grupo de transformaciones es la representación de la Geometría proyectiva ordinaria del espacio.} En esta Geometría no se podrá usar el punto como elemento del plano, ya que los puntos, para el grupo de transformaciones escogido no forman un cuerpo ({\S}5), sino que se escogerán como elementos las circunferencias.

Por lo que se refiere a la segunda extensión de la que hemos hablado, hay que preguntarse en primer lugar, por la naturaleza del grupo correspondiente de transformaciones. Se trata de encontrar transformaciones tales que todo [haz de planos cuyo eje es tangente a la esfera] se convierta en un [haz] que tenga también esta disposición. Podremos, para abreviar el lenguaje, transformar en primer lugar la cuestión por dualidad, y además descender un grado en el número de las dimensiones; tendremos así que encontrar las transformaciones puntuales del plano que, en cada tangente de una cónica dada, hagan corresponder una tangente a la misma cónica. Para lograrlo, consideremos el plano, y la cónica que está situada en él, como la proyección de una cuádrica hecha desde un punto de vista que no está sobre la superficie y de tal manera que la cónica sea la curva de contorno aparente. A las tangentes a la cónica les corresponden las generatrices de la superficie y el problema se reduce a encontrar el conjunto de las transformaciones puntuales que reproducen la superficie, permaneciendo las generatrices como generatrices.

De tales transformaciones las hay tantas como se quiera, pues es suficiente con considerar el punto de la superficie como intersección de las generatrices de cada sistema y transformar en el mismo, de una manera cualquiera, cada uno de esos sistemas. Entre estas transformaciones se encuentran, en particular, las que son lineales: son las únicas que vamos a considerar. Si teníamos, en efecto, que vérnoslas, no con una superficie, sino con una variedad de varias dimensiones, representada por una ecuación cuadrática, únicamente las transformaciones lineales subsistirían, las otras desaparecerían\endnote{ Si se proyecta estereográficamente la variedad, se obtiene el siguiente teorema conocido: \textit{Por fuera de las transformaciones del grupo de los radios vectores recíprocos, no existe, en los campos con varias dimensiones (y ya en el espacio), ninguna transformación puntual conforme. En el plano, por el contrario, existen una infinidad de ellas. }Véanse una vez más los trabajos citados de Lie.}.

Llevadas sobre el plano por proyección (no estereográfica), estas transformaciones lineales que reproducen la superficie se convierten en transformaciones puntuales con dos determinaciones, tales que a cada tangente a la cónica de contorno aparente corresponde de nuevo una tangente, pero a cualquier otra recta corresponde, en general, una cónica que tiene un doble contacto con la cónica de contorno aparente. Se puede muy bien caracterizar ese grupo de transformaciones basando sobre ésta última una determinación métrica proyectiva. Las transformaciones tienen entonces la propiedad de cambiar puntos que, en el sentido de la determinación métrica, están a una distancia nula el uno del otro, o puntos que están a una misma distancia constante uno del otro, en otros para los cuales tendrán lugar las mismas cosas.

Todas estas consideraciones pueden extenderse a un número cualquiera de variables; en particular, pueden ser empleadas para la cuestión planteada al comienzo, y relativa a la esfera y al plano, que es entonces tomado como elemento. En este caso, se puede dar al resultado una forma particularmente intuitiva, porque el ángulo que forman dos planos, en el sentido de la determinación métrica basada sobre la esfera, es igual al ángulo que forman, en sentido ordinario, las circunferencias de intersección en la esfera. 

Obtenemos pues sobre la esfera, y a continuación sobre el plano, un grupo de transformaciones circulares que tienen la propiedad de \textit{transformar círculos tangentes (formando un ángulo nulo) y círculos que cortan a otro bajo un mismo ángulo, respectivamente en círculos que satisfacen las mismas condiciones. }A este grupo de transformaciones pertenecen las transformaciones lineales sobre la esfera, y las transformaciones por radios vectores recíprocos en el plano\endnote{ [Las fórmulas siguientes harán mucho más claras las consideraciones del texto. Sea $$ x_{1}^{2} + x_{2}^{2} + x_{3}^{2} + x_{4}^{2}= 0 $$ la ecuación, en coordenadas tetraédricas ordinarias, de la esfera que es llevada estereográficamente al plano. Las $x$ que satisfacen esta ecuación de condición adquieren entonces para nosotros la significación de coordenadas tetracíclicas en el plano, y $$u_{1 }x_{1 }+ u_{2 }x_{2 }+ u_{3} x_{3 }+ u_{4 }x_{4 }= 0 $$ se convierte en la ecuación general del círculo en el plano. Si se calcula el radio de ese círculo, se encuentra el radical cuadrado $$ \sqrt{u_{1}^{2} + u_{2}^{2 }+ u_{3}^{2 }+ u_{4}^{2}} $$ que representaremos por $iu_{5}$. Podemos ahora considerar los círculos como elementos del plano. Entonces el grupo de los radios vectores recíprocos se ofrece como el conjunto de las transformaciones lineales homogéneas de $u_{1}$, $ u_{2}$, $ u_{3}$, $ u_{4}$, tales que $$u_{1}^{2} + u_{2}^{2 }+ u_{3}^{2 }+ u_{4}^{2} $$ se reproduzca en un cierto factor. El grupo más extenso que corresponde a la Geometría de la esfera de Lie, se compone, por su parte, de las transformaciones lineales de las cinco variables $u_{1}$, $ u_{2}$, $ u_{3}$, $ u_{4}$, $u_{5}$ que, salvo por un factor, reproducen $$u_{1}^{2} + u_{2}^{2 }+ u_{3}^{2 }+ u_{4}^{2 }+u_{5}^{2} $$]}.

La Geometría del círculo que se puede fundar sobre este grupo es la análoga de la \textit{Geometría de la esfera} propuesta por Lie para el espacio, y que parece de una importancia excepcional en las investigaciones sobre la curvatura de las superficies. Comprende la Geometría de los radios vectores recíprocos, en el sentido en que ésta comprende a su vez, la Geometría elemental.

Las transformaciones circulares (esféricas) que acabamos de obtener tienen, en particular, la propiedad de transformar circunferencias (esferas) tangentes en otras igualmente tangentes. Considerando todas las curvas (superficies) como envoltorios de circunferencias (esferas), puede observarse que dos curvas (superficies) tangentes serán siempre transformadas en curvas (superficies) igualmente tangentes. Las transformaciones en cuestión pertenecen pues a la categoría que estudiaremos más tarde en general, de las \textit{transformaciones de contacto, }es decir de transformaciones tales que el contacto de las figuras sea una propiedad invariante. Las transformaciones circulares mencionadas en primer lugar en este parágrafo, junto a las cuales pueden situarse transformaciones esféricas análogas, no son transformaciones de contacto.

Las dos especie de extensiones de las que sólo nos hemos ocupado para la geometría de los radios vectores recíprocos, pueden realizarse todavía de una manera análoga para la Geometría del espacio reglado y, en general, para el estudio proyectivo de una variedad caracterizada por una ecuación cuadrática; es lo que ya hemos indicado, y sobre lo cual no tiene sentido volver una vez más aquí. 



\subsection*{8. Enumeración de otros métodos que tienen como base un grupo de transformaciones puntuales.}

La Geometría elemental, la de los radios vectores recíprocos, e incluso la Geometría proyectiva cuando se hace abstracción de las transformaciones por dualidad que aportan con ellas un cambio del elemento del espacio, no son sino ejemplos particulares entre los numerosos métodos de tratamiento imaginables, donde se toman como base grupos de transformaciones puntuales. No señalaremos aquí más que los tres métodos siguientes, que, con aquellos que acabamos de nombrar, comparten este carácter. Aunque estos métodos estén todavía lejos de estar desarrollados, en el mismo grado que la Geometría proyectiva, en disciplinas que les sean propias, es de todos modos fácil de reconocer que se sitúan en las investigaciones modernas\endnote{ [Mientras que, en los ejemplos anteriores, se trataba de grupos con un número limitado de parámetros, llegamos ahora a la consideración de los grupos llamados \textit{infinitos}.]}.

\bigskip

\textbf{1. }\textbf{El grupo de las transformaciones racionales.}

En relación con las transformaciones racionales, hay que distinguir cuidadosamente si ellas son racionales para todos los puntos del campo en el cual se opera, como el espacio, o el plano, etc., o bien si lo son únicamente para los puntos de un conjunto perteneciente al campo, como una superficie, una curva. Sólo las primeras son aplicables, si se trata de construir, en el sentido entendido hasta aquí, una Geometría del espacio, del plano; las últimas, desde el punto de vista en el que nos hemos situado, sólo adquieren importancia si se trata de estudiar la Geometría sobre una superficie, una curva dadas. La misma distinción se aplica para el \textit{analysis situs} del que vamos a ocuparnos dentro de un momento.

Sin embargo, las investigaciones realizadas aquí y allí hasta ahora tienen esencialmente que ver con las transformaciones del segundo tipo. Como no se propone en ellas el estudio de la Geometría sobre la superficie, ni la curva, sino que se trata más bien de encontrar criterios para que dos superficies, dos curvas, puedan ser transformadas una en otra, esas investigaciones se escapan al dominio de las que vamos a considerar aquí\endnote{ [Ellas se vinculan, de otro modo y de la manera más feliz, con nuestras consideraciones, lo que yo todavía no sabía en 1872. Dada una forma algebraica cualquiera (curva, superficie, etc.) trasladémosla, introduciendo como coordenadas las relaciones $$ \varphi _{1 }: \varphi _{2 }: \dots : \varphi_{p} = du_1 : du_{2 }: \dots : du_{p } $$ donde $u_{1 },u_{2 },\dots, u_{p}$ son las integrales abelianas de primera especie vinculadas a la curva, en un espacio de orden superior. No hay más que tomar como base de las consideraciones relativas a este espacio el grupo de las transformaciones lineales homogéneas de las $\varphi $. Véanse diversos trabajos de los Sres Brill, Noether y Weber, así como mi reciente Memoria: \textit{Zur Theorie der Abel'schen Functionen, }en el tomo {\sc xxxvi} de los \textit{Math. Annalen.}]}. El esquema general expuesto en este trabajo no comprende [incluye] ciertamente la totalidad de las investigaciones matemáticas: únicamente ciertos enfoques se encuentran aquí reunidos bajo un mismo punto de vista.

De una Geometría de las transformaciones racionales tal como debe ofrecerse cuando se toman como base las transformaciones del primer tipo, no existe hasta aquí más que los principios. En el campo del primer nivel, sobre la recta, estas transformaciones racionales son idénticas a las transformaciones lineales, y no producen a continuación nada nuevo. En el plano se conocen, es cierto, todas las transformaciones racionales (transformaciones de Cremona); se sabe que ellas resultan de la composición de transformaciones cuadráticas. Se conocen también caracteres invariantes de las curvas planas, su género, la existencia de módulos; pero estas consideraciones no se han desarrollado todavía verdaderamente en una Geometría del plano en el sentido en que lo entendemos aquí. Por lo que al espacio se refiere, toda la teoría no ha hecho sino nacer. No se conocen hasta aquí más que un pequeño número de transformaciones racionales y se las utiliza para vincular por representación superficies desconocidas con superficies conocidas. 

\bigskip

\textbf{2.} \textbf{El analysis situs}

 En lo que se llama el \textit{analysis situs,} se estudia la invariancia frente a transformaciones que resultan de la composición de transformaciones infinitamente pequeñas. Como ya hemos dicho, es necesario todavía distinguir aquí si debemos considerar el campo total, por ejemplo el espacio, como sometido a la transformación, o solamente un conjunto que está desprendido del mismo, es decir una superficie. Son las transformaciones del primer tipo que podrían tomarse como fundamento de una Geometría del espacio. Su grupo estaría constituido de manera completamente diferente de los considerados hasta aquí. Como comprende todas las transformaciones que resultan de transformaciones puntuales, infinitamente pequeñas y supuestas reales, se limita por sí mismo, por origen, a los elementos reales del espacio, y corresponde al dominio de la función con definición arbitraria. Se puede muy bien extender ese grupo de transformaciones adjuntándolo a las transformaciones homográficas reales, que modifican también los elementos al infinito. 

\bigskip

\textbf{3. }\textbf{El grupo de todas las transformaciones puntuales.} 

Si, relativamente a ese grupo, ninguna superficie posee propiedades individuales, ya que cada una de ellas puede ser transformada en cualquier otra por transformaciones del grupo, existen sin embargo elementos de orden más elevado para el estudio de los cuales el grupo puede ser ventajosamente empleado. Con la manera de entender la Geometría que constituye la base de este trabajo, poco importa que estos elementos hayan sido considerados hasta aquí no tanto como elementos geométricos que únicamente como elementos analíticos, que, fortuitamente, encontraban una aplicación geométrica y que, estudiándolos se hayan empleado métodos (como precisamente transformaciones puntuales arbitrarias) apenas concebidas únicamente en nuestros días como transformaciones geométricas. A estos elementos analíticos pertenecen, en primer lugar las expresiones diferenciales homogéneas, y después las ecuaciones con derivadas parciales. Parece sin embargo, como lo mostrará el parágrafo siguiente, que, para el estudio general de estas últimas, el grupo de transformaciones de contacto sea todavía preferible.

La proposición fundamental de la Geometría que tiene como base el grupo de todas las transformaciones puntuales, es que, \textit{una tal transformación es siempre, para una parte infinitamente pequeña del espacio, equivalente a una transformación lineal.} Los desarrollos de la Geometría proyectiva son pues aplicables a lo infinitamente pequeño, cualquiera que pueda ser, por otra parte, el grupo tomado como base de tratamiento de las variedades, y es \textit{ese un notable carácter del método proyectivo.}



Ha sido cuestión anteriormente de la relación que existe entre los modos de tratamiento que descansan sobre grupos que se comprenden el uno al otro; daremos aquí también un ejemplo de la teoría general del {\S}2. Podemos preguntarnos cómo hay que concebir, desde el punto de vista de ``el conjunto de las transformaciones puntuales'', las propiedades proyectivas, y queremos hacer aquí abstracción de las transformaciones por dualidad, que constituyen propiamente parte del grupo de la Geometría proyectiva. La cuestión no difiere de esta otra: ?`por qué condición el grupo de las transformaciones lineales se desprende del conjunto de las transformaciones puntuales? Lo que caracteriza a las primeras, es que a todo plano le corresponde un plano. Son aquellas de las transformaciones puntuales que no cambian el conjunto de planos, o lo que es una consecuencia de esto, el conjunto de rectas. \textit{De la Geometría basada en las transformaciones puntuales se deduce, por adjunción del conjunto de los planos, la Geometría proyectiva, de las misma manera que de la Geometría proyectiva, por adjunción del círculo imaginario al infinito, se deduce la Geometría elemental.} En particular, debemos, desde el punto de vista de las transformaciones puntuales, concebir la cualidad de una superficie de ser algebraica y desde un cierto orden como una relación invariante respecto del conjunto de los planos. Es lo que deviene enteramente manifiesto, cuando, con Grassmann, se conecta la generación de los elementos algebraicos a su construcción por medio de la regla. 



\subsection*{9. El grupo de las transformaciones de contacto} 



Hace ya mucho tiempo que se han considerado, en casos particulares, transformaciones de contacto; Jacobi incluso ha hecho uso, en investigaciones analíticas, de la transformación de contacto más general. Sin embargo ellas no han sido puestas en las filas de las concepciones geométricas más corrientes, más que por recientes trabajos de Lie\endnote{ Ver en particular el trabajo ya citado: \textit{Ueber partielle Differentialgleichungen und Complexe (Math. Annales, }t. {\sc v}). Los desarrollos dados en el texto relativos a las ecuaciones con derivadas parciales son esencialmente debidos a comunicaciones orales de Lie. Ver su nota: \textit{Zur Theorie pariteller Differenctialgleichungen (Gottienger Nachrischten,} octubre 1872).}. No es pues en absoluto inútil exponer aquí claramente lo que es una transformación de contacto, limitándonos, como siempre, al espacio puntual de tres dimensiones.

Por transformación de contacto hay que entender desde el punto de vista analítico, toda transformación en la que los valores de las variables $x, y, z$ y las derivadas parciales $dz/dx = p, dz/dy = q$, son expresadas en función de cantidades análogas $x',y',z',p',q'$. Es evidente que en general superficies que se tocan son así transformadas en superficies que se tocan, y es eso lo que justifica el nombre de \textit{transformación de contacto.} Si partimos del punto como elemento del espacio, las transformaciones de contacto se dividen en tres clases: las que, en la triple infinidad de puntos, hacen corresponder de nuevo puntos (son las transformaciones puntuales que acabamos de considerar), las que los transforman en curvas, finalmente las que los transforman en superficies. No hay que considerar esta división como esencial, pues, al hacer uso de otros elementos del espacio en número triplemente infinito, como los planos, encontramos una vez más una división en tres grupos; pero ella no coincide con la división que se obtiene partiendo de puntos.

Si se aplican a un punto todas las transformaciones de contacto, se obtienen la totalidad de los puntos, de las curvas y de las superficies. Es necesaria pues la totalidad de los puntos, de las curvas y de las superficies, para formar un cuerpo de nuestro grupo. Podemos concluir esta regla general, que en el sentido de las transformaciones de contacto, es defectuoso tratar una cuestión (como, por ejemplo, la teoría de las ecuaciones con derivadas parciales que inmediatamente vamos a estudiar) empleando coordenadas de punto o de plano, porque justamente entonces los elementos del espacio escogidos como base no forman un cuerpo.

Pero, si queremos atenernos a los métodos habituales, la introducción, como elemento del espacio, de todos los individuos comprendidos en el cuerpo en cuestión, no es practicable, pues su número es un número infinito de veces infinito. De ahí se desprende la necesidad de introducir como \textit{elemento del espacio}, en estas consideraciones, no el punto, ni la curva, ni la superficie, sino claramente el \textit{elemento de superficie}, es decir, el sistema de valores $x, y, z, p, q$. Por una transformación de contacto cualquiera, cada elemento de superficie se convierte en otro; el quíntuple infinito de elementos de superficie forma pues un cuerpo.

Desde este punto de vista, hay que concebir el punto, la curva y la superficie a la vez como agregados de elementos de superficie, y como agregados de una doble infinidad de esos elementos. La superficie está, en efecto, envuelta por $\infty ^{2}$ elementos, un igual número son tangentes a una curva, y es también el número de los que pasan por un punto. Pero estos agregados, doblemente infinitos, de elementos tienen todavía una propiedad característica común. Si, para dos elementos de superficie consecutivos $x, y, z, p, q $ y $x + dx, y + dy, z + dz, p + dp, q + dq$, se tiene $$ dz - pdx - qdy = 0 $$ los llamaremos \textit{asociados en posición.} Entonces el punto, la curva, la superficie son, los tres, \textit{conjuntos, doblemente infinitos, de elementos cada uno de los cuales está asociado en posición con los que, en número simplemente infinito le son vecinos.} El punto, la curva, la superficie son caracterizados así de una misma manera, y es igualmente así que deben ser representados analíticamente si se quiere tomar como base el grupo de las transformaciones de contacto.

La asociación en posición de dos elementos consecutivos es una relación invariante relativa a toda transformación de contacto. Pero, recíprocamente, se pueden definir también las transformaciones de contacto \textit{como las sustituciones de cinco variables $x, z, y, p, q$, tales que la relación $$ dz - pdx - qdy = 0 $$ sea transformada en ella misma.} En estas investigaciones, el espacio debe ser así mirado como una variedad de cinco dimensiones, y debemos estudiar esta variedad, tomando como grupo fundamental el conjunto de las transformaciones de las variables, que dejan inalterada una relación diferencial determinada.

Los conjuntos representados por una o varias ecuaciones entre las variables, es decir \textit{las ecuaciones con derivadas parciales de primer orden y sus sistemas, }se ofrecen primeramente como temas de estudio. Una cuestión fundamental es saber cómo, de los conjuntos de elementos que satisfacen las ecuaciones dadas, pueden deducirse series simplemente, doblemente infinitas de elementos, tales que cada uno de sus elementos esté asociado en posición con el vecino. A una cuestión semejante se reduce, por ejemplo, el problema de la resolución de una ecuación con derivadas parciales de primer orden. Podemos formularla así: deducir de la cuádruple infinidad de elementos, que satisfacen a la ecuación, todos los conjuntos doblemente infinitos de la naturaleza indicada. En particular, el problema de la solución completa adquiere, desde ese momento, esta forma precisa: dividir la cuádruple infinidad de elementos que satisfacen a la ecuación en una doble infinidad de tales conjuntos.

No podemos tener aquí la intención de llevar más lejos estas consideraciones sobre las ecuaciones con derivadas parciales; remito al respecto a los trabajos de Lie citados. Notemos solamente aún que desde el punto de vista de las transformaciones de contacto, una ecuación con derivadas parciales de primer orden, no tiene ningún invariante, y que cada una de ellas puede ser transformada en cualquier otra, y que por consiguiente, en particular, las ecuaciones lineales no se distinguen ya de otro modo de las otras. Solo cuando volvemos al punto de vista de las transformaciones puntuales se presentan distinciones.

Los grupos de las transformaciones de contacto, de las transformaciones puntuales, y en definitiva, de las transformaciones proyectivas, pueden caracterizarse de una manera común que no puedo silenciar\endnote{ Debo estas definiciones a una observación de Lie}. Ya hemos definido las transformaciones de contacto como aquellos que conservan la asociación en posición de dos elementos consecutivos. Las transformaciones puntuales tienen, en cambio, la propiedad característica de transformar elementos de recta consecutivos asociados en posición en otros de la misma naturaleza. Finalmente las transformaciones homográficas y por dualidad conservan la asociación en posición de dos elementos conexos. Por elemento conexo entiendo el conjunto de un elemento de superficie y de un elemento de recta contenido en él; elementos conexos consecutivos se dice que están asociados en posición cuando no solamente el punto, sino también el elemento de recta de uno está contenido en el elemento de superficie del otro. La denominación (por otra parte provisional) de \textit{elemento conexo} se vincula a los seres nuevamente introducidos en geometría por \textit{Clebsch}\endnote{ \textit{Gött. Ablhandlungen, }t. {\sc xvii}, 1872: \textit{Ueber eine Fundamentalaufgabe der Invariantentheorie, }y también \textit{Gött. Nachrichten, }nº 22, 1872: \textit{Ueber ein Grundgebilde der analytytischen Geometrie del Ebene.}}, que son definidos por una ecuación conteniendo a la vez coordenadas de punto, de plano y de recta, y cuyo análogo en el plano ha recibido por parte de \textit{Clebsch} el nombre de conexa. 

\subsection*{10. Sobre las variedades de un número cualquiera de dimensiones.}

Ya hemos hecho notar en varias ocasiones que al vincularnos, en lo que hemos dicho hasta aquí, en las nociones del espacio, no hacemos más que atenernos al deseo de facilitar los desarrollos de las nociones abstractas apoyándolos en ejemplos claros. En el fondo, estas consideraciones son independientes de la representación sensible, y pertenecen al dominio general de estudios matemáticos que se conocen como la \textit{teoría de variedades} con varias dimensiones o brevemente (según Grassmann) la \textit{teoría de la extensión (Ausdehnungnslehre)}. La manera en que hay que proceder para relacionar lo que hemos dicho del espacio con la pura noción de variedad se concibe por sí misma. Observemos únicamente, una vez más, que, en el estudio abstracto, tenemos, frente a la Geometría, la ventaja de poder escoger enteramente el grupo de transformaciones que queremos tomar como base, mientras que en Geometría un grupo completamente determinado, el grupo principal, estaba dado \textit{a priori}.

No abordaremos ya aquí, y sólo muy ligeramente, más que los tres mé\-todos de tratamientos siguientes: 

\begin{enumerate}

\item \textit{Método de tratamiento proyectivo o álgebra moderna (Teoría de los invariantes). } Su grupo se compone del conjunto de las transformaciones lineales y por dualidad de las variables empleadas para representar el elemento de la variedad: es la generalización de la Geometría proyectiva. Ya hemos hecho notar cómo este método encuentra utilidad en la discusión de lo infinitamente pequeño de una variedad que tiene una dimensión más. Comprende los dos métodos de tratamiento que todavía tenemos que señalar, en el sentido de que su grupo comprende los grupos que hay que tomar como bases en estos.

\item \textit{La variedad con curvatura constante.} En \textit{Riemann} la noción de una tal variedad resulta de la noción más general de una variedad en la cual está dada una expresión diferencial de las variables. El grupo se compone en ellas del conjunto de las transformaciones de las variables que dejan inalterada la expresión dada. Se llega de otra manera a la noción de una variedad con curvatura constante cuando se establece, en el sentido proyectivo, una determinación métrica basada en una ecuación cuadrática dada entre las variables. Este método, en oposición al de Riemann, permite la generalización de que las variables pueden suponerse complejas; podemos a continuación limitar la variabilidad del campo real. A~esta rama pertenece la serie ampliada de investigaciones que hemos abordado en {\S}{\S}5,6,7.

\item \textit{La variedad plana.} Riemann designa por variedad plana a una variedad con curvatura constante nula. Su teoría es la generalización inmediata de la Geometría elemental. Su grupo puede, como el grupo principal en Geometría, desprenderse del grupo proyectivo, manteniendo fija una figura representada por dos ecuaciones, una lineal y la otra cuadrática. Si queremos adaptarnos a la forma en la cual la teoría es presentada habitualmente, debemos distinguir entre lo real y lo imaginario. La Geometría elemental misma figura en el primer rango en esta teoría y después vienen las generalizaciones recientemente desarrolladas de la teoría ordinaria de la curvatura, etc.

\end{enumerate} 

\subsection*{Observaciones finales}

Para terminar situaremos dos observaciones que están en relación íntima con lo que hemos dicho hasta aquí: la primera concierne al algoritmo que hay que utilizar para la representación de las nociones desarrolladas hasta aquí; en la segunda indicaremos algunos problemas que parecería importante y fecundo tratar según los puntos de vista que hemos dado.

Se ha reprochado frecuentemente a la Geometría analítica el que ponga por delante, mediante la introducción del sistema de coordenadas, elementos arbitrarios, y este reproche alcanza igualmente a toda manera de tratar de las variedades en las cuales el elemento viene caracterizado por los valores de variables. Si este reproche no estaba sino demasiado justificado por la manera defectuosa con la que se manejaba, sobre todo antiguamente, el método de las coordenadas, se desvanece de todos modos desde el momento en que ese método se emplea racionalmente. Las expresiones analíticas que pueden presentarse en el estudio de una variedad en el sentido de un grupo deben ser, en razón de su significación, independientes del sistema de coordenadas, en tanto que este sigue siendo arbitrario; y se trata simplemente de poner también en evidencia esta independencia \textit{en las fórmulas.} El álgebra moderna muestra que eso es posible y la manera en que se hace; la noción de invariante, fundamental aquí, y de la que se trata es en ella en efecto puesta de relieve de la manera más evidente. Tiene una ley de formación general y perfecta de las expresiones invariantes y se restringe a no operar más que con estas. Es lo que resulta necesario que proporcione aún el tratamiento analítico cuando otros grupos diferentes del grupo proyectivo son tomados como base\endnote{ Por ejemplo, para el grupo de las rotaciones del espacio de tres dimensiones alrededor de un punto fijo, un tal algoritmo viene dado por los cuaternios. }. Es necesario claramente, en efecto, que el algoritmo se adapte a lo que se desea, que se lo utilice como una expresión clara y precisa de la concepción, o que se lo quiera utilizar para penetrar fácilmente en campos todavía no explorados.

Nos vemos conducidos a plantear los problemas de los que todavía quisiéramos hablar por una comparación entre las ideas que hemos expuesto y lo que se conoce como la \textit{teoría de las ecuaciones de Galois}.

En la teoría de Galois como aquí, todo el interés reside en los grupos de transformaciones. Pero los objetos con los cuales se relacionan las transformaciones son muy diferentes: allí nos las tenemos que ver con un número limitado de elementos distintos, aquí con un número indefinido de elementos de un conjunto continuo; pero se puede, gracias a la identidad de la noción de grupo, llevar más lejos la comparación\endnote{ Recordaré aquí que Grassmann, ya en la introducción de la primera edición de su \textit{Ausdehnungslehre }(1844), ha establecido un paralelismo entre el Análisis combinatorio y la\textit{ Teoría de la extensión (Ausdehnungslehre).}}, y lo indicaremos aquí tanto más gustosamente como que así se encuentra caracterizada la posición que es necesario atribuir a ciertas investigaciones empezadas, por Lie y yo mismo\endnote{ Ver nuestra Memoria: \textit{Ueber diejenigen ebenen Curven, welche durch ein geschlossenes System von einfach unendlich vertauschbaren linearen Transformationen in sich übergehen }(\textit{Math. Annalen, }t. {\sc iv}).}, en el sentido de los puntos de vista desarrollados aquí.

En la teoría de Galois tal como se expone, por ejemplo, en el \textit{Traité d'Al\-gèbre supérieur }de\textit{ Serret}\endnote{ Cf. Joseph-Alfred Serret, \textit{Cours d'Algèbre supérieure, }4${}^{a}$ ed., 1877-1879, reed. por Ed. Jacques Gabay.} o en el \textit{Traité des substitutions }de \textit{C. Jordan}\endnote{ Cf. Camille Jordan, \textit{Traité des substitutions et des équations algébriques, }1870, reed. por Ed. Jacques Gabay.}, lo que constituye propiamente el objeto de las investigaciones, es la teoría misma de los grupos o de las sustituciones; la teoría de las ecuaciones se deduce de ello como aplicación. Por analogía, quisiéramos una \textit{teoría de las transformaciones,} una teoría de los grupos que pueden engendrarse mediante transformaciones de una naturaleza dada. Las nociones de conmutatividad, de semejanza, etc., encontrarían empleo como en la teoría de las sustituciones. El tratamiento de una variedad sacado de la consideración de un grupo fundamental de transformaciones aparecería como una aplicación de la teoría de las transformaciones. 

 En la teoría de las ecuaciones son primeramente las funciones simétricas de los coeficientes las que ofrecen un interés, pero a continuación son las expresiones que permanecen inalteradas, sino para todas las permutaciones de las raíces, al menos para un gran número de ellas. En el tratamiento de una variedad con un grupo tomado como fundamental, quisiéramos en primer lugar, por analogía, determinar los cuerpos ({\S}6), las figuras que permanecen inalteradas por todas las transformaciones del grupo; pero hay figuras que no admiten todas las transformaciones del grupo, sino solamente algunas de ellas, y estas figuras, en el sentido del tratamiento basado sobre el grupo, son particularmente interesantes, gozan de propiedades notables. Así, por ejemplo, en el sentido de la Geometría ordinaria, se distinguen cuerpos simétricos irregulares, superficies de revolución y helicoidales. Si uno se sitúa en el punto de vista de la Geometría proyectiva y se pide en particular que las transformaciones que conducen las figuras a ellas mismas sean permutables, se llega a las figuras consideradas por Lie y por mi en la Memoria citada, y al problema general que se plantea allí en el {\S}6. En los {\S}{\S}1,3 se encuentra la determinación de todos los grupos que contienen una infinidad de transformaciones lineales, permutables en el plano; en definitiva, es una parte de la teoría general de las transformaciones lo que hemos pretendido formular aquí y es fundamentalmente de esto de lo que acabamos de hablar\endnote{ Debo rehusarme aquí a mostrar cuán fructuosa es, para la Teoría de las ecuaciones diferenciales, la consideración de las transformaciones infinitamente pequeñas. En el {\S}7 del trabajo citado, Lie y yo hemos mostrado que: ecuaciones diferenciales ordinarias que admiten las mismas transformaciones infinitamente pequeñas presentan las mismas dificultades de integración. Lie, en diferentes lugares y, en particular, en la Memoria citada más arriba (\textit{Math. Annalen, }t. {\sc v}), ha hecho ver, con varios ejemplos, cómo deben ser aplicadas estas consideraciones para las ecuaciones con derivadas parciales (ver también en particular las Comunicaciones a la Academia de Christiania, mayo de 1872).

 [Puedo indicar hoy el hecho de que los dos problemas mencionados en el texto han continuado dirigiendo precisamente una gran parte de los trabajos ulteriores de Lie y mios. Por lo que concierne a Lie, hemos de citar sobre todo su \textit{Teoria de los grupos continuos de transformaciones}, cuya exposición sistemática es objeto hasta aquí de dos volúmenes (Leipzig, t. {\sc i}, 1888; t. {\sc ii}, 1890). Entre mis investigaciones posteriores al presente escrito, puedo señalar las que se refieren a los cuerpos regulares, a las funciones modulares elípticas y, en general, sobre las funciones uniformes que admiten transformaciones lineales. Ya, en 1884, expuse las primeras en una obra especial: \textit{Vorlesungen über das Ikosaeder und die Auflösung der Gleichungen vorn fünften Grade }(Leipzig); poco después ha aparecido el primer volumen de una exposición (para la que M. Fricke me prestó una ayuda esencial) de la \textit{Theorie des fonctions modulaires elliptiques} (Leipzig, 1890).]}. 

\newpage 

\subsection*{Notas}

\noindent\textbf{Nota I. Sobre la oposición, en la Geometría moderna, entre los métodos sintético y analítico.}

 En la actualidad, no debe ya considerarse como esencial la diferencia entre las \textit{Geometrías sintética }y \textit{analítica }modernas, pues las materias estudiadas y el modo de discusión han llegado a ser ahora enteramente semejantes. Así hemos escogido el término \textit{Geometría proyectiva, }para designarlas a ambas en el texto. Si el método sintético procede más mediante la intuición del espacio, dando así a sus primeras teorías elementales un atractivo particular, el campo de una tal intuición no está por ello cerrado al método analítico, y pueden concebirse las fórmulas de la Geometría analítica como una expresión clara y precisa de relaciones geométricas. Por otro lado, no hay que menospreciar, de cara a investigaciones posteriores, el beneficio que procura, superando de alguna manera el pensamiento, un algoritmo apropiado. De todos modos, no hay que abandonar la prescripción de que una cuestión matemática no debe ser considerada como completamente agotada hasta que no ha devenido como intuitivamente evidente; descubrir por medio del Análisis, es dar un paso muy importante, pero sólo es dar un primer paso. 

 

\noindent\textbf{Nota II. Escisión de la Geometría moderna en disciplinas.}

 Si, por ejemplo, se observa como el físico matemático rechaza generalmente la ventaja que sacaría, en muchos casos, de una intuición proyectiva incluso poco desarrollada; como, por otro lado, aquel que cultiva la Geometría proyectiva aborda poco la rica mina de verdades matemáticas de donde ha nacido la teoría de la curvatura de las superficies, nos vemos obligados a pensar el estado actual del estudio de la Geometría como muy imperfecto, y, según todas las apariencias, como transitorio. 

 \noindent\textbf{Nota III. Sobre la importancia de la intuición del espacio.}

 Cuando, en el texto, hablamos de la intuición del espacio como de algo accesorio, lo hacemos en razón de la naturaleza puramente matemática de las consideraciones a formular: para estas aquella intuición no tiene sino el valor de un método que hace las cosas sensibles, valor que, por otra parte, desde el punto de vista pedagógico, debe ser estimado como de gran valor. Así un modelo geométrico es, desde este punto de vista, de lo más interesante y de lo más instructivo.

 Pero es de otra cosa de lo que se trata cuando hablamos de la importancia de la intuición del espacio en general. Yo la considero como subsistente por sí misma. Existe una Geometría propiamente dicha que no puede, como las investigaciones que nos han ocupado, dejar de ser más que una forma sensible de consideraciones abstractas. Hay que concebir las figuras del espacio en la plena verdad de su forma y (lo que constituye el lado matemático) apercibir sus relaciones como consecuencias evidentes de los postulados de la intuición del espacio. Para esta Geometría, un modelo, que sea aplicado y examinado o solamente representado con fuerza, no es un medio para alcanzar el objetivo, sino la cosa misma.

 Cuando situamos así, con una existencia propia, la Geometría junto a las Matemáticas puras y sin que ella dependa de aquellas, hacemos nada menos que algo nuevo. Es sin embargo deseable poner de nuevo una vez más, explícitamente, este punto en evidencia, puesto que las investigaciones recientes lo pierden casi completamente de vista. Igualmente así, recíprocamente, los métodos de investigación nuevos han sido raramente aplicados, cuando habrían podido serlo, al estudio de las relaciones de forma de los seres del espacio, y, sin embargo, en esta vía, parece que sean particularmente fecundos. 



 \noindent \textbf{Nota IV. Sobre las variedad de un número cualquiera de dimensiones.}

 Que el espacio considerado como lugar de puntos sólo tenga tres dimensiones, es lo que, desde el punto de vista matemático, no es necesario discutir; pero no podría en adelante, desde el mismo punto de vista, impedirse a cualquiera afirmar que hay cuatro o un número cualquiera, pero que solamente podemos \textit{percibir} tres. La teoría de las variedades en varias dimensiones, tal como se desprende cada vez más de las investigaciones matemáticas modernas es, por naturaleza, completamente independiente de semejante afirmación. Sin embargo, se ha establecido una manera de hablar que, sin duda, se deriva de esta idea. En lugar de elementos de un conjunto continuo, se habla de los puntos de un espacio superior, etc. En sí misma, esta manera de expresarse es bastante buena porque, recordando las concepciones geométricas, facilita su inteligencia [comprensión]. Pero la misma ha tenido la molesta consecuencia de que, para muchos, las investigaciones sobre las variedades de [en] varias dimensiones son pensadas como haciendo uno con las ideas que acabamos de recordar acerca de la naturaleza del espacio. Nada es menos fundado que esta creencia. Si estas ideas fueran justas, esas investigaciones matemáticas hallarían inmediatamente una aplicación geométrica; pero su valor como su fin, plenamente independientes de ellas, se encuentran en su naturaleza puramente matemática.

 Muy diferente es la manera indicada por Plücker de considerar el verdadero espacio como una variedad con un número cualquiera de dimensiones introduciendo como elemento (ver {\S}5 del texto) una figura (curva, superficie, etc.) que depende de un número cualquiera de parámetros. 

 En el \textit{Ausdehnungslehre} de Grassmann (1844) se encuentra desarrollada por primera vez esta manera de ver, donde se considera el elemento de una variedad de un número cualquiera de dimensiones como el análogo del punto del espacio. Grassmann no se preocupa en modo alguno de las ideas que hemos recordado sobre la naturaleza del espacio; estas ideas se remontan a observaciones hechas pasando por Gauss y se han expandido después de las investigaciones de Riemann sobre las variedades de varias dimensiones, investigaciones con las cuales se encuentran mezcladas.

 Las dos maneras de ver, tanto la de Grassmann como la de Plücker, tienen sus ventajas particulares; ambas pueden utilizarse, tanto la una como la otra, con provecho. 

 \noindent\textbf{Nota V. Sobre la llamada Geometría no euclidiana}

 Como han mostrado recientes investigaciones, la Geometría métrica proyectiva de la que tratamos en el texto coincide esencialmente con la Geometría métrica que se obtiene cuando se rechaza el postulado de las paralelas y que, bajo el nombre de \textit{Geometría no euclídea}, es objeto en la actualidad de frecuentes debates y discusiones. Si, en lo que precede, no hemos utilizado, en general, esta expresión, es por una razón que se vincula con las consideraciones de la nota anterior [nota IV]. Suelen asociarse al nombre de Geometría no euclidiana multitud de ideas que no tienen nada de matemático, aceptadas con tanto entusiasmo por un lado como la repulsión que provocan del otro, y con las cuales, en todos los casos, nuestras consideraciones exclusivamente matemáticas no tienen nada que ver en absoluto. Mediante las consideraciones que siguen hemos querido aportar algún esclarecimiento a esta distinción.

 Las investigaciones en cuestión sobre la teoría de las paralelas y sus desarrollos sucesivos tienen una importancia matemática precisa por dos lados.

 Muestran en primer lugar, y podemos considerar esta cuestión como definitivamente zanjada, que el axioma de las paralelas no es una consecuencia matemática de los axiomas generalmente situados antes del mismo, sino que es la expresión de un hecho intuitivo esencialmente nuevo que quedó intacto en las investigaciones que le precedieron. Semejante discusión podía y debía realizarse, incluso en otros lugares que en el ámbito de la Geometría, en relación con cada axioma; se ganaría con ello con respecto a la situación respectiva de estos [en la teoría].

 En segundo lugar, estas investigaciones nos han proporcionado una noción matemática preciosa, la de una variedad con curvatura constante. Ella está ligada, como observaremos y desarrollaremos más ampliamente en el {\S}10, de la manera más estrecha, con la determinación métrica proyectiva desarrollada independientemente de toda teoría de las paralelas. Si, en sí mismo, el estudio de esta determinación métrica ofrece un gran interés matemático y permite numerosas aplicaciones, comprende además, como caso particular (caso límite), la determinación métrica dada en la Geometría y enseña a considerarla desde un punto de vista más elevado.

 Absolutamente independiente de estas consideraciones es la cuestión de saber sobre que descansa el axioma de las paralelas, si debe ser considerado como dado de una forma absoluta, cosa que algunos desean, o como establecido empíricamente y solamente de manera aproximada por la experiencia, cosa que pretenden otros. Si hubiera razones para aceptar esta última manera de ver, las investigaciones matemáticas en cuestión nos mostrarían entonces como debe construirse una Geometría más exacta. Pero es esa evidentemente una cuestión filosófica que afecta a los principios más generales de nuestro entendimiento. No interesa al matemático \textit{como tal, }y puede anhelar que sus investigaciones no sean consideradas como dependientes de la respuesta que, de un lado o del otro, pueda dársele. 



\noindent\textbf{Nota VI. La Geometría del espacio reglado como estudio de una variedad con curvatura constante.}



 Conectando una con otro la Geometría del espacio reglado y la determinación métrica proyectiva en una variedad de cinco dimensiones, debemos prestar atención al hecho de que las rectas no nos ofrecen (en el sentido de la determinación métrica) más que los elementos al infinito de la variedad. Resulta así necesario examinar cuál es el valor de una determinación métrica proyectiva para sus elementos al infinito; vamos a desarrollar aquí esta cuestión para descartar las dificultades que se oponen a la concepción de la Geometría del espacio reglado como Geometría métrica. Conectamos estos desarrollos con el ejemplo intuitivo que ofrece la determinación métrica proyectiva basada en una superficie de segundo grado.

 Dos puntos tomados arbitrariamente en el espacio tienen, relativamente a la superficie, un invariante absoluto: la relación inarmónica que forman con los dos puntos de intersección de la recta que los une y de la superficie; pero, si los dos puntos vienen a situarse en la superficie, la relación inarmónica tiende a cero independientemente de la posición de los puntos, excepto en el caso en que los dos puntos vengan a situarse sobre una generatriz, en cuyo caso se hace indeterminado; es el único caso particular al que da lugar su posición relativa si no coinciden; tenemos así el siguiente teorema:

\begin{quote}\it 

La determinación métrica proyectiva que se puede basar en el espacio en una superficie de segundo grado no proporciona ninguna determinación métrica para la Geometría sobre esta superficie.

\end{quote} 

Con esto se relaciona el hecho de que se puede, mediante transformaciones lineales de la superficie en ella misma, llevar tres cualesquiera de sus puntos a coincidir con otros tres\endnote{ Estas relaciones están alteradas en la geometría métrica ordinaria; por dos puntos al infinito, hay ciertamente un invariante absoluto. La contradicción que podría encontrarse así al contar las transformaciones lineales que admite la superficie al infinito se levantan al considerar que las traslaciones y las transformaciones de semejanzas, que están en el número de estas transformaciones, no alteran en modo alguno el infinito.}

Para tener en la superficie misma una determinación métrica, es necesario restringir el grupo de las transformaciones, y eso se logra manteniendo fijo un punto cualquiera del espacio (o su plano polar). Supongamos primeramente que el punto no esté en la superficie. Desde este punto se la proyecta entonces sobre un plano, lo que produce una cónica como curva de contorno aparente. Sobre esta cónica se basa, en el plano, una determinación métrica proyectiva que se lleva a continuación sobre la superficie. Es una determinación métrica propiamente dicha, con curvatura constante, de lo que resulta el siguiente teorema:

\begin{quote}\it 

Una determinación métrica con curvatura constante se obtiene sobre la superficie desde el momento en que se mantiene fijo un punto que no está sobre la superficie.

\end{quote} 

Encontramos de la misma manera que:

\begin{quote}\it 

Tomando como punto fijo un punto de la superficie misma, se obtiene sobre ésta una determinación métrica con curvatura nula.

\end{quote}

 Para todas estas determinaciones métricas sobre la superficie, las generatrices son líneas de longitud nula. Las expresiones de elemento de arco de la superficie no difieren pues, en las diferentes determinaciones, más que por un factor constante. No hay sobre la superficie un elemento de arco absoluto, pero se puede muy bien hablar del ángulo que forman sobre la superficie dos direcciones.

 Ahora todos estos teoremas y todas estas consideraciones pueden ser a continuación aplicadas para la Geometría del espacio reglado. Para el espacio reglado mismo no existe \textit{a priori }ninguna determinación métrica propiamente dicha. Sólo se obtiene una cuando se mantiene fijo un complejo lineal, y entonces ella tiene una curvatura constante o nula según que el complejo sea general o particular (una recta). A la elección de ese complejo está también ligada la existencia de un elemento de arco absoluto. Cualquiera que sea esa elección, la distancia de dos rectas infinitamente próximas que se cortan es nula, y se puede también hablar del ángulo que forman entre sí dos rectas infinitamente próximas a una recta dada\endnote{ Véase la Memoria: \textit{Ueber Liniengeometrie und metrische Geometrie }(\textit{Math. Annalen, }t. {\sc v}, p. 271).}. 

 

\noindent\textbf{Nota VII. Sobre la interpretación de las formas binarias.} 



 Mostraremos aquí qué representación simple se puede, por medio de la interpretación de $x + iy$ sobre la esfera, obtener para los sistemas formas que se vinculan con la forma binaria cúbica y con la forma binaria bicuadrática.

 Una forma binaria cúbica $f$ tiene una covariante cúbica $Q$, una cuadrática $\Delta $ y un invariante\endnote{ Ver los capítulos de Clebsch referentes a la cuestión: \textit{Theorie der binären Formen.}} $R$. Con $f $ y $Q$ se forma toda una serie de covariantes de sexto grado $$ Q^2+\lambda Rf^2 $$ entre los cuales igualmente $\Delta^{3}$. Se puede demostrar\endnote{ Por la consideración de las transformaciones lineales de $f $ en ella misma. Ver \textit{Math. Ann., }{\sc iv}, p. 352.} que cada covariante de la forma cúbica se descompone en tales sistemas de seis puntos. $\lambda $ puede tomar valores complejos, hay una doble infinidad de los mismos.

 El conjunto de formas así definido puede ser representado sobre la esfera de la manera siguiente\endnote{ Ver también Beltrami: \textit{Ricerche sulla Geometría delle forme binarie cubiche }(\textit{Memorie Acc. Bologna; }1870). }: mediante una transformación lineal conveniente, llevemos los tres puntos representados por $f$ en tres puntos equidistantes sobre un gran círculo. Ese gran círculo puede ser tomado como ecuador; las longitudes de los tres puntos $f $ situados sobre él son $0^0$, $120^o$, $240^o$. $Q$ es entonces representado por los puntos del ecuador cuyas longitudes son $60^o$, $180^{o}$, $300^{o}$, $\Delta$ lo es por los dos polos. Cada forma $Q^{2}+\lambda Rf^{2}$ es representada por seis puntos cuya latitud y longitud están contenidas en la tabla siguiente, donde $\alpha$ y $\beta$ designan números cualesquiera $$ \begin{array}{c|c|c|c|c|c|} \alpha &\alpha &\alpha &-\alpha &-\alpha &-\alpha \\ \beta& 120+\beta& 240+\beta&-\beta& 120-\beta& 240-\beta \end{array} $$ Es interesante ver, examinando la sucesión de estos sistemas de puntos sobre la esfera, cómo se deducen de ello $f$ y $Q$ contados dobles, y $\Delta$ contado triple.

 Una forma bicuadrática tiene un covariante $H$ también bicuadrático, un covariante de sexto grado $T$, y dos invariantes $i$ y $j$. El conjunto de formas bicuadráticas $i H + \lambda j f$, que corresponden todas al mismo $T$, es particularmente notable; a este conjunto pertenecen los tres factores cuadráticos en los cuales se puede descomponer $T$, cada uno de ellos se cuenta doble.

 Tracemos ahora, por el centro de la esfera, tres ejes rectangulares $Ox$, $Oy$, $Oz$. Los seis puntos de intersección con la esfera figuran la forma $T$. Designando por $x, y, z$ las coordenadas de un punto cualquiera de la esfera, los cuatro puntos que corresponden a una bicuadrática $iH +\lambda j f$ son dados por la tabla $$ \begin{array}{rrr} x& y & z \\ x & -y& -z \\ -x & y & -z \\ -x & -y & z \end{array} $$

 Esos cuatro puntos son siempre los vértices de un tetraedro simétrico cuyos lados opuestos están divididos en dos partes iguales por los ejes del sistema de coordenadas; el papel que juega T, como resolvente de $i H + \lambda j f$, en la teoría de las ecuaciones bicuadráticas es así puesto en evidencia. 



 \newpage

\theendnotes 

\end{document} 

