\documentclass[a4paper, 12pt]{article}

%%%%%%%%%%%%%%%%%%%%%%Paquetes
\usepackage[spanish]{babel}  
\usepackage[utf8]{inputenc}
\usepackage{tcolorbox}
\usepackage{cmbright}  %%%%%%% El tipo de letra
\usepackage{setspace}
\onehalfspacing  %%%%%%%%%%% Espacio y medio de interlineado
\parskip=1em  %%%%%%%%%%%% Separacion entre parrafos
%%%%%%%%%%%%%%%%%%%%%%



%%%%%%%%%%%%%%%%%
\title{La Irrazonable Eficacia de las Matemáticas}
\author{E. Wigner}
\date{}
%%%%%%%%%%%%%%%%%

\begin{document}

\begin{tcolorbox}[colback=blue!5!white,colframe=blue!75!black]

\vspace{-1.8cm}
\textbf \maketitle

\end{tcolorbox}

\bigskip


\begin{quote}

{\it Las matemáticas, consideradas correctamente, poseen no solamente verdad, sino una suprema belleza fría y austera, como la de una escultura, que no apela a ningún aspecto de nuestra más débil naturaleza, y que carece de los primorosos atavíos de la pintura o de la música, aunque es de una pureza sublime y capaz de una perfección rigurosa como solamente puede exhibir el arte más elevado. El verdadero espíritu del deleite, de la exaltación, del sentimiento de ser más que humano, que es la piedra de toque de la más alta perfección, ha de buscarse en las matemáticas al igual que en la poesía.}

\end{quote}

\hfill BERTRAND RUSSELL, Study of Mathematics


\bigskip


Existe un relato acerca de dos amigos que habían sido compañeros de clase durante sus estudios de escuela secundaria y que hablan acerca de sus trabajos actuales. Uno de ellos se ha convertido en un estadístico y se ocupa de las tendencias de la población. Muestra un ejemplar publicado de su trabajo a su antiguo compañero. El trabajo comienza, como es usual, con la distribución gaussiana, y el estadístico explica a su amigo el significado de los símbolos relativos a la población real, a la población promedio, etcétera. Su compañero se mostraba algo incrédulo y no estaba muy seguro de que su amigo no le estuviera tomando el pelo. ``¿Cómo puedes saber eso?'' indagó. ``¿Y qué símbolo es este de aquí?'' ``Ah, ---contestó el estadístico--- se trata de $\pi$''. ``¿Y eso qué es?'' ``La razón de la circunferencia a su diámetro''. ``Vaya, ahora estás llevando la broma demasiado lejos ---dijo su antiguo compañero--- pues estoy seguro de que la población no tiene nada que ver con la circunferencia''.

Como es natural, nos sentimos inclinados a sonreír ante la ingenuidad del antiguo compañero de clase. No obstante, cuando escuché esta historia, tuve que admitir un sentimiento de escalofrío porque, con seguridad, la reacción del condiscípulo deja traslucir solamente el sentido común más llano. Quedé aún más confundido cuando, no muchos días más tarde, alguien me expresó su desconcierto [1 La observación que se cita a continuación se debe a F. Werner, en sus tiempos de estudiante en Princeton] por el hecho de que hacemos una selección bastante estrecha cuando elegimos los datos que han de verificar nuestras teorías. ``¿Cómo podemos estar seguros de que si establecemos una teoría que enfoque su atención en los fenómenos que desdeñamos y que desdeñe algunos de los fenómenos que ahora reclaman nuestra atención, no podemos construir otra teoría que tenga poco en común con la presente pero que sin embargo explique tantos fenómenos como ella?'' Debemos admitir que no tenemos evidencia definitiva de que no exista una teoría tal.

Las dos historias precedentes ilustran los dos puntos de vista principales objeto del presente discurso. El primer punto es que los conceptos matemáticos se revelan en conexiones completamente inesperadas. Es más, con frecuencia permiten una descripción sorprendentemente precisa de los fenómenos involucrados en tales conexiones. En segundo lugar, y precisamente debido a dicha circunstancia, y puesto no que entendemos las razones de su utilidad, no podemos saber si una teoría formulada en términos de conceptos matemáticos es la única correcta. Estamos en una posición comparable a la de alguien al que se le ha entregado un manojo de llaves y que, teniendo que abrir varias puertas de modo sucesivo, acierta siempre con la llave correcta al primer o segundo intento. Se volverá escéptico en relación con la unicidad de la coordinación entre las llaves y las puertas.



La mayor parte de lo que se dirá sobre estas cuestiones no será nuevo; se le habrá ocurrido probablemente de una forma u otra a la mayoría de los científicos. Mi intención principal es iluminarlas desde diversas vertientes. El primer punto es que la enorme utilidad de la matemática en las ciencias naturales es algo rayano en el misterio, y que no existe ninguna explicación racional para ello. En segundo lugar, justamente a causa de esta portentosa utilidad de los conceptos matemáticos, surge la cuestión de la unicidad de nuestras teorías físicas. Con el fin de establecer el primer punto, el de que la matemática desempeña un papel de importancia irrazonable en la física, será útil decir algunas palabras sobre la pregunta ¿qué es la matemática?, seguida de la ¿qué es la física?, a continuación, cómo la matemática se incorpora a las teorías físicas, y por último, por qué el éxito de la matemática en su papel en la física parece ser tan desconcertante. Mucho menos se dirá acerca del segundo punto: la unicidad de las teorías de la física. Una respuesta apropiada a esta cuestión requeriría un trabajo teórico y experimental elaborado que no ha sido llevado a cabo hasta ahora.

\subsection*{¿Qué es la matemática?}

Alguien dijo una vez que la filosofía es el abuso de una terminología que se inventó precisamente con ese propósito. [2. Esta frase está citada aquí del libro de  W. Dubislav ``Die Philosophie der Mathematik in der Gegenwart'' (Berlin: Junker and Dunnhaupt Verlag, 1932), p. 1.] En el mismo sentido, yo diría que la matemática es la ciencia de operaciones expertas con conceptos y reglas inventados justamente con dicho fin. La matemática pronto se marginaría de teoremas interesantes si éstos se tuvieran que formular en términos de los conceptos que aparecen en los axiomas. Es más, si bien es una verdad incuestionable que los conceptos de la matemática elemental y en particular de la geometría elemental fueron formulados para describir entidades directamente sugeridas por el mundo real, ello no parece ser cierto en lo que se refiere a conceptos más avanzados, en particular los que representan un papel tan importante en la física. Así, las reglas para las operaciones con pares de números están diseñadas obviamente para dar los mismos resultados que las operaciones con fracciones que aprendimos primero sin referencia a ``parejas de números''. Las reglas para las operaciones con series, es decir, con números irracionales, pertenecen todavía a la categoría de reglas que fueron determinadas cuidando de reproducir las reglas de las operaciones con cantidades que ya nos eran conocidas. Conceptos matemáticos mucho más avanzados, tales como los números complejos, las diversas álgebras, los operadores lineales, los conjuntos de Borel (y esta lista podría continuar casi indefinidamente), fueron ideados por ser asuntos adecuados en los cuales el matemático puede demostrar su ingenio y sentido de la belleza formal. De hecho, la definición de tales conceptos, con la noción de que se pueden aplicar a ellos consideraciones ingeniosas e interesantes, es la primera demostración de la destreza del matemático que los define. La profundidad del pensamiento implícita en la formulación de los conceptos matemáticos se justifica después por la destreza con la que se emplean. El gran matemático saca provecho por completo, casi implacablemente, del dominio del razonamiento permisible y roza el no permisible. El que su temeridad no le conduzca a un terreno pantanoso de contradicciones es un milagro en sí mismo: es ciertamente difícil de creer que nuestra capacidad de razonamiento haya sido conducido, por el proceso darviniano de la selección natural, a la perfección que parece poseer. No es éste, sin embargo, nuestro objetivo presente. El punto principal que recordaremos más tarde es que el matemático podría formular solamente un conjunto de teoremas interesantes sin definir conceptos más allá de los que están contenidos en los axiomas, y que los conceptos que están fuera de los contenidos en los axiomas se definen con vistas a permitir sutiles operaciones lógicas que apelan a nuestro sentido estético, tanto en cuanto tales operaciones como también en cuanto a sus resultados de gran generalidad y sencillez. [3 M. Polanyi, en su ``Personal Knowledge'' (Chicago: University of Chicago Press, 1958), dice: ``Todas esas dificultades no son sino consecuencia de nuestro rechazo de tratar de ver que la matemática no puede definirse sin el reconocimiento de su característica más obvia, es decir, que es interesante''  (p 188).]



Los números complejos proporcionan un ejemplo particularmente llamativo de lo anterior. Nada en nuestra experiencia, ciertamente, sugiere la introducción de tales cantidades. En realidad, si a un matemático se le pide que justifique su interés en los números complejos, indicará con cierta indignación los muchos y bellos teoremas de la teoría de ecuaciones, de las series de potencias y de las funciones analíticas en general, que deben su origen a la introducción de los números complejos. El matemático no desea abandonar su interés en estos los logros más bellos de su talento.  [4 El lector podría estar interesado, en relación con esto, en los comentarios bastante irritados de Hilbert acerca del intuicionismo, que ``tratan de destrozar y de desfigurar las matemáticas'' Abh. Math. Sem., Univ. Hamburg, 157 (1922), or Gesammelte Werke (Berlin: Springer, 1935), p. 188.]

\subsection*{¿Qué es la física?}


El físico está interesado en descubrir las leyes de la naturaleza inanimada. Con el fin de comprender esta frase, es necesario analizar el concepto ``ley de la naturaleza''.

El mundo que nos rodea es de una complejidad desconcertante y el hecho más obvio en relación con ello es que no podemos predecir el futuro. A pesar de que el chiste atribuye solamente al optimista la opinión de que el futuro es incierto, éste tiene razón en este caso: el futuro es impredecible. Es un milagro, como ha señalado Schroedinger, que a pesar de la perturbadora complejidad del mundo, puedan descubrirse en los fenómenos ciertas regularidades. Una regularidad tal, descubierta por Galileo, es que dos piedras, dejadas caer a la vez desde la misma altura, alcanzan el suelo al mismo tiempo. Las leyes de la naturaleza conciernen a tales regularidades. La regularidad de Galileo es un prototipo de un conjunto mayor de regularidades. Se trata de una regularidad sorprendente, y ello por tres razones.

La primera razón por la que es sorprendente es que se cumple no solamente en Pisa, y en la época de Galileo, sino que es cierta en todos los lugares de la Tierra, siempre ha sido cierta, y siempre será cierta. La propiedad de la regularidad es una propiedad reconocida de invariancia y, como tuve ocasión de señalar hace algún tiempo, sin principios de invariancia similares a los que están implícitos en la generalización anterior de la observación de Galileo, la física no hubiera sido posible. La segunda característica sorprendente es que la regularidad de la que estamos tratando es independiente de muchísimas condiciones que podrían tener efecto sobre la misma. Es válida con independencia de que llueva o no, de que el experimento se lleve a cabo en una habitación o desde la Torre Inclinada, de si la persona que deja caer las rocas es hombre o mujer. Es válida incluso en el caso de que las dos rocas se dejen caer, simultáneamente y desde la misma altura, por dos personas distintas. Existen, como es obvio, otras innumerables condiciones que son del todo intrascendentes en lo que hace a la validez de la regularidad de Galileo. La irrelevancia de tantas circunstancias que podrían ejercer un papel en el fenómeno observado ha sido calificada también de invariancia. Esta invariancia, sin embargo, es de un tipo distinto del precedente, puesto que no puede formularse como un principio general.  La exploración de las condiciones que ejercen o no su influencia sobre un fenómeno es parte de la primera exploración de un campo de actividad. La destreza y el ingenio del experimentador le harán ver los fenómenos que dependen de un conjunto relativamente reducido de condiciones relativamente fáciles de llevar a cabo y de reproducir. [5 En relación con esto véase el ensayo gráfico de M. Deutsch, Daedalus 87, 86 (1958). A. Shimony ha llamado mi atención sobre un pasaje semejante de la obra de C. S. Peirce ``Essays in the Philosophy of Science'' (New York: The Liberal Arts Press, 1957), p. 237.] En el caso presente, la restricción de Galileo de sus observaciones a cuerpos relativamente pesados fue el paso más importante en este aspecto. Es cierto de nuevo que si no hubiera fenómenos que fueran independientes de todas excepto un conjunto realizablemente pequeño de condiciones, la física hubiera sido imposible.

Los dos puntos anteriores, aunque altamente significativos desde el punto de vista del filósofo, no son los que más sorprendieron a Galileo, ni tampoco contienen una ley específica de la naturaleza. La ley de la naturaleza está contenida en la afirmación de que el tiempo que tarda un objeto pesado en caer desde una altura determinada es independiente del tamaño, material y forma del cuerpo que cae. En el marco de la segunda ``ley'' de Newton, esto equivale a la afirmación de que la fuerza gravitatoria que actúa sobre un cuerpo que cae es proporcional a su masa pero independiente del tamaño, composición y forma del cuerpo que cae.

El argumento anterior intenta recordarnos, en primer lugar, que no es en absoluto natural que existan ``leyes de la naturaleza'', y mucho menos que seamos capaces de descubrirlas. [6 E. Schroedinger, en su ``What Is Life?'' (Cambridge: Cambridge University Press, 1945), p. 31, dice que este segundo milagro podría estar muy bien más allá del entendimiento humano]. El que esto escribe tuvo la ocasión, hace cierto tiempo, de llamar la atención sobre la serie de capas de ``leyes de la naturaleza'', cada una de las cuales contiene leyes más generales y más incluyentes que la previa, y su descubrimiento constituye una penetración más profunda en la estructura del universo que las capas previamente reconocidas. Sin embargo, el punto que resulta más significativo en el presente contexto es que dichas leyes de la naturaleza contienen, en sus consecuencias más remotas, solamente una parte pequeña de nuestro conocimiento del mundo inanimado. Todas las leyes de la naturaleza son afirmaciones condicionales que permiten una predicción de algunos sucesos futuros sobre la base del conocimiento del presente, con la excepción de que algunos aspectos del estado presente del mundo, en la práctica la inmensa mayoría de los determinantes del estado presente del mundo, son irrelevantes desde el punto de vista de la predicción. La irrelevancia es significativa en el sentido del segundo punto tratado en relación con el teorema de Galileo.  [7 Creemos innecesario mencionar que el teorema de Galileo, tal como se ha enunciado en el texto, no agota el contenido de sus observaciones en relación con las leyes de la caída de los cuerpos.]



En lo que se refiere al estado presente del mundo, tal como la existencia de la Tierra en la que vivimos y en la cual se llevaron a cabo los experimentos de Galileo, la existencia del Sol y de la totalidad de nuestro entorno, las leyes de la naturaleza no dicen absolutamente nada. Es en consonancia con esto, en primer lugar, que se pueden utilizar las leyes de la naturaleza para predecir acontecimientos futuros solamente bajo circunstancias excepcionales, cuando se conocen todos los factores relevantes del estado presente del mundo. En correspondencia con esto también la construcción de máquinas, cuyo funcionamiento se puede prever, constituye el logro más espectacular del físico. En tales máquinas el físico crea una situación en la cual se conocen todas las coordenadas relevantes, de tal modo que puede predecirse el comportamiento de la máquina. Los radares y los reactores nucleares son ejemplos de tales máquinas.

La finalidad principal de la argumentación anterior es señalar que las leyes de la naturaleza son siempre afirmaciones condicionales y que se refieren solamente a una parte muy pequeña de nuestro conocimiento del mundo. Así, la mecánica clásica, que es el prototipo mejor conocido de una teoría física, proporciona las derivadas segundas de las coordenadas de la posición de todos los cuerpos, en base al conocimiento de las posiciones, etc. de tales cuerpos. No proporciona información sobre la existencia, las posiciones presentes o las velocidades de dichos cuerpos. Debería mencionarse, en aras a la precisión, que descubrimos hace unos treinta años que incluso las afirmaciones condicionales no pueden ser del todo precisas, puesto que dichas afirmaciones son leyes de probabilidad que nos permiten solamente apuestas inteligentes acerca de las propiedades futuras del mundo inanimado, basadas en el conocimiento de su estado presente. No nos permiten hacer afirmaciones categóricas, ni tampoco afirmaciones condicionales categóricas acerca del estado presente del mundo. La naturaleza probabilística de las ``leyes de la naturaleza'' se manifiesta por sí misma también en el caso de las máquinas, y se puede verificar, al menos en el caso de los reactores nucleares, cuando funcionan a muy baja potencia. Sin embargo, la limitación adicional del alcance de las leyes de la naturaleza que se deriva de su carácter probabilístico no representa ningún papel en el resto de la discusión.

\subsection*{El papel de la matemática en las teorías físicas}

Habiendo recordado la esencia de la matemática y la física, deberíamos estar en una mejor posición para pasar revista al papel de la matemática en las teorías físicas.

Naturalmente, utilizamos la matemática en la física cotidiana para evaluar los resultados de las leyes de la naturaleza, para aplicar las afirmaciones condicionales a las condiciones particulares que resultan prevalecer o bien nos interesan. Con el fin de que ello sea posible, las leyes de la naturaleza deben estar formuladas previamente en lenguaje matemático. Sin embargo, el papel de evaluar las consecuencias de teorías ya establecidas no es el más importante de la matemática en la física. La matemática o, más bien, la matemática aplicada, no es tanto la dueña de la situación en esta función, sino que sirve meramente como herramienta.

 

La matemática representa también, sin embargo, un papel más soberano en la física. Esto estaba ya implícito en las afirmaciones efectuadas al discutir el papel de la matemática aplicada, según las cuales las leyes de la naturaleza deben haber sido formuladas en el lenguaje de la matemática para que puedan ser objeto del uso de la matemática aplicada. La declaración de que las leyes de la naturaleza están escritas en el lenguaje de la matemática fue realizada adecuadamente hace trescientos años;[8 Se atribuye a Galileo] es ahora más cierta que nunca antes. Con el fin de mostrar la importancia que los conceptos matemáticos poseen en la formulación de las leyes de la física, recordemos por ejemplo los axiomas de la mecánica cuántica tal como fueron formulados explícitamente por el gran físico Dirac. Hay dos conceptos básicos en la mecánica cuántica: estados y observables. Los estados son vectores del espacio de Hilbert, los observables operadores autoadjuntos de dichos vectores. Los valores posibles de las observaciones son los valores característicos de los operadores, pero debemos detenernos aquí para no sumergirnos en una relación de los conceptos matemáticos desarrollados en la teoría de los operadores lineales.

Es cierto, naturalmente, que la física elige ciertos conceptos matemáticos para la formulación de las leyes de la naturaleza, y seguramente utiliza solamente una fracción de todos los conceptos matemáticos. Es cierto asimismo que los conceptos elegidos no fueron seleccionados arbitrariamente de una lista de términos matemáticos, sino que se desarrollaron, en muchos si no en todos los casos, independientemente por el físico y luego se reconocieron como concebidos con anterioridad por el matemático. No es cierto, sin embargo, lo que se dice con frecuencia, y es que ello había de ser así puesto que la matemática utiliza los conceptos más simples y que por tanto están destinados a aparecer en cualquier formalismo. Como vimos antes, los conceptos de la matemática no se eligen por su sencillez conceptual, aunque series de pares de números están lejos de ser los conceptos más simples, sino por su tendencia a manipulaciones inteligentes y a razonamientos notables y brillantes. No olvidemos que el espacio de Hilbert de la mecánica cuántica es el espacio de Hilbert complejo, con un producto escalar hermítico. Es seguro que para la mente despreocupada los números complejos están lejos de lo natural y lo sencillo, y no pueden resultar sugeridos por las observaciones físicas. Más aún, el uso de números complejos no es en este caso un truco de cálculo de la matemática aplicada, sino que está muy cerca de ser una necesidad en la formulación de las leyes de la mecánica cuántica. Finalmente, ahora comienza a revelarse que no solamente los números complejos sino que también las llamadas funciones analíticas están destinadas a ejercer un papel decisivo en la formulación de la teoría cuántica. Me refiero a la teoría de las relaciones de dispersión, en rápido desarrollo.

Es difícil evitar la impresión de que aquí nos enfrentamos a un milagro, completamente comparable en su asombrosa naturaleza al milagro de que la mente humana sea capaz de enlazar un millar de razonamientos sin caer en contradicciones, o a los dos milagros de la existencia de leyes de la naturaleza y de la capacidad de la mente humana para adivinarlas. La observación que más se acerca a una explicación del surgimiento de los conceptos matemáticos en física que conozco es la declaración de Einstein de que las únicas teorías físicas que deseamos aceptar son las bellas. Hay que estar alerta para discutir que los conceptos de la matemática, que invitan al ejercicio de tanto ingenio, tienen la cualidad de la belleza. Sin embargo, la observación de Einstein puede explicar más bien las propiedades de teorías que estamos dispuestos a creer y no hace referencia a la precisión intrínseca de la teoría. Volveremos, por consiguiente, a esta última cuestión.

\subsection*{¿Es en realidad sorprendente el éxito de las teorías físicas?}

Una posible explicación del uso de la matemática por parte del físico para formular sus leyes de la naturaleza es la de que en cierto sentido es una persona irresponsable. Como resultado, cuando encuentra una conexión entre dos cantidades que semejan una conexión bien conocida de la matemática, concluye que la conexión es la tratada en la matemática simplemente porque no conoce otra conexión parecida. No es la intención de la presente discusión refutar la acusación de que el físico es una persona irresponsable. Quizás lo sea. Importa sin embargo señalar que la formulación matemática de la con frecuencia cruda experiencia del físico conduce en un extraño número de casos a una descripción asombrosamente precisa de un conjunto grande de fenómenos. Esto muestra que el lenguaje matemático es más que recomendable como el único lenguaje que podemos hablar; muestra que se trata, en un sentido verdaderamente real, del lenguaje correcto. Consideremos unos pocos ejemplos.

El primer ejemplo es el frecuentemente citado del movimiento planetario. Se consiguió establecer bastante bien las leyes la caída de los cuerpos, como resultado de experimentos llevados a cabo principalmente en Italia. Tales experimentos no podían ser muy precisos en el sentido según el cual entendemos actualmente la precisión, en parte debido al efecto de la resistencia del aire y en parte debido a la imposibilidad, en aquella época, de medir intervalos de tiempo cortos. A pesar de ello, no sorprende que, como resultado de sus estudios, los científicos italianos adquirieran familiaridad con los modos según los cuales los objetos viajan a través de la atmósfera. Fue Newton quien más tarde relacionó la caída libre de los cuerpos con el movimiento de la Luna, advirtiendo que la parábola de la trayectoria de una piedra lanzada sobre la Tierra y la trayectoria circular de la Luna en el cielo son casos particulares del mismo objeto matemático de una elipse, y postuló la ley universal de la gravitación sobre la base de una única, y en aquel tiempo muy aproximada, coincidencia numérica. Filosóficamente, la ley de la gravitación tal como fue formulada por Newton era rechazable para su época y para él mismo. Empíricamente, estaba basada en muy escasas observaciones. El lenguaje matemático en el que estaba formulada contenía el concepto de una derivada segunda, y los que hemos intentado dibujar un círculo osculatriz de una curva sabemos que la segunda derivada no es un concepto muy inmediato. La ley de la gravitación que Newton estableció con reluctancia y que pudo verificar con una precisión de cerca de un 4\% demostró ser precisa en menos de una diezmilésima por ciento y se asoció tan cercanamente con la idea de la precisión absoluta que sólo recientemente los físicos se han vuelto lo bastante audaces como para inquirir las limitaciones de esa precisión. [9 Véase, por ejemplo, R. H. Dicke, Am. Sci., 25 (1959).] Ciertamente, el ejemplo de la ley de Newton, tantas veces citado, debe mencionarse primero como un ejemplo monumental de una ley, formulada en términos que parecen sencillos al matemático, que ha demostrado ser precisa más allá de las expectativas razonables. Permítasenos recapitular nuestra tesis en este ejemplo: en primer lugar, la ley, desde el momento en que en ella aparece una segunda derivada, es solamente sencilla para el matemático, no para el sentido común ni para el hombre corriente de mentalidad no matemática; en segundo lugar, es una ley condicional de alcance bastante limitado. No explica nada acerca de la Tierra que atrae a las piedras de Galileo, ni acerca de la forma circular de la órbita de la Luna, ni en relación con los planetas del sistema solar. La explicación de esas condiciones iniciales se deja al geólogo y al astrónomo, que tienen con ellas una dura tarea.

El segundo ejemplo pertenece a la mecánica cuántica elemental ordinaria. Se originó cuando Max Born advirtió que algunas de las reglas de cálculo dadas por Heisenberg estaban formuladas de modo idéntico que las reglas del cálculo con matrices, establecidas hacía mucho tiempo por los matemáticos. Born, Jordan y Heisenberg se propusieron entonces reemplazar por matrices las variables posición e impulso de las ecuaciones de la mecánica clásica. Aplicaron las reglas de la mecánica de matrices a unos pocos problemas muy idealizados y los resultados fueron bastante satisfactorios. No obstante, no había, en aquella época, evidencia racional de que su mecánica de matrices pudiera resultar correcta bajo condiciones más realistas. En realidad, dijeron ``la mecánica tal como se ha propuesto aquí debería ser ya correcta en sus trazos esenciales''. De hecho, la primera aplicación de su mecánica a un problema real, el del átomo de hidrógeno, fue hecha varios meses más tarde por Pauli. Esta aplicación proporcionó resultados en acuerdo con la experiencia. Ello fue satisfactorio pero todavía inexplicable porque las reglas de cálculo de Heisenberg estaban sacadas de problemas que incluían la antigua teoría del átomo de hidrógeno. El milagro ocurrió solamente cuando la mecánica de matrices, y una teoría matemática equivalente a ella[1], se aplicó a problemas para los cuales las reglas de cálculo de Heisenberg no eran significativas. Las reglas de Heisenberg presuponían que las ecuaciones clásicas del movimiento tenían soluciones con ciertas propiedades periódicas; y las ecuaciones del movimiento de los dos electrones del átomo de helio, o del número todavía mayor de electrones de átomos más pesados, simplemente no tienen tales propiedades, de modo que las reglas de Heisenberg no pueden aplicarse en tales casos. Sin embargo, el cálculo del nivel de menor energía del helio, tal como lo realizaron hace algunos meses Kinoshita en Cornell y Bazley en el Bureau of Standards, coincide con los datos experimentales dentro de la precisión de las observaciones, que es de una parte en diez millones. Con seguridad en este caso hemos ``obtenido algo'' de las ecuaciones que no pusimos en ellas.

Lo mismo es cierto para las características cualitativas de los ``espectros complejos'', es decir, los espectros de los átomos más pesados. Quisiera recordar una conversación con Jordan, quien me dijo, cuando se derivaron las características cualitativas de los espectros, que un desacuerdo con las reglas derivadas de la teoría de la mecánica cuántica y las establecidas por la investigación empírica hubiera proporcionado la última oportunidad para realizar un cambio en el marco de la mecánica de matrices. En otras palabras, Jordan pensaba que quedaríamos, al menos temporalmente, faltos de ayuda si se hubiera producido un desacuerdo en la teoría del átomo de helio. Esta había sido, en esa época, desarrollada por Kellner y por Hilleraas.  El formalismo matemático era demasiado costoso e irremplazable, de tal modo que  si el milagro relativo al helio antes mencionado no hubiera ocurrido, se hubiera producido una verdadera crisis. Con seguridad, la física se hubiera sobrepuesto a dicha crisis de un modo u otro. Es cierto, por otra parte, que la física tal como actualmente la conocemos no hubiera sido posible sin una recurrencia constante de milagros semejantes al del átomo del helio, que es quizás el más asombroso milagro que ha tenido lugar en el curso del desarrollo de la mecánica cuántica elemental, pero con mucho no el único. De hecho, el número de milagros análogos está limitado, según nuestra opinión, solamente por nuestra voluntad de indagar otros semejantes. La mecánica cuántica tenía en su haber, sin embargo, muchos otros éxitos igualmente deslumbrantes que nos proporcionaba la convicción firme de que era lo que llamamos correcta.

El último ejemplo es el de la electrodinámica cuántica, o la teoría del desplazamiento de Lamb. Mientras que la teoría de la gravitación de Newton tiene todavía conexiones obvias con la experiencia, ésta entró en la formulación de la mecánica matricial solamente en la forma refinada o sublimada de las prescripciones de Heisenberg. La teoría cuántica del desplazamiento de Lamb, tal como fue concebido por Bethe y establecido por Schwinger, es una teoría puramente matemática y la única contribución directa del experimento fue mostrar la existencia de un efecto mensurable. El acuerdo con el cálculo es mejor que una parte en un millar.

Los tres ejemplos anteriores, que se podrían multiplicar casi indefinidamente, deberían ilustrar la idoneidad y la precisión de la formulación matemática de las leyes de la naturaleza en términos de conceptos elegidos para su manipulación, siendo las ``leyes de la naturaleza'' de una precisión casi fantástica pero de un alcance estrictamente limitado. Propongo referirnos a la observación que dichos ejemplos ilustran como la ley empírica de la epistemología. Junto con las leyes de la invariancia de las teorías físicas, es un fundamento indispensable de las mismas. Sin las leyes de la invariancia las teorías físicas podían haber quedado sin fundamento alguno; si la ley empírica de la epistemología no fuera correcta, nos faltaría el estímulo y la confianza que son necesidades emocionales sin las cuales las ``leyes de la naturaleza'' no podrían haber sido exploradas con éxito. El  Dr. R. G. Sachs, con el cual he discutido la ley empírica de la epistemología, la calificó de artículo de fe del físico teórico, y se trata seguramente de eso. Sin embargo, lo que él llamó nuestro artículo de fe puede apoyarse bien por los muchos ejemplos reales además de los tres antes mencionados.

\subsection*{La unicidad de las teorías de la física}

La naturaleza empírica de las observaciones precedentes me parece evidente por sí misma. Está claro que no es una ``necesidad del pensamiento'', y que no debería ser necesario, con el fin de demostrarlo, indicar el hecho de que se aplican solamente a una parte muy pequeña de nuestro conocimiento del mundo inanimado. Es absurdo creer que la existencia de expresiones matemáticamente simples para la segunda derivada de la posición es evidente por sí misma, cuando no existen expresiones semejantes para la propia posición o para la velocidad. Es por lo tanto sorprendente la prontitud con la que fue dado por hecho el maravilloso regalo contenido en la ley empírica de la epistemología. La capacidad de la mente humana para construir una serie de 1000 conclusiones y permanecer en lo ``correcto'', antes mencionada, es otro regalo similar.

Cada ley empírica tiene la cualidad inquietante de que uno no conoce sus limitaciones. Hemos visto que hay regularidades en los sucesos del mundo que nos rodea que pueden formularse en términos de conceptos matemáticos con una precisión prodigiosa. Hay, por otra parte, aspectos del mundo en relación con los cuales no creemos en la existencia de ninguna regularidad precisa. Les damos el nombre de condiciones iniciales. La cuestión que se presenta es si las diversas regularidades, esto es, las diversas leyes de la naturaleza que serán descubiertas, se fusionarán en una única unidad consistente, o al menos se aproximarán de modo asintótico a una fusión de ese tipo. Alternativamente, es posible que haya siempre leyes de la naturaleza que no tengan nada en común con otras. En la actualidad esto es así, por ejemplo, con las leyes de la herencia y de la física. Es incluso posible que algunas de las leyes de la naturaleza resulten en conflicto entre sí en cuanto a sus implicaciones, pero que cada una convenza lo bastante en su propio dominio de forma que no se esté dispuesto a abandonarlas. Debemos resignarnos a tal estado de cosas, o bien podría desvanecerse nuestro interés por aclarar el conflicto entre las diversas teorías. Podríamos perder el interés en ``la verdad definitiva'', esto es, en una representación que sea una fusión consistente en una única unidad de pequeñas representaciones, formadas sobre los diversos aspectos de la naturaleza. 

Puede resultar conveniente ilustrar las alternativas mediante un ejemplo. Ahora tenemos en la física dos teorías de gran potencia e interés: la teoría de los fenómenos cuánticos y la teoría de la relatividad. Estas dos teorías tienen sus raíces en grupos de fenómenos que se excluyen mutuamente. La teoría de la relatividad se aplica a los cuerpos macroscópicos, tales como las estrellas. El suceso de la coincidencia, esto es, en último análisis la colisión, es el suceso primario de la teoría de la relatividad y define un punto en el espacio-tiempo, o al menos definiría un punto si las partículas que colisionan fueran infinitamente pequeñas. La teoría cuántica tiene sus raíces en el mundo microscópico y, desde este punto de vista, el suceso de la coincidencia, o de la colisión, incluso si se produce entre partículas sin extensión espacial, no es primario y no está en absoluto aislado en el espacio-tiempo. Las dos teorías operan con distintos conceptos matemáticos: el espacio de cuatro dimensiones de Riemann y el espacio de infinitas dimensiones de Hilbert, respectivamente. Hasta el momento, las dos teorías no han podido unificarse, es decir, que no existe una formulación matemática para la cual las ambas teorías resulten como aproximaciones. Todos los físicos creen que una unión de las dos teorías es inherentemente posible, y que la hallaremos. No obstante, es posible imaginar también que no se pueda hallar una unión de las dos teorías. Este ejemplo ilustra las dos posibilidades, de unión y de conflicto, mencionadas antes, ambas concebibles.

Con el fin de obtener una indicación de cuál es la alternativa que cabe esperar en definitiva, podemos pretender ser un poco más ignorantes de lo que somos y colocarnos en un nivel más bajo de conocimiento del que actualmente poseemos. Si podemos hallar una fusión de nuestras teorías en este nivel menor de inteligencia, podemos esperar confiadamente que hallaremos una fusión de nuestras teorías en nuestro nivel real de inteligencia. Por otra parte, si llegáramos a teorías mutuamente contradictorias a un cierto nivel de conocimiento, la posibilidad de la permanencia de teorías conflictivas no puede tampoco excluirse. El nivel de conocimiento y de ingenio es una variable continua y es improbable que una variación relativamente pequeña de esta variable continua cambie la representación alcanzable del mundo de inconsistente a consistente. [10 Este extracto fue escrito después de mucha vacilación. Estoy convencido de que es útil, en los debates epistemológicos, abandonar la idealización de que el nivel de la inteligencia humana tiene una posición singular en una escala absoluta. En algunos casos puede resultar incluso útil considerar el logro posible en el nivel de inteligencia de otras especies. Sin embargo, también me doy cuenta de que mis pensamientos a lo largo de las líneas indicadas en el texto son demasiado breves y no están sujetos a la suficiente evaluación crítica como para resultar confiables] Considerado desde este punto de vista, el hecho de que algunas de las teorías que sabemos que son falsas proporcionan resultados tan asombrosamente precisos es un factor adverso. Si tuviéramos menos conocimiento, el grupo de fenómenos que tales teorías ``falsas'' explican nos parecería lo bastante extenso como para ``demostrar'' dichas teorías. No obstante, consideramos ``falsas'' dichas teorías por la razón de que, en último análisis, son incompatibles con otras representaciones más globales y, si se descubre la cantidad suficiente de tales falsas teorías, estarían obligadas a entrar en conflicto entre ellas. De modo semejante, es posible que las teorías que consideramos que están ``verificadas'' por un número de coincidencias numéricas que nos parece ser lo bastante grande, son falsas porque están en conflicto con una teoría más global que está más allá de nuestras posibilidades de descubrimiento. Si esto fuera cierto, deberíamos esperar conflictos entre nuestras teorías tan pronto como su número crezca más allá de un cierto punto y tan pronto como cubran un número grande de grupos de fenómenos. En contraste con el artículo de fe del físico teórico antes mencionado, esta es la pesadilla del teórico.

Consideremos unos cuantos ejemplos de teorías ``falsas'' que proporcionan, en vista de su falsedad, descripciones alarmantemente precisas de grupos de fenómenos. Con alguna buena voluntad, podemos descartar parte de la evidencia que esos ejemplos deparan. El éxito de las ideas pioneras de Bohr sobre el átomo fue siempre bastante ajustado, y lo mismo se aplica a los epiciclos de Tolomeo. Nuestro ventajoso punto de vista actual nos da una descripción precisa de todos los fenómenos que dichas teorías primitivas podían describir. Lo mismo no es cierto para la así llamada teoría del electrón-libre, que proporciona una descripción maravillosamente precisa de muchas, si no de la mayoría, de las propiedades de los metales, semiconductores y aislantes. En particular, explica el hecho, nunca comprendido de modo apropiado sobre la base de la ``teoría actual'', de que los aislantes muestran una resistencia específica a la electricidad que puede ser 1026 veces mayor que la de los metales. De hecho, no existe evidencia experimental que demuestre que la resistencia no es infinita bajo las condiciones en las que la teoría del electrón-libre nos lleva a esperar una resistencia infinita. Sin embargo, estamos convencidos de que la teoría del electrón-libre es una burda aproximación que debería reemplazarse, en la descripción de todos los fenómenos relativos a los sólidos, por una descripción más exacta.

Desde nuestro ventajoso punto de vista actual, la situación presentada por la teoría del electrón-libre es irritante porque no parece presagiar ninguna de las inconsistencias que no podamos superar. La teoría del electrón-libre despierta dudas acerca de hasta qué punto deberíamos creer en las coincidencias numéricas entre la teoría y el experimento como evidencia de la corrección de una teoría. Estamos acostumbrados a tales dudas. 

Una situación mucho más difícil y confusa se presentaría si pudiéramos, algún día, establecer una teoría de los fenómenos de la conciencia, o de la biología, que fuera tan coherente y convincente como nuestras actuales teorías del mundo inanimado. Las leyes de la herencia de Mendel y el trabajo siguiente sobre los genes podrían formar muy bien el comienzo de una teoría de ese tipo en cuanto concierne a la biología. Es más, es completamente posible que se pueda hallar un razonamiento abstracto que muestre que hay conflicto entre esa teoría y los principios aceptados por la física. El razonamiento podría ser de naturaleza tan abstracta que no resultara posible resolver el conflicto, a favor de una o de la otra teoría, mediante un experimento. Tal situación pondría bajo una gran tensión nuestra fe en nuestras teorías y en nuestra creencia en la realidad de los conceptos que formamos. Nos daría un profundo sentido de frustración en nuestra búsqueda de lo que llamo ``la verdad definitiva''. La razón de que una situación tal sea concebible es que, fundamentalmente, no conocemos por qué nuestras teorías funcionan tan bien. Por consiguiente, su precisión no puede demostrar su certeza y consistencia. Efectivamente, creo que algo bastante comparable a la situación antes descrita existe si se confrontan las leyes presentes de la herencia y de la física.




Permítaseme terminar con una nota alegre. El milagro de la idoneidad del lenguaje de las matemáticas para la formulación de las leyes de la física es un regalo maravilloso que ni comprendemos ni merecemos[2]. Deberíamos estar agradecidos por ello y esperar que siga siendo válido en la investigación futura y que se extienda, para bien o para mal, para nuestro placer o incluso para nuestra confusión, a ramas más amplias del saber.





\end{document}