\documentclass[a4paper, 12pt]{article}

%%%%%%%%%%%%%%%%%%%%%%Paquetes
\usepackage[spanish]{babel}  
\usepackage[utf8]{inputenc}
\usepackage{tcolorbox}
\usepackage{cmbright}  %%%%%%% El tipo de letra
\usepackage{setspace}
\onehalfspacing  %%%%%%%%%%% Espacio y medio de interlineado
\parskip=1em  %%%%%%%%%%%% Separacion entre parrafos
%%%%%%%%%%%%%%%%%%%%%%



%%%%%%%%%%%%%%%%%
\title{Gödel y lo Límites de la Lógica}
\author{J. Dawson}
\date{}
%%%%%%%%%%%%%%%%%

\begin{document}

\begin{tcolorbox}[colback=blue!5!white,colframe=blue!75!black]

\vspace{-1.8cm}
\textbf \maketitle

\end{tcolorbox}

\bigskip





Kurt Gödel, genio de la matemática, se consagró en su obra a la racionalidad. Paradojas de la vida, tuvo que luchar con ésta en su intimidad.

Ante el encerado, Kurt Gödel tiene un aspecto formal, reservado y un tanto desnutrido. Pero ni ese rostro ni los escritos han calado en el público, si exceptuamos un puñado de filósofos y lógicos-matemáticos. De sus teoremas de completitud derivan consecuencias decisivas para los fundamentos de las matemáticas y de las ciencias de la computación. Su peripecia vital y su obra responden a una tenaz búsqueda de la racionalidad en todo. Un ansia que deja al descubierto el trasfondo recurrente de una inestabilidad mental.

Gödel demostró que los métodos matemáticos aceptados desde tiempos de Euclides eran inadecuados para descubrir todas las verdades relativas a los números naturales. Su descubrimiento minó los fundamentos sobre los que se había construido la matemática hasta el siglo XX, acicateó a los pensadores para buscar otras posibilidades y engendró un vivaz debate sobre la naturaleza de la verdad. Las innovadoras técnicas de Gödel, aplicables sin dificultad en algoritmos de cómputo, echaron también los cimientos de las ciencias de computación modernas.

Nacido el 28 de abril de 1906 en Brno, ciudad de Moravia, Gödel fue el menor de los dos hijos de Rudolf y Marianne Gödel, expatriados alemanes cuyas familias estuvieron asociadas con la industria textil de la ciudad. Entre los antepasados de Gödel no encontramos profesores ni intelectuales; la educación de su padre no fue más allá de estudios de comercio. Pero Rudolf Gödel, ambicioso y tenaz, logró salir adelante, llegando a director gerente primero, y a copropietario más tarde, de una de las grandes fábricas de hilados de Brno. Ganó dinero suficiente para comprar una casa en uno de los barrios elegantes y enviar a sus hijos a escuelas privadas de habla alemana. Los chicos lograron excelentes resultados en sus estudios.

En toda su trayectoria escolar, primaria y secundaria, sólo una vez recibió Kurt una calificación inferior a la máxima en una asignatura (¡en matemáticas!). Pero no mostraba signos precoces de genialidad. Era un niño inquisitivo, tanto, que fue apodado der Herr Warum (``el señor Por qué''); también, introvertido, sensible y enclenque. A eso de los ocho años contrajo unas fiebres reumáticas. Aunque al parecer no le dejaron secuelas duraderas, le mantuvieron apartado de la escuela por algún tiempo; quizás alentaron su enfermiza preocupación por la salud y la dieta, que se fue reforzando con los años.

En 1924, tras graduarse en el Realgymnasium, una escuela técnica de Brno, Gödel abandonó su país natal para matricularse en la Universidad de Viena. A ese centro había acudido, cuatro años antes, su hermano para estudiar medicina. La economía vienesa se encontraba por entonces en ruinas. La universidad, empero, retenía su viejo esplendor. Gracias a ella, a pesar de las privaciones materiales, Viena dio cobijo en el período de entreguerras a un impresionante florecimiento de las ciencias, las artes y la filosofía.

Gödel ingresó en la universidad con la intención de seguir la carrera de física. Pero al poco, impresionado por las lecciones de los profesores Philipp Furtwängler y Hans Hahn, se orientó hacia la matemática. Muy pronto destacó por su talento. A los dos años de su matriculación fue invitado a asistir a las sesiones de un seminario de debates que Hahn y el filósofo Moritz\linebreak Schlick habían fundado dos años antes. El grupo, que llegaría a ser famoso con el nombre de Círculo de Viena, se inspiraba en los escritos de Ernst Mach, un campeón del racionalismo, convencido de que todas las cosas podían explicarse mediante la lógica y la observación empírica, sin recurrir a entidades metafísicas.

El Círculo puso a Gödel en contacto con Rudolf Carnap, filósofo de la ciencia, y Karl Menger, matemático. Le ayudó a familiarizarse con la bibliografía de la lógica matemática y de la filosofía. En particular, el Círculo se hallaba enfrascado en los escritos de Ludwig Wittgenstein, cuya preocupación por el metalenguaje (en qué medida el lenguaje puede hablar acerca del lenguaje) pudo haber inducido a Gödel a sondear cuestiones similares en matemática. Algunos de los miembros del Círculo, entre ellos Carnap, Hanh y el físico Hans Thirring, estaban investigando los fenómenos parapsicológicos, asunto por el que también Gödel mostraba agudo interés. (Años más tarde, Gödel le haría notar a un amigo íntimo, el economista Oskar Morgenstern, que en el futuro sería tenido por fenómeno extraño que los científicos del siglo XX hubieran descubierto las partículas físicas elementales y ni siquiera se les hubiera ocurrido considerar la posibilidad de factores psíquicos elementales.)

Gödel, sin embargo, no compartía la visión positivista del Círculo de Viena, que desarrolló y generalizó las ideas de Mach. Era, por contra, un platónico, convencido de que, además del mundo de los objetos, existe un mundo de los conceptos al que los humanos tienen acceso por intuición. Para él, un enunciado debía tener un ``valor de verdad'' bien definido ---ser verdadero o no serlo--- tanto si había sido demostrado como si era susceptible de ser refutado o confirmado empíricamente. Desde su propio punto de vista, tal filosofía constituía una ayuda para su excepcional penetración en las matemáticas.

Aunque Gödel era un observador atento y muy lúcido, rara vez contribuía a las discusiones del Círculo, a menos que tratasen de matemáticas. Tímido y reservado, tenía pocos amigos íntimos. (Le agradaba, sin embargo la compañía femenina y, según parece, las mujeres le encontraban francamente atractivo.) Después de 1928 sólo en raras ocasiones asistía a las reuniones del grupo; en cambio, participaba activamente en un coloquio matemático organizado por Menger. Las actas del coloquio se publicaban en un anuario, que Gödel ayudaba a redactar, y al que posteriormente habría de contribuir con más de una docena de artículos.

Durante este período, Gödel adquirió súbitamente estatura internacional en lógica matemática. Dos fueron, en particular, las publicaciones responsables de su prominencia. Una de ellas, su tesis doctoral, presentada en Viena en 1929, y publicada al año siguiente. La otra, su tratado {\it Sobre las proposiciones formalmente indecidibles de Principia Mathematica y sistemas afines}, publicada en alemán en su {\it Habilitationsschrift} (la memoria de cualificación para el ejercicio de la docencia universitaria) en 1932.

En su tesis doctoral, {\it La completitud de los axiomas del cálculo funcional de primer orden}, resolvía un problema pendiente, que David Hilbert y Wilhelm Ackermann habían planteado en un libro que escribieron conjuntamente en 1928, {\it Grundzüge der theoretischen Logik} ({\it Fundamentos de la Lógica Teórica}). La cuestión consistía en si las reglas al uso, enunciadas en el libro, para la manipulación de expresiones que contengan conectivas lógicas (``y'', ``o'', y similares) y cuantificadores (``para todo'' y ``existe'', aplicadas a variables que recorren números o conjuntos) permitirían, adjuntados a los axiomas de una teoría matemática, la deducción de todas y sólo todas las proposiciones que fueran verdaderas en cada estructura que cumpliera los axiomas. En lenguaje llano, ¿sería realmente posible demostrar todo cuanto fuera verdadero para todas las interpretaciones válidas de los símbolos?

Se esperaba que la respuesta fuese afirmativa, y Gödel confirmó que así era. Su disertación estableció que los principios de lógica desarrollados hasta aquel momento eran adecuados para el propósito al que estaban destinados, que consistía en demostrar todo cuanto fuera verdadero basándose en un sistema dado de axiomas. No demostraba, sin embargo, que todo enunciado verdadero referente a los números naturales pudiera demostrarse a partir de los axiomas aceptados de la teoría de los números.

Entre dichos axiomas, propuestos por el matemático italiano Giuseppe Peano en 1899, figura el principio de inducción. Este axioma afirma que cualquier propiedad que sea verdadera para el número cero, y que se cumpla para el número natural $n+1$ siempre que sea verdadera para $n$, tiene que ser verdadera para todos los números naturales. El axioma, al que algunos llaman ``principio dominó'' ---porque si cae el primero, caerán derribados todos los demás--- podría parecer evidente por sí mismo. Sin embargo, los matemáticos lo encontraron problemático, porque no se circunscribe a los números propiamente dichos, sino a propiedades de los números. Se consideró que tal enunciado de ``segundo orden'' era demasiado vago y poco definido para servir de fundamento a la teoría de los números naturales.

Por tal motivo, se refundió el axioma de inducción y se le dio la forma de un esquema infinito de axiomas similares concernientes a fórmulas específicas, en vez de referirse a propiedades generales de los números. Pero estos axiomas ya no caracterizan unívocamente los números naturales, como demostró el lógico noruego Thoralf Skolem algunos años antes del trabajo de Gödel: existen también otras estructuras que los satisfacen.

El teorema de completud de Gödel enuncia que es posible demostrar todos aquellos enunciados que se siguen de los axiomas. Existe, sin embargo, una dificultad: si algún enunciado fuese verdadero para los números naturales, pero no lo fuese para otro sistema de entidades que también satisface los axiomas, entonces no podría ser demostrado. Ello no parece constituir un problema serio, porque los matemáticos confiaban en que no existieran entidades que se disfrazasen de números para diferir de ellos en aspectos esenciales. Por este motivo, el teorema de Gödel que vino a continuación provocó auténtica conmoción.

En su artículo de 1931, Gödel demostraba que ha de existir algún enunciado concerniente a los números naturales que es verdadero, pero no puede ser demostrado. (Es decir, que existen objetos que obedecen a los axiomas de la teoría de números y, no obstante, en otros aspectos dejan de comportarse como números.) Se podría eludir este ``teorema de incompletud'' si todos los enunciados verdaderos fueran tomados como axiomas. Sin embargo, en ese caso, la decisión de si ciertos enunciados son verdaderos o no se torna problemática a priori. Gödel demostró que siempre que los axiomas puedan ser caracterizados por un sistema de reglas mecánicas, resulta indiferente cuáles sean los enunciados tomados como axiomas. Si son verdaderos para los números naturales, algunos otros enunciados verdaderos acerca de los números naturales seguirán siendo indemostrables.

En particular, si los axiomas no se contradicen entre sí, entonces, ese hecho mismo, codificado en enunciado numérico, será ``formalmente indecidible''\linebreak ---esto es, ni demostrable ni refutable--- a partir de dichos axiomas. Cualquier demostración de consistencia habrá de apelar a principios más fuertes que los propios axiomas.

Este último resultado apenó muchísimo a Hilbert, quien había contemplado un programa para fijar los fundamentos de las matemáticas por medio de un proceso ``autoconstructivo'', mediante el cual la consistencia de teorías matemáticas complejas pudiera deducirse de la consistencia de otras más sencillas y evidentes. Gödel, por otra parte, no consideraba que sus teoremas de incompletud demostrasen la inadecuación del método axiomático, sino que hacían ver que la deducción de teoremas no puede mecanizarse. A su modo de ver, justificaban el papel de la intuición en la investigación matemática.

Los conceptos y los métodos introducidos por Gödel en su artículo sobre la incompletud desempeñan un papel central en la teoría de recursión, que subyace a toda la informática moderna. Generalizaciones de sus ideas han permitido la deducción de diversos otros resultados relativos a los límites de los procedimientos computacionales. Uno de ellos es lo irresoluble del ``problema de la detención'', que consiste en decidir, para un ordenador arbitrario provisto de un programa y de unos datos arbitrarios, si llegará a detenerse o si quedará atrapado en un bucle infinito. Otro es la demostración de que ningún programa que no altere el sistema operativo de un ordenador será capaz de detectar todos los programas que sí lo hagan (virus).

Gödel pasó el año académico 1933--34 en Princeton, en el recién fundado Instituto de Estudios Avanzados, donde disertó sobre sus resultados de incompletud. Fue invitado a volver al año siguiente, pero al poco de regresar a Viena sufrió una grave crisis mental. Se recuperó a tiempo para retornar a Princeton en el otoño de 1935; al mes de su llegada sufrió una recaída, y no volvió a impartir enseñanza hasta la primavera de 1937, en Viena.

Por ser confidencial el historial médico de Gödel, la diagnosis de su mal sigue siendo desconocida. Sus problemas parecen haber comenzado con hipocondría: estaba obsesionado por su dieta y por sus hábitos intestinales. Durante veinte años llevó un registro diario de su temperatura corporal y de su consumo de leche de magnesia. Temía sufrir un envenenamiento accidental; con los años, le aterraba ser objeto de una intoxicación deliberada. Esta fobia le llevó a no querer tomar alimentos, con la consiguiente desnutrición. Lo que no le impedía ingerir píldoras de diversa condición para un imaginario problema cardíaco.

Salvo en los problemas de crisis, los problemas mentales de Gödel entorpecieron muy poco su trabajo. La persona que le mantuvo en activo fue Adele Porkert, a quien conoció en un local nocturno de Viena durante sus años de estudiante. Porkert, seis años mayor que Gödel, católica y divorciada, con el rostro desfigurado por una ``flor'' de nacimiento, trabajaba de bailarina. Los padres de Gödel la tenían por motivo de escándalo. Pero ellos no desmayaron en su mutuo afecto, y más de una vez, sirviéndole de catadora de alimentos, Adele contribuyó a paliar los temores de Gödel, cada vez más fuertes, de que buscaban envenenarle. Tras un largo noviazgo, se casaron en septiembre de 1938, justo antes de que Gödel retornase a los EEUU, donde disertó en el Instituto de Estudios Avanzados y en la Universidad de Notre Dame sobre los apasionantes resultados que había obtenido en teoría de conjuntos.

Tal logro entrañaba la resolución de algunos de los aspectos más controvertidos de la teoría de colecciones de objetos. A finales del siglo XIX, el matemático alemán Georg Cantor había introducido la noción de tamaño (``cardinal'') para conjuntos infinitos. Según tal concepto, un conjunto $A$ tiene menor cardinal que un conjunto $B$ si, cualquiera que sea la forma en que a cada elemento de $A$ otro le sea asignado en $B$, quedan siempre elementos de $B$ que no tienen correspondiente. Valiéndose de esta noción, Cantor demostró que el conjunto de los números naturales es menor que el conjunto de todos los números reales (el conjunto de todos los números decimales). Cantor conjeturó también que entre un conjunto y otro no existen conjuntos de tamaño intermedio, enunciado que llegó a ser conocido como la hipótesis del continuo.

En 1908, Ernst Zermelo, formuló una lista de axiomas para la teoría de conjuntos. Entre ellos se encontraba el axioma de elección, el cual (en una de sus versiones) afirma que dada una colección infinita de conjuntos disjuntos, cada uno de los cuales contiene al menos un elemento, existe un conjunto que contiene exactamente un elemento de cada uno de los conjuntos de la colección. Aunque su aspecto parece incuestionable ---¿por qué no habríamos de ser capaces de extraer un elemento de cada conjunto?--- el axioma de elección entraña una multitud de consecuencias contrarias a la intuición. De él se deduce, por ejemplo, la posibilidad de descomponer una esfera en un número finito de piezas, que separadas y vueltas a ensamblar aplicando tan sólo movimientos rígidos, formen una nueva esfera de volumen doble que la primera.

El axioma de elección desencadenó la polémica. Los matemáticos sospechaban ---correctamente, como luego se vería--- que ni el axioma de elección ni la hipótesis del continuo podían deducirse de los otros axiomas de la teoría de conjuntos. Y temían que las demostraciones fundadas en dichos principios pudieran generar contradicciones. Gödel, sin embargo, demostró que ambos principios eran coherentes con los restantes axiomas.

Los resultados de Gödel en teoría de conjuntos resolvieron una de las cuestiones que Hilbert había planteado en 1900 en una alocución célebre pronunciada en el Congreso Internacional de Matemáticas. Sólo por ello constituían un gran logro; no bastaron, empero, para asegurarle un puesto académico permanente. Durante el año que pasó en el Instituto de Estudios Avanzados y en Notre Dame, expiró su autorización para la docencia en las universidades austriacas. Y cuando volvió a Viena para reunirse con su esposa, en el verano de 1939, fue reclamado para un reconocimiento médico militar y declarado apto para el servicio en las fuerzas armadas nazis.

Hasta entonces, Gödel parecía haber permanecido indiferente ante los pavorosos acontecimientos que se estaban produciendo en Europa. Aunque interesado por la política, e informado de los acontecimientos, permaneció curiosamente insensible ante ellos. Su falta de compromiso con sus semejantes pudo haberle impedido apreciar la gravedad de lo que estaba ocurriendo. Parecía ajeno a la suerte que estaban corriendo sus colegas y sus profesores, judíos muchos de ellos, y siguió sumido en su trabajo mientras el mundo que le rodeaba se hacía pedazos. Por fin, acabó comprendiendo que con el mundo que se hundía también se estaba hundiendo él.

En aquella situación desesperada, sin empleo y a punto de ser reclutado, solicitó el apoyo del Instituto de Estudios Avanzados para que le ayudaran a obtener visados de salida para sí mismo y para su mujer. Sus esfuerzos tuvieron éxito. En enero de 1940 ambos emprendieron un largo viaje hacia el este en el ferrocarril transiberiano. Desde Yokohama continuaron por barco hasta San Francisco. Llegaron a Princeton
a mediados de marzo.

Gödel ya no volvería a salir de los EEUU. Tras una serie de nombramientos anuales se le admitió como miembro permanente del claustro en 1946. Dos años después obtuvo la ciudadanía estadounidense. (En aquella ocasión, el juez que le tomó juramento cometió el desafortunado error de pedirle su opinión sobre la Constitución de los EEUU, y desencadenó como respuesta una disertación en toda regla sobre sus contradicciones.) Pero Gödel no fue ascendido a catedrático hasta 1953, el mismo año en que fue elegido miembro de la Academia Nacional de Ciencias. Tal demora se debió, en parte, a las dudas que planteaba su estabilidad mental con sus constantes temores sobre posibles emanaciones de gases tóxicos en su refrigerador. Durante aquellos años, su amigo Albert Einstein se preocupó de Gödel lo más que pudo; todos los días daban un paseo.

Tras su emigración a los EEUU, abandonó el trabajo en teoría de conjuntos y se orientó hacia la filosofía y hacia la teoría de la relatividad. En 1949 demostró que eran compatibles con las ecuaciones de Einstein universos donde se pudiera viajar retrógradamente en el tiempo. En 1950 disertó sobre estos resultados en el Congreso Internacional de Matemáticos, y al año siguiente pronunció la prestigiosa {\it Disertación Gibbs} en la asamblea anual de la Sociedad Matemática Americana. Pero en el intervalo entre estas dos intervenciones públicas estuvo a punto de morir por una úlcera sangrante, que descuidó hasta un estadio peligrosamente avanzado, tal era la desconfianza que sentía hacia los médicos.

El último de sus artículos publicados en vida apareció en 1958. Después, se sumió en la introversión, cada vez más demacrado, paranóide e hipocondríaco. Su última aparición pública aconteció en 1972, al recibir un doctorado honorífico por la Universidad Rockefeller. Tres años después le fue otorgada la Medalla Nacional de Ciencias, pero Gödel disculpó su asistencia por razones de salud.

El 1 de julio de 1976, alcanzados los 70 años, edad de jubilación obligatoria, Gödel se convirtió en profesor emérito del Instituto. Sus responsabilidades empero no disminuyeron, porque su esposa, que durante tantos años le había alimentado y protegido, había sufrido pocos meses antes un ataque cardíaco que la dejó inválida. Ahora le correspondía a él cuidarla. Y así lo hizo, con devoción, hasta julio de 1977, cuando ella hubo de someterse a una operación de urgencia y permaneció hospitalizada durante casi seis meses.

Por aquellas fechas, Morgenstern, el amigo que había contribuido a cuidar de Gödel tras fallecer Einstein en 1955, murió de cáncer. Gödel tuvo entonces que luchar por sí solo contra su cada vez más acusada paranoia. Solo frente a ella, su declive entró en barrena. Temeroso de ser envenenado dejó de comer y acabó muriendo por desnutrición el 14 de enero de 1978.

Gödel publicó excepcionalmente poco en vida ---menos que ninguno de los otros grandes matemáticos, si se exceptúa a Bernhard Riemann---, pero la influencia de sus escritos ha sido enorme. Sus trabajos han afectado prácticamente a todas las ramas de lógica moderna. Durante el decenio pasado, otros artículos suyos han sido traducidos desde la obsoleta taquigrafía alemana que él utilizaba, y publicados póstumamente en el tercer volumen de sus {\it Collected Works}. Sus contenidos, entre los que figura su formalización del argumento ontológico de la existencia de Dios, han empezado también a llamar la atención.


\end{document}