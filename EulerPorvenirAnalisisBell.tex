\documentclass[a4paper, 12pt]{article}

%%%%%%%%%%%%%%%%%%%%%%Paquetes
\usepackage[spanish]{babel}  
\usepackage[utf8]{inputenc}
\usepackage{tcolorbox}
\usepackage{cmbright}  %%%%%%% El tipo de letra
\usepackage{setspace}
\onehalfspacing  %%%%%%%%%%% Espacio y medio de interlineado
\parskip=1em  %%%%%%%%%%%% Separacion entre parrafos
%%%%%%%%%%%%%%%%%%%%%%



%%%%%%%%%%%%%%%%%
\title{Euler. El Porvenir del Análisis}
\author{E. T. Bell}
\date{}
%%%%%%%%%%%%%%%%%

\begin{document}

\begin{tcolorbox}[colback=blue!5!white,colframe=blue!75!black]

\vspace{-1.8cm}
\textbf \maketitle

\end{tcolorbox}

\bigskip





\begin{quote}\it

	La historia muestra que los jefes de naciones que han favorecido el cultivo de la Matemática, la fuente común de todas las ciencias exactas, son también aquellos cuyos reinos han sido los más brillantes y cuyas glorias son las más durables.

\end{quote}

\hfill Michel Chasles


\bigskip

«Euler calculaba sin aparente esfuerzo como los hombres respiran o las águilas se sostienen en el aire» (como dijo Arago), y esta frase no es una exageración de la inigualada facilidad matemática de Léonard Euler (1707-1783), el matemático más prolífico de la historia y el hombre a quien sus contemporáneos llamaron, «la encarnación del Análisis». Euler escribía sus grandes trabajos matemáticos con la facilidad con que un escritor fluido escribe una carta a un amigo íntimo. Ni siquiera la ceguera total, que le afligió en los últimos 17 años de su vida, modificó esta fecundidad sin paralelo. En efecto, parece que la pérdida de la visión agudizó las percepciones de Euler en el mundo interno de su imaginación.

La extensión de los trabajos de Euler no ha sido exactamente conocida hasta 1936, pero se calcula que serían necesarios de sesenta a ochenta grandes volúmenes en cuarto para la publicación de todos sus trabajos. En 1909, la Asociación Suiza de Ciencias Naturales emprendió la publicación de los diversos trabajos de Euler, con la colaboración económica de muchas personas y de sociedades matemáticas de todo el mundo, ya que Euler pertenece a todo el mundo civilizado y no solo a Suiza. El cálculo de los probables gastos (alrededor de 80.000 dólares en la moneda de 1909), tuvo que modificarse por el descubrimiento de numerosos e insospechados manuscritos de Euler, realizado en San Petersburgo (Leningrado).

La carrera matemática de Euler comienza el año de la muerte de Newton. No podía elegirse una época más propicia para un genio como el de Euler. La Geometría Analítica (que se hizo pública en el año 1637) llevaba en uso 90 años, el Cálculo alrededor de 50, y la ley de la gravitación universal de Newton, la clave de la astronomía física, había sido presentada al público matemático hacía 40 años. En cada uno de estos campos había sido resuelto un vasto número de problemas aislados, habiéndose hecho ciertos ensayos de unificación, pero no existía ningún estudio sistemático que abarcara todo el complejo de las Matemáticas puras y aplicadas. En particular, los poderosos métodos analíticos de Descartes, Newton y Leibniz no habían sido aun explotados hasta el límite de lo posible, especialmente en Mecánica y Geometría.

El Álgebra y la Trigonometría, en un nivel inferior, podían ser ahora objeto de una sistematización y ampliación, especialmente la última. En el dominio de Fermat del análisis diofántico y de las propiedades de los números enteros comunes no era posible, ni todavía lo es, esa «perfección temporal»; pero también aquí Euler demostró ser maestro. En efecto, una de las características más notables del genio universal de Euler, fue su sin igual competencia en las principales direcciones de la Matemática, la continua y la discontinua.

Como algorista, Euler jamás ha sido sobrepasado y probablemente no hay quien se le aproxime, como no sea Jacobi. Un algorista es un matemático que idea «algoritmos» para la solución de problemas de tipos especiales. Como un ejemplo muy sencillo aceptamos (o probamos) que todo número real positivo tiene una raíz cuadrada real. ¿Cómo será calculada la raíz? Se conocen varios métodos; un algorista idea métodos practicables. Además, en el análisis diofántico, y también en el Cálculo integral, la solución de un problema puede no ser hallada hasta que haya sido hecha alguna ingeniosa (muchas veces simple) sustitución de una o más de las variables por funciones de otras variables; un algorista es un matemático al que se le ocurren de un modo natural esos ingeniosos trucos. No existe un modo uniforme de proceder; los algoristas, como los versificadores fáciles, nacen, no se hacen.

Actualmente es moda despreciar a los «simples algoristas»; sin embargo, cuando un verdadero gran algorista, como el hindú Ramanuan, surge inesperadamente, hasta los analistas expertos le consideran como un don del cielo: su visión sobrenatural respecto a fórmulas al parecer no relacionadas, revela sendas ocultas que conducen desde un territorio a otro y los analistas encuentran nuevas tareas al ser abiertos nuevos campos. Una algorista es «un formalista» que ama las fórmulas bellas por sí mismas.

Antes de continuar con la pacífica pero interesante vida de Euler, debemos mencionar dos circunstancias de su época que fomentaron su prodigiosa actividad y le ayudaron a darle una dirección.

En el siglo XVIII las Universidades no eran los centros principales de investigación en Europa. Pudieron hacer mucho más de lo que hicieron de no haber sido por su tradición clásica y su incomprensible hostilidad hacia la ciencia. La Matemática, por  ser suficientemente antigua, era considerada respetable, pero la Física, más reciente, era sospechosa. Además, un matemático en una Universidad de la época tenía que emplear gran parte de su esfuerzo en la enseñanza elemental; sus investigaciones, si las hacía, constituían un lujo no aprovechable, precisamente como en el tipo medio de las actuales instituciones americanas de enseñanza superior. Los miembros de las Universidades británicas podían hacer lo que quisieran. Pocos, sin embargo, querían hacer algo, y lo que hacían o dejaban de hacer no afectaba a su forma de vivir. En ese estado de laxitud o de abierta hostilidad, no había razón para que las Universidades condujeran a la ciencia, y realmente no la conducían.

Este papel era desempeñado por las diversas Academias reales mantenidas por gobernantes generosos y de gran visión. Los matemáticos deben una extraordinaria gratitud a Federico el Grande de Prusia y a Catalina la Grande de Rusia por su gran liberalidad. Hicieron posible todo un siglo de progresos matemáticos en uno de los períodos más activos de la historia científica. En el caso de Euler, Berlín y San Petersburgo constituyen el nervio de la creación matemática. Estos dos focos creadores fueron inspirados por la inquieta ambición de Leibniz. Las Academias trazadas siguiendo los planes de Leibniz dieron a Euler la ocasión de ser el matemático más prolífico de todos los tiempos; así, en cierto sentido, Euler fue el nieto de Leibniz.

La Academia en Berlín se había ido marchitando durante cuarenta años cuando Euler, inspirado por Federico el Grande, le dio nueva vida; y la Academia de San Petersburgo, que Pedro el Grande no llegó a organizar de acuerdo con el programa de Leibniz, quedó firmemente fundada por su sucesor.

Estas Academias no eran comparables a las actuales, cuya principal función es premiar con el nombramiento de académico a aquellos individuos que se distinguen por la obra realizada. Eran organizaciones que pagaban a sus miembros principales para que se dedicaran a la investigación científica. Los sueldos y otros gajes eran lo suficientemente elevados para permitir que vivieran con cierta comodidad el académico y su familia. La familia de Euler se componía en cierta época de al menos 18 personas, y, sin embargo, le fue posible sostenerla decentemente. Por si esto fuera poco, los hijos de los académicos del siglo XVII, si eran dignos de ello, sabían que gozaban de una fácil iniciación en el mundo.

Esto nos lleva a una segunda influencia dominante sobre la vasta producción matemática de Euler. Los gobernantes que pagaban generosamente los sueldos, deseaban ver retribuidos sus afanes y su dinero con alguna cosa, aparte de la cultura abstracta, pero debe hacerse notar que cuando tales gobernantes se creían suficientemente pagados, no insistían en que sus académicos gastaran el resto de su vida dedicados a la labor «productiva». Euler, Lagrange y los otros académicos gozaban de libertad para hacer lo que quisieran. Tampoco se ejercía ninguna presión por el hecho de que los resultados obtenidos no pudieran ser usados inmediatamente para fines prácticos. Más sabios que muchos directores de institutos científicos actuales, los gobernantes del siglo XVIII tan sólo insinuaban algunas veces lo que necesitaban, pero dejaban que la ciencia siguiera su curso. Parece que se dieron cuenta instintivamente de que la llamada investigación «pura» puede dar lugar también a cosas que más pronto o más tarde tienen aplicación práctica.

A este juicio general hay que hacer una importante excepción que no conforma ni desecha la regla. Sucedió que en los tiempos de Euler el problema más sobresaliente de la investigación matemática, coincidía por casualidad, con el que probablemente era el problema práctico esencial de la época: el dominio de los mares. La nación cuya técnica en la navegación superara a la de todos sus competidores, sería inevitablemente la reina de los mares. La navegación es, en último análisis, un problema de determinar exactamente la posición en el mar a cientos de millas de la tierra, y aquellos marinos que mejor lo consiguieran, podrían elegir el lugar más favorable para una batalla naval. Gran Bretaña, como todos saben, gobierna los mares y los gobierna debido, en no pequeño grado, a la aplicación práctica que sus navegantes supieron hacer de las investigaciones matemáticas puras referentes a la mecánica celeste, durante el siglo XVIII.

El fundador de la navegación moderna es Newton, aunque jamás este tema le diera un dolor de cabeza, y aunque a juzgar por lo que sabemos, nunca puso sus plantas sobre la cubierta de un barco. La posición en el mar se determina por la observación de los cuerpos celestes, (algunas veces se incluyen los satélites de Júpiter), y conocida la ley universal de Newton pueden determinarse, con suficiente paciencia, con un siglo de anterioridad, las posiciones de los planetas y las fases de la Luna, con cuyos datos quienes deseen gobernar los mares pueden dejar establecidos sus cálculos en los almanaques náuticos, lo que les permitirá componer los cuadros de las futuras posiciones.

En tal empresa práctica la Luna ofrece un problema particularmente difícil, el de los tres cuerpos que se atraen de acuerdo con la ley de Newton. Este problema se repetirá muchas veces al llegar el siglo XX. Euler fue el primero en dar una solución calculable para el problema de la luna («la teoría lunar»). Los tres cuerpos a que nos referimos son la Luna, la Tierra y el Sol. Hacemos notar que se trata de uno de los problemas más difíciles de toda la Matemática. Euler no lo resolvió, pero su método de cálculo aproximado (sustituido actualmente por mejores métodos), fue suficientemente práctico para permitir a un calculador inglés redactar las tablas de la Luna que habría de utilizar el Almirantazgo Británico. El calculador recibió 5000 libras (una bonita suma para aquel tiempo), y se votó para Euler un sueldo de 300 libras como retribución por su método.

Leonard (o Leonhard) Euler, hijo de Paul Euler y de Marguerite Brucker, es probablemente el hombre de ciencia más grande que Suiza ha producido. Nació en Basilea el 15 de abril de 1707, pero al año siguiente sus padres se trasladaron a la cercana aldea de Riechen, donde su padre era el pastor calvinista. Paul Euler, un excelente matemático, discípulo de Jacob Bernoulli, quiso que Leonard siguiera sus pasos y le sucediera en la iglesia de la aldea, pero por fortuna cometió el error de enseñarle al muchacho la Matemática.

El joven Euler conoció pronto lo que quería hacer. De todos modos obedeció a su padre e ingresó en la Universidad de Basilea para estudiar teología y hebreo. En Matemática se hallaba suficientemente avanzado para atraer la atención de Johannes Bernoulli, que generosamente daba al joven una lección semanal. Euler empleaba el resto de la semana preparando la siguiente lección, con el objeto de que el número de problemas que tuviera que plantear a su profesor fuera el menor posible. Pronto, su inteligencia y marcada capacidad fueron observadas por Daniel y Nicolaus Bernoulli, quienes se hicieron buenos amigos de Euler.

Leonard pudo seguir estos estudios hasta obtener su título de maestro en 1724, teniendo 17 años. En ese momento, su padre insistió en que debía abandonar la Matemática y dedicarse totalmente a la teología. Mas el padre cedió cuando los Bernoulli le dijeron que su hijo estaba destinado a ser un gran matemático, y no el pastor de Riechen. Aunque la profecía se cumplió, la precoz educación religiosa de Euler influyó sobre toda su vida, y nunca pudo deshacerse de una partícula de su fe calvinista. En efecto, a medida que los años pasaban viró en redondo hacia donde su padre intentó dirigirle; dirigía los rezos familiares y de ordinario los terminaba con un sermón.

El primer trabajo independiente de Euler fue realizado cuando tenía 19 años. Se dice que este primer esfuerzo revela tanto el punto fuerte como el débil de la obra subsiguiente de Euler. La Academia de París propuso el tema de las arboladuras de los barcos como problema correspondiente al año 1727. Euler no ganó el premio, pero recibió una mención honorífica. Más tarde se resarció de esta pérdida ganando el premio doce veces. Su punto fuerte era el Análisis, la Matemática técnica; su punto débil, la falta de relación de su obra con las aplicaciones prácticas. Esto no puede sorprender cuando recordamos las bromas tradicionales referentes a la no existente Marina suiza. Euler pudo haber visto una o dos barcas en los lagos suizos, pero no había visto aún un barco. Ha sido criticado, algunas veces justamente, por dejar que su Matemática se alejara del sentido de la realidad. El universo físico era una ocasión que se daba a Euler para aplicar la Matemática, y si el universo no estaba de acuerdo con su análisis era el universo el que estaba errado.

Dándose cuenta de que había nacido para la Matemática, Euler se preparó para ser profesor en Basilea. No habiendo logrado su propósito, continuó sus estudios movido por la esperanza de unirse a Daniel y Nicolaus Bernoulli en San Petersburgo. Sus amigos se habían ofrecido generosamente para encontrarle un cargo bien rentado en la Academia

En esta fase de su carrera, parece que le era indiferente a Euler la elección del tema, siempre que se tratara de algo científico. Cuando los Bernoulli le hablaron de la posibilidad de que tuviera un puesto en la sección médica de la Academia de San Petersburgo, Euler se dedicó a estudiar fisiología en Basilea y asistió a cursos de medicina. Pero hasta en este campo, no podía alejarse de la Matemática. La fisiología del oído le sugirió la investigación matemática del sonido, que, a su vez le llevó al estudio de la propagación de las ondas, y así sucesivamente. Los Bernoulli eran hombres que cumplían su palabra. Euler fue llamado a San Petersburgo en 1727, incorporado a la Sección Médica de la Academia. Por una sabia disposición todos los miembros extranjeros estaban obligados a admitir los discípulos que siguieran las enseñanzas. El gozo del pobre Euler pronto se desvaneció. El día que puso el pie en suelo ruso moría la liberal Catalina I.

Catalina, amante del Pedro el Grande, antes de ser su esposa, parece haber sido una mujer de mente amplia por más de un concepto, y fue ella la que en su reinado, de sólo dos años, llevó a la práctica el deseo de Pedro de establecer la Academia. A la muerte de Catalina, el poder pasó a manos de una facción brutal, durante la minoría del joven Zar (que quizá para su bien murió antes de que pudiera comenzar a reinar). Los nuevos gobernantes de Rusia consideraron la Academia como un lujo costoso y durante algunos meses contemplaron la posibilidad de suprimirla, repatriando a los miembros extranjeros. Este era el momento en que Euler llegó a San Petersburgo. En la confusión del momento nada se dijo respecto al cargo para el cual Euler había sido llamado, y entonces ingresó en la sección matemática, después de haber aceptado en su desesperación, el nombramiento de teniente naval.

Más tarde las cosas marcharon mejor y Euler comenzó a trabajar. Durante seis años, no se separó de sus trabajos, no sólo por la pasión absorbente que sentía para la Matemática, sino también porque no se atrevía a dedicarse a una vida social normal por temor a los espías que había por todas partes.

En 1733 Daniel Bernoulli volvió a la libre Suiza, cansado de la santa Rusia, y Euler, teniendo 26 años, ocupó su puesto en la sección matemática en la Academia. Suponiendo que permanecería en San Petersburgo durante el resto de su vida, Euler decidió casarse y hacer las cosas lo mejor que pudiera. Su esposa era hija del pintor Gsell a quien Pedro el Grande había llevado a Rusia. Las condiciones políticas empeoraron, y Euler sintió la desesperada ansia de escapar. Pero con la rápida llegada de los hijos en rápida sucesión, Euler se sintió más atado que antes, refugiándose en una incesante labor. Algunos biógrafos atribuyen la fecundidad incomparable de Euler a esta primera permanencia en Rusia; la prudencia le forzó a este hábito de trabajo incesante.

Euler fue uno de los grandes matemáticos que podía trabajar en cualquier condición. Amaba los niños (tuvo trece, aunque cinco de ellos murieron siendo pequeños), y podía dedicarse a sus trabajos teniendo a alguno de sus hijos sentado sobre sus rodillas y a los restantes jugando en torno de él. La facilidad con que resolvía los problemas más difíciles es increíble.

Muchas son las anécdotas que se cuentan de su constante flujo de ideas. No hay duda de que algunas son exageraciones, pero se dice que Euler podía terminar un trabajo matemático en la media hora que transcurría desde que era llamado a la mesa hasta que comenzaba a comer. En cuanto terminaba un trabajo era colocado sobre el montón de hojas que esperaba la impresión. Cuando se necesitaba material para los trabajos de la Academia, el impresor elegía una hoja del montón de papeles. En consecuencia, se observa que la fecha de publicación no suele corresponder a la de la redacción. Este desorden todavía se hacía mayor debido a la costumbre de Euler de volver muchas veces sobre el mismo tema para aclararlo o ampliar lo que ya había escrito. Por tanto, una serie de trabajos sobre un determinado tema suele ser interrumpida por otras investigaciones sobre temas diferentes.

Cuando el joven Zar murió, Anna Ivanovna (sobrina de Pedro), fue Emperatriz en el año 1730, y por lo que se refiere a la Academia las cosas se aclararon considerablemente. Pero bajo el gobierno indirecto del amante de Anna, Ernest John de Biron, Rusia sufrió una de las épocas de terror más tremendo de su historia, y Euler se dedicó durante diez años a una labor silenciosa. Por entonces sufrió su primera gran desventura. Deseoso de obtener el premio  de la Academia de París, para el cual se había propuesto un problema astronómico que exigió a los matemáticos más conspicuos varios meses de labor, Euler trabajó tanto que lo resolvió en tres días. Pero el prolongado esfuerzo le produjo una enfermedad de cuyas consecuencias perdió la visión del ojo derecho.

Haremos notar que la crítica moderna, que se ha dedicado a desacreditar todas las anécdotas interesantes de la historia de los matemáticos, demostró que el problema astronómico no tuvo la menor responsabilidad en la pérdida del ojo de Euler. Pero como la crítica erudita (o cualquiera otra) debe saber mucho acerca de la llamada ley de causa y efecto, el misterio debería ser resuelto por el espíritu de David Hume (un contemporáneo de Euler). Con esta precaución narraremos una vez más la famosa historia de Euler y el ateo (o quizá sólo panteísta) filósofo francés Denis Diderot (1713-1784), si bien nos apartamos algo del orden cronológico, pues el suceso tuvo lugar durante la segunda permanencia de Euler en Rusia.

Invitado por Catalina la Grande para visitar su corte, Diderot se ganaba el sustento intentado convertir al ateísmo a los cortesanos, pero Catalina encargó a Euler de que tapara la boca al infatuado filósofo. Esto era fácil, pues la Matemática era chino para Diderot. De Morgan cuenta lo sucedido en su clásico {\it Budgel of Paradoxes}, 1872:

\begin{quote}\small

 Diderot fue informado de que un docto matemático estaba en posesión de una demostración algebraica de la existencia de Dios, y que la expondría ante toda la corte si él deseaba oírla. Diderot consintió amablemente... Euler avanzó hacia Diderot y dijo gravemente en un tono de perfecta convicción:

--- Señor, $\displaystyle\frac{a+b^n}{n}=x$ , por tanto Dios existe. Replique.

Humillado por la risa no contenida que saludó a su embarazoso silencio, el pobre hombre pidió permiso a Catalina para volver inmediatamente a Francia, permiso que le fue graciosamente concedido.

\end{quote}

No contento con esta obra maestra, Euler añadió gravemente las pruebas solemnes de que Dios existe, y de que el alma no es una sustancia material. Se dice que ambas pruebas fueron incorporadas a los tratados de teología de la época. Se trata probablemente de flores escogidas de la faceta matemática no práctica de su genio.

La Matemática no es lo único que absorbió las energías de Euler durante su permanencia en Rusia. Siempre que fue solicitado para ejercer sus talentos matemáticos en terrenos alejados de la Matemática pura, accedió a la solicitación. Euler escribió los manuales matemáticos elementales para las escuelas rusas, el departamento oficial de geografía, ayudó a reformar el sistema de pesas y medidas, etc. Estas fueron algunas de sus actividades. Aparte de esta obra ajena a la Matemática, Euler continuó sus investigaciones favoritas.

Una de las obras más importantes de este período fue el tratado de 1736 sobre Mecánica. Obsérvese que a la fecha de publicación le falta un año para coincidir con el centenario de la publicación de la Geometría Analítica de Descartes. El tratado de Euler hizo para la Mecánica lo que el de Descartes hizo para la Geometría, liberarla de las cadenas de la demostración sintética, haciéndola analítica. Los {\it Principia} de Newton pudieron haber sido escritos por Arquímedes; la Mecánica de Euler no pudo ser escrita por un griego. Por primera vez el gran poder del Cálculo infinitesimal fue dirigido hacia la Mecánica, y entonces comienza la era moderna para esa ciencia básica. Euler fue superado en esta dirección por su amigo Lagrange, pero el mérito de haber dado el paso decisivo corresponde a Euler.

A la muerte de Anna, en 1740, el gobierno ruso se hizo más liberal, pero Euler ya estaba fatigado, y aceptó con satisfacción la invitación de Federico el Grande para que se incorporara a la Academia de Berlín. La reina viuda tomó cariño a Euler e intentó sonsacarle. Todo lo que pudo obtener fueron monosílabos.

\noindent --- ¿Por qué no queréis hablarme?--- le preguntó.

\noindent --- Señora ---replicó Euler--- vengo de un país donde al que habla se le ahorca.

Los 24 años siguientes de su vida transcurrieron en Berlín, y no fueron muy felices, pues Federico hubiera preferido a un pulido cortesano en lugar del sencillo Euler. Aunque Federico creía que su deber era fomentar la Matemática, se desvió de ese deseo. Pero demostró que apreciaba en muchos los talentos de Euler al proponerle problemas prácticos: el sistema monetario, la conducción de aguas, los canales de navegación, sistemas y cálculos de pensiones, entre otros.

Rusia jamás olvidó a Euler completamente y mientras estuvo en Berlín le pagó parte de su sueldo. A pesar de que su familia era numerosa, Euler vivió prósperamente y poseía una casa de campo cerca de Charlottenburg, además de su casa de Berlín. Durante la invasión rusa en 1760, la casa de campo de Euler fue saqueada. El general ruso declaró que «no hacía la guerra a la ciencia», e indemnizó a Euler con una cantidad superior a la que representaba el verdadero daño. Cuando la Emperatriz Isabel oyó hablar de la pérdida de Euler, le envió una cuantiosa suma, aparte de la más que suficiente indemnización.

Una causa de la falta de popularidad de Euler en la corte de Federico fue su incapacidad para oponer argumentos a las cuestiones filosóficas, que le eran totalmente desconocidas. Voltaire, que empleó gran parte de su tiempo adulando a Federico, se divertía, con los otros brillantes verbalistas que rodeaban al Emperador, en hacer caer al infeliz Euler en los enredos metafísicos. Euler lo admitía sin resistencia, y se unía a los demás en sus risas por sus ridículos disparates. Pero Federico se irritaba cada vez más, y pensó en un filósofo más agudo para encabezar su Academia y entretener a su corte.

D'Alembert fue invitado a Berlín para examinar la situación. Él y Euler tenían ligeras diferencias respecto a la Matemática; pero D'Alembert no era el hombre a quien un entredicho personal enturbiaba el juicio, y contestó a Federico diciéndole que sería un ultraje colocar a cualquier otro matemático por encima de Euler. Esta respuesta tan sólo dio lugar a que Federico se irritara más que antes, y las condiciones se hicieron intolerables para Euler. Pocas probabilidades de triunfo esperaban a sus hijos en Prusia, y a la edad de 50 años (en 1776), volvió a hacer su equipaje y se dirigió nuevamente a San Petersburgo, invitado cordialmente por Catalina la Grande.

Catalina recibió al matemático como si fuera un noble y le hizo preparar una espléndida casa  para él y los 18 miembros de su familia, cediéndole uno de sus propios cocineros.

Por esta época Euler comenzó a perder (por una catarata), la visión del ojo que le quedaba. La progresión de su ceguera fue seguida con alarma y consternación por Lagrange, D'Alembert y otros eminentes matemáticos de la época. Euler mismo sentía aproximarse su ceguera con serenidad. No hay duda de que su profunda fe religiosa le ayudaba a enfrentarse con lo que era superior a él. Pero no se resignaba al silencio y a la oscuridad, e inmediatamente se dedicó a reparar lo irreparable.

Antes de que se apagara el último rayo de luz, se habituó a escribir sus fórmulas con yeso en una gran pizarra. Luego, sus hijos, (particularmente Alberto) actuaban de amanuenses, y el padre dictaba las palabras y explicaba las fórmulas. En lugar de disminuir, su producción matemática aumentó.

Toda su vida, Euler gozó de una memoria fenomenal. Sabía de memoria la {\it Eneida} de Virgilio, y aunque desde su juventud rara vez había vuelto a releer la obra, podía siempre decir cuál era la primera y la última línea de cada página de su ejemplar. Su memoria era visual y auditiva. También tenía una capacidad prodigiosa para el cálculo mental, no sólo del tipo aritmético, sino también del tipo más difícil exigido en el Álgebra superior y en el Cálculo infinitesimal. Las fórmulas principales de toda la Matemática existente en su época, estaban cuidadosamente grabadas en su memoria.

Como un ejemplo de esta capacidad, Condorcet cuenta que dos de los discípulos de Euler habían sumado una complicada serie convergente (para un valor particular de la variable) con 17 términos, y sólo estaban en desacuerdo en una cifra del lugar decimoquinto del resultado. Para decidir cuál era la suma exacta Euler realizó todo el cálculo mentalmente, y su respuesta fue la exacta. Esta memoria venía en su ayuda para consolarle de su ceguera. La teoría lunar ---el movimiento de la Luna--- el único problema que había producido dolores de cabeza a Newton, recibió por entonces una completa solución al ser tratada por Euler. Todo el complicado análisis fue hecho de memoria.

A los cinco años de haber vuelto a San Petersburgo le aconteció otro desastre. En el gran fuego de 1771 su casa y todos sus muebles quedaron destruidos, y gracias al heroísmo de su sirviente suizo (Peter Grimm, o Grimmon), Euler logró salvarse con el riesgo de su propia vida, Grimm pudo arrastrar a su amo, ciego y enfermo, fuera de las llamas. La biblioteca se quemó pero gracias a la energía del Conde Orloff fueron salvados todos los manuscritos de Euler. La Emperatriz Catalina prontamente reparó todos los daños, y Euler volvió a sus trabajos.

En 1776 (cuando tenía 69 años) Euler sufrió una gran pérdida con la muerte de su mujer. Al año siguiente volvió a casarse. La segunda mujer, Salomé Abigail Gsell, era media hermana de la primera. Su gran tragedia fue el fracaso de una operación para restablecer la visión de su ojo izquierdo, el único del que podían abrigarse esperanzas. La operación había sido eficaz, y la alegría de Euler fue inenarrable. Pero se presentó una infección, y después de un prolongado sufrimiento, Euler volvió a sumirse en la obscuridad.

Al examinar retrospectivamente la enorme producción de Euler podemos sentirnos inclinados a creer que cualquier hombre de talento podría haber hecho una gran parte de ese trabajo casi tan fácilmente como Euler. Pero un examen de la Matemática actual pronto nos mostrará nuestro error. En el estado presente, la Matemática, con sus bosques de teorías, no es más complicada que antes, si consideramos el poder de los métodos que ahora tenemos a nuestra disposición, y de que Euler no disponía, y ahora está ya madura para un segundo Euler. En su época, Euler sistematizó y unificó los resultados parciales y los teoremas aislados, desbrozando el terreno y asociando todas las cosas de valor con la fácil capacidad de su genio analítico. Mucho de lo que hoy se enseña en los cursos elementales de Matemática se debe prácticamente a Euler, por ejemplo, la teoría de secciones cónicas y cuadráticas en el espacio tridimensional desde el punto de vista unificado proporcionado por la ecuación general de segundo grado. Además, la cuestión de las anualidades, y todos los problemas que en ella se deducen (seguros, pensiones a la vejez, etc.), fueron planteados en la forma que hoy los estudiosos conocen con el nombre de «teoría matemática de las inversiones» de Euler.

Como Arago señala, una causa del gran e inmediato triunfo de Euler como maestro se debe a su falta total de falso orgullo. Cuando ciertos trabajos de mérito intrínseco relativamente escaso eran necesarios para aclarar otras investigaciones anteriores y más importantes, Euler no dudaba en realizarlos, sin temor de que disminuyera su reputación.

Hasta en la faceta creadora Euler combinó la enseñanza con el descubrimiento. Sus grandes tratados de 1748, 1755, y 1768-70, sobre el cálculo ({\it Introductio in analysin infinitorum}; {\it Institutiones calculi differentialis}; {\it Institutiones calculi integralis}) se hicieron rápidamente clásicos, y continuaron, durante tres cuartos de siglo, inspirando a los jóvenes que iban a ser grandes matemáticos. Pero fue en su obra sobre el cálculo de variaciones ({\it Methodus inveniendi lineas curvas maximi minimive propietate gaudentes}, 1744), donde Euler se reveló, por primera vez, como un matemático de primera categoría. 

Ya hemos hablado del gran paso de Euler cuando hizo analítica a la Mecánica; cualquier estudioso de la Dinámica rígida está familiarizado con el análisis de las rotaciones de Euler, por sólo citar un detalle de sus progresos. La Mecánica analítica es una rama de la Matemática pura, de modo que Euler no estuvo tentado en este caso, como en alguna de sus fugas hacia el terreno práctico, de escapar por la tangente para volver al infinito campo del Cálculo puro. Las más grandes críticas que los contemporáneos de Euler hicieron de su obra, se referían a su ingobernable impulso de calcular simplemente por el objeto de realizar un bello análisis. Alguna vez es posible que haya carecido de la suficiente comprensión de la situación real, y haya intentado reducirla al cálculo sin ver lo que existía de verdad en ella. De todos modos, las ecuaciones fundamentales del movimiento de los fluidos, que se usan actualmente en hidrodinámica, son de Euler. Supo ser práctico cuando la situación práctica era digna de su meditación.

Una peculiaridad del análisis de Euler debe ser mencionada en este lugar, pues fue la causa de una de las principales direcciones de la Matemática del siglo XIX. Era su reconocimiento de que a no ser que una serie infinita sea convergente, su uso no es seguro. Por ejemplo, por una larga división encontramos:

\vspace{-1em}
\[
\frac{1}{x-1}= \frac{1}{x}+\frac{1}{x^2}+\frac{1}{x^3}+ \cdots
\]
y la serie se continúa indefinidamente. Para $x = 1/2$, se tiene:

\vspace{-1em}
\[
-2 = 2 + 2^2 + 2^3 + 2^4 + \cdots = 2 + 4 + 8 + 16 + \cdots
\]

El estudio de la convergencia  nos enseña a evitar absurdos como éste. Lo curioso es que aunque Euler reconoció la necesidad de ser cauteloso al tratar con procesos infinitos, fue incapaz de tener esa cautela en gran parte de su obra. Su fe en el Análisis era tan grande que podía algunas veces buscar una «explicación absurda» para hacer aceptable un absurdo evidente.

Debemos añadir aún que pocos han igualado o se han aproximado a Euler en la cantidad de sólidos y nuevos trabajos de importancia esencial. Los que gustan de la Aritmética, reconocerán a Euler, en el análisis diofántico, méritos análogos a los que pueden atribuirse a Fermat y al mismo Diofanto. Euler fue el primero, y posiblemente el más grande de los universalistas matemáticos.

Euler no fue solamente un matemático; en literatura y en todas las otras ciencias, incluyendo la biología, era también muy ducho. Pero hasta cuando gozaba recitando su {\it Eneida}, Euler no podía menos de buscar en ella un problema merecedor de ser abordado por su genio matemático. El verso «El ancla desciende, la quilla que avanzaba se detiene», le estimula a trabajar sobre el movimiento del barco en tales circunstancias. Su curiosidad omnívora le llevó a estudiar durante cierto tiempo astrología, pero demostró que no la había digerido cuando cortésmente se negó a establecer el horóscopo del príncipe Iván al ordenársele hacerlo así en 1740, advirtiendo que los horóscopos correspondían al astrónomo de la corte; y el pobre astrónomo tuvo que hacerlo.

Una obra del período en que estuvo en Berlín revela a Euler como un escritor lleno de gracia, aunque también demasiado piadoso: las conocidas {\it Cartas a una Princesa Alemana}, escritas para dar lecciones de Mecánica, Óptica, Física, Astronomía, Sonido, etc., a la sobrina de Federico la Princesa de Anhalt-Dessau. Las famosas cartas se hicieron muy populares, circulando en forma de libro en siete idiomas. El interés del público por la ciencia no es de tan reciente desarrollo como algunas veces estamos inclinados a imaginarnos.

Euler mantuvo una mente viril y poderosa, hasta el momento de su muerte, que tuvo lugar cuando tenía 77 años, el 18 de septiembre de 1783. Después de haberse divertido una tarde calculando las leyes del ascenso de los globos, sobre su pizarra, como de ordinario, cenó con Lexell y su familia. «El planeta de Herschel» (Urano) era un descubrimiento reciente; Euler bosquejó el cálculo de su órbita. Poco después pidió a su nieto que se acercara. Mientras jugaba con el niño y bebía una taza de té, sufrió un ataque. La pipa cayó de su mano, y con las palabras «Me muero», «Euler cesó de vivir y de calcular».




\end{document}