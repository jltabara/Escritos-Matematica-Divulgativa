\documentclass[a4paper, 12pt]{article}

%%%%%%%%%%%%%%%%%%%%%%Paquetes
\usepackage[spanish]{babel}  
\usepackage[utf8]{inputenc}
\usepackage{tcolorbox}
\usepackage{cmbright}  %%%%%%% El tipo de letra
\usepackage{setspace}
\onehalfspacing  %%%%%%%%%%% Espacio y medio de interlineado
\parskip=1em  %%%%%%%%%%%% Separacion entre parrafos
%%%%%%%%%%%%%%%%%%%%%%



%%%%%%%%%%%%%%%%%
\title{El Matemático}
\author{J. Von Neumann}
\date{}
%%%%%%%%%%%%%%%%%

\begin{document}

\begin{tcolorbox}[colback=blue!5!white,colframe=blue!75!black]

\vspace{-1.8cm}
\textbf \maketitle

\end{tcolorbox}

\bigskip

Un análisis sobre la naturaleza del trabajo intelectual es una tarea difícil en cualquier campo, incluso en los que no están tan apartados del área central de nuestro común esfuerzo intelectual humano como lo son todavía las matemáticas. Una discusión sobre la naturaleza de cualquier esfuerzo intelectual es difícil \textit{per se}; en todo caso, más difícil que el mero ejercicio de ese particular esfuerzo intelectual. Es más difícil comprender el mecanismo de un aeroplano y las teorías de las fuerzas que lo elevan y lo mueven, que montar en él, ser elevado y transportado por él o incluso gobernarlo. Es excepcional que uno fuera capaz de adquirir el conocimiento de un proceso sin haber adquirido anteriormente una profunda familiaridad con su funcionamiento, con su uso, sin antes haberlo asimilado de una manera instintiva y empírica.

Así, cualquier análisis sobre la naturaleza de un esfuerzo intelectual en cualquier campo es difícil, a menos que se presuponga una fácil y rutinaria familiaridad con este campo. En las matemáticas esta limitación llega a ser muy grave, si el análisis debe mantenerse en un plano no matemático. Entonces mostrará necesariamente algunos aspectos muy malos; se aceptarán principios que nunca podrán ser comprobados propiamente y se hace inevitable una cierta superficialidad general. Es su relación peculiar con las ciencias naturales, o, más generalmente, con toda ciencia que interprete la experiencia en un nivel superior al descriptivo.

Sé muy bien las limitaciones con que voy a hablar y me disculpo de antemano. Asimismo, las oponiones que voy a expresar probablemente no son plenamente compartidas por muchos matemáticos ---ustedes adquirirán las impresiones 
e interpretaciones no muy bien sistematizadas de un hombre--- y yo puedo prestarles muy poca ayuda para decidir hasta qué punto son acertadas.
Sin embargo, a pesar de todos los inconvenientes, debo admitir que el hacer el intento y hablarles a ustedes de la naturaleza del esfuerzo intelectual en matemáticas es una tarea interesante y desafiadora. Espero no hacerlo del todo mal.

En mi opinión, el hecho más esencialmente característico de las matemáticas 
es su relación enteramente peculiar con las ciencias naturales, o, más generalmente, con toda ciencia que interprete la experiencia en un nivel superior al puramente descriptivo.

La mayoría de la gente, matemáticos o no, convendrán en que las matemáticas no son una ciencia empírica, o al menos en que se practican de una manera que difiere en varios aspectos decisivos de las técnicas de las ciencias empíricas. Y, aún así, su desarrollo está estrechamente ligado al de las ciencias naturales. Una de sus ramas principales, la geometría, empezó de hecho como ciencia natural y empírica. Algunas de las mejores inspiraciones de las matemáticas modernas (yo creo que las mejores) provienen claramente de las ciencias naturales. Los métodos matemáticos invaden y dominan las secciones «teóricas» de las ciencias naturales. En las modernas ciencias empíricas se han convertido en un criterio principal de éxito el que hayan llegado a ser accesibles al método matemático o a métodos físicos casi matemáticos. Ciertamente, a través de todas las ciencias naturales se ha hecho cada vez más evidente una ininterrumpida cadena de sucesivos seudomorfismos, todos ellos tendiendo hacia las matemáticas, y casi identificados con la idea de progreso científico. La biología ha sido invadida gradualmente por la química y la física, la química por la física experimental y teórica, y la física por formas muy matemáticas de física teórica.

Existe una duplicidad completamente peculiar en la naturaleza de las matemáticas. Uno tiene que darse cuenta de esta duplicidad, aceptarla y asimilarla en sus conceptos del asunto. Este doble aspecto es el de las matemáticas y no creo que sea posible otra visión simplificada y unitaria de ellas sin sacrificar su esencia.

Por lo tanto, no intentaré ofrecerles una versión unitaria. Intentaré describir, lo mejor que pueda, el fenómeno múltiple que constituye las matemáticas.

Es innegable que algunas de las mejores inspiraciones de las matemáticas ---de parte de ellas, que son matemáticas tan puras como pueda imaginarse--- proceden de las ciencias naturales. Mencionaré los dos hechos más sobresalientes.

El primer ejemplo es, como tenía que ser, geométrico. La geometría era la parte principal de las matemáticas antiguas. Todavía es, con alguna de sus ramificaciones, una de las secciones principales de las matemáticas modernas. No puede haber duda de que su origen en la antigüedad fue empírico y que empezó de manera semejante a la física teórica de hoy. Aparte todas las demás evidencias, el mismo nombre «geometría» lo indica. El tratamiento con postulados de Euclides representa un gran paso fuera del empirismo, pero no es del todo simple el defender la postura de que éste fue el paso decisivo y final para producir una separación absoluta. El hecho de que la axiomatización de Euclides no cumpla en algunos puntos menores, las exigencias modernas de absoluto rigor axiomático es de menor 
importancia en este aspecto. Lo más esencial es esto: otras disciplinas, que son indudablemente empíricas, como la mecánica y la termodinámica, se presentan generalmente con un tratamiento más o menos postulativo, que en la presentación de algunos autores se distingue difícilmente del procedimiento de Euclides. El clásico de la física teórica de nuestro tiempo, los \textit{Principia} de Newton, era, en la forma literaria así como en la esencia de algunas de sus partes más críticas, un libro muy parecido a los de Euclides. Naturalmente, en todos estos ejemplos hay detrás de la presentación en forma de postulados la perspicacia física que apoya los postulados y la comprobación experimental que sostiene los teoremas. Pero bien puede argumentarse que es posible una interpretación similar de Euclides, especialmente desde el punto de vista de la antigüedad, antes de que la geometría hubiera adquirido su actual estabilidad y autoridad bimilenaria, de la que está claramente falto el moderno edificio de la física teórica.

Además, aun cuando la desempirización de la geometría ha progresado gradualmente desde Euclides, nunca ha llegado a ser completa, ni siquiera en los tiempos modernos. La discusión de la geometría no euclídea constituye un buen ejemplo de esto. Y también de la ambivalencia del pensamiento matemático. Ya que la mayor parte de la discusión transcurre en un plano muy abstracto, trata el problema puramente lógico de si el «quinto postulado» de Euclides era o no una consecuencia de los otros, y el conflicto formal fue acabado por el ejemplo, puramente matemático de F. Klein, que hizo ver cómo un trozo de plano euclídeo podía hacerse no euclídeo definiendo otra vez formalmente ciertos conceptos básicos. Y el estímulo empírico aún estaba aquí desde el comienzo hasta el fin. La primera razón de por qué, de entre todos los postulados de Euclides, se discutía el quinto era claramente el carácter no empírico del concepto de todo el plano infinito que interviene allí y sólo allí. La idea de que, por lo menos en un importante sentido ---y a pesar de todos los análisis logicomatemáticos---, la decisión en pro o en contra de Euclides puede tener que ser empírica, pasó ciertamente por la mente del más eximio de los matemáticos, Gauss. Y después Bolyai, Lobatchevsky, Riemann y Klein han obtenido \textit{more abstracto} lo que hoy en día consideramos como solución formal de la controversia primitiva; sin embargo, el empirismo ---o mejor la física--- tuvo la última palabra. El descubrimiento de la relatividad general impuso una revisión de nuestras perspectivas sobre la relación geométrica de una manera enteramente nueva y con una distribución completamente nueva del puro énfasis matemático. Finalmente, un toque más para completar la imagen de contraste. Este último desarrollo se produjo en la misma generación que vio la completa desempirización y abstracción del método axiomático de Euclides en manos de los modernos matemáticos de lógica axiomática. Y estas dos actitudes aparentemente contradictorias son compatibles en una mente matemática; así, Hilbert hizo importantes aportaciones a la geometría axiomática y a la relatividad general.

El segundo ejemplo es de cálculo, o mejor de análisis, que proviene de él. El cálculo fue la primera obra de las matemáticas modernas y es difícil sobreestimar su importancia. Creo que define mejor que cualquier otra cosa el comienzo de
las matemáticas modernas, y el sistema de análisis matemático, que es su desarrollo lógico, constituye todavía el mayor avance técnico del pensamiento exacto.

Los orígenes del cálculo son claramente empíricos. Los primeros intentos de integración de Kepler fueron formulados como «dolicometría» ---medida de cuñas--- esto es, volumetría para cuerpos con superficies curvas. Es decir, geometría, pero posteuclídea y, en la época en cuestión, geometría no axiomática, empírica. Esto Kepler lo sabía de sobra. El principal esfuerzo y los mayores descubrimientos, los de Newton y Leibniz, eran de origen explícitamente físico. Newton inventó el cálculo «de fluxiones» con propósitos esencialmente mecánicos ---de hecho, desarrollaba más o menos juntas las dos disciplinas, cálculo y mecánica---. Las primeras formulaciones de cálculo no eran ni siquiera rigurosamente matemáticas. ¡Una formulación inexacta, semifísica, fue la única disponible durante más de ciento cincuenta años después de Newton! ¡Y, sin embargo, durante este período se produjeron algunos de los más importantes avances de análisis a pesar de estas bases inexactas y matemáticamente inadecuadas! Algunas de las principales mentes matemáticas del período no eran nada rigurosas, como Euler; pero otras, las principales, lo eran, como Gauss o Jacobi. El desarrollo era muy confuso y ambiguo y su relación con el empirismo no estaba ciertamente de acuerdo con nuestras ideas actuales (o euclideas) de abstracción y rigor. Con todo, ningún matemático desearía excluirlo del conjunto ---¡este período produjo matemáticos tan eminentes como cualquier otro!--- E incluso después de que el predominio del rigor fue esencialmente restablecido con Cauchy, se produjo con Riemann una recaída muy peculiar en los métodos semiempíricos. La personalidad científica de Riemann es uno de los ejemplos más ilustrativos de la doble naturaleza de las matemáticas, así como la controversia de Riemann y Weierstrass, pero me introduciría demasiado en materias técnicas si entrara en detalles específicos. Desde Weierstrass, el análisis parece haber llegado a ser completamente abstracto, riguroso y no empírico. Pero esto no es absolutamente cierto. La controversia acerca de los «fundamentos» de las matemáticas y de la lógica, que se produjo durante las dos últimas generaciones, disipó muchas ilusiones sobre el asunto.

Esto me conduce al tercer ejemplo, que es fundamental para el diagnóstico. Sin embargo, este ejemplo trata de la relación de las matemáticas con la filosofía o epistemología más que con las ciencias naturales. Ilustra de una manera muy llamativa que el mismo concepto de rigor matemático «absoluto» no es inmutable. La variabilidad del concepto de rigor muestra que, aparte de la abstracción matemática, debe existir alguna otra cosa en el modo de ser de las matemáticas. Al analizar la controversia acerca de los «fundamentos», no he sido capaz de convencerme a mí mismo de que el veredicto deba ser en favor de la naturaleza empírica de esta componente adicional. El proceso en favor de tal interpretación es bastante fuerte, al menos en algunas fases del estudio. Pero no lo considero del todo convincente. Sin embargo, dos cosas son claras. Primero, que entra de manera esencial algo no matemático, relacionado de alguna manera con las ciencias empíricas o con la filosofía o con ambas a la vez; y su carácter no empírico sólo podría afirmarse si se supone que la filosofía (o más específicamente la epistemología), pueden existir en independencia de la experiencia. (Y está suposición es sólo necesaria, pero no suficiente.) Segundo, que el origen empírico de las matemáticas está fuertemente sostenido por casos como nuestros dos primeros ejemplos (geometría y cálculo), aparte de la mejor interpretación de la controversia acerca de los «fundamentos».



Al analizar la variabilidad del concepto de rigor matemático, deseo dar una importancia principal a la controversia de los «fundamentos» antes mencionada. Sin embargo, me gustaría considerar en primer lugar de manera breve un aspecto secundario de la cuestión. Este aspecto también confirma mi argumento, pero lo considero secundario, porque probablemente es menos conclusivo que el análisis de la controversia de los «fundamentos». Me refiero a los cambios de «estilo» matemático. Es bien sabido que el estilo con que se escriben las demostraciones matemáticas ha experimentado considerables fluctuaciones. Es mejor hablar de fluctuaciones que de tendencias, porque en ciertos aspectos la diferencia entre lo actual y ciertos autores de los siglos XVIII o XIX es mayor que entre lo actual y Euclides. Por otra parte, en otros aspectos ha habido una notable constancia. En los campos en que hay diferencias, éstas son principalmente de presentación, que pueden eliminarse sin introducir nuevas ideas. Sin embargo, en muchos casos las diferencias son tan amplias que uno empieza a dudar si los autores que «presentan sus casos» de formas tan divergentes pueden estar separados solamente por diferencias de estilo, gusto o educación; de si realmente pueden haber tenido las mismas ideas acerca de lo que constituye el rigor matemático. Finalmente, en los casos extremos (por ejemplo, en muchos de los trabajos de análisis de finales del siglo XVIII, antes citados), las diferencias son tan fundamentales que sólo pueden corregirse, si es posible, con la ayuda de nuevas y profundas teorías, que tardaron cien años en desarrollarse. Algunos de los matemáticos que trabajaron con tales procedimientos, para nosotros no rigurosos (o algunos de sus contemporáneos que los criticaron), conocían bien su falta de rigor. O para ser más objetivos: sus propias aspiraciones de lo que debía ser un proceso matemático estaban más de acuerdo que sus acciones con nuestros actuales conceptos. Pero otros ---como Euler, el mayor virtuoso del período--- parecen haber actuado con perfecta buena fe y haber quedado completamente satisfechos de sus obras.

Sin embargo, no deseo insistir más sobre este asunto. En cambio, me dedicaré a un suceso perfectamente definido, la controversia acerca de los «fundamentos de las matemáticas». A finales del siglo {XIX y a principios del XX una nueva rama de matemáticas abstractas, la teoría de conjuntos de G. Cantor, provocó nuevas dificultades. Esto es, ciertos razonamientos conducían a contradicciones y, aun cuando estos razonamientos no estaban en la parte central y «útil» de la teoría de conjuntos y siempre podían ser criticados fácilmente con ciertos criterios formales, no obstante no quedaba claro por qué debían considerarse teóricamente menos que las partes «afortunadas» de la teoría. Aparte de la perspicacia \textit{ex post} de que conducían realmente al desastre, no quedaba claro qué motivo \textit{a priori}, qué filosofía consistente de la situación, permitía segregarlos de estas partes de la teoría de conjuntos que se deseaba salvar. Un estudio más atento de los \textit{merita} del caso, emprendido principalmente por Russell y Weyl y concluido por Brouwer, mostró que
la manera en que no sólo la teoría de conjuntos sino también la mayoría de las matemáticas modernas utilizaban los conceptos de «validez general» y de «existencia» era recusable filosóficamente. Un sistema matemático que estaba libre de  estos rasgos indeseables, el «intuicionismo», fue desarrollado por Brouwer. En este sistema
 no se presentaban las dificultades y contradicciones de la teoría de  conjuntos. Sin embargo, un buen cincuenta por ciento de las matemáticas modernas, en sus partes más vitales ---y hasta entonces indiscutibles---, especialmente el análisis, eran también afectadas por esta «purga»; o se volvían no válidas o tenían que  ser justificadas por consideraciones secundarias muy complicadas. Y en este último proceso, a menudo se perdía apreciablemente en cuanto a validez general  y elegancia en la deducción. No obstante, Brouwer y Weyl consideraron necesario que  se
revisara el concepto de rigor matemático de acuerdo con estas ideas.


Es difícil sobrestimar la significación de estos sucesos. ¡En la tercera década del siglo XX, dos matemáticos ---los dos de primera magnitud y tan plenamente conscientes de lo que son las matemáticas, o de cuál es su objeto, o de lo que tratan, como el que más--- propusieron de hecho que el concepto de rigor matemático, de lo que constituye una demostración exacta, debía cambiarse! Los desarrollos que siguieron son igualmente dignos de notarse.


\begin{enumerate}

\item  Sólo muy pocos matemáticos quisieron aceptar los nuevos y exigentes modelos para uso diario. Sin embargo, muchos admitieron que Weyl y Brouwer estaban \textit{prima facie} en lo cierto, pero continuaron transgrediendo estos modelos, esto es, haciendo sus matemáticas en la vieja y «fácil manera» ---probablemente con la esperanza de que alguien encontrara, en otro tiempo, la respuesta a la crítica intuicionista y que, por lo tanto, les justificase \textit{a posteriori}---.


\item Hilbert salió al paso con una ingeniosa idea para justificar las matemáticas «clásicas» (es decir, preintuicionistas): Incluso en el sistema intuicionista es posible dar una explicación rigurosa de cómo operan las matemáticas clásicas, esto es, puede describirse cómo trabaja el sistema clásico, aunque no pueden justificarse sus procedimientos. Por lo tanto, cabía demostrar de manera intuionista que los procedimientos clásicos no podían conducir nunca a contradicciones, a conflictos entre sí. Estaba claro que tal demostración sería muy difícil, pero había ciertas indicaciones de cómo podía intentarse. ¡Si este proyecto hubiera dado resultado habría suministrado una justificación notable de las matemáticas clásicas a partir del sistema intuicionista opuesto! Al menos, esta interpretación se habría legitimado en un sistema de filosofía matemática que la mayoría de matemáticos querían aceptar.


\item  Después de casi una década de intentos para realizar este programa, Gödel suministró un resultado más notable. Este resultado no puede enunciarse con absoluta precisión sin ciertas cláusulas e hipótesis que son demasiado técnicas para formularlas aquí. Sin embargo, su significado esencial era éste: Si un sistema matemático no conduce a contradicción, entonces este hecho no puede demostrarse con procedimientos de ese sistema. La demostración de Gödel satisfizo el criterio de rigor matemático más estricto, el intuicionista. Su influencia en el programa de Hilbert es algo controvertible, por razones que de nuevo son
demasiado técnicas para esta ocasión. Mi opinión personal, que es compartida por muchos otros, es que Gödel demostró que el programa de Hilbert es esencialmente sin esperanza.


\item  Una vez desaparecida la principal esperanza de justificación de las matemáticas clásicas ---en el sentido de Hilbert o de Brouwer y Weyl--- muchos matemáticos decidieron usar ese sistema sea lo que fuere. Después de todo, las matemáticas clásicas daban resultados que eran elegantes y útiles y, aunque nunca se podría estar de nuevo absolutamente seguro de su precisión, al menos tenían un fundamento tan sólido como, por ejemplo, la existencia del electrón. He aquí que si uno quería aceptar las ciencias, podía asimismo aceptar el sistema de matemáticas clásicas. Tales opiniones llegaron a ser aceptables incluso para algunos de los primitivos defensores del sistema intuicionista. Actualmente la controversia acerca de los «fundamentos» no está acabada, pero parece improbable que nadie, excepto una pequeña minoría, abandone el sistema clásico.

\end{enumerate}



He narrado tan detalladamente la historia de esta controversia, porque pienso que constituye la mejor advertencia contra el dar demasiado por supuesto el inmutable rigor de las matemáticas. Esto sucedió en el transcurso de mi propia vida y yo sé con cuán humillante facilidad mi concepto sobre la verdad absoluta matemática  ha cambiado durante estos episodios ¡y cómo ha cambiado sucesivamente tres veces!

Espero que los tres ejemplos anteriores ilustren bien una mitad de mi tesis ---que muchas de las mejores inspiraciones matemáticas proceden de la experiencia y que difícilmente es posible creer en la existencia de un concepto de rigor matemático absoluto, inmutable y disociado de toda experiencia humana---. Estoy intentando tomar una actitud poco culta en este asunto. Cualesquiera que sean las preferencias filosóficas o epistemológicas que puedan tenerse respecto a esto, las actuales experiencias de las asociaciones matemáticas con su sujeto dan muy poca base para suponer de la existencia de un concepto a priori de rigor matemático. Sin embargo, mi tesis también tiene una segunda mitad y ahora voy a volverme hacia esta parte.

Es muy difícil para un matemático el creer que las matemáticas son una ciencia puramente empírica o que todas las ideas matemáticas proceden de materias empíricas. Permítanme considerar primero la segunda parte de la afirmación. Existen varias partes importantes de las matemáticas modernas en que no puede señalarse el origen empírico, o, si puede señalarse, es tan remoto que queda claro que el sujeto ha sufrido una completa metamorfosis desde que fue separado de sus raíces empíricas. El simbolismo del álgebra fue inventado para uso exclusivamente matemático, pero puede afirmarse razonablemente que tuvo fuertes vínculos empíricos. Sin embargo, el álgebra moderna, «abstracta», se ha desarrollado más y más en direcciones que tienen aún menos conexiones empíricas. Lo mismo puede decirse de la topología. Y en todos estos campos el criterio subjetivo de éxito del matemático, de valoración de su esfuerzo, es muy reservado, estético y libre (o casi libre) de conexiones empíricas. (Sobre esto insistiré más adelante.) En teoría de conjuntos esto es todavía más claro. La «potencia» y la «ordenación» de un conjunto infinito pueden ser la generalización de conceptos numéricos finitos, pero su forma infinita (especialmente la «potencia») difícilmente tienen ninguna relación con este mundo. Si no deseara evitar los tecnicismos, podría documentar esto con numerosos ejemplos de teoría de conjuntos ---el problema del «axioma de elección», la comparatividad de «potencias» infinitas, «el problema del continuo», etcétera---. Las mismas observaciones se aplican asimismo a la teoría de funciones reales y a la teoría de conjuntos de puntos reales. La geometría diferencial y la teoría de grupos dan dos sorprendentes ejemplos: fueron ciertamente concebidas como disciplinas abstractas, no aplicadas, y casi siempre cultivadas con ese espíritu. Después de una década en un caso y de un siglo en el otro, resultaron muy útiles para la física. Y todavía se continúan principalmente con el citado espíritu abstracto, de no aplicación.


Podrían multiplicarse los ejemplos de estas condiciones y de sus diversas combinaciones, pero en lugar de esto prefiero volver al primer punto, que he indicado antes: ¿Las matemáticas son una ciencia empírica? O, con mayor precisión: ¿Las matemáticas se desarrollan actualmente de la misma manera que se desarrollan las ciencias empíricas? O, con mayor generalidad: ¿Cuál es la relación normal del matemático a su objeto? ¿Cuáles son sus criterios de éxito, de elección? ¿Qué influencias, qué consideraciones controlan y dirigen su esfuerzo?


Veamos, entonces, en qué aspectos la manera en que el matemático trabaja normalmente difiere del modo de trabajo de las ciencias naturales. La diferencia entre éstas, por una parte, y las matemáticas, por otra, va creciendo claramente, a medida que se pasa de las disciplinas teóricas a las experimentales, y de las experimentales a las descriptivas. Comparemos, por lo tanto, las matemáticas con la categoría que está más próxima a ellas: las disciplinas teóricas. Y escojamos la que está más próxima a las matemáticas. Espero que no me juzgarán con mucho rigor si no consigo controlar la \textit{hybris} matemática y añado que es la que ha alcanzado un desarrollo más alto entre todas las ciencias teóricas ---es decir, la física teórica---. Las matemáticas y la física teórica tienen, de hecho, buena parte en común. Como antes he indicado, el sistema geométrico de Euclides fue el prototipo de presentación axiomática de la mecánica clásica, y tratamientos similares dominan la termodinámica fenomenológica, así como ciertos aspectos del sistema electrodinámico de Maxwell, y también de la relatividad especial. Además, la postura de que la física teórica no explica los fenómenos, sino que sólo los clasifica y correlaciona, es hoy en día adoptada por muchos físicos. Esto significa que el criterio de éxito para tal teoría es simplemente el de poder, mediante un esquema de clasificación y de correlación sencillo y elegante, abarcar muchos fenómenos, que sin este esquema parecerían complicados y heterogéneos, y el de si este esquema abarca fenómenos que no se consideraron, o incluso que no se conocían en el tiempo
que este esquema fue desarrollado. (Estas dos últimas afirmaciones expresan, naturalmente, el poder de unificación y de predicción de la teoría.) Ahora bien, este criterio, tal como aquí se expresa, es, en gran parte, de naturaleza estética. Por esta razón, es muy parecido a los criterios de éxito matemático, que, como verán, son casi completamente estéticos. De este modo, estamos ahora comparando las matemáticas con la ciencia empírica que está más unida a ellas, y con la que tienen, como espero haber demostrado, mucho en común con la física teórica. Las diferencias en el actual \textit{modus procedendi} son, no obstante, grandes y básicas. Los objetivos de la física teórica son, en su mayor parte, dados desde «fuera», en muchos casos por las necesidades de la física experimental. Casi siempre proceden de la
	necesidad de resolver una dificultad. Los éxitos de predicción y unificación vienen, generalmente, después. Si se nos permite un símil, los progresos (predicciones y 	unificaciones) se producen durante la investigación, que va precedida necesarianente de una lucha contra alguna dificultad preexistente (generalmente una contradicción aparente dentro del sistema que ya existe). Parte del trabajo del físico teórico es la investigación de las obstrucciones que prometen una posibilidad de «abrirse camino». Como he dicho, estas dificultades se producen generalmente en la experimentación, pero algunas veces son contradicciones entre diferentes partes del cuerpo de doctrina aceptado. Los ejemplos son, desde luego, numerosos.
	
El experimento de Michelson, que condujo a la relatividad especial, y las dificultades de ciertos potenciales de ionización y de ciertas estructuras espectroscópicas, que condujeron a la mecánica cuántica, son ejemplos del primer caso; el conflicto entre la relatividad especial y la teoría de la gravitación de Newton, que condujo a la relatividad general, es un ejemplo del segundo caso, más raro. En cualquiera de ellos, los problemas de física teórica están dados objetivamente y, mientras que los criterios que rigen la explotación de un éxito son, como he indicado antes, principalmente estéticos, aun así la parte del problema y lo que he llamado antes el primitivo «abrirse camino», son hechos firmes y objetivos. Por lo tanto, el objeto de la física teórica fue, casi siempre, enormemente concentrado; en casi todos los tiempos, la mayor parte del esfuerzo de los físicos teóricos estaba concentrado sólo en uno o dos campos muy limitados ---ejemplos de ello son la teoría cuántica en la segunda década de este siglo y principios de la tercera, y las partículas elementales y la estructura del núcleo, a partir de la tercera---.

La situación en las matemáticas es enteramente diferente. Las matemáticas están sujetas a gran número de subdivisiones, que difieren bastante unas de otras en carácter, estilo, objetivos e influencia. Presentan lo más opuesto a la extrema concentración de la física teórica. Un buen físico teórico, hoy día, todavía puede tener un conocimiento viable de la mitad de su materia. Dudo que cualquier matemático actual tenga, a lo sumo, relación con más de un cuarto. Problemas dados «objetivamente» e «importantes», pueden presentarse después de que se haya llegado relativamente lejos en una subdivisión de las matemáticas y si se ha retrocedido seriamente ante una dificultad. Pero, incluso entonces, el matemático es esencialmente libre de tomarlo o dejarlo y dedicarse a otra cosa, mientras que un problema «importante» de física teórica es, por lo general, un conflicto, una contradicción que «debe» resolverse. El matemático tiene gran variedad de campos a que dedicarse, y disfruta de una libertad muy considerable en lo que hace con ellos. Llegamos al punto decisivo: creo que es correcto decir que sus criterios de selección, y también los de éxito, son, principalmente, estéticos. Comprendo que esta afirmación es polémica, y que es imposible «demostrarla» o justificarla hasta el último
extremo sin analizar numerosas pruebas específicas y técnicas. Esto requriría de nuevo un tipo de discusión sumamente técnico; no es la ocasión propicia. Baste  decir que el carácter estético es incluso más relevante que en el ejemplo que antes
he mencionado en el caso de la física teórica. Uno espera que un teorema matemático o una teoría matemática, no sólo describa y aclare de manera sencilla y elegante numerosos casos particulares, a priori dispares. También espera «elegancia» en su aspecto «arquitectónico» y estructural. Facilidad al establecer el problema, gran dificultad en captarlo y en todos los intentos de aproximarse a él, entonces otra vez algún giro muy sorprendente por el cual la aproximación, o alguna parte de la aproximación, llegue a ser fácil, etc. También, si las deducciones son indebidamente largas o complicadas, debe estar implicado algún principio general sencillo, que «explique» las complicaciones y rodeos, reduzca la aparente arbitrariedad  a unas pocas motivaciones simples, que sirvan de guía. Estos criterios son claramente los de todo arte creativo, y la existencia de algún estrato empírico, motivo profano en el origen ---a menudo en un origen muy remoto desarrollado por progresos estéticos y continuado en una multitud de variantes laberínticas---; todo esto está mucho más emparentado con la atmósfera de arte puro y simple que con la de las ciencias empíricas.

Notarán que no he mencionado siquiera una comparación de las matemáticas con las ciencias experimentales o con las descriptivas. Aquí las diferencias de método y de atmósfera general son demasiado obvias.

Creo que es una aproximación relativamente buena a la verdad, la cual es lo bastante complicada como para permitir solamente aproximaciones, que la ideas matemáticas se originan en lo empírico, aunque la genealogía sea, a veces larga y oscura. Pero, una vez concebidas así, el asunto empieza a vivir una vida peculiar propia, y es mejor compararla a lo creativo, gobernado por motivos casi enteramente estéticos, que a cualquier otra cosa y, en particular, a una ciencia empírica. 
Sin embargo, hay otro punto que creo necesario subrayar. Cuando una disciplina matemática se aparta mucho de su fuente empírica, o, aún más, si está durante una segunda y una tercera generación inspirada sólo indirectamente por las ideas que proceden de la «realidad», está amenazada de peligros muy graves. Se vuelve más y más en esteticismo puro, más y más en \textit{l'art pour l'art}. Esto no es necesariamente malo, si el campo está rodeado de materias correlacionadas, las cuales todavía tienen conexiones empíricas más próximas, o si la disciplina está bajo la influencia de hombres con un criterio extraordinariamente bien desarrollado. Pero existe un grave peligro de que la materia sea desarrollada a lo largo de la línea de mínima resistencia, que la corriente, tan alejada de su origen, quede separada en multitud de ramas insignificantes, y que la disciplina se convierta en una masa desorganizada de detalles y complejidades. En otras palabras, a gran distancia de su origen empírico, o después de muchas reproducciones «abstractas», un tema matemático está en peligro de degeneración. Al principio, el estilo es generalmente clásico; cuando presenta signos de transformarse en barroco, aparece la señal de peligro. Sería fácil dar ejemplos, trazar evoluciones específicas del barroco y del barroco exagerado; pero esto, de nuevo, sería demasiado técnico.

En cualquier caso, siempre que se alcance este punto, me parece que el único remedio es el retorno rejuvenecedor a la fuente: la reinyección de ideas más o menos directamente empíricas. Estoy convencido de que ésta es una condición necesaria para conservar el frescor y la vitalidad de la materia, y que esto seguirá siendo igualmente cierto en el futuro.


\end{document}
