\documentclass[a4paper, 12pt]{article}

%%%%%%%%%%%%%%%%%%%%%%Paquetes
\usepackage[spanish]{babel}  
\usepackage[utf8]{inputenc}
\usepackage{tcolorbox}
\usepackage{cmbright}  %%%%%%% El tipo de letra
\usepackage{setspace}
\onehalfspacing  %%%%%%%%%%% Espacio y medio de interlineado
\parskip=1em  %%%%%%%%%%%% Separacion entre parrafos
%%%%%%%%%%%%%%%%%%%%%%



%%%%%%%%%%%%%%%%%
\title{Aplicación del Análisis a las demás Ciencias}
\author{F. Iñiguez}
\date{}
%%%%%%%%%%%%%%%%%

\begin{document}

\begin{tcolorbox}[colback=blue!5!white,colframe=blue!75!black]

\vspace{-1.8cm}
\textbf \maketitle

\end{tcolorbox}

\bigskip



SEÑORES:


 La ciencia, en el presente siglo, por sus numerosos descubrimientos y por la utilidad de sus aplicaciones prácticas, se ha conquistado una autoridad tan sólida como universal. Pero la importancia de la ciencia no consiste tan sólo en los problemas que le pertenecen cuyas soluciones ha encontrado, ni en las necesidades de la vida que ha aliviado o satisfecho: consiste principalmente en la luz vivísima con que ilumina cuestiones, antes oscuras y embrolladas, hoy claras y accesibles para toda inteligencia que no se halle debilitada por la preocupación. Aunque cada una con jurisdicción propia y bien deslindada, la ciencia y la filosofía se compenetran en parte; no es posible a la ciencia, como creen algunos, prescindir de la filosofía; ésta tiene por objeto las primeras nociones de la razón humana, el pensamiento mismo y sus leyes, la esencia de las cosas, los principios supremos del conocimiento y de  la existencia; y como el punto de partida de toda ciencia es un principio del pensamiento puro, anterior y superior a toda experiencia, a la filosofía pertenece señalar la solidez del principio adoptado. Y luego, en el desarrollo sucesivo de las cuestiones que la ciencia estudia, a la filosofía corresponde también apreciar la legitimidad de las inducciones y deducciones, de los razonamientos, en una palabra, que sirven para establecer las verdades sucesivamente eslabonadas. Vemos, pues, que ni en sus comienzos, ni en sus desarrollos, es posible a la ciencia prescindir de las enseñanzas de la filosofía.

A su vez la filosofía no puede separarse de la ciencia, no puede olvidar las afirmaciones de ésta sobre muchos puntos que se relacionan muy estrechamente, ya con la extensión del campo de la metafísica, al cual abre nuevos horizontes, ya con la solución misma de los problemas que ésta estudia, vedándole ciertos caminos, que seguramente la conducirían al absurdo, señalándole otros de resultado cierto más probable. Indicaré muy a la ligera algunos de estos puntos de contacto entre una y otra rama del saber humano.

Penetrando la ciencia en el mundo físico, ha descubierto las leyes mecánicas que rigen los movimientos de las grandes masas y las acciones y reacciones de los elementos atómicos. Analizando detenidamente los fenómenos del mundo material e inorgánico, refiriendo unos como efectos a otros como causas, ha llegado, por síntesis sucesivas, a afirmar que los mencionados fenómenos se reducen todos en definitiva a simples movimientos de un cierto número de elementos materiales, llamados átomos, ponderables unos, regidos por leyes de atracción mutua, imponderables los otros y en continuo estado de recíproca repulsión. Los movimientos de estos átomos, ya de traslación en conjunto, ya de vibración tan sólo, y sus recíprocas influencias, bastan para explicar todos los fenómenos, aun aquellos en los cuales el movimiento es menos perceptible, como el sonido y la luz, la electricidad y el magnetismo, el calor y las acciones químicas. Investigando después cómo se agrupan los átomos para formar los cuerpos y cuál es su naturaleza íntima, ha descubierto que los cuerpos no son masas continuas, que su extensión no está ocupada totalmente por los átomos, sino que éstos se hallan separados unos de otros, mantenidos a distancia por sus acciones mutuas; y en cuanto a la naturaleza de estos elementos materiales, la ciencia afirma, tal es al menos la opinión que reúne hoy mayor suma de autoridad, que los átomos no son otra cosa que simples puntos geométricos, sin extensión, que sirven cada uno de asiento a una fuerza cuyos puntos de aplicación se hallan en los demás. Pero, sea lo que quiera de la extensión de los átomos, lo que importa aquí consignar es que la ciencia enseña que el universo no es una masa continua, sino que se halla formado por la reunión de elementos distintos, resultado que se impone a la filosofía, obligándola a considerar el mundo como limitado en el espacio, puesto que, siendo una suma de elementos, no puede ser infinito en extensión.

Estudiando detenidamente las leyes a que obedecen la propagación y sucesivas transformaciones de la energía, ha descubierto la ciencia que tales transformaciones no se verifican circularmente, restableciéndose en períodos sucesivos el estado inicial, sino que tienen lugar siempre en un mismo sentido, transformándose paulatinamente la energía dinámica en energía calorífica. Y como se demuestra, por procedimientos y métodos cuya exactitud se halla a cubierto de toda crítica, que las leyes físicas son, por todas partes, idénticas, el resultado anterior, aplicado al sistema del mundo, manifiesta de un modo terminante que el universo tiene fin; que del mismo modo que se halla limitado en el espacio lo está también en el tiempo; afirmación que condena al absurdo y a la nada a todo sistema filosófico, cuya base sea la eternidad del universo.

Pero no es la limitación, por decirlo así, de la filosofía la única consecuencia que se deduce de los ejemplos citados: la existencia de los átomos lleva consigo la necesidad de que sean estudiadas su esencia y sus influencias mutuas; la limitación del universo en el tiempo hace surgir los problemas de la creación y del destino futuro de los seres que en ella se comprenden: cuestiones todas que, si por su naturaleza pertenecen a la metafísica, no es menos cierto que se plantean como consecuencia de las últimas afirmaciones científicas.

Los fenómenos materiales, como queda dicho, se reducen todos a movimientos regidos por las leyes de la dinámica. Pero la conciencia propia nos revela con evidencia que en nuestros movimientos y aun en nuestras sensaciones hay algo que depende de nosotros mismos; algo que nos coloca sobre el mundo físico, donde todo sucede de un modo necesario; algo, en fin, que nos permite modificar en parte los movimientos atómicos y disponer de ellos convenientemente para lograr fines elegidos de antemano. Tal resultado nos conduce a afirmar que en nuestras acciones existe un principio distinto de los átomos, un principio que goza de espontaneidad y libertad. La existencia de este principio, del espíritu, da origen a una multitud de cuestiones de todos conocidas, que no he de enunciar ahora; sí he de haceros notar cómo surge aquí también una nueva conjunción entre la filosofía, a quien corresponde estudiar la naturaleza del espíritu, y la ciencia, a cuya jurisdicción pertenecen los fenómenos materiales donde el espíritu interviene.

Podría extenderme más en este asunto, tratado con erudición suma por el reverendo P. Carbonelle\footnote{Les confins de la science et de la philosophie. Bruxelas.}; pero bastan los ejemplos citados para dejar probada suficientemente la afirmación hecha antes, y para que se comprenda también la razón que guía hoy a los filósofos más notables, al buscar en los conocimientos científicos la base para sus teorías. Tan generalmente sentida es actualmente la autoridad de la ciencia, que todos quieren ampararse a su sombra: y se da el caso, por todo extremo curioso, de que los defensores de los mayores absurdos se presentan como los legítimos representantes del progreso científico.

Es, por consiguiente, de importancia suma investigar a qué debe la ciencia la perfección de que procede su autoridad; y a poco que, con tal objeto, se examine su historia, se ve que la fama, justamente conquistada, es consecuencia de la exactitud de los métodos, y principalmente del empleo del cálculo. Es tal la confianza que inspiran las matemáticas como medio de análisis, que se quiere hacer aplicación de ellas a todo género de estudios: ya no se utilizan tan sólo en la astronomía, en la física, en aquellas ciencias, en fin, donde los fenómenos estudiados son completamente mecánicos; aplícanse también a la fisiología, a la psicología, a las ciencias sociales. La consideración de tal universalidad me inclinó a proponer el tema que, aprobado por la Mesa de la sección, he de desarrollar en esta noche, por obligación del cargo con que me honrasteis, y que versa sobre la aplicación del análisis matemático a las demás ciencias. Mi principal objeto es dar ocasión para que aquellos que se dedican a estudios relacionados con tema tan importante, nos ilustren con sus conocimientos. Tened, pues, indulgencia para mi trabajo; corregid sus errores y llenad sus vacíos; así el resultado será perfecto en lo posible, y por mi parte quedaré una vez más obligado a rendiros el tributo de gratitud y reconocimiento de que por tantos motivos os soy ya deudor.

\bigskip

\centerline{***}

Todo el valor de las matemáticas, como medio de investigación científica, consiste principalmente en la seguridad de su razonamiento y en su poder analítico. Por complicadas que sean las relaciones que existen entre las variables diversas que intervienen en un fenómeno, las matemáticas conducen a la inteligencia, con seguridad, hasta las conclusiones más extremas y dan la manera de señalar a cada elemento la parte que en el fenómeno le corresponde. Las matemáticas son además un lenguaje cuya claridad y concisión supera a las de cualquier otro. «No hay expresión retórica, ha dicho un sabio eminente, que pueda compararse en elegancia con una fórmula matemática.» Pero en esta propiedad, a que las matemáticas deben tanta parte de su importancia, consiste también la mayor dificultad de su empleo; porque en las verdades matemáticas no caben aproximaciones, son todo lo que son o no son nada. Si pierden precisión los términos usados, las proposiciones enunciadas dejan de ser ciertas; no siendo posible, por lo tanto, variar en lo más mínimo el significado de las palabras y modos de expresión adoptados por los matemáticos: el olvidar esto conduce frecuentemente a resultados absurdos, o cuando menos inútiles, a los que, engañados por la significación que en el lenguaje ordinario tienen los términos mencionados, no los emplean en su verdadero sentido.

Compréndese ya con lo dicho lo que son las matemáticas como medio de investigación y demostración científicas: un lenguaje claro y preciso, un modo de razonar breve y seguro y un medio poderoso de análisis. Pero la base que les sirve de punto de partida les es extraña por completo, consistiendo ya en una ley experimental, ya en una hipótesis muy probable; y nada darán las matemáticas que no esté contenido en esa ley o esa hipótesis, deduciéndose de aquí con cuánto cuidado debe precederse en la elección del principio fundamental que en una ciencia cualquiera ha de dar entrada al cálculo.

El análisis matemático, dice Moutier\footnote{Moutier. La Thermodynamique. París, 1885, pág. 189.}, es como un buen molino; si ponemos en él buen trigo, obtendremos excelente harina. Por otra parte, el depósito más abundante del mejor trigo valdría bien poco si no dispusiéramos de molinos que lo trasformasen. Lo propio sucede con los resultados de la experiencia, que son aquí el grano; mucho valen de por sí, pero sin el cálculo sería imposible conocer todo lo que en ellos se encierra.

Tratemos ya de comprender de qué manera penetran las matemáticas en las demás ciencias. Un examen, siquiera sea muy rápido, de aquellas que por su precisión y por sus procedimientos parecen a primera vista una rama de las matemáticas puras, será muy provechoso para nuestro objeto, pues a la vez que nos permitirá apreciar en su justo valor la importancia y la índole del procedimiento matemático, nos proporcionará el criterio seguro para juzgar otras ramas del conocimiento a las que tal método pretende aplicarse. La astronomía es el mejor ejemplo que podemos elegir; la física también nos dará alguna enseñanza; examinemos a grandes rasgos la historia de la primera.

\bigskip

\centerline{***}

El espectáculo que los cielos nos ofrecen ha sido siempre demasiado magní\-fico para sustraerse a la atención de los hombres. La idea primitiva acerca del universo no podía, sin embargo, ser más rudimentaria: una inmensa cúpula sobre un plano horizontal. Los viajes hicieron comprender bien pronto que la base de tal cúpula no está tan cerca como acusan las apariencias: el cambio de aspecto del cielo, al trasladarse el observador de un punto a otro bastante distante sobre la superficie de la tierra, manifestó claramente que la bóveda estrellada no se limita a cubrir el horizonte, sino que envuelve por completo a nuestro globo. El paso de un concepto a otro debió ser rápido, verificándose tan pronto como la especie humana ocupó una extensión ya considerable sobre la superficie terrestre, pues basta la contemplación del cielo para hacer tal descubrimiento.

Las necesidades de la agricultura y de la navegación trasformaron a los astrónomos de contempladores en observadores: pronto descubrieron que hay astros aparentemente fijos y astros que ocupan sucesivamente posiciones distintas; observaron los eclipses de sol y de luna y registraron sus fechas; y comparando éstas, al cabo de algún tiempo notaron que tales fenómenos se suceden con sujeción a un orden determinado.

Hasta llegar aquí, no era posible una astronomía con carácter científico; pero una vez conocidos los movimientos del sol y de la luna, la causa de sus eclipses y de las fases de la última, los planetas y sus revoluciones, la esfericidad de la tierra y su tamaño, fué ya llegado el momento de pensar en las leyes de los movimientos celestes.

Pocas noticias nos quedan de los astrónomos primitivos, pero bastan para comprender cómo unos observadores privados de todo medio auxiliar, llegaron con sólo su ingenio a hacer descubrimientos tan importantes. Para observar con  cierta facilidad habían dividido las esfera celeste en constelaciones, mereciendo atención preferente las doce del Zodiaco, dentro de las cuales se verifican siempre los movimientos del sol, la luna y los planetas entonces conocidos.

Los primeros datos numéricos que encontramos en la astronomía antigua son los relativos a la duración del año, corregida sucesivamente hasta llegar casi a la exactitud; la del mes, acomodada al movimiento lunar, y los períodos de larga duración, al cabo de los cuales, restablecidas las posiciones relativas del sol y de la luna, los eclipses se repiten en el mismo orden, y es posible anunciarlos con seguridad suficiente. La semana, período el más antiguo y sin duda el más universal, puesto que en todos los pueblos, desde la China hasta el Atlántico, era la misma, tuvo un origen a la vez astronómico y religioso.

No habían pasado de aquí los conocimientos astronómicos entre los chinos y los caldeos, los indios y los egipcios.

Los griegos consideraron la astronomía como ciencia puramente especulativa: no eran observadores, y sus teorías en general, lejos de ser provechosas, han sido más bien una rémora para el progreso científico.

Hasta la escuela de Alejandría, la astronomía práctica no gozaba de mayor prosperidad que la teórica; el gnomon, varilla o columna vertical, que servía para medir la extensión de la sombra proyectada por el sol al mediodía, era el aparato, único casi, empleado por los astrónomos: los de la mencionada escuela, fundada 300 años antes de Jesucristo, observaban ya sistemáticamente: empleaban aparatos ideados con un fin matemático; fijaban la posición de las estrellas por longitudes y latitudes, y análogamente los puntos, de la tierra, y en fin, poseían una geometría bastante adelantada, y una trigonometría, si no tan perfecta como la actual, lo suficiente para las necesidades de sus cálculos. Con tales medios, pudieron apreciar mejor los detalles de los movimientos celestes, y hacer mediciones que, aun distando mucho de la realidad, permitieron comprender que la distancia que separa a la tierra de las estrellas es inmensamente grande con relación a la que media entre ella y los planetas. Estos nuevos conocimientos hicieron que se modificase el concepto primitivo del universo, pues no era posible ya considerar a los astros todos como fijos sobre una esfera cristalina; los datos adquiridos acerca del movimiento de los planetas permitieron idear un sistema astronómico, el primero que en la historia de la ciencia merece nombre de tal.

La única rama de las matemáticas que entonces tenía algún desarrollo era la geometría, y ella sola intervino en el sistema que lleva, como es sabido, el nombre de Ptolomeo, y data del siglo II de nuestra era. Muy imperfecto ciertamente, lo que era debido en parte a las escuelas filosóficas griegas, según las cuales, el único movimiento posible para los astros es el circular y uniforme, realizaba, sin embargo, un fin científico, en cuanto, por combinaciones de círculos, permitía representar y someter al cálculo los movimientos celestes.

Lo dicho hasta aquí nos permite ya ver cómo por el empleo de las matemáticas, hechos observados aisladamente se agrupan constituyendo teoría y adquieren carácter científico.  Para lograrlo no bastó observar, medir y comparar magnitudes, distancias y tiempos; se necesitó, además, una doble hipótesis, a saber: que la tierra se halla inmóvil en el centro del universo, y que los movimientos de los astros son uniformes y circulares. Ambas eran falsas, pero, así y todo, fueron base de progreso. Porque las hipótesis, cuando están bien establecidas, no olvidando su carácter, son uno de los auxiliares más poderosos de la ciencia. El espíritu humano, dice Laplace\footnote{Oeuvres completes de Laplace. París, 1884. Tomo VI, pág. 420.}, necesita de hipótesis para relacionar entre sí los fenómenos y determinar sus leyes; limitando las hipótesis a este uso, evitando el atribuirles realidad y verificándolas sin cesar por nuevas observaciones, se llega por fin a las verdaderas causas, o a lo menos es posible suplirlas y deducir de los fenómenos observados, los que circunstancias dadas, deben ocasionar.
 
Los árabes extendieron por Occidente los conocimientos de la escuela de Alejandría; perfeccionaron con sus trabajos los datos de la observación; pero no modificaron absolutamente nada de los principios. Con aparatos más delicadamente construidos, y con el trascurso del tiempo, único elemento de que los astrónomos no disponen a su arbitrio, fueron descubriéndose nuevas particularidades de los movimientos planetarios; pero cada nuevo detalle que la observación hacía percibir en el movimiento de un planeta, obligaba generalmente a añadir un círculo más a los muchos que desde el principio se necesitaron para hacer concordar la observación con la teoría. Así sucedió que la acumulación de círculos, y de círculos sobre círculos, llegó a ser tal, que la inteligencia se abrumaba al contemplarlos. Cuéntase que, con tal motivo, dijo en una ocasión nuestro Rey sabio ante su consejo de astrónomos: «Si Dios me hubiese consultado al hacer el mundo, las cosas habrían resultado mejor hechas;» frase notable que revela cuánto desconfiaban los astrónomos de su propia obra, al comparar tanta complicación con la sencillez que caracteriza todas las de la naturaleza.


Dos causas se oponían al progreso científico en la Edad Media: consistía la una en que no preocupaba a los hombres de ciencia la idea de causa, y la otra en el aislamiento en que vivían. Satisfacíanse con poder explicar las particularidades de los fenómenos y no trataban de elevarse al conocimiento de las causas que los engendran; no había comercio de ideas entre los sabios y cada uno se veía en la necesidad de serlo todo, geómetra, físico, astrónomo. Uno y otro obstáculo desaparecen en los comienzos de la edad moderna; ya la inteligencia no se satisface con hechos y quiere conocer sus causas, y por otra parte, la división del trabajo, tan necesaria para todo progreso, queda perfectamente lograda con el establecimiento de las sociedades científicas. Con razón se ha llamado del renacimiento a aquel período notable de la historia; parece que nueva vida anima a la humanidad; los genios se suceden sin interrupción, créanse ciencias hasta entoces ignoradas, o se perfeccionan rápidamente las ya conocidas: el progreso es general.

La astronomía participó también de aquella universal transformación. Ya el cardenal Cusa había resucitado las ideas de los pitagóricos sobre el movimiento de la tierra; pero  la gloria de haber perfeccionado el sistema astronómico se debe a Copérnico. Gracias al talento y a la energía del sabio canónigo de Thorn, la tierra ocupó en el sistema planetario el lugar que le corresponde; su movimiento de rotación explicó con toda sencillez el movimiento diurno de los astros; su traslación en torno del sol, unida a la de los otros planetas explicó las particularidades más raras que éstos ofrecen, las estaciones y retrogradaciones; en fin, la nueva hipótesis, y digo hipótesis, porque nada más era aún la idea del movimiento de la tierra, ofreció una base segura para calcular la relación numérica que existe entre las distancias de los cuerpos del sistema solar, cosa no lograda hasta entonces. El sistema de Copérnico corrigió así una de las inexactitudes fundamentales del sistema de Ptolemeo, la relativa a la inmovilidad de la tierra, pero dejó subsistir la que se refiere al movimiento circular, siendo notable que una inteligencia tan independiente como la de Copérnico, no supiera sustraerse a la influencia, aún dominante, de la filosofía griega, defecto que se nota en muchas partes de su obra.

Comparando ahora los dos sistemas, el de Ptolemeo y el de Copérnico, bajo el punto de vista que se refiere a nuestro tema, podéis ver claramente una de las particularidades que, al principio señalé sobre el poder analítico de las matemáticas. La parte de éstas que interviene en uno y otro sistema, es la misma, la geometría tan sólo; la diferencia de resultados consiste únicamente en la de los puntos de partida. Supuesta la tierra inmóvil, era necesario explicar las particularidades todas del movimiento como fenómenos propios exclusivamente de los planetas; en cuanto a distancias relativas de éstos, las hipótesis admitidas nada encerraban, y las matemáticas nada podían deducir. Admitiendo, al contrario, con Copérnico el doble movimiento de la tierra, se hacía preciso considerar los fenómenos de los movimientos planetarios como reales en parte, y en parte como aparentes, por reflejarse en ellos, en cierto modo, el movimiento de la tierra; además, trasladándose ésta de un extremo a otro de su órbita, daba una línea de apoyo para referir a ella las distancias planetarias. No se conocía aún la extensión de dicha línea; pero sí podían calcularse las relaciones numéricas que existen entre ella y todas sus análogas, y más tarde, sería posible, por una sencilla operación de aritmética, pasar de magnitudes relativas a magnitudes absolutas, tan pronto como el valor del diámetro de la órbita terrestre fuese conocido.

Posteriormente a Copérnico aparece la figura de Tico-Brahe. Su sistema astronómico fué seguramente un retroceso, pues, creyendo establecer otro distinto, no hizo en realidad más que resucitar el mismo de Ptolemeo. Pero en cambio, observador infatigable y mimado de los poderosos, tuvo cuanto se necesitaba para dejar a sus sucesores una preciosa colección de observaciones astronómicas de notable exactitud.

Le heredó Keplero, hombre de ingenio sutil y de constancia imperturbable, aunque de ideas algún tanto extraviadas por ilusiones metafísicas, que le perjudicaron bastante, ocupándole mucho tiempo en trabajos estériles y retrasando sus descubrimientos. Al fin, de una parte el conocimiento de la elipse, hipérbola y parábola, de las secciones cónicas, en una palabra, estudiadas ya por los geómetras griegos, y de otra el estudio minucioso de las observaciones de Tico-Brahe, le condujeron a afirmar que la curva realmente descrita por los planetas en sus revoluciones no es el círculo, sino la elipse, y que el sol no se halla en el centro, sino en uno de los focos de dicha elipse.

Diez y ocho siglos, nada menos, fueron necesarios para que así quedase corregido el segundo error fundamental de los astrónomos de Alejandría, hijo de las ideas dominantes en la filosofía griega. Esto indica cuánto perjudican, cuánto pueden retrasar el adelanto de las ciencias, las preocupaciones de los que a su estudio se dedican.

Para que un filósofo ---dice Laplace\footnote{Lugar citado, pág. 441.}--- sea útil al progreso científico, es necesario que reúna una imaginación profunda, una gran severidad en el razonamiento y en las experiencias, y que se vea atormentado a la vez por el deseo de elevarse a las causas de los fenómenos y por el temor de engañarse al aceptar las que él mismo les asigna, regla muy sabia, cuyo desconocimiento u olvido ha hecho que mueran en el descrédito multitud de especulaciones, que teniendo apariencias de realidad, se ha encontrado, al someterlas al análisis, no ser más que parto de la imaginación.



Abandonados ya los cielos sólidos de los antiguos, preocupaba a los astró\-nomos el conocimiento de la causa que retiene a los planetas en sus órbitas. Copérnico vislumbró ya la fuerza de la gravitación, Keplero llegó a formularla con suficiente claridad, pero su establecimiento definitivo pertenece a Newton.

Anteriormente a él, Descartes había tratado de explicar la causa del movimiento de los planetas, ideando, con tal motivo, su sistema de los torbellinos, obra por mil conceptos extravagante e impropia de la inteligencia de su autor.

Cuando Newton se entregaba a sus trabajos más importantes, las matemáticas habían alcanzado ya gran desarrollo. El álgebra estaba formada, se había descubierto la trigonometría de senos; Descartes había ideado la geometría analítica; eran conocidos los principios del cálculo infinitesimal; Galileo había descubierto las leyes de la caída de los cuerpos y echado los cimientos de la mecánica; Huygens había encontrado las leyes de la transmisión del movimiento, había estudiado la fuerza centrífuga y dado a conocer la teoría del movimiento sobre las curvas; Hoocke había hecho ver que el movimiento de los planetas es el resultado de una fuerza de proyección tangencial respecto de la órbita y otra atractiva dirigida hacia el Sol; Picard, en fin, había medido un arco de meridiano, que permitía calcular con suficiente aproximación el radio terrestre.




Newton imaginó que la fuerza atractiva del sol respecto de los planetas, y de éstos sobre sus satélites, debe ser análoga a la que hace descender los cuerpos en la superficie de la tierra: el cálculo confirmó su teoría, permitiéndole establecer que los cuerpos celestes se hallan dotados de fuerza de atracción recíproca, proporcionales a las masas, y que varían en razón inversa de los cuadrados de las distancias, único principio sobre el que descansa desde entonces toda la mecánica celeste.


Considerando luego las fuerzas de los astros como resultantes de las acciones de sus moléculas, extendió a éstas la ley, por él descubierta, que ha sido después constantemente confirmada, y que es la más general que se conoce.

Descubierta esta ley, las matemáticas se enseñorearon por completo de los espacios celestes: su sola aplicación ha permitido explicar todos los fenómenos del movimiento planetario; hacer descubrimientos notables como el del planeta Neptuno, y dar la demostración experimental del movimiento de rotación de la tierra, que dejó así de ser hipótesis, como sucedió más tarde con el movimiento de traslación, cuando teniendo bastante alcance el anteojo y precisión suficiente la aplicación del mismo a medir ángulos, se descubrió el fenómeno de la aberración.

Los adelantos de la física moderna y el perfeccionamiento de los telescopios han permitido a los astrónomos extender sus investigaciones hasta las estrellas llamadas fijas: el resultado ha sido encontrar en todas partes las mismas leyes, no sólo mecánicas, sino físicas y químicas. Entre los servicios prestados por la física a la astronomía es notable por su trascendencia la explicación dada por la termodinámica de la pérdida de velocidad de la rotación terrestre, debida al choque sobre las costas del agua elevada en las mareas, descubrimiento importante, por cuanto fijando un límite a la antigüedad de la tierra, lo señala también a los partidarios de ciertas teorías, que exigen como indispensable base un tiempo indefinido.

Aunque el principio único de la gravitación universal permite explicar racionalmente los movimientos celestes, sin embargo, su aplicación es penosa, a causa de que cada astro tiene ciertos elementos numéricos, constantes, que es preciso determinar para introducirlos en las fórmulas generales. Las leyes de Kepler establecen relaciones entre algunos de estos elementos, pero las que ligan a la mayoría de ellos son desconocidas. Las teorías cosmogónicas modernas tienden a llenar este vacío, pero la de Laplace fué estéril bajo este punto de vista, circunstancia nada extraña, puesto que recientemente se ha demostrado la falsedad de una de las hipótesis fundamentales. A la teoría de Laplace ha sucedido la de Faye, tan reciente, que los astrónomos no han tenido tiempo aún de deducir de ella consecuencias útiles, limitándose a demostrar que satisface plenamente a las exigencias de la observación.

La teoría de los errores es un descubrimiento importantísimo, que no es posible omitir, cuando se trata de la aplicación de las matemáticas a las ciencias de observación.

Los observadores antiguos ponían todo el esmero posible en la construcción e instalación de sus aparatos, y tomaban luego como exactas sus indicaciones; los modernos no descuidan nada de lo que tanto atendían sus antecesores, pero saben que, apesar de los adelantos de las artes, todo aparato, por perfecto que sea, tiene ligeros defectos, que influyen sobre el resultado de las observaciones con él ejecutadas. Averiguan, pues, qué errores encierra el aparato, determinan la magnitud de los mismos con cuidado escrupuloso, y así pueden siempre corregir la influencia de tales errores en los resultados de la observación. Pero, además de estos errores, llamados constantes, cuya causa es conocida, existen otros debidos a una porción de concausas imposibles de determinar, debiéndose a ellos que una misma operación, ejecutada varias veces, dé resultados distintos; pues bien; los matemáticos han encontrado medios para deducir de un cierto número de resultados afectados de algún error, los valores más probables de los elementos buscados, señalando a la vez el límite del error cometido. Y como en la práctica no se necesitan resultados exactos, sino aproximados, según los casos, se comprende la importancia de una teoría, que, no sólo hace lo que acabo de decir, sino que, señalada de antemano una cantidad como límite del error aceptable en un resultado, indica qué número de observaciones deben hacerse, qué exactitud deben tener los aparatos y en qué orden se deben suceder los trabajos. Todo esto da de sí el método moderno de observación, hijo de la teoría famosa de las probabilidades, que habiendo comenzado por ser nada más que pasatiempo de geómetras, ha llegado a convertirse en una de las ramas más fecundas de las matemáticas.



No son hasta ahora muchas las relaciones descubiertas entre nuestro pequeño mundo solar y el inmenso de las estrellas llamadas fijas; ha podido, sí, saberse que sus elementos químicos son los mismos y que la ley de gravitación existe del mismo modo en todo el universo. Observando los diversos grupos o aglomeraciones de materiales cósmicos que en forma nebulosa aparecen diseminados en el espacio, se ha encontrado que todos ellos pertenecen al sistema de la vía láctea, que no hay, como se creía, varios universos, varias acumulaciones de estrellas análogas a la que encierra nuestro mundo, y que esas nebulosas de luz blanquecina, donde el telescopio no revela astro alguno, son y serán siempre lo mismo que ahora, gases tan sólo, si otro cuerpo celeste no las arrastra y las condensa sobre sí mismo; desengaño grande, para los que en todo quieren ver una serie de evoluciones constantemente renovadas.

Analizando, en fin, las condiciones de habitabilidad de los mundos, y no olvidando que las leyes naturales son las mismas en todo el universo, la ciencia enseña que, entre el conjunto de soles que pueblan la inmensidad, los que son centro de un sistema planetario análogo al nuestro, constituyen la excepción, y que, aun dentro de nuestro mismo sistema, si hay planetas habitados además de la tierra, lo son el menor número. Como se ve, la ciencia ni afirma ni niega la posibilidad de que los cuerpos celestes se hallen habitados; lo que sí establece es que la inmensa mayoría de ellos ni han tenido ni tienen, ni tendrán habitantes, reduciendo así a otros tantos sueños de la imaginación ciertos trabajos, por otra parte muy apreciables, en los cuales sus autores llegan hasta dibujar la forma de los seres organizados que pueblan los astros, y aun a detallar el carácter y las costumbres de las humanidades, según las llaman, que en ellos habitan.

Volviendo ahora la vista al cuadro rápidamente bosquejado del progreso astronómico, encontraremos importantes consecuencias, con relación al tema que nos ocupa. Los astrónomos primitivos recogen multitud de hechos; los de Alejandría imaginan el primer sistema del mundo asentado sobre dos hipótesis falsas, y como conocimientos auxiliares  emplean la geometría y una trigonometría de cuerdas: el sistema, como deductivo, resulta infecundo, a causa de la falsedad de sus principios fundamentales, y lejos de ceñirse los descubrimientos nuevos a las previsiones de los astrónomos, son éstos los que trabajan para hacer al sistema acomodarse a la realidad. Hemos visto a Copérnico corregir uno de los errores de Ptolemeo, dejando subsistir el otro; con esto el sistema se simplifica, pero su fecundidad habría sido bien escasa; hasta Keplero toda ley astronómica es desconocida. Keplero revela al mundo las leyes por él descubiertas, y el sistema planetario llega así a su mayor sencillez geométrica: la base que ofrece es ya real, y la aplicación de las matemáticas puede dar resultados positivos. Los astrónomos, partiendo de las leyes de Keplero, comprenden que se hallan ante un problema mecánico, y Newton les da la ley de la fuerza única que deben considerar. Ocúpanse desde entonces en formular el problema bajo todos sus aspectos y en idear medios teóricos y prácticos para resolverlo; las Matemáticas manifiestan su poder como medio de análisis; en Astronomía geométrica se llega a un grado de perfección ideal, como lo prueban fenómenos con tanta frecuencia predichos y que jamás dejan de presentarse en la hora, minuto y segundo anunciados; en Astronomía mecánica la perfección de los resultados no es menor, pero los procedimientos son penosísimos por el relativo atraso de los medios teóricos.

Resulta, pues, que hasta tanto que, por una observación minuciosa, donde toda causa de error haya sido prevista y separada su influencia, se hayan valuado las cantidades que intervienen en una clase de fenómenos y después se hayan encontrado relaciones constantes entre los diversos elementos que intervienen, y, en fin, se hayan establecido afirmaciones definitivas o hipotéticas, pero de carácter fundamental, la aplicación de las matemáticas resulta casi estéril, y la mayor parte de las veces conduce al absurdo.

Tal es, convertida en regla general, la enseñanza que se saca de la ciencia de observación que se halla a mayor altura. Esto mismo se desprende del estudio histórico de la Física.


\bigskip

\centerline{***}







Mientras los fenómenos físicos se atribuyeron a esencias misteriosas o a causas ocultas, la ciencia no dió un paso, y si algún conocimiento se poseía, hallábase siempre rodeado de nebulosidades metafísicas. Así encontramos la distinción de cuerpos graves y cuerpos ligeros, el horror al vacío, la investigación de la lámpara filosófica, y tantas otras aberraciones, producto de la fantasía. Se fueron descubriendo algunos hechos aislados, que la naturaleza revelaba por feliz coincidencia las más de las veces, las menos por ser inteligentemente consultada. Si alguna parte de la física progresó, siquiera fuese con lentitud, debióse a que no es preciso hacer hipótesis sobre la naturaleza de la causa productora de los fenómenos a que se refería el progreso realizado. Así se ve que los fenómenos antes y con más extensión conocidos, fueron los luminosos de reflexión y refracción, y es que, para explicar los más comunes, basta conocer las leyes geométricas de su producción.

La física no adelanta de una manera visible hasta la edad moderna; comprenden los físicos que en todos los fenómenos es necesario buscar la cantidad, que todos ofrecen algo que exige medida. Y miden la presión atmosférica inventando el barómetro; descubren el fenómeno general de la dilatación de los cuerpos por el calor; miden la velocidad de la luz y del sonido; algo más tarde la intensidad de las atracciones eléctricas y magnéticas, etc., etc., que no he de necesitar, dirigiéndome a un auditorio tan ilustrado como el del Ateneo, enumerar los fenómenos todos. Sírvanme los citados para hacer constar que las matemáticas entran en la física por el número, por la medida de ciertos elementos reconocidos como cantidades. Medidas éstas, la comparación se establece y las leyes físicas van poco a poco revelándose: ya es la de la caída de los cuerpos; ya la que relaciona el volumen de los gases con las presiones que sufren; ya la de dilatación; ya la que relaciona los calores específicos con los pesos atómicos, y así sucesivamente las demás leyes particulares, que poco a poco van también uniéndose para formar las teorías de los diferentes grupos de fenómenos.

El desarrollo de la física ha sido más lento que el de la astronomía, como consecuencia de la dificultad de plantear y de resolver los problemas que le pertenecen. Los astronómicos se plantean bien, y con más o menos dificultad, se resuelven; pero en la física existen fenómenos cuyo planteamiento se ignora aún, y por consiguiente, su resolución.
Sin embargo, en lo que va de siglo, y siempre por la influencia de las matemáticas, el modo de ser de la física se ha trasformado por completo, su campo se ha ensanchado, y su influencia se extiende a todas las otras ciencias naturales, rozándose también en algunos puntos con la filosofía.

La teoría de la luz, la más perfecta de las teorías físicas, no se limita a explicar todos los fenómenos luminosos como consecuencias de un solo principio, sino que proporciona medios auxiliares seguros, para resolver multitud de cuestiones interesantes. ¿Quién sino la óptica, por citar un ejemplo, ha dado el medio de conocer el estado y composición de los astros, y, lo que admira más aún, de medir sus movimientos en dirección de la recta misma que los une con la tierra? Pero la verdadera conquista de nuestro siglo ha sido la termodinámica. El conocimiento de la trasformación del trabajo en calor, y del calor en trabajo, y de las leyes que rigen estas trasformaciones son descubrimientos, que, por su fecundidad en consecuencias y en aplicaciones prácticas, deben colocarse a la altura misma del principio de la gravitación universal. Entre estas consecuencias debo recordar por su importancia la que explica el aumento de duración del día, ocasionado por las mareas, ya citada antes, y la llamada de disipación de la energía, o imposibilidad de la trasformación completa del calor en trabajo. Ambas son de suma trascendencia; la primera nos indica que la tierra ha tenido principio, la segunda que el universo todo tendrá fin, y las dos caen con todo su peso sobre la filosofía materialista. No digan, pues, los materialistas que la ciencia está de su parte, no; renuncien a su apoyo; grande es la pérdida, pero... resígnense, que la resignación es el único remedio de todo lo que no lo tiene.

La termodinámica ha derramado además viva luz sobre senderos antes oscuros para la ciencia; los fenómenos eléctricos se resistían a toda tentativa hecha para explicarlos mecánicamente; pero hoy el problema es abordado con esperanzas de éxito completo, si bien es cierto que al presente no ha sido todavía logrado. Los fenómenos químicos, la teoría de la constitución de los gases, son también ya en gran parte del dominio matemático.

Los fenómenos ópticos y las formas cristalinas que los minerales ofrecen, permiten simplificar bastante el estudio de la mineralogía, y de esperar es que el problema de la agrupación de los elementos moleculares para formar los cuerpos sea resuelto por la aplicación tan sólo de las leyes de la Mecánica.


\bigskip

\centerline{***}

Lo dicho hasta aquí, señores, no es otra cosa que una serie de triunfos obtenidos por las matemáticas. Hemos visto cómo, poco a poco, han ido penetrando en las ciencias físicas y cómo han sido en ellas el principal elemento de progreso. No es, por tanto, de extrañar que, considerando su gran poder analítico, se trate hoy de utilizarlas en aquellos estudios que por su índole especial han sido hasta aquí considerados como incompatibles con todo lo que suponga medidas y relaciones de cantidad. Hoy se hacen esfuerzos para aplicar las matemáticas a las ciencias sociales, a la fisiología, a la psicología misma, no faltando quien pretenda haber descubierto por tal medio la solución exacta de los más arduos problemas sociales y aun teológicos.

Al examinar estas novísimas aplicaciones de las matemáticas, entramos ya en terreno movedizo, agitado constantemente por la lucha; los resultados examinados hasta aquí resisten a toda crítica, por severa que sea; los que ahora trato de analizar no ofrecen el mismo carácter, y algunos son tan poco firmes en sus fundamentos, que pronto se ve en ellos la inexactitud, derrumbándose al soplo del análisis todo el edificio, por grande que sea su apariencia de solidez.




Ciertamente que apenas habrá rama alguna de los conocimientos humanos que no tenga su fase matemática. Desde que en una ciencia encontramos algo que se mide; desde que en ella se enuncian leyes por medio de números, allí han entrado las matemáticas. Pero en el empleo de éstas es necesario precaverse contra multitud de causas de error, porque es muy fácil ser seducido por apariencias de exactitud y de lógica y, olvidando el método propiamente científico, apartarse de la verdadera ciencia, cuyas afirmaciones tienen por base sólida, y a la vez por confirmación, los hechos de la experiencia, para caer en la ciencia ideal, cuyas conclusiones, como dice Berthelot, tienen por principal fundamento las opiniones individuales y la libertad. En la ciencia ideal, o no hay método, o si lo hay es directamente opuesto al verdadero método experimental; lejos de tener en cuenta las relaciones inmediatas de los fenómenos y de seguir la cadena de hierro del determinismo científico, llega por saltos a conclusiones extremas; se dispensa del análisis minucioso de los hechos, condición indispensable de toda inducción legítima; no se cuida de someter sus afirmaciones a la contraprueba de la experiencia, única garantía de toda certeza inductiva; sistematiza sin cesar, transforma hipótesis gratuitas en teorías definitivas, y, en fin, se sale del terreno propio, pretendiendo llegar al conocimiento de la esencia de las cosas y de su propia finalidad. Multitud de producciones de autores modernos podrían servirnos de ejemplo de equivocaciones tamañas; ya un psicólogo, pretendiendo reducir el espíritu a una entidad dinámica, nos dice que es la resultante de las actividades celulares, olvidando que la resultante carece de existencia sustancial y que tan sólo es una creación de los matemáticos; ya es un fisiólogo que quiere valuar la inteligencia por el peso de la sustancia gris del cerebro o por las protuberancias del cráneo; ya un geólogo, quien estableciendo hipótesis sobre hipótesis, llega a afirmar categóricamente la duración de cada período atravesado por la tierra en su evolución sucesiva.

Al tratar, pues, de examinar las aplicaciones de las matemáticas a las ciencias que más se resisten a tal medio de investigación, preciso es que no olvidemos los peligros y que 
tengamos en cuenta las enseñanzas que, en cuanto al procedimiento, se desprenden del estudio de los progresos astronómicos y físicos.

\bigskip


\centerline{***}

Tratemos, en primer lugar, de los fenómenos que nos ofrecen los seres organizados. En los vegetales observamos desde luego fenómenos químicos, fenómenos de organización y fenómenos mecánicos. Examinados estos tres grupos de fenómenos, ningún agente sustancial se revela en ellos distinto de las actividades atómicas. La termodinámica aplicada al estudio de los fenómenos químicos, ha permitido a Berthelot formular multitud de leyes a que tales fenómenos obedecen: los estudios matemáticos encuentran el camino abierto, y de esperar es que llegarán a aclarar tan importante rama de la ciencia, hasta ahora llena de dudas y misterios.

Los fenómenos de organización, por los cuales las moléculas orgánicas, formadas por la acción química, se agrupan y constituyen los elementos de los tejidos, por más que en nada revelan otra cosa que una complicada acción física, se sustraen hasta ahora a toda investigación matemática. Claramente se acusan algunos fenómenos conocidos, como los ósmicos; pero no es fácil aislarlos y señalar la parte que en la nutrición les corresponde. Esta complicación que presentan los fenómenos orgánicos, ofreciendo en conjunto todas las acciones físicas, sin permitir aislar unas de otras, es lo que principalmente los distingue de los fenómenos del mundo inorgánico: en éstos, las fuerzas que actúan son escasas en número y sus leyes bien conocidas, los fenómenos son más sencillos, y es fácil, en general, asignar a cada acción componente la parte que le corresponde en el fenómeno resultante: en los seres orgánicos, por el contrario, las causas son múltiples, los fenómenos se nos presentan en conjunto, no siendo posible hasta ahora su descomposición en otros más elementales, que, aislados, pueden ser sometidos a medida para descubrir sus leyes numéricas. Por todo esto, la Mecánica, que tantas cosas ha explicado, permanece muda ante el problema de la organización de la materia; ni se ve aún el medio de llegar a intentar siquiera dar una explicación matemática de los fenómenos de la vida.

No sucede así afortunadamente con los movimientos mecánicos: de muchos de ellos se conoce la causa, como en el heliotropismo, por ejemplo, y no sería difícil formular sus leyes.

Los mismos fenómenos que en los vegetales encontramos en el mundo animal, y puede aplicárseles sin quitar ni añadir cosa alguna, cuanto de aquellos queda dicho. Pero en los animales encontramos además otros fenómenos que les pertenecen exclusivamente, y son los fenómenos voluntarios. En cuanto diga sobre este punto, me referiré exclusivamente al hombre, por ser en él donde se han hecho los principales estudios que con este punto especial se relacionan, a causa de la gran importancia, tanto fisiológica como psicológica, de los descubrimientos a que tal estudio puede conducir.

En primer lugar, es incuestionable que los movimientos voluntarios indican la existencia de un agente especial, distinto de las actividades atómicas. En los fenómenos por éstas producidos todo está matemáticamente determinado, y basta el conocimiento de las leyes de la dinámica y del estado inicial para explicar todas las fases sucesivas que han de atravesar; pero en los movimientos voluntarios hay algo que se sustrae a tal determinación: la conciencia, cuyo testimonio es aquí irrecusable, nos dice de un modo evidente que somos dueños de ejecutar o no tales o cuales acciones mecánicas, y aun de graduarlas, en conformidad con el fin que deseamos conseguir. Vano ha sido todo cuanto se ha hecho para negar la existencia del espíritu; éste se revela claramente, y aparece como una fuerza consciente y espontánea, doblemente espontánea, puesto que lo es en el momento y en el grado de la acción. Cierto es que su unión con el cuerpo le priva, en parte, de su espontaneidad, en cuanto pone límites a su acción; pero no lo es menos que su naturaleza es distinta de la de los átomos, puesto que las acciones de éstos se hallan siempre perfectamente definidas por ley matemática en función de sus mutuas distancias. Esta naturaleza del espíritu, de sér sustancial, hace que no pueda ser considerado como producto de la actividad del cerebro, o como fuerza viva del mismo, según afirman algunos materialistas; pues aparte de la impropiedad de los términos y del absurdo matemático que algunos encierran, tales modos de ser tienen existencia pasajera, función de tiempo, que se opone a la identidad permanente que se revela en la existencia del espíritu: ni puede ser tampoco, como quiere un ilustre consocio nuestro, fuerza de tensión de la gran molécula a que él asimila el cuerpo humano, puesto que la fuerza de tensión tampoco es nada que tenga individualidad ni permanencia.

Mas apesar de ser tal la naturaleza del espíritu y tan distinta de la propia de los átomos, el modo de obrar sobre éstos y el de recibir las acciones de los mismos, son sin duda alguna dinámicos, y se hallan por consiguiente sujetos a medida, como toda fuerza. No es extraño, por lo mismo, que tantos esfuerzos se hagan por estudiar cuantitativamente las acciones psíquicas, cuyo estudio, aunque en mi concepto no pasa aún de las primeras tentativas, constituye ya una especialidad llamada, como todos sabéis, psico-física. La ciencia en sí no es una quimera, porque cierto es que tanto la acción del espíritu sobre el cuerpo, causa ocasional de los movimientos voluntarios, como la recíproca del cuerpo sobre el espíritu, condición necesaria de las sensaciones, tienen su parte mecánica, susceptible de ser medida, aunque los procedimientos para lograrlo sean aún desconocidos.

Lo que sí se conoce bastante bien es el proceso de los movimientos voluntarios; pero la acción, que en ellos corresponde al espíritu, es tan pequeña, que no puede compararse ni a la parte que, en un tiro de arma de fuego, pertenece al fulminante en la impulsión del proyectil. Mas la pequeñez de la acción no se opone a su existencia, y todos convienen en que consiste en movimientos comunicados a las moléculas cerebrales, de donde se deduce que podría medirse por la cantidad de movimiento producido. No se concibe aún la manera de averiguar su valor, pues se trata de un movimiento que se sustrae a toda apreciación directa y que tampoco puede calcularse indirectamente, puesto que no se conoce magnitud alguna que le sea proporcional, ni relación que lo ligue a otros movimientos susceptibles de ser medidos.

Por estas dificultades, sin duda, los psico-fisiólogos y creo que la denominación invertida sería más propia estudian con preferencia las relaciones cuantitativas entre las impresiones y las sensaciones. Los trabajos por ellos realizados, hechos de una manera sistemática y por procedimientos distintos, no carecen de interés. Sin embargo, examinados atentamente, pronto llama la atención su inexactitud matemática y las contradicciones que revelan los métodos empleados.

Parten los psicómetras del principio de que las sensaciones tienen intensidad y son comparables, sobre todo si son análogas, lo cual equivale a decir que las sensaciones son cantidades, pues esos son precisamente los caracteres de la cantidad. Que las sensaciones son magnitudes, es innegable, puesto que varían de intensidad; en cuanto a ser comparables, no es posible afirmarlo hoy, no habiéndose descubierto aún el medio de medirlas, condición indispensable para que una magnitud se convierta en cantidad. Preciso es recordar, sin embargo, que la medición directa no es esencial para que las cantidades sean expresadas numéricamente; basta su medición indirecta, por medio de otras cantidades con las cuales tengan relación conocida. Así, por ejemplo, en mecánica, no se miden directamente las fuerzas, sino las cantidades de movimiento, o las aceleraciones de los móviles sobre que actúan, según el caso, con cuyas cantidades se admite que son proporcionales. Algo parecido se trata de investigar en la psicofísica, puesto que el objeto que se persigue es averiguar la relación que existe entre la excitación y la sensación. La operación es bastante difícil a causa de la complejidad del fenómeno mismo, puesto que la excitación física se convierte primero en excitación sensorial, ésta en excitación nerviosa, y ésta, en fin, en los procesos centrales que acompañan a la sensación. Cada trasformación sucesiva del fenómeno, cada uno de sus períodos, que sin duda son totalmente mecánicos, se verifica según leyes que se desconocen y que, sin embargo, deben ser necesarias para la exacta apreciación de los fenómenos que se investigan. No se sabe en qué medida la excitación física desarrolla la sensorial; la corriente nerviosa, aunque fenómeno de movimiento, es desconocida también; y desconocida es, por último, la forma del movimiento determinado en los centros nerviosos, el cual es, según todas las probabilidades, apreciado por el espíritu en forma de sensación. El conocimiento matemático de esta última parte del proceso fisiológico, sería por consiguiente de importancia capital, mas no por eso se hallaría resuelto el problema, pues sabida es la parte que en las sensaciones tiene la atención, y la experiencia demuestra que aquélla es una acción dinámica del espíritu sobre las células nerviosas, que las modifica a veces hasta el punto de suspender temporalmente su actividad, resultando, según esto, que tales células, en el momento de la sensación, se hallan sometidas a una doble acción dinámica, cada una de cuyas componentes ha de tener su influencia en el movimiento resultante.



Contribuyen también a dificultar la apreciación exacta de los fenómenos que analizamos, de una parte la influencia del organismo restante sobre el sistema nervioso, y de otra la acción del mundo externo: ambos actúan incesantemente y tienen que modificar de una manera variada el proceso fisiológico. No obstante, si fuese posible sospechar siquiera la ley del fenómeno, en lo que tiene de propio, las matemáticas darían el medio de calcular con suficiente aproximación el resultado, con independencia de dichas acciones extrañas; y digo esto, porque la misma variación continua del estado del organismo y del mundo externo, convierte su influencia en una cosa idéntica a lo que en las observaciones físicas se llama errores accidentales, sujetos a ley conocida y calculables con exactitud suficiente por el método de mínimos cuadrados.

Tales son los caracteres del problema que persigue la psicofísica: trata de medir cuantitativamente una serie de fenómenos, de los cuales sólo es directamente medible el primero, y apreciable por la conciencia el último. Dos son los principios que sirven de fundamento a los métodos de medición empleados: fijados los límites inferior y superior, entre los cuales varía la percepción de las sensaciones y de las variaciones de las mismas, y que llaman respectivamente umbral y altura de la excitación, establecen las dos proposiciones siguientes, según Wundt, que resume este género de trabajos\footnote{Psychologie Physiologiquie. París, 1886.}:

\begin{enumerate}

\item  La sensibilidad a la excitación es proporcional al valor recíproco de los umbrales de la excitación.

\item  La receptividad a la excitación es proporcional al valor directo de la altura de la excitación.

\end{enumerate}

Estas dos leyes son de carácter empírico, y, además, carecen de sentido matemático, porque mientras no sean definidas numéricamente de una manera precisa la sensibilidad y la receptividad a la excitación, no es exacto ni absurdo, sino que carece de significado decir que son proporcionales con otras cantidades.

Sobre tan débil base descansan las fórmulas matemáticas que se emplean en las operaciones psicométricas. La determinación de las cantidades que se trata de comparar se hace con esmero indudablemente, y en cuanto a esto, preciso es convenir en que los psicómetras están en el buen camino: las proposiciones que hoy sientan podrán no tener valor, pero sus experiencias no serán perdidas para el día en que se llegue a encontrar el verdadero método que ha de conducir al establecimiento de las relaciones dinámicas que existen entre el mundo físico y el psíquico. Por ahora las propiedades descubiertas tienen el mismo defecto que las fundamentales, y aun cuando con el tiempo se llegase a demostrar que son verdaderas, no por eso tendrían hoy valor científico; serían tan sólo felices conjeturas, algo parecido a las ideas de los pitagóricos sobre el sistema del mundo. Así, por ejemplo, la ley de Weber, que establece que «la energía de la excitación debe crecer en progresión geométrica para que la energía de la sensación crezca en progresión aritmética» será cierta o no lo será; pero cuando no se ha demostrado que la función que liga los valores de la excitación y de la sensación es de forma logarítmica, la ley de Weber no tiene
sentido. Este defecto se nota constantemente en la psico-física, siempre que emplea la palabra proporcional: así, observando los psicómetras que, cuanto mayor es lo que llaman umbral diferencial, menor es la sensibilidad diferencial, afirman que ambas son inversamente proporcionales, olvidando que no basta que, cuando aumenta una cantidad, disminuya otra, para establecer entre ellas proporcionalidad, sino que es preciso que tales aumentos y disminuciones se verifiquen según leyes determinadas.

En fin, la psico-física, en su aspecto matemático es, no una ciencia, sino un género de estudios que comienza, y que comienza bien: está en el período de investigación de números, llega a vislumbrar relaciones, pero no todavía a establecerlas con exactitud, enunciando leyes; dista bastante de formular hipótesis fundamentales y de constituir verdaderas teorías.


\bigskip

\centerline{***}

Mucho se roza con el asunto que por el momento nos ocupa el orden de estudios que tiene por objeto establecer relaciones entre el desarrollo del cerebro y el de las facultades del espíritu. Quieren los que a ello se dedican encontrar correspondencia entre uno y otro, y para lograrlo se fijan ya en las protuberancias del cráneo, supuestas en correspondencia con desarrollos locales del cerebro, ya en el peso de la masa encefálica, y principalmente de la sustancia gris. Lo contradictorio de los resultados obtenidos prueba que las hipótesis de que parten no tienen suficiente grado de exactitud. Quizás consiste en que la perfección del cerebro no se halla en su volumen ni en su cantidad de masa, sino en el número de células diversas que contiene. Destinado a trasmitir al espíritu las impresiones o excitaciones externas y dar así pábulo a la formación de ideas, es claro que cuanto mayor sea el número de sus células con caracteres dinámicos propios y diversos; cuanto mayor sea su diferenciación; en una palabra, cuanto más completa sea en él la división del  trabajo, con más facilidad y precisión y en mayor número aparecerán las ideas.

El cerebro así considerado aparecería como un organismo, según la expresión de nuestro respetable presidente, cuyo funcionamiento sería tanto más perfecto, cuanto mayor fuese el número y variedad de sus órganos. Imaginemos un instrumento músico, un arpa, por ejemplo, provista de cuerdas bastantes para dar todas las notas principales y sus intermedias en la extensión necesaria, y comparémosla con otra, en la cual no habiendo el número suficiente de cuerdas, cada una de éstas se ve obligada a dar notas distintas, variando su tensión por medio de pedales convenientemente colocados; claro es que la primera será más perfecta que la segunda, bajo el punto de vista de la riqueza y exactitud de las notas y de la duración de sus cuerdas, puesto que todas ellas conservan constante su tensión, al paso que en la segunda será muy difícil el arreglo y manejo de los pedales para obtener con precisión la nota deseada, y por otra parte, el frecuente cambio de tensión en cada cuerda hará cambiar pronto su grado de elasticidad, y como resultado, el arpa quedará desafinada. Lo propio sucederá en el cerebro; si cuenta con células correspondientes a un número considerable de variaciones de la excitación, las ideas aparecerán con prontitud y claridad y sin considerable fatiga, al paso que si las células no son en número suficiente, las ideas aparecerán oscuras, necesitarán un tiempo considerable para su formación y una atención enérgica, que variando la tensión, digámoslo así, de los elementos del cerebro, agotará pronto su energía, trayendo el cansancio y la consiguiente necesidad de reposo. Quizás este modo de ver explicaría también por qué las impresiones violentas producen perturbaciones mentales, puesto que el efecto de tales impresiones sería alterar las energías específicas de las células, produciendo algo parecido a desafinaciones de un instrumento músico, imposibles a veces de corregir, susceptibles de arreglo en otros casos, con más o menos trabajo. Y siendo así, no sería extraño que la autopsia nada descubriese en el cerebro de tal modo alterado.

Pero comprendo, señores, que todo esto no es sino señalar una solución más que puede tener el problema, hacer hipótesis, dejar el campo a la imaginación, pero no afirmar la solución verdadera, aduciendo los razonamientos teóricos que la establecen y las experiencias que la confirman.

Las ideas modernas acerca de los fenómenos físicos, han hecho cambiar de fase a una cuestión considerada siempre como misteriosa de un orden especial, la de la acción del alma sobre el cuerpo. Hoy que ya está abandonada la idea de la materia como cosa distinta de la fuerza, considerado el espíritu como una de éstas, si bien consciente y espontánea, el modo de ser de su acción sobre el organismo no es un misterio de orden distinto que el de la acción de las demás fuerzas unas sobre otras. Es, sin embargo, curioso explicarse de algún modo cómo y por qué un organismo sano responde con exactitud a los mandatos de la voluntad, siendo así que el espíritu no conoce el punto especial del sistema nervioso que debe ser excitado para lograr el fin apetecido. Demos para conseguirlo un poco de libertad a la imaginación, que justo es indemnizarle de algún modo el abandono y como prisión constante a que la venimos sujetando.

Imaginemos, pues, que la energía dinámica del espíritu no se aplica directamente a puntos especiales del cerebro, sino que su acción consiste en producir en el éter movimientos vibratorios de ritmo distinto para cada manifestación de un acto imperativo de la voluntad: sucederá entonces que, teniendo ritmo propio las células cerebrales, entrarán en vibración tan sólo aquellas que lo tengan idéntico al determinado por la acción dinámica del espíritu, produciendo por consecuencia el fenómeno fisiológico deseado, al paso que las demás se limitarán a trasmitir o anular el oleaje etéreo, como las cuerdas y demás objetos vibrantes repiten los sonidos propios, permaneciendo unidos bajo la acción de otro cualquiera, y como sucede también y de un modo idéntico en los fenómenos luminosos. Y si este oleaje etéreo se trasmite al medio ambiente que rodea al individuo, ¿no es cierto que encontrarían en ello solución satisfactoria multitud de fenómenos, como las sugestiones y otros varios del mismo orden? Conveniente sería poder probar que estas corrientes existen, y encontrar en esa especie de comunicación de dos espíritus algo muy parecido a lo que sucede cuando las ondas sonoras hacen vibrar la placa de un teléfono, y como consecuencia desarrollan una corriente eléctrica que, a su vez, hace vibrar la placa de otro teléfono, restituyendo al aire las ondas que de él recibiera el teléfono primero. Emitida esta hipótesis, confinemos de nuevo a la imaginación en su forzado retiro, y veamos cómo las matemáticas han entrado también en esta cuestión ardua de la acción del espíritu.



La idea y su desarrollo se deben al geómetra Bousinesq. Trata de conciliar el determinismo mecánico con el libre albedrío, y para ello se sirve de una teoría ingeniosísima, aunque no tan verdadera como ingeniosa, en mi concepto. Observa este geómetra que un carácter de los elementos moleculares en los organismos es su instabilidad química, y de consiguiente la facilidad con que ceden a las fuerzas que los solicitan. Deduce de aquí que los movimientos de tales moléculas han de ser sobre curvas variadas, las cuales tendrán multitud de elementos comunes, de elementos de osculación, según la palabra consagrada: estudia en seguida el movimiento de un punto sobre curvas de tales condiciones y encuentra que, en circunstancias especiales, basta una fuerza nula para lanzar al móvil por una de las curvas osculatrices en el punto donde se halla aquél. Y como evidentemente, entre todos los órganos, el cerebro es el que posee en grado máximo la propiedad mencionada, Bousinesq deduce que allí deben presentarse constantemente los puntos singulares por él estudiados, ofreciendo así, de un modo continuo ocasión a la voluntad para ejercer su acción, sin alterar por eso en lo más mínimo la energía mecánica del sistema que forman los elementos materiales del organismo. Examinada con detenimiento esta ingeniosa idea, se encuentra, de una parte, que el movimiento de las moléculas en esos puntos singulares de sus trayectorias es únicamente indeterminado, y de otra parte, que la movilidad misma de las moléculas cerebrales ha de oponerse a que lleguen a tales puntos con velocidad nula ---tal es la condición exigida--- dadas las múltiples influencias que sufren.





Mas, aunque no sea del todo cierta la teoría de Bousinesq, despréndese de su estudio que bastará en muchos casos una fuerza, ya que no nula, la cual nada haría, infinitamente pequeña, para producir un movimiento inicial en determinado sentido, el cual movimiento, por acciones sucesivas de los órganos, y por descargas, digámoslo así, de la energía potencial acumulada, llegará a ocasionar por fin efectos mecánicos considerables. Lo cual, además de muchas otras cosas, nos explicaría satisfactoriamente por qué pequeñísimas dosis de un medicamento producen la curación de una enfermedad, según la experiencia asegura.

\bigskip

\centerline{***}

También a la Medicina se han aplicado las matemáticas, y entre nosotros tenemos una importante producción de este género, si no de Medicina, en el sentido estricto de la palabra, al menos de una de sus ciencias fundamentales. Ya sabéis todos, con estos ligeros datos, que me refiero a la obra de un autor ilustradísimo y estimado por todos nosotros, titulada {\it Plan de reforma de la Patología general}. Tal vez atendiendo al objeto especial de estudio, donde en dicha obra se hacen aplicaciones matemáticas, hubiera debido examinarla al tratar de la Fisiología; pero la especialidad de la obra me inclinó a considerarla separadamente.

Intenta su autor dar una noción mecánica de la vida, e investiga la ecuación matemática de la misma. Los peligros que antes indiqué, a que se hallan expuestas empresas tales cuando no ha llegado el momento oportuno de acometerlas, partiendo de base firme, se revelan perfectamente en la obra mencionada; su autor, apesar de su erudición vastísima y de su clara inteligencia, no ha logrado salvarlos. No me ocuparé en examinar los primeros desarrollos matemáticos que en la repetida obra se encuentran; quiero analizar tan sólo la ecuación formulada como expresión matemática de la vida.

Establécese como principio que la vida es una resultante de dos energías distintas; la energía cósmica y la energía individual, en cuya suposición hay ya un error capital, como luego veremos; admitámosla, sin embargo, por el momento. Analízanse en seguida los modos diversos, según los cuales pueden combinarse ambas energías, para dar la expresión de la resultante, y procediendo por exclusión, se llega a establecer que no hay forma admisible más que la de producto; en consecuencia, llamando $V$ a la vida, $I$ a la energía indivividual y $C$ a la cósmica, la ecuación obtenida es $V=IC$. Pero no se ha tenido en cuenta que, en la composición de las fuerzas, no es dado atender únicamente a la intensidad de las mismas, sino que es preciso considerar además su dirección y su punto de aplicación. El problema tiene, por consiguiente, dos partes, ambas necesarias, puesto que la magnitud y la dirección de las fuerzas son inseparables en todo problema de composición y descomposición de aquéllas. Pero hay además otra circunstancia importantísima, y es que sólo en casos muy especiales, que la Mecánica establece con toda claridad, tienen las fuerzas resultante única: en el caso general, la mecánica demuestra que no es posible reducir las fuerzas de un sistema a una sola resultante, sino a dos situadas en planos distintos, siendo también muy digno de notarse que en ningún caso la resultante tiene un valor igual al producto de las intensidades de las componentes.

De todo esto se deduce ya que es inadmisible la ecuación mencionada como expresión matemática de la vida.

El error fundamental a que me refería antes consiste en haber considerado el problema que se trata de resolver como un problema de estática. Y, aun considerado así, el camino que es necesario seguir para resolverlo, es muy otro que el seguido por el autor; porque un cuerpo vivo no es en modo alguno lo que se llama en mecánica un sistema rígido, en cuyo caso, su estudio se reduciría únicamente a composición y descomposición de fuerzas; a lo más, sería un sólido invariable, y para establecer las condiciones de su equilibrio, sería preciso acudir al llamado principio de las velocidades virtuales, siendo importante también no olvidar que no se hallaría una sola, sino seis ecuaciones como expresión matemática de la vida, prescindiendo de las muy numerosas que serían necesarias para expresar la invariabilidad del sistema considerado. Pero evidentemente, un cuerpo vivo no es, en modo alguno, un sólido invariable; es un sistema de elementos materiales en movimiento constante, y por consiguiente, para establecer las condiciones de su existencia sería necesario conocer antes las leyes a que tales móviles obedecen, bajo todas las acciones que pueden solicitarlos. Mas considérese que el sistema solar, reducido a número muy escaso de elementos, de condiciones dinámicas perfectamente conocidas, conduce a ecuaciones tales, que sólo con enorme trabajo y sirviéndose de mil recursos, pueden ser resueltas, y se comprenderá la complicación del estudio de un sér vivo así considerado. Por lo que hoy puede sospecharse, el estudio matemático de la vida habrá de partir de las leyes establecidas en la termodinámica. Algo se ha hecho ya en este sentido, pero es aún muchísimo lo que falta para establecer una teoría algún tanto completa.

Para terminar este examen sólo me resta decir que, afortunadamente, la parte matemática es un detalle tan sólo en la obra que nos ha ocupado: la abundante doctrina que contiene, nada pierde cuando su exposición se despoja de todo aparato matemático. El trabajo no será, por consiguiente, inútil para la ciencia de nuestro país, sino todo lo contrario, y no podía suceder otra cosa, dadas las condiciones de su autor.

\bigskip

\centerline{***}


Avancemos un poco más, y tratemos de examinar las aplicaciones de las matemáticas a los problemas sociales. Dos dificultades principales tienen las cuestiones de este género para ser tratadas por medio de las matemáticas, dado que la aplicación de éstas, como tantas veces queda repetido, exige el conocimiento exacto de las leyes, conforme a las cuales varían las cantidades que han de estudiarse. La primera dificultad consiste en que intervienen muchos factores desconocidos, o cuya influencia no es de forma matemática determinada; la segunda depende de que en asuntos tales entra como factor importante la voluntad del hombre, con poder suficiente para alterar en un momento dado, por miras de cualquier género, los resultados que lógicamente debieran esperarse.

La primera dificultad, aunque grande, es más fácil de vencer que la segunda, en multitud de circunstancias. No son éstas las que se refieren a causas que se presentan de un modo repentino e inesperado, y que, favorables o adversas, alteran por completo la vida de un pueblo, dando al traste con todo cálculo establecido; son las que obran de un modo continuo, y, por lo mismo, aunque desconocidas, sus efectos pueden considerarse como otros tantos errores accidentales, y ser calculados con aproximación. La teoría de las probabilidades es aquí el recurso supremo, y siempre que su aplicación se haga conforme a lo que los fundamentos de la teoría exigen, no cabe duda que los resultados serán confirmados por la experiencia. Por eso se hace uso de las probabilidades en multitud de cuestiones, como los seguros de todo género, las leyes de población de los estados, etc., etc.

La exactitud en la resolución matemática de todas estas cuestiones depende del conocimiento que se tenga de la ley que en su acción siguen las causas desconocidas; porque la experiencia enseña, siempre que pueden compararse resultados de igual género y en número suficiente, que las discrepancias que algunos ofrecen, respecto de los que son considerados como regulares, no se hallan esparcidas acá y allá, sin orden ni concierto, sino que sus valores guardan relación con el número de veces que las mismas se presentan. De aquí la posibilidad de estudiar la ley de producción de todos los casos posibles, y, una vez encontrada, establecer la fórmula matemática que ha de servir de fundamento seguro a toda clase de cálculos relativos al asunto estudiado. Fijémonos, como ejemplo, en los seguros sobre la vida: claro es que si una Compañía estableciese un número muy escaso de seguros sobre la vida de otros tantos individuos, la operación se reduciría a un verdadero juego de azar, donde saldría ganando aquel a quien la suerte quisiera favorecer. Pero supongamos que la misma Compañía hace seguros en número muy considerable: entonces ya cabe afirmar que los resultados han de acomodarse a lo previsto, si la ley de mortalidad es conocida en el país donde la Compañía funciona. Porque la vida de un individuo, en relación con la edad del mismo, puede ser larga o corta, sin que sobre ella pueda hacerse, en general, cálculo alguno acertado: pero, dado un número considerable de individuos, es seguro que entre todos ofrecerán todos los casos posibles, en el orden y número que la ley de mortalidad afirma.

Esta condición de las cuestiones que ahora examinamos, de no poder ser sometidas a regla en cada caso particular, sino en el conjunto de todos, si es posible, es condición que se encuentra casi siempre, y su olvido ha conducido a muchos a muy extrañas consecuencias.

Han sido siempre estudios matemático-sociales muy preferidos, los que se refieren a la población de las naciones. Intervienen en ellos multitud de factores, cuya influencia es distinta y muy variadas las leyes de su acción: sin embargo, las matemáticas, partiendo de las leyes de nacimiento y mortalidad, han llegado a establecer la ecuación que representa la ley a que las variaciones de la población obedecen, ecuación que puede servir de guía en muchos casos, si bien en otros permanece muda, consecuencia del pecado original de empirismo que la afecta. Una ley notable y célebre que establece la relación que existe entre la población y los medios de subsistencia es la tan conocida de Malthus, cuya adopción dió lugar al sistema especial llamado maltusismo: esta ley es también empírica, y sólo las estadísticas pueden probar si es cierta o no. Sin embargo, es preciso notar que los inventos se suceden sin interrupción y son cosas que no pueden preverse; y como multitud de ellos dan por resultado el descubrimiento de medios nuevos de vida, quita esto mucha importancia a la ley citada.

Y ya que tratamos de la ley de población en función progresiva del tiempo, natural es que pensemos en la misma ley en función regresiva de la misma variable. Establecida la 
ecuación que rige el desarrollo progresivo de la humanidad, es de importancia preguntarse cuánto tiempo sería necesario para que, con el coeficiente actual de crecimiento, una pareja primitiva hubiera dado la población actual del mundo. La ecuación está fácilmente resuelta, pues no es otra que la que sirve en las cuestiones de interés compuesto y sus análogas, y llama mucho la atención la escasísima cifra que da por resultado, al compararla con los miles y miles de años que los geólogos atribuyen a la humanidad.

La cuestión de la riqueza es también una de las más especialmente estudiadas por los matemáticos: entre las obras dedicadas a este asunto, llama la atención por ser puramente matemática, la de Cournot titulada: {\it Teoría de las riquezas}. En ella se encuentran reducidas a fórmulas las cuestiones más importantes; pero muy luego se echa de ver que, dentro de aquel aparato matemático, generalmente lógico en el razonamiento, existe un gran defecto, y es que las funciones son puramente simbólicas y difícilmente se podrá hallar en cada caso su verdadera forma algebraica, siempre a causa de la complejidad de las leyes que relacionan los diversos elementos, del número de éstos, de ser muchos desconocidos o surgir inesperadamente, y de la parte que en estas cuestiones tiene la voluntad que no se sujeta a ley. ¿Porque, quién ignora las verdaderas locuras a que la competencia conduce muchas veces a las empresas y a los particulares? ¿A qué no dan lugar la mala fe y las pasiones, cuando se unen a la posibilidad de satisfacerlas?

Un ensayo notable de aplicación de las matemáticas a las ciencias sociales y a otros géneros de conocimientos, ha sido hecho recientemente, y algunas de sus partes han sido ya discutidas en esta cátedra; me refiero a la nueva ciencia titulada {\it Genética}. Basta hojear la obra del mismo nombre para comprender que su autor es muy versado en las matemáticas puras, y que conoce a fondo las cuestiones a que intenta aplicar las leyes del cálculo. Pero muy pronto se ve también la influencia de la metafísica, que, cuando obra fuera del terreno propio, tiene la virtud de esterilizar todo estudio. Las consideraciones metafísicas imperan principalmente en los fundamentos de la Genética, olvidando que no es esa la manera de establecer una ciencia matemática aplicada: giros especiales del discurso hacen aparecer como nuevos inventos cosas que hace mucho tiempo se conocen y se aplican en la ciencia; en fin, en el paso de la Mecánica a la Genética se encuentra, en lugar de un razonamiento lógico, el dominio de ideas preconcebidas, defecto que se nota en muchas partes de la obra.

Esta opinión que se forma tan pronto como se leen las primeras páginas de la {\it Genética}, se corrobora luego al examinar las llamadas leyes objetivas genéticas, donde lo débil y aun lo ilógico y poco matemático del razonamiento y la inexactitud de la base fundamental, hacen ver, más bien que teoremas deducidos, conceptos a priori matemáticamente formulados. El descubrimiento de lo que la Genética llama la LEY, al cual se atribuye una importancia tal, que se consigna la fecha del descubrimiento, considerándolo de origen divino, es, en mi concepto, una ilusión no más de la especial metafísica que domina en la nueva ciencia.

Siendo tal la debilidad e inexactitud de los principios fundamentales de la Genética, las aplicaciones que vienen luego no pueden tener valor matemático. Cuando se hacen trabajos de este género, es preciso no perder de vista que las matemáticas, al ser aplicadas a otras ciencias, no lo son sino en cuanto las leyes de relación de las cantidades que han de estudiarse se hallan comprobadas prácticamente. Las matemáticas puras estudian todas las formas posibles de las funciones, sus relaciones y propiedades: si estudiando después prácticamente las cantidades que intervienen en un fenómeno de orden cualquiera, se encuentra que la función que liga unos a otros los elementos que en él intervienen es de forma conocida, aplicando cuanto la teoría tiene establecido, se deducirán multitud de consecuencias que, traducidas al lenguaje ordinario, harán conocer otras tantas particularidades del fenómeno estudiado. Si dicha función es de forma nueva, los matemáticos se encargarán de hacer su estudio teórico, y las consecuencias que vayan deduciendo serán sucesivamente aplicadas, como en el caso anterior, a la cuestión práctica. Pero en uno y otro caso, ha sido preciso conocer de antemano por la observación atenta del fenómeno, una relación entre los elementos cuantitativos que en él intervienen. Por olvidar estas reglas que la experiencia ha elevado a la categoría de leyes, la Genética no resulta una obra verdaderamente matemática: es, sí, un poderoso alarde de ingenio, y los aficionados al género de estudios que en ella se hacen encontrarán con su lectura ocasión de ejercitarse con fruto en importantes investigaciones.

\bigskip

\centerline{***}


No puede decirse que haya verdaderas aplicaciones de las matemáticas a la historia, si bien ésta recibe de ellas importantes servicios, por ejemplo, en la depuración de ciertas fechas dudosas. Pero de la aplicación de la teoría de las probabilidades a determinar el grado de confianza que merecen los hechos referidos por los historiadores, se deducen importantes consecuencias, y no puedo ceder a la tentación de dedicarle unas líneas de mi trabajo.

Liagre, en su {\it Cálculo de Probabilidades}\footnote{Liagre. {\it Calcul des probalités}, Bruxelas, 1879, pág. 66.}, se propone como ejercicio la cuestión siguiente: Supongamos que un hecho ha sido trasmitido sucesivamente por 20 personas, de modo que la primera lo haya referido a la segunda, ésta a la tercera, y así de las demás. Admitiendo que la veracidad de cada relación del hecho sea 9/10, es decir, que de cada diez circunstancias se altere la exactitud de una sola, lo cual no es suponer demasiado, el cálculo nos dice que se puede apostar ocho contra uno a que el hecho, tal como se refiere definitivamente, no es cierto. Dedúcese de aquí en primer término cuán grande deberá ser la escrupulosidad del historiador en acomodarse a la verdad de los hechos que refiere, sin permitirse alterar sus detalles por ningún género de consideraciones. También se deduce una consecuencia bien triste, y es cuánta razón hay en muchas ocasiones para desconfiar de la historia, viendo las contradicciones que manifiestan hechos de importancia referidos por historiadores distintos, reconocidos como suficientemente serios y veraces. El mismo ejemplo nos indica también cuan sabiamente obran las autoridades al no permitir que se altere en lo más mínimo el texto de documentos por cualquier concepto importantes.

\bigskip

\centerline{***}


Del rápido examen que acabo de hacer de las principales aplicaciones de las matemáticas a los estudios científicos, resulta que aquellas ciencias donde los fenómenos son bastante conocidos para poder establecer relaciones cuantitativas entre sus elementos, dan fácil entrada al cálculo, bajo cuya influencia es rápido el progreso, explicándose todas las particularidades observadas, y descubriéndose muchas otras que ni se sospechaba existiesen. En esta clase de ciencias, la forma de las funciones es perfectamente conocida, y de ahí la utilidad y fecundidad del análisis matemático aplicado a ellas.

En cambio, las ciencias cuyos fenómenos cuantitativos son tales que sus elementos se han resistido hasta aquí a toda medida, no han podido dar entrada al cálculo; cuando se ha tratado de dar forma matemática a su estudio, las funciones establecidas se han reducido a puros símbolos, sin indicar siquiera la posibilidad de conocer su forma, y por consiguiente, estériles para el progreso de la ciencia.

Cuando se dice, como Newton, los cuerpos se atraen proporcionalmente a las masas y en razón inversa de los cuadrados de sus distancias, el punto de partida queda establecido con toda claridad, las consecuencias no tardan en venir, y las fórmulas encontradas se aplican a todos los casos, sin más que poner en lugar de los símbolos generales los valores particulares correspondientes, dando así lugar al conocimiento de otras tantas leyes de los fenómenos estudiados. Cuando sólo se dice que una cantidad es función de otra, esto no significa más sino que entre ambas existe dependencia, susceptible quizás de expresarse matemáticamente; por consiguiente, que debe buscarse la forma de la función, midiendo 
las cantidades y comparando sus variaciones, hasta poder afirmar o, al menos, presumir la ley que las une, y entonces, sólo entonces, el procedimiento matemático dará resultados positivos.

\bigskip

\centerline{***}


Voy a terminar, señores, repitiéndoos el ruego que al principio os hice: yo sé muy bien todo lo deficiente que es mi trabajo, y comprendo cuánto pueden contribuir a su desarrollo aquellos que entre nosotros se dedican al cultivo de ramas especiales de la ciencia. Al suplicarles que lo hagan, creo interpretar los deseos de todo el Ateneo; esperemos, pues, confiadamente; la aplicación de las matemáticas a las demás ciencias encontrará expositores más elocuentes que el que esta noche ha tenido que tratar tema tan interesante; entonces las matemáticas brillarán con todo el esplendor a que por su importancia y por su influencia universal tienen derecho.




\end{document}