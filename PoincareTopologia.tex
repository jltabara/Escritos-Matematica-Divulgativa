\documentclass[a4paper, 12pt]{article}

%%%%%%%%%%%%%%%%%%%%%%Paquetes
\usepackage[spanish]{babel}  
\usepackage[utf8]{inputenc}
\usepackage{tcolorbox}
\usepackage{cmbright}  %%%%%%% El tipo de letra
\usepackage{setspace}
\onehalfspacing  %%%%%%%%%%% Espacio y medio de interlineado
\parskip=1em  %%%%%%%%%%%% Separacion entre parrafos
%%%%%%%%%%%%%%%%%%%%%%



%%%%%%%%%%%%%%%%%
\title{Poincaré y la Topología}
\author{P. Alexandrov}
\date{}
%%%%%%%%%%%%%%%%%

\begin{document}

\begin{tcolorbox}[colback=blue!5!white,colframe=blue!75!black]

\vspace{-1.8cm}
\textbf \maketitle

\end{tcolorbox}

\bigskip


A la pregunta sobre cuál es la relación de Poincaré con la topología puede contestarse con una sola frase: él la fundó; pero también se puede contestar con un ciclo de conferencias en las que se expongan, más o menos detalladamente, los resultados topológicos obtenidos por Poincaré. Según la primera de estas dos aproximaciones a mi tarea, puedo considerarla completamente resuelta. Para la segunda aproximación, naturalmente, no tengo tiempo. Es necesario entonces buscar  una solución de compromiso, intermedia, que será fallida como todas las soluciones de compromiso, y en todo caso más próxima a la primera variante que a la segunda. Esta solución sólo puede constituir un intento de renovar y refrescar la emoción y admiración por el acto de grandiosa creación científica realizado por el 
gran geómetra francés, en una área cuya influencia sobre todo el conocimiento matemático no sólo superó todas las predicciones de sus contemporáneos, sino que 
continúa aumentando año a año.

Poincaré vivió en la época romántica de la historia de la ciencia matemática, cuando por primera vez
(por él mismo y por F. Klein) fue demostrada la consistencia de la geometría no euclideana, como consecuencia de lo cual nuestra visión de la geometría y del 
concepto mismo de espacio geométrico se ampliaron en 
forma inaudita, cuando las nuevas ideas geométricas acababan de encontrar aplicación (entre otros, en los trabajos del mismo Poincaré), en la teoría especial de la relatividad, teoría capaz de hacer vacilar nuestros conceptos acerca de la estructura del mundo, conceptos que parecían inamovibles desde los tiempos de
Galileo y Newton. En una época en que, de las profundidades abstractas de la misma matemática, surgía una teoría que en opinión de los más destacados matemáticos se encontraba fuera de su ciencia, y posiblemente incluso fuera de la ciencia en general, la teoría de
los conjuntos, que produjo en la matemática una revolución tan significativa como la revolución que produjo en la física la teoría de la relatividad. Por sus gustos matemáticos y tradiciones heredadas, Poincaré era un representante de la matemática clásica, la gran escuela matemática del análisis creada por Lagrange, Laplace y Cauchy. Poincaré era un representante del análisis matemático en el sentido más amplio de esta palabra, que incluye la teoría de funciones, todos los aspectos de las ecuaciones diferenciales, y la ``física matemática'' en su sentido más amplio. Y la universalidad de Poincaré como matemático se reflejó en la forma como él creó la nueva rama de la matemática: la topología. Para Poincaré, la topología, ante todo y sobre todo, era un instrumento poderoso para la solución de los problemas que surgían en las ramas clásicas de la matemática. Estos eran, en primer lugar, la teoría de funciones de variable compleja, cuya relación  íntima con la geometría había sido prevista sólo en  germen por Riemann, y que Poincaré fue el primero en
comprender en toda su profundidad; la teoría de las  ecuaciones diferenciales, que para Poincaré era inseparable de la mecánica celeste; la misma geometría.  Pero, comprendiendo el poder de los métodos topológicos en la ``matemática clásica'' y frecuentemente previéndolos allí donde en su tiempo estos métodos todavía no podían ser aplicados con toda su fuerza, Poincaré descubrió para la matemática un mundo entero de nuevos problemas, los problemas de carácter ``cualitativo'', es decir, precisamente topológico; un mundo entero inaccesible, por su misma esencia, no sólo a los métodos sino también, si es posible expresarse así, a la visión del mundo de la matemática ``clásica'', en cuyo centro se encontraban las fórmulas y los cálculos (es decir, la técnica para operar con fórmulas). De  esta forma, el más grande representante de la matemática clásica, Poincaré, como ningún otro ``rompió desde dentro'' sus tradiciones y abrió acceso hacia ella no sólo a nuevos métodos de investigación, sino lo  que puede ser aún más importante, a nuevas formas de ver las cosas e interesarse por ellas.

Aclararemos un poco lo que acabamos de decir. Toda creación matemática tiene en última instancia a la intuición matemática como su fundamento. Una reflexión continua, tenaz y concentrada, conduce a fin de cuentas a una (más o menos momentánea) visión de las leyes en cuya búsqueda estaban encaminados nuestro  trabajo y nuestra reflexión. El fin de la investigación subsiguiente (frecuentemente muy minuciosa) es la verificación de la visión de nuestra intuición, la cual (si se confirma en esta comprobación) resultará ser el verdadero núcleo del resultado obtenido. Nadie ha expuesto este mecanismo de la creación matemática mejor que Poincaré en sus libros {\it Ciencia y método} y {\it Ciencia e hipótesis}. Pero el carácter de la intuición matemática no es el mismo en todos los casos ni para todos los matemáticos. La intuición de Jacobi no se parecía a la de Hilbert, y la de Weierstrass era  distinta de la de Poincaré.
Probablemente existe algo que podríamos llamar ``intuición de la fórmula'', la capacidad de predecir el resultado de una transformación muy compleja (por ejemplo, en el análisis tensorial). Existe una intuición de carácter algebraico-lógico, la visión (y la anticipación) de complejas relaciones lógicas (por ejemplo, en la teoría de los conjuntos, en el álgebra abstracta). Y finalmente (mejor dicho, ante todo), existe la intuición geométrica; a veces hablan de ella como si fuera la única que existiera en matemáticas, la única ``auténtica'' intuición matemática. Yo creo que esto no es 
cierto, y que en realidad existen diferentes formas  de intuición matemática, incluso más allá de los pocos ejemplos mencionados. La intuición ``topológica'' es considerada frecuentemente como un caso particular de la geométrica; sin embargo, este caso particular manifiesta por sí mismo tal riqueza y variedad de posibilidades, y por otra parte, se diferencia tanto de otras  formas de intuición geométrica, que probablemente sea necesario separarla en una categoría especial. La intuición del topólogo no está ligada a líneas rectas, a transformaciones en perspectiva y otras formas tan fundamentales, por ejemplo, para el geómetra proyectivo. La intuición topológica es la intuición de la forma y disposición de las figuras en su forma más pura, la pura ``{\it Freude an der Gestal}'' como decía Klein. Esta es la
más geométrica de toda la variedad de intuiciones geométricas.

Poincaré dominaba esta forma de intuición como
ningún matemático de su tiempo o de las épocas precedentes; es posible que sólo Riemann pudiera competir  con él en este campo, pero Riemann no alcanzó a desarrollarla con la amplitud y variedad de aplicaciones  con que lo hizo Poincaré.

La intuición topológica invade la mayoría de  los trabajos más notables de Poincaré; la teoría de  las funciones automorfas y la uniformización (el triunfo supremo de la aproximación ``Riemanniana'' a la teoría de las funciones de variable compleja); la teoría cualitativa de las ecuaciones diferenciales que es,  posiblemente, la mejor ilustración de la forma tan novedosa con que sabía ver Poincaré los objetos más clásicos de la matemática y la problemática tan original a que sabía someterlos. Y, finalmente, por supuesto,
todo el ciclo de sus trabajos propiamente topológicos.

No otra cosa, sino precisamente una manifestación de su genial intuición topológica, es el hecho  de que Poincaré viera en el concepto de homología el hilo fundamental para todo el desarrollo posterior de la topología. La primera formulación de este concepto (en la memoria fundamental de 1895, {\it Analysis situs}) apelaba precisamente a la evidencia geométrica inmediata, bajo la cual solamente algunos años más tarde fue construida una base lógica rigurosa.

Igualmente intuitivo, aunque formulado en forma completamente rigurosa era el segundo concepto básico introducido por Poincaré: el concepto de grupo  fundamental. Al introducir este concepto, Poincaré se convirtió en el iniciador de toda aquella inmensa corriente que resultó ser la topología homotópica, cuyo	ser ulterior desarrollo debemos ante todo a Brower, H.  Hopf, Gurevich y una larga serie de matemáticos posteriores. Aquí, es necesario señalar que la definición de los grupos de homotopía, es decir, de los grupos  que generalizan el concepto de grupo fundamental a un
número arbitrario de dimensiones, fue formulada por  primera vez en 1932 por el famoso topólogo checo Chec, quien, es cierto, no los sometió a una investigación  posterior; esto último, como es bien sabido, es un mérito que corresponde a V. Gurevich.

Entre las más notables y más rápidamente elaboradas partes de la topología homotópica se cuenta la  teoría de los campos vectoriales y polivectoriales y  sus singularidades, que está íntimamente ligada con la teoría de los puntos fijos de las transformaciones continuas.

El fundador de esta teoría es, nuevamente, Poincaré. Las primeras definiciones y hechos, en particular el concepto fundamental de índice de singularidad del campo vectorial, fueron establecidos por Poincaré ya  en los años ochenta, en sus trabajos en teoría cualitativa de ecuaciones diferenciales, es decir, antes de  la creación de sus trabajos propiamente topológicos.

Actualmente es difícil exagerar la fundamental importancia de estas ideas y resultados de Poincaré  para todo el desarrollo posterior no sólo de la teoría de las ecuaciones diferenciales, sino de todo el análisis matemático contemporáneo.

En particular, en lo que se refiere a los teoremas de existencia de puntos fijos para unas u otras  transformaciones continuas, Poincaré comprendía la significación de estos teoremas como medios de demostración de teoremas de existencia en análisis. Esto es evidente al menos en los enormes esfuerzos que él hizo para demostrar su ``último teorema geométrico'': sobre la existencia de un punto fijo para una clase definida de transformaciones continuas de un anillo circular plano en sí mismo. Este último trabajo de Poincaré produce  en el lector una, yo diría, trágica impresión; en su  corta introducción el autor escribe que nunca publicaría una obra tan imperfecta. En efecto, Poincaré no logró demostrar el resultado fundamental (el último teorema geométrico de Poincaré) al cual estaba dedicado  el trabajo. A pesar de ello, Poincaré consideró posible y necesario publicar los resultados parciales que había obtenido, en vista de la importancia del tema y también en vista de que, como él decía, a su edad ya  no tenía esperanza de obtener la solución completa del problema. En realidad, Poincaré en esta época tenía 57 años de edad, y éste no era naturalmente el problema sino la grave enfermedad que ya le había afectado (en esta época casi imposible de operar) y a causa de la  cual falleció un año más tarde.

En su forma general, el ``último teorema geométrico de Poincaré'' fue demostrado poco después de su muerte por el entonces joven matemático norteamericano J.D. Birkhof, quien de inmediato obtuvo  celebridad por este resultado. Pero actualmente para nosotros es
importante constatar cuán profundamente pudo Poincaré prever la relevancia de los teoremas topológicos del tipo ``teorema de punto fijo'' para el análisis y para
la mecánica celeste, y reconocerlo como el fundador 
del ``método de los puntos fijos''.

La fuerza de la intuición geométrica de Poincaré lo conducía a veces a despreciar el pedante rigor
en las demostraciones. Aquí hay además otro aspecto. Encontrándose constantemente bajo el flujo de una cantidad de ideas en las ramas más diversas de la matemática, Poincaré ``no alcanzaba a ser riguroso''; a menudo se conformaba cuando su intuición le daba la certeza de que la demostración de un teorema u otro podía
llevarse hasta el más impecable rigor lógico y dejaba a otros su elaboración. Entre los ``otros'' había matemáticos del más elevado rango.

 Citaré una carta de  Poincaré a Brouwer (que data del último año de vida de Poincaré y, por cuanto yo sé, aún no ha sido publicada), la cual, según creo, ilustra muy bien las ideas
que acabo de expresar.

\begin{quote} \small

Querido Colega.

\indent Le estoy muy agradecido por su carta, pero no veo porqué duda usted de que la correspondencia entre las dos variedades sea analítica; los módulos de las superficies de Riemann pueden expresarse analíticamente como función de las constantes de los grupos fuchsianos; es cierto que es necesario darles a algunas variables solamente valores reales, pero las funciones de estas variables
reales no pierden en lo absoluto su carácter analítico.

 También es posible que usted vea la dificultad en que una de estas variedades depende, no de las constantes del grupo, sino de los invariantes. Si mal no recuerdo, yo consideraba una variedad que dependía de las constantes de las permutaciones fundamentales del grupo; al grupo le corresponderá entonces una infinidad discreta de puntos de esta variedad; dividí después esta variedad en variedades parciales, de tal manera que a un grupo le correspondiera un punto por cada una de estas variedades parciales (de la misma manera como se divide el plano en paralelogramos de períodos o al círculo fundamental en polígonos fuchsianos). Creo que el carácter analítico de la correspondencia permanecerá finalmente inalterado.

 En lo qué se refiere a la variedad de las superficies de Riemann, es posible encontrar dificultades si se las considera en la forma como lo hizo Riemann. Se puede, por ejemplo, preguntarse si el conjunto de estas dos superficies no forma dos variedades separadas. La dificultad desaparece si se consideran estas superficies desde el punto de vista de Klein; la continuidad, la ausencia de singularidades, la posibilidad de pasar de una variedad a otra continuamente se transforman entonces en verdades casi intuitivas.

 Le pido a usted me disculpe por la forma tan desarticulada y desordenada en que le expongo estas explicaciones; no espero que le satisfagan porque las he expuesto muy mal; pero creo que le darán a usted la posibilidad de precisar los puntos que le causan dificultad, de manera que yo pueda después satisfacerlo.


 Estoy muy satisfecho de tener la ocasión de entrar en contacto
con un hombre de su valor.

\hfill Su más fiel colega Poincaré.

Fecha: según el sello de correos, el 10 de diciembre de 1911.

\end{quote}

La carta citada es interesante no sólo como una ilustración del estilo creativo de Poincaré. Demuestra, además, que Poincaré tenía en gran valor ciertos trabajos matemáticos, a saber, los trabajos topológicos de Brouwer. Aquí únicamente puede tratarse de los trabajos de Brouwer que pertenecen al bienio 1909-1911. Evidentemente, Poincaré no solamente conocía bien estos trabajos al final del año 1911 (cuando escribió esta carta), sino que valoraba su profundidad. Además, los trabajos topológicos de Brouwer están escritos en forma muy difícil y en un estilo que no es clásico en  absoluto. Por lo tanto, Poincaré, incluso en el último año de su vida, encontró suficientes energías y ansias de saber como para dominar resultados matemáticos y métodos pertenecientes a una manera completamente distinta de crear, ¡un rasgo propio sólo de los más grandes científicos! Esta fue la cualidad que acompañó a Poincaré durante toda su vida. 

En 1883 G. Cantor construyó el conjunto perfecto y nunca denso (en un segmento) que lleva su nombre (``El discontinuo de Cantor''). Este fue un descubrimiento genial,
no solamente por la importancia que el conjunto de Cantor adquirió en toda la matemática, sino también porque introdujo en la matemática una nueva construcción, en nada semejante a lo que la ciencia conocía hasta entonces. Cantor vió una forma geométrica que se salía
de los límites de lo que se consideraban los dominios de la intuición geométrica, ampliando los horizontes de las posibilidades mismas de nuestra imaginación espacial. El demostró, por primera vez, que estas posibilidades pueden extenderse a formaciones que pertenecen a la misma teoría de conjuntos cuya pertenencia incluso a la matemática era discutida por destacados y respetados matemáticos (por ejemplo, Kroneker). Pero Poincaré no fue solamente uno de los primeros matemáticos que aceptó el descubrimiento de Cantor; fue el primero que lo aplicó en investigaciones concretas analíticas como dicen los químicos, {\it in statu nascendi}, en el momento mismo del nacimiento de este nuevo ser matemático, en nada parecido a toda la vieja ciencia.

Muchos matemáticos han hecho algunas construcciones notables, siguiendo el nuevo camino de intuición geométrica que abrió Cantor: Brouwer construyó sus primeros ejemplos continuos indescomponibles; Antoin, sus curvas sorprendentes cuyo grupo del espacio complementario es diferente de cero; Alexander, sus esferas cornadas y así todo un conjunto de otros investigadores. Pero el primer paso lo dió Cantor y Poincaré fue el primero en comprender no solamente la significación de este primer paso, sino su fecundidad para el análisis matemático, y con ello, para toda la matemática. Indiquemos finalmente que, como lo demuestra la última memoria de Poincaré, el dominaba la técnica de la teoría geométrica de los conjuntos hacia el final de su vida, tal y como ésta se había formado para
ese tiempo.

Regresemos al concepto de homología introducido por Poincaré. Como he señalado, este concepto fue presentado en la primera memoria topológica de Poincaré, en el célebre {\it Analisis Situs}, en forma intuitiva. Sin  embargo, esta aproximación insuficientemente rigurosa tuvo, por así decirlo, la consecuencia de hecho de que sirvió de ocasión para las fundamentadas críticas del matemático noruego Heergaard. La cuestión está en que en su primera memoria Poincaré no puso debida atención al fenómeno de torsión, limitándose en lo fundamental a los números de Betti. Pero completó brillantemente esta omisión en sus publicaciones siguientes en topología, {\it Complementos al ``Analisis Situs''}. Al hacer esto, Poincaré se situó en el punto de vista combinatorio, introduciendo el concepto de partición simplicial
(triangulación de una variedad, es decir, el concepto
de complejo simplicial) y creó, de esta manera, el método fundamental de la topología combinatoria. Probablemente Poincaré consideraba intuitivamente claro que las características introducidas por él de las variedades (y en general de los poliedros), no pueden depender de la elección de una u otra triangulación del poliedro. Sin embargo, como es bien sabido, este hecho es un teorema profundo y difícil de la topología. Para su demostración además del concepto de una subdivisión arbitrariamente fina de una triangulación, que Poincaré naturalmente dominaba, eran necesarios además, el concepto (que se apoya en el de subdivisión) de aproximación simplicial (es decir, lineal a trozos) de una transformación continua (que es una generalización de la aproximación de una curva continua mediante una quebrada inscrita), y uno u otro equivalente del concepto de grado de una trasformación (es decir, la multiplicidad con la cual, bajo su transformación continua dada digamos de un simplex $X$ sobre un simplex $Y$ o una variedad $X$ sobre otra variedad $Y$ de la misma dimensión, la variedad $Y$ está recubierta por la imagen de la variedad~$X$). Estos dos conceptos fundamentales fueron introducidos por Brouwer en 1911, es decir la víspera de la muerte de Poincaré. Con su ayuda, Brouwer demostró sus célebres teoremas sobre la invariancia topológica del número de dimensiones de una variedad $n$-dimensional, y sobre las invariancias de los puntos internos para conjuntos contenidos en ella; el teorema general de Jordan (en el caso $n$-dimensional); los teoremas sobre puntos fijos y otros. Sin embargo no demostró el teorema mismo sobre la invariancia de las características homológicas de los poliedros de Brouwer, aunque dominaba todos los medios necesarios para ello. Esto lo obtuvo por primera vez, en 1915, el célebre topólogo Alexander.

La demostración del teorema de invariancia fue el primer paso fundamental en el desarrollo ulterior de la teoría de homología fundada por Poincaré. El paso siguiente, a diferencia del primero, no estuvo ligado con la superación de dificultades matemáticas concretas, pero tuvo, a pesar de ello, una importancia fundamental. Fue realizado por la célebre algebrista Emmy Noether (en 1925-1926) y consiste en el cambio de las características homológicas consideradas por Poincaré, los números  de Betti y los coeficientes de torsión por el concepto único de grupo de Betti (o, como ahora prefieren llamarlo, el grupo de homologías). Algunos destacados topólogos, por ejemplo, Lefshetz, al principio veían con escepticismo esta innovación propuesta por Noether, considerándola solamente como una formalidad esencial entre hablar directamente del grupo de Betti de un poliedro o del conjunto de características numéricas que lo definen completamente (su rango, es decir, los números de Betti, y sus coeficientes de torsión). Sin embargo, investigaciones más cercanas demostraron que no se trataba únicamente de palabras.

En particular y ante todo, bajo el enfoque anterior, sin el concepto de los grupos de homologías, era imposible el desarrollo de una de las teorías más notables de la topología: la teoría de la dualidad topológica, cuyos primeros fundamentos fueron construidos por el mismo Poincaré y la cual se desarrolló posteriormente en nuevas direcciones y aspectos en los trabajos de Alexander
y, posteriormente, en toda su profundidad, en los de Pontriaguin y otros matemáticos.

Sin el concepto de grupo de Betti era imposible concebir los dos desarrollos posteriores de la teoría de la homología. El primero consistente en el traslado de los conceptos homológicos a formas geométricas más generales que los poliedros, ante todo a los compactos. Ello fue posible como consecuencia del aparato general de aproximación de las formaciones topológicas más complejas (compactos, bicompactos y espacios topológicos más generales) mediante construcciones combinatorio-topológicas: los complejos, lo cual fue realizado, a partir de 
1926, por el autor de esta exposición mediante los así llamados espectros proyectores (sometidos posteriores a diferentes generalizaciones y variaciones). Este proceso de aproximación está basado en el concepto, introducido por el mismo autor, de nervio de un recubrimiento de un espacio dado y proporciona la posibilidad de trasladar, prácticamente a cualquier espacio topológico, los conceptos fundamentales de la topología combinatoria\footnote{A propósito, las primeras fuentes de mi concepto de nervio están en aquello que Poincaré llamaba poliedro conjugado con uno dado}.

Otro progreso fundamental en la teoría de la homología fue la introducción por J. Alexander y A.N. Kolmogorov en 1934-1935 de las homologías ``superiores'' llamadas ahora cohomologías. De los grupos de homologías y cohomologías, cuyo campo de definición y aplicación constantemente se ampliaba, surgió finalmente una nueva disciplina matemática: el álgebra homológica, que determinó fundamentalmente el carácter de una parte muy significativa de la matemática...

Incluso en la más breve exposición del tema de mi presentación actual, no podría callar el célebre artículo de divulgación de Poincaré: ``¿Por qué el espacio tiene tres dimensiones?''  publicado en la revista francesa {\it Revue de Métaphisique et de Morale}.

Este artículo es notable porque en él, en una forma más bien literaria que rigurosamente científica, se plantea el problema, y se expone la idea, de uno de los conceptos de la topología conjuntista: el problema y la idea de la definición general e inductiva de dimensión $n$. La idea de Poincaré consiste en que si el espacio tiene dimensión $n$, entonces es posible dividirlo en partes (arbitrariamente pequeñas) mediante subespacios de dimensión $n-1$. El primer matemático que le dió forma rigurosa y definitiva a estas afirmaciones de Poincaré fue Brouwer (en su trabajo de 1913). De esta manera, Brouwer
fue el fundador de la amplia rama de la topología conjuntista desarrollada por Menger y (principalmente) por P. S. Uryson y conocida actualmente como teoría general de la dimensión. En este momento para nosotros lo importante es, sin embargo, subrayar que la idea del concepto general de dimensión se remonta a Poincaré y nos da una nueva demostración de la singular fuerza de su intuición geométrica, que abarca esta vez la rama de los conceptos de la teoría de conjuntos. Señalemos finalmente que la teoría general de la dimensión obtuvo su pleno desarrollo después de que la así llamada teoría homológica fue construida por mí en 1928-1932; esta teoría subordinaba el concepto de dimensión al concepto de homología e incluía, de esta manera, a la teoría de la dimensión como una parte de la topología homológica general.

Inicié mi exposición con la advertencia de que Poincaré vivió en una época en que en la matemática nacían ideas que admiraban nuestra inteligencia con la fuerza de su aplicabilidad al conocimiento del mundo, así como por su capacidad para ampliar, casi instantáneamente, el horizonte de la misma matemática, por su belleza y perfección interna.
 Las ideas matemáticas que nacen en nuestro tiempo son igualmente poderosas (si no más), y bellas; y sin embargo, no podrían desarrollarse, si en su cuna no tuvieran a las ideas de Poincaré.
 
En su célebre {\it Exposición del sistema del mundo}, Laplace dijo que la astronomía, por la grandeza de su objeto y la perfección de sus teorías, es el mejor monumento erigido al entendimiento humano, la más bella manifestación de su intelecto.

Los matemáticos como Poincaré nos impulsan a extender las palabras de Laplace también a la matemática, y a darle a ésta derecho de competir con la astronomía en la grandeza de su objeto y en cualquier caso, en la perfección de sus teorías.



\end{document}