\documentclass[a4paper, 12pt]{article}

%%%%%%%%%%%%%%%%%%%%%%Paquetes
\usepackage[spanish]{babel}  
\usepackage[utf8]{inputenc}
\usepackage{tcolorbox}
\usepackage{cmbright}  %%%%%%% El tipo de letra
\usepackage{setspace}
\onehalfspacing  %%%%%%%%%%% Espacio y medio de interlineado
\parskip=1em  %%%%%%%%%%%% Separacion entre parrafos
%%%%%%%%%%%%%%%%%%%%%%



%%%%%%%%%%%%%%%%%
\title{Sobre las Hipótesis que Sirven de Base a la Geometría}
\author{B. Riemann}
\date{}
%%%%%%%%%%%%%%%%%

\begin{document}

\begin{tcolorbox}[colback=blue!5!white,colframe=blue!75!black]

\vspace{-1.8cm}
\textbf \maketitle

\end{tcolorbox}

\bigskip




\subsection*{Plan de estudio}


Se sabe que la Geometría admite como dado \textit{a priori} no sólo el concepto de espacio, sino  también las primeras ideas fundamentales de las construcciones en el espacio.  Puesto que no da de estos conceptos sino definiciones nominales, las determinaciones esenciales se introducen bajo forma de axiomas.  Las relaciones mutuas de estos datos primitivos permanecen envueltas en el misterio; no se percibe bien si están necesariamente ligadas entre sí, ni hasta que punto lo están, ni siquiera \textit{a priori} si pueden estarlo.

Desde Euclides hasta Legendre, por no citar sino al más ilustre de los reformadores modernos de la Geometría, nadie entre los matemáticos ni entre los filósofos ha venido a aclarar este misterio.  La razón es que el concepto general de las magnitudes de dimensión múltiple, que comprende como caso particular las magnitudes extensas, no ha sido jamás objeto de ningún estudio.  En consecuencia, yo me he planteado primero el problema de construir, partiendo del concepto general de magnitud, el concepto de magnitud de dimensiones múltiples.  Se deducirá que una magnitud de dimensiones múltiples es subceptible de diferentes relaciones métricas, y que el espacio no es, por tanto, más que un caso particular de una magnitud de tres dimensiones.  Se deduce de esto, necesariamente, que las proposiciones de la geometría no pueden deducirse del concepto general de magnitud, sino que las propiedades por las cuales el espacio se distingue de toda otra magnitud imaginable de tres dimensiones no pueden ser deducidas más que por la experiencia.  De aquí surge el problema de buscar los hechos más simples por medio de los cuales
puedan establecerse las relaciones métricas del espacio, problema que, por la naturaleza misma del objeto, no está completamente determinado; pues se pueden indicar varios sistemas de hechos simples, suficientes para  la determinación de las relaciones métricas del espacio. El más importante, para nuestro objeto actual, es el que Euclides ha tomado por base. Estos hechos, como todos los hechos posibles, no son necesarios; no tienen sino una certeza empírica y constituyen las hipótesis. Se puede, pues, estudiar su probabilidad, que es ciertamente muy considerable dentro de los límites de observación, y juzgar por esto del grado de seguridad de la extensión de estos hechos fuera de estos mismos límites,  tanto en el sentido de los inconmensurablemente grandes, como en el de los inconmensurablemente pequeños.


\subsection*{Concepto de magnitud de $n$ dimensiones}


Ensayando a tratar el primero de estos problemas, relativo al desa\-rrollo del concepto de magnitud de dimensiones múltiples, me creo obligado a solicitar la indulgencia de los lectores, puesto que estoy menos ejercitado en los trabajos filosóficos de esta naturaleza, en donde la dificultad reside más en la concepción que en la construcción, y en donde excepto algunas breves indicaciones dadas por Gauss en su segunda memoria sobre los residuos bicuadráticos, en las \textit{gelehrte Anzeigen} de Göttingen y en su memoria de jubilación, y de algunas investigaciones filosóficas de Herbart, no he podido ayudarme de ningún trabajo anterior.

\bigskip

\S 1. \hspace{3 mm}Los conceptos de magnitud no son posibles más que allí  en donde un concepto general permite diferentes modos de determinación. Según que sea, o no, posible pasar de uno de estos modos de determinación a otro de una manera continua, forman una variedad continua o una variedad discreta; cada uno en particular de estos modos de determinación se llama en el primer caso un punto, en el segundo un elemento de esta variedad. Los conceptos en que los modos de determinación forman
una variedad discreta son muy frecuentes, pues dando objetos cualesquiera se encuentra siempre, por lo menos en las lenguas cultivadas, un concepto que los comprende (y los matemáticos están por consecuencia en el derecho, en la teoría de las magnitudes discretas, de tomar por punto de partida la condición de que los objetos dados se consideren como de la misma especie). Por el contrario, las ocasiones que pueden hacer nacer los conceptos en que los modos de determinación forman una variedad continua, son tan raros en la vida ordinaria, que los lugares de los objetos sensibles y los colores son casi los únicos conceptos simples cuyos modos de determinación forman una variedad de varias dimensiones. Solamente en la alta matemática las ocasiones para la formación y el desarrollo de estos conceptos son más frecuentes.

Una parte de una variedad, separada del resto por una marca o por un límite, se llama \textit{quantum}. La comparación de los quantos, desde el punto de vista de la cantidad, se efectúa por las magnitudes discretas, contando; o por las magnitudes continuas por medio de la medida. La medida consiste en una superposición de magnitudes para compararlos; es necesario, pues, para medir, tener un medio de transportar la magnitud que sirve de patrón de medida sobre las otras. Si falta este medio, no se podrán comparar entre sí dos magnitudes, más que si una de ellas es una parte de la otra, y aun en este caso, no se podrá decidir más que la cuestión de más grande o más pequeño, y no la de razón numérica. Las investigaciones a las que puede dar lugar este caso forman una rama general de la teoría de magnitudes, independiente de las determinaciones métricas, en la cual ellas no son consideradas como existentes independientemente de la posición, ni  como expresables por medio de una unidad, sino como regiones en una variedad. Tales investigaciones se han hecho necesarias en varias partes de las matemáticas, especialmente para el estudio de las funciones analíticas de varios valores, y es sobre todo a causa de su imperfección que el célebre teorema de Abel, así como los trabajos de Lagrange, Pfaff, Jacobi, sobre la teoría general de las ecuaciones diferenciales, han permanecido estériles mucho tiempo. En esta rama general de la teoría de las magnitudes extensas, en donde no se supone nada más que lo que ya se encierra en el concepto de estas magnitudes, nos bastará para fijar nuestro objeto de estudio sobre dos puntos: relativo el primero sobre la generación del concepto de una variedad de varias dimensiones; el segundo respecto al medio de referir las determinaciones de lugar en una variedad dada a determinaciones de cantidad, y es este último punto el que hará claramente sobresalir el carácter esencial de una extensión de $n$ dimensiones.
   
\bigskip



\S 2.\hspace{3 mm} Dando un concepto en que los modos de determinación forman una variedad continua, si se pasa, de una manera determinada, de un modo de determinación a otro, los modos de determinación recorridos formarán una variedad extendida en un solo sentido, cuyo carácter esencial es que en esta variedad no se puede, partiendo de un punto, ir de una manera continua más que en dos direcciones, hacia adelante o hacia atrás. Imaginemos ahora que esta variedad se transporta a su vez sobre otra variedad completamente distinta, y ésta todavía de una manera determinada, es decir, tal que cada uno de sus puntos se transporte en un punto determinado de la otra variedad; el conjunto de los modos de determinación así obtenido formará una variedad de dos dimensiones; se obtendrá análogamente una variedad de tres dimensiones si se concibe que una variedad de dos dimensiones se transporta de una manera determinada sobre otra completamente distinta, y es fácil ver cómo se puede continuar esta construcción. Si en lugar de considerar el concepto como determinable se considera su objeto como variable, se podrá designar esta construcción como la composición de una variabilidad de $n + 1$ dimensiones por medio de una variabilidad de una sola dirección.
   
\bigskip
   
\S 3.\hspace{3 mm} Voy ahora a mostrar que recíprocamente cómo una variabilidad en que el campo está dado, se puede descomponer en una variabilidad de una dimensión y una variabilidad de un número de dimensiones menor. Supongamos para esto una porción variable de una variedad de una dimensión contada a partir de un punto fijo, de manera que estos valores sean comparables entre sí; supongamos que esta porción tenga, para cada punto de la variedad dada, un valor determinado, cambiando con este punto de una manera continua; o en otros términos, imaginemos, en el interior de la variedad dada, una función continua del lugar, función que no sea constante a lo largo de una porción de esta variedad. Todo este sistema de puntos, para los cuales la función tiene un valor constante, forma entonces una variedad continua de menor número de dimensiones que la variedad dada.  Estas variedades, si se hace variar la función, se transforman de una manera continua unas en otras; se podrá, pues, admitir que una de ellas engendra las otras, y esto podrá tener lugar, en general, de manera que cada punto de una se transporte en un punto determinado de la otra. Los casos de excepción, cuyo estudio es importante, pueden dejarse aquí de lado. Por tanto, la determinación de posición en una variedad dada se reduce a una determinación de magnitud y a una determinación de posición en una variedad de menor número de dimensiones. Ahora es fácil ver que esta última variedad tiene $n-1$ dimensiones cuando la variedad dada tiene~$n$. Repitiendo $n$ veces este procedimiento, la determinación de posición en una variedad de $n$ dimensiones se reducirá a $n$ determinaciones de magnitud, y así la determinación de posición en una variedad dada, cuando esto es posible, se reduce a un número finito de determinaciones de cantidad. Pero existen variedades en las cuales la determinación de la posición exige, no un número finito, sino una serie infinita, o una variedad continua de determinaciones de magnitudes. Tales son, por ejemplo, las variedades formadas por las determinaciones posibles de una función en una región dada, por las formas posibles de una figura en el espacio, etc.
   
\subsection*{Relaciones métricas de las que es susceptible una variedad de $n$ dimensiones, en la hipótesis de que las líneas poseen una longitud independiente de su posición, y siendo toda línea medible por cualquier otra línea}


Después de haber construido el concepto de una variedad de $n$ dimensiones, y encontrar como carácter esencial de tal variedad la propiedad de que la determinación de la posición puede reducirse a $n$ determinaciones de magnitud, llegamos al segundo de los problemas enunciados más arriba, conocer el estudio de las relaciones métricas de que es susceptible tal variedad, y de las condiciones suficientes para la determinación de estas relaciones métricas. Estas relaciones métricas no pueden estudiarse más que como conceptos de magnitudes abstractas y su dependencia no
se puede representar por fórmulas. En ciertas hipótesis, sin embargo, son descomponibles en relaciones que tomadas separadamente, son susceptibles de una representación geométrica y, por tanto, es posible expresar geométricamente los resultados del cálculo. Así, para llegar a un terreno sólido, no se puede, verdaderamente, evitar en las fórmulas las consideraciones abstractas, pero por lo menos los resultados del cálculo se podrán representar bajo forma geométrica. Los fundamentos de estas dos partes de la cuestión se establecen en la célebre memoria de Gauss
\textit{Disquisitiones generales circa superficies curvas}.

\bigskip

\S 1.\hspace{3 mm} Las determinaciones métricas exigen la independencia entre las magnitudes  y la posición, lo cual se puede realizar de varias maneras. La hipótesis que se ofrece primero, y que yo desarrollaré aquí, es aquella en la cual la longitud de las líneas es independiente de su posición, siendo, por tanto, cada línea medible por medio de cada otra. La determinación de la posición se reduce a determinaciones de magnitudes, y la posición de un punto en la variedad dada, siendo de $n$ dimensiones; por tanto, expresable por medio de $n$ magnitudes variables $x_1,x_2, \dots ,x_n$, la determinación de una línea se obtendrá estableciendo que las cantidades $x$ son dadas como funciones de una variable. El problema consiste entonces en establecer una expresión matemática de la longitud de una línea, y para esto es preciso considerar las cantidades. No trataré este problema más que bajo ciertas restricciones, y me limitaré primero a las líneas en las cuales las relaciones entre los incrementos $dx$  de las variables $x$ correspondientes varían de una manera continua. Entonces se puede concebir las líneas descompuestas en elementos en el sentido en que las  razones de las cantidades $dx$  puedan mirarse como constantes, y el problema se reduce entonces a establecer para cada punto una expresión general del elemento lineal $ds$ partiendo de ese punto, expresión que contendrá así las cantidades $x$ y las cantidades $dx$. Admito, en segundo lugar, que la longitud del elemento lineal, abstracción hecha de las cantidades de segundo orden, permanece invariable cuando todos los puntos de este elemento sufren un mismo desplazamiento infinitamente pequeño, lo que implica al mismo tiempo que si todas las cantidades $dx$ decrecen en una misma razón, el elemento lineal varía igualmente en esa razón. Admitidas estas hipótesis, el elemento lineal podrá ser una función homogénea cualquiera de primer grado de las cantidades $dx$, que permanecerá invariable cuando se cambia de signo toda las cantidades $dx$, y en la cual las constantes arbitrarias serán $n$ funciones continuas de las cantidades $x$. Para encontrar los casos más simples, buscaré primero la expresión del elemento lineal para las variedades de $n-1$ dimensiones, que son equidistantes en todos sus puntos del origen; es decir, buscaré una función continua de la posición que distinga unas de otras. Esta función deberá crecer o decrecer en todas las direcciones a partir del origen; admitiré, que crece en todas las direcciones, y así tendrá un mínimo en el origen. Es necesario entonces, si sus cocientes diferenciales de primero y segundo orden son finitos, que la diferencial de primer orden se anule, y que la de segundo no sea jamás negativa; admitiré que permanezca siempre positiva. Esta expresión diferencial de segundo orden permanece, pues, constante cuando $ds$ permanece constante y crece en la razón de los cuadrados, cuando las cantidades $dx$ y, por tanto, también $ds$ varían en una misma razón; ella será, pues,  igual constante a $ds^2$, y por consecuencia, $ds$ igual a la raíz cuadrada de una función entera homogénea de segundo grado, siempre positiva, de las cantidades $dx$, en la cual los coeficientes son funciones continuas de las cantidades $x$. Para el espacio, si se expresa la posición del punto en coordenadas rectangulares, se tiene

\[
\textstyle ds = \sqrt{\Sigma(dx)^2}
\]

\noindent el espacio está, pues, comprendido en este caso el más simple de todos. El caso más simple después de éste comprende las variedades en las cuales el elemento lineal será expresado por la raíz cuarta de una expresión diferencial de cuarto grado. El estudio de esta clase más general no exigirá principios esencialmente diferentes, pero sí en cambio mucho tiempo, y no contribuirá mucho, relativamente, a aclarar la teoría del espacio, tanto más que los resultados no se podrán expresar geométricamente. Me limitaré a las variedades en las cuales el elemento lineal se expresa por la raíz cuadrada de una expresión diferencial de segundo grado. Tal expresión puede transformarse en otra semejante, reemplazando las $n$ variables indepedientes por funciones de $n$ variables nuevas independientes. Pero por este medio no se podrá
transformar una expresión cualquiera en otra, pues la expresión contiene $n(n+1)/2$ coeficientes, que son funciones arbitrarias de
las variables independientes; por la introducción de nuevas variables, no se podrá satisfacer más que a $n$ relaciones y, por tanto, no se podrán igualar más que $n$ de  los coeficientes a las cantidades dadas. Los coeficientes restantes están entonces completamente determinados por la naturaleza misma de la variedad que se trata de representar, y así la determinación de sus razones métricas exige $n(n-1)/2$ funciones de posición. Las variedades en
las cuales el elemento lineal puede, como en el plano y en el espacio, reducirse a la forma $\sqrt{\Sigma(dx)^2}$, solamente forman un caso particular de las variedades que nosotros estudiamos aquí, merecen un nombre especial y yo llamaré, en consecuencia, a las variedades en las cuales el cuadrado del elemento lineal se puede reducir a una suma de cuadrados de diferenciales completas variedades planas. Para poder ahora revisar las diversidades esenciales de todas las variedades susceptibles de representarse bajo la forma considerada, es preciso dejar de lado las diversidades que provienen del modo de representación.

\bigskip

§ 2. Para esto, imaginamos que a partir de un punto dado se construye el sistema de líneas de más corta distancia que pasan por este punto: la posición de un punto indeterminado podrá fijarse entonces por medio de la dirección inicial de la línea de más corta distancia sobre la cual se encuentre, y de una distancia contada sobre esta línea a partir del origen, y en consecuencia,
expresarse por medio de las razones $dx^0$ de las cantidades $d x$ sobre esta línea de más corta distancia y por medio de su longitud~$s$. Introduzcamos ahora, en lugar de $d x^0$, expresiones lineales $d\alpha$, formadas con estas cantidades, y tales que el valor inicial del cuadrado del elemento sea igual a la suma de los cuadrados de estas expresiones, de tal suerte que las variables independientes sean la magnitud $s$ y las razones de las cantidades $d\alpha$; y reemplacemos, en fin, las $d\alpha$ por las cantidades $x_1,x_2, \dots, x_n$ que les sean proporcionales, y cuya suma de cuadrados sea igual a $s_2$. Si se introducen estas magnitudes, entonces para valores infinitamente pequeños de $x$ el cuadrado lineal será igual a $dx^2$ el término del orden siguiente en este cuadrado será igual a una
función homogénea de segundo grado en las $n(n-1)/2$  magnitudes $(x_1dx_2-x_2dx_1)$ $(x_1dx_3-x_3dx_1)\dots$, es decir que será un infinitamente pequeño de cuarto orden; de tal manera que se obtiene una magnitud finita dividiendo este término por el cuadrado del triángulo infinitamente pequeño, cuyos vértices corresponden a los sistemas de valores $(0,0, \dots)$ $(x_1,x_2, x_3, \dots )$ $(dx_1, dx_2,dx_3, \dots)$ de las variedades. Este término conserva el mismo valor mientras que las cantidades $x$ y $dx$ están contenidas en las mismas formas lineales binarias, o mientras que las dos líneas de más corta distancia, desde los valores $0$ hasta los valores $x$ y desde los valores $x$ hasta los valores $dx$, estén en el mismo elemento superficial y no dependan, en consecuencia, más que de la posición y de la dirección de este elemento. Este término es evidentemente igual a cero cuando la variedad representada es plana, es decir, cuando el cuadrado del elemento lineal es reducible a $\Sigma dx^2$, y puede, por consecuencia, ser considerado como la medida en la variedad se aleja de la planicie (\textit{Ebenheit}) en este punto y en esta dirección superficial. Multiplicándola por $- 3/4$ es igual a la cantidad que Gauss ha llamado la medida de curvatura de una superficie. Para determinar las relaciones métricas de una variedad de $n$ dimensiones, susceptible de una representación bajo la forma supuesta, se encuentra que $n(n-1)/2$ funciones de posición son necesarias; si, pues, se da en cada punto la medida de la curvatura según $n(n-1)/2$ direcciones  superficiales, se podrá determinar por su medio las relaciones métricas de la variedad, siempre que entre estos valores no existan relaciones idénticas, relaciones que efectivamente no existen en general. Las relaciones métricas de estas variedades en donde el elemento lineal está representado por la raíz cuadrada de una expresión diferencial de segundo grado, pueden así expresarse de manera independiente de la elección de las magnitudes variables. Se puede todavía para este objeto seguir una marcha semejante en el caso de las variedades en donde el elemento lineal se expresa menos simplemente, por ejemplo por medio de la raíz cuadrada de una expresión diferencial de cuarto grado. Entonces el elemento lineal no será, en general, reducible a la forma de la raíz cuadrada de una suma de cuadrados de expresiones diferenciales, y por consecuencia como expresión del cuadrado del elemento lineal, lo que se separa de un plano será un infinitamente pequeño de segundo orden, mientras que en las variedades consideradas precedentemente, esta separación será un infinitamente pequeño de cuarto orden. Esta propiedad de estas últimas variedades puede llamarse planaridad en las partes infinitesimales. Pero la propiedad de estas variedades más importante para nuestro objeto actual, y la única por la cual hemos estudiado aquí estas variedades, es aquella que consiste en que las relaciones de las variedades de dos dimensiones pueden representarse geométricamente por superficies, y que las variedades de mayor número de dimensiones pueden referirse a las superficies que encierran. Esto exige todavía una corta explicación.


\bigskip
 
\S  3. \hspace{3 mm} En la manera de concebir las superficies en las relaciones métricas intrínsecas, en las cuales sólo se consideran las longitudes de los caminos trazados sobre las superficies, se mezcla siempre la idea de la porción relativa a los puntos situados fuera de ellos. Pero se puede hacer abstracción de las relaciones exteriores cuando se hace sufrir a estas superficies cambios tal que la longitud de las líneas que están situadas sobre ellas permanezca invariable, es decir, cuando se suponen flexibles sin extensión y se consideran como de la misma especie todas las superficies así obtenidas. Así, por ejemplo, dos superficies cilíndricas o cónicas cualesquiera se considerarán como equivalentes a un plano, porque pueden aplicársele por simple flexión, sus relaciones métricas intrínsecas permaneciendo invariables y todas las proposiciones que se refieren a estas relaciones, es decir, toda la planimetría permanece. Son, por el contrario, esencialmente no equivalentes a la esfera, que no se puede transformar sin extensión
en un plano. Según la investigación precedente, las relaciones métricas intrínsecas en una magnitud de dos dimensiones, cuando el elemento lineal puede expresarse por la raíz cuadrada de una expresión diferencial de segundo grado, como sucede en las superficies, están caracterizadas en cada punto por la medida de la curvatura. Se puede dar a esta cantidad, en el caso de las superficies, una interpretación sensible a los ojos, estableciendo que es el producto de las dos curvaturas de la superficie en el punto considerado, o todavía que su producto para un triángulo infinitamenete pequeño formado de líneas de la más corta distancia es igual a la mitad del exceso de la suma de los ángulos de este
triángulo, medidos en partes del radio sobre dos ángulos rectos.

La primera definición supondrá el teorema que el producto de dos radios de curvatura permanece invariable cuando la superficie sufre una simple flexión; la segunda supondrá que, para la misma posición, el exceso de la suma de los ángulos de un triángulo infinitamente pequeño sobre los ángulos rectos es proporcional al área del triángulo. Para dar una representación de la medida de curvatura de una variedad de $n$ dimensiones en un punto dado y siguiendo una dirección superficial dada pasando por este punto, es necesario partir de que una línea de la más corta distancia, partiendo de un punto, está completamente determinada cuando se da la dirección inicial. Según esto, se obtiene una superficie determinada, prolongando según líneas de la más corta distancia todas las direcciones iniciales, partiendo de un punto dado y situadas sobre el elemento superficial dado, y esta superficie tiene, en el punto dado, una medida de curvatura determinada, que es al mismo tiempo la medida de curvatura de la variedad de $n$ dimensiones en el punto dado y según la dirección superficial dada.


\bigskip

\S 4.\hspace{3 mm} Antes de pasar a las aplicaciones al espacio, es necesario todavía presentar algunas consideraciones sobre las variedades en las cuales el cuadrado del elemento lineal puede representarse por una suma de cuadrados de diferenciales. 

En una variedad plana de $n$ dimensiones, la medida de curvatura en cada punto y en dicha dirección es nula; o, seguir la discusión precedente, es necesario, para determinar las relaciones métricas, saber que en cada punto ella es nula según $n(n-1)/2$ direcciones superficiales en que las medidas de curvatura son independientes entre sí. Las variedades en que la medida de curvatura es por toda ella igual a cero, pueden considerarse como un caso particular de las variedades en que la medida de curvatura es por toda ella constante. El carácter común de estas variedades en que la medida de la curvatura es constante, puede también expresarse diciendo que las figuras allí pueden moverse sin sufrir extensión. Pues es evidente que las figuras no podrán ser susceptibles de traslaciones y de rotaciones arbitrarias si la medida de curvatura no es la misma en cada punto y en todas las direcciones. Por otra parte, las relaciones métricas de la variedad están completamente determinadas por la medida de la curvatura; pues las relaciones métricas alrededor de un punto y en todas las direcciones son exactamente las mismas que alrededor de otro punto y, por tanto, se puede a partir de este punto ejecutar las variedades en donde la medida de la curvatura es constante, se pueden dar a las figuras una posición arbitraria cualquiera. Las relaciones métricas de estas variedades dependen solamente del valor de la medida de curvatura, y en cuanto a la representación analítica, observaremos que si se designa este valor por $\Sigma$, se podrá dar a la expresión del elemento lineal la forma

\[
\frac{1}{1+\frac{\alpha}{4} \Sigma x^2}\sqrt{\Sigma dx^2}
\]

\bigskip



\S 5. \hspace{3 mm}  Para aclarar lo que precede con un ejemplo geométrico, consideremos las superficies de curvatura constante. Es fácil ver que las superficies en que la curvatura es constante y positiva pueden siempre aplicarse sobre una esfera en que el radio es igual a la unidad dividida por la raíz cuadrada de la medida de la curvatura; pero para poder considerar toda entera la variedad de estas superficies, demos a una de ellas la forma de una esfera y  a las otras la forma de superficies de revolución tomándola a lo largo del ecuador. Las superfices de mayor curvatura que la esfera tocarán la esfera interiormente y tomarán una forma semejante a la parte exterior de una superficie anular, la más alejada del eje de esta superficie. Ellas serán aplicables sobre zonas de esferas de radio menor, pero recubriendo estas zonas más de una vez. Las superficies de menor medida de curvatura positiva se obtendrán cortando sobre dos superficies esféricas de mayor radio un uso limitado por dos semicírculos máximos, y uniendo entre ellos las líneas de sección. La superficie de medida cilíndrica teniendo por base el ecuador; las superfices de curvatura negativa tocarán este cilindro exteriormente y tendrán una forma semejante a la parte interior de una superficie anular.
   
Si se consideran estas superficies como el lugar en donde se puede mover un segmento superficial, lo mismo que el espacio es el lugar en donde se pueden mover los cuerpos; el segmento superficial será móvil sin extensión en todas estas superficies. Las superficies de curvatura positiva podrán siempre tener una forma tal, que los segmentos superficiales puedan moverse sin
flexión, y esta forma será la de una esfera; pero esto no puede suceder en las de curvatura negativa. Además de la propiedad de los segmentos superficiales de ser indepedientes del lugar, la
superficie de curvatura nula posee también la propiedad que la dirección es independiente del lugar, \enlargethispage{\baselineskip} propiedad que no existe en las otras superficies.


\subsection*{Aplicación al espacio}


\S 1.\hspace{3 mm} Después del estudio anterior sobre la determinación de las relaciones métricas de una magnitud de $n$ dimensiones, se pueden indicar ahora las condiciones suficientes y necesarias para la determinación de las relaciones métricas del espacio cuando se admite como hipótesis que las líneas son independientes de su posición, y que el elemento lineal es expresable por la raíz cuadrada de una expresión diferencial de segundo grado, es decir, que el espacio es una magnitud plana en sus partes infinitesimales.

Ellas se pueden primeramente expresar exigiendo que la curvatura en cada punto sea nula según tres direcciones superficiales y, por tanto, las relaciones métricas del espacio están determinadas si la suma de los ángulos de un triángulo es igual a dos rectos por toda ella.

Si se supone en segundo lugar, como Euclides, una existencia independiente de la posición, no sólo para las líneas sino también para los cuerpos, se deduce que la curvatura es constante por toda ella, y entonces la suma de los ángulos está determinada para todos los triángulos cuando lo está en uno solo.
   
En fin, se podrá todavía en tercer lugar, en vez de admitir que la longitud de las líneas es independiente del lugar y de la dirección, suponer que su longitud y su dirección son independientes del lugar. En este punto de vista, los cambios de lugar o las diferencias de posición son magnitudes complejas expresables por medio de tres unidades.

\bigskip



\S 2.\hspace{3 mm} En el curso de las consideraciones que acabamos de presentar, hemos abordado separadamente las relaciones de extensión o de región de las relaciones métricas, y hemos encontrado que para las mismas relaciones de extensión se pueden concebir diferentes relaciones métricas; hemos buscado a continuación los sistemas de determinaciones métricas simples, por medio de las cuales las relaciones métricas del espacio están completamente determinadas y de donde las proposiciones concernientes a esta relaciones son consecuencias necesarias. Nos queda ahora por examinar cómo, y con qué grado y con qué extensión estas hipótesis se confirman por la experiencia. Desde este punto de vista existe entre las simples relaciones de extensión y las relaciones métricas esta diferencia esencial, que en las primeras, en donde los casos posibles forman una variedad discreta, los resultados de la experiencia no son, en verdad, jamás completamente ciertos, pero no son inexactos; mientras que en los segundos, en donde los casos posibles forman una variedad continua, toda determinación de la experiencia es siempre inexacta, por muy grande que pueda ser la posibilidad de su aproximación. Esta circunstancia es importante cuando se trata de extender estas determinaciones empíricas más allá de los límites de observación en lo inconmensurablemente grande o en lo inconmensurablemente pequeño; pues las segundas relaciones evidentemente pueden ser cada vez más inexactas cuando se salen de los límites de observación, lo cual no sucede para las primeras.
     
Cuando se extienden las construcciones del espacio o lo inconmensurablemente grande, es necesario hacer la distinción entre ilimitado e infinito: el primero pertenece a las relaciones de 
extensión, el segundo a las relaciones métricas. Que el espacio sea una variedad ilimitada de tres dimensiones, es una hipótesis que se aplica en todas nuestras concepciones del mundo exterior, que nos sirve para completar en cada instante el dominio de nuestras percepciones efectivas y para construir los lugares posibles de un objeto y que se encuentra constantemente verificado en todas estas aplicaciones. La propiedad del espacio de ser ilimitado posee, pues, más grande certeza empírica que ninguna otra externa a la experiencia. Pero la infinidad del espacio no es de ninguna manera la consecuencia; por el contrario, si se suponen los cuerpos independientes de la posición y se atribuye al espacio una curvatura constante, el espacio será necesariamente finito si esta curvatura tiene un valor positivo por pequeño que sea. Prolongando según líneas de la más corta distancia las direcciones iniciales situadas en un elemento superficial, se obtendrá una superficie ilimitada de curvatura constante, es decir, una superficie que en una variedad plana de tres dimensiones toma la forma de una superfice esférica y que será en consecuencia finita.


\bigskip

\S 3.\hspace{3 mm} Las cuestiones sobre lo incomensurablemente grande son cuestiones inútiles para la explicación de la naturaleza. Pero no ocurre lo mismo con las cuestiones sobre lo incomensurab!emente pequeño. Es la exactitud con la cual seguimos los fenómenos en lo infinitamente pequeño, sobre lo que reposa esencialmente nuestro conocimiento de sus relaciones de causalidad. Los progresos en los últimos siglos en el conocimiento de la naturaleza mecánica dependen casi únicamente de la exactitud de la construcción que se ha hecho posible gracias a la invención del análisis infinitesimal, y a los principios simples descubiertos por Arquímedes, Galileo y Newton, y de los cuales se sirve la física moderna. Pero en las ciencias naturales, en donde faltan los principios simples para tales construcciones, se pretende reconocer la relación de causalidad siguiendo los fenómenos en la extensión muy pequeña, tan lejos como permite el microscopio. Las cuestiones sobre las relaciones métricas del espacio en lo incomensurablemente pequeño no son, pues, cuestiones supérfluas.

Si se supone que los cuerpos existen independientemente del lugar, la curvatura es constante y resulta entonces de las medidas astronómicas que no puede ser diferente de cero; en todos los casos, será preciso que su valor recíproco sea una superficie en presencia de la cual el alcance de nuestros telescopios será como nulo. Pero si esta independencia entre los cuerpos y el lugar no existe, entonces las relaciones métricas reconocidas en lo grande no sirven para concluir nada del infinitamente pequeño; entonces la medida de curvatura de cada punto puede tener según tres direcciones un valor arbitrario, por lo que la curvatura total de toda porción medible del espacio no difiere sensiblemente de cero; se pueden introducir relaciones todavía más complicadas, cuando se supone que el elemento lineal se puede representar por la raíz cuadrada de una expresión diferencial de segundo grado. Parece que los conceptos empíricos sobre los cuales se fundan las determinaciones métricas de la extensión, el concepto del cuerpo sólido y el
de rayo luminoso, cesan de subsistir en lo infinitamente pequeño.
Es pues legítimo suponer que las relaciones métricas del espacio
en lo infinitamente pequeño no son conformes a las hipótesis de
la Geometría, lo cual habrá que admitir en el momento en que por
ello se obtiene una explicación más simple de los fenómenos.

La cuestión de la validez de las hipótesis de la Geometría en lo infinitamente pequeño está ligada con la cuestión del principio íntimo de las relaciones métricas en el espacio. En esta última cuestión, que se puede considerar como perteneciente a la doctrina del espacio, se encuentra la aplicación de la observación precedente, que, en una variedad discreta, el principio de las relaciones métricas está ya contenido en el concepto de esta variedad, mientras	que, en una variedad continua, este principio ha de venir de fuera. 	Es necesario, pues, o que la realidad sobre la cual está fundada el
espacio forme una variedad discreta, o que el fundamento de las
relaciones métricas se busque fuera de él, en las fuerzas de ligazón
que actúan en él.
	
La respuesta a estas cuestiones no se puede obtener más que
partiendo de los fenómenos verificados hasta ahora por la experiencia, y que Newton ha tomado por base, y aportando a esta
concepción las modificaciones sucesivas exigidas por los hechos que
no pueda explicar. Las investigaciones partiendo de conceptos generales, como el estudio que acabamos de hacer, no puede tener
otra utilidad que evitar que este trabajo no sea obstaculizado por
visiones demasiado estrechas, y que el progreso en el conocimiento
de la dependencia mutua de las cosas no encuentre un obstáculo
en los prejuicios tradicionales.	
	
Esto nos conduce a los dominios de otra ciencia, a los dominios
de la física, en donde el objeto al cual está destinado este trabajo
no nos permite penetrar hoy.





\end{document}