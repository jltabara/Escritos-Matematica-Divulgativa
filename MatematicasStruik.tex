\documentclass[a4paper, 12pt]{article}

%%%%%%%%%%%%%%%%%%%%%%Paquetes
\usepackage[spanish]{babel}  
\usepackage[utf8]{inputenc}
\usepackage{tcolorbox}
\usepackage{cmbright}  %%%%%%% El tipo de letra
\usepackage{setspace}
\onehalfspacing  %%%%%%%%%%% Espacio y medio de interlineado
\parskip=1em  %%%%%%%%%%%% Separacion entre parrafos
%%%%%%%%%%%%%%%%%%%%%%



%%%%%%%%%%%%%%%%%
\title{Las Matemáticas. Su Origen y Desarrollo}
\author{D. Struik}
\date{}
%%%%%%%%%%%%%%%%%

\begin{document}

\begin{tcolorbox}[colback=blue!5!white,colframe=blue!75!black]

\vspace{-1.8cm}
\textbf \maketitle

\end{tcolorbox}

\bigskip

\tableofcontents

\newpage

\section*{Los orígenes}\addcontentsline{toc}{section}{Los orígenes}

\setcounter{page}{1}

Nuestras concepciones matemáticas se formaron como resultado de un prolongado
proceso social e intelectual, cuyas raíces se esconden en el remoto pasado. Sus
orígenes pueden buscarse en el período neolítico, cuando los hombres, en lugar
de limitarse a buscar y conservar sus alimentos, se convirtieron en productores
de los mismos, sentándose los cimientos de la agricultura, la domesticación de
ganado y, eventualmente, el trabajo de los metales. Los vestigios de algunas
actividades de la Edad de Piedra ---productos de carpintería, tejido, cestería y
alfarería--- prueban que esas actividades pudieron estimular el desarrollo de
concepciones geométricas. Los procedimientos primitivos de contar, reducidos
primeramente a la forma  «uno, dos, muchos», condujeron gradualmente a una
manera más precisa de denotar los números desde uno hasta diez, y
posteriormente, hasta números más grandes. Durante la Edad de Piedra hubo un
tráfico considerable, que estimuló también el arte de contar. El estudio de los
lenguajes africanos o indoamericanos revela sistemas de numeración, a veces en
escala decimal o en una escala de cinco, doce o veinte. Aquí y allá aparecen
fracciones simples. Esto condujo a la noción abstracta de número y forma. Cuando
en Sumeria y Egipto aparecen los primeros documentos escritos, revelan ya algún
dominio de las reglas aritméticas simples. Se advierten tales reglas cuando
cuatro aparece como dos más dos, diecinueve como veinte menos uno, veinte como
dos veces diez. Los lenguajes muestran también estas relaciones en su manera de
expresar los números cardinales. Hicieron su aparición los dibujos en vestidos,
cestas y alfombras; se tensaron, ligaron y anudaron cuerdas, y surgieron algunas
figuras regulares como símbolos protectores en los ritos mágicos.

Este proceso se vio acelerado en el siguiente período de desarrollo cultural, el
de las primitivas civilizaciones de Oriente: Egipto, India, Babilonia y China.
Aquí, la matemática apareció ya como una ciencia, pero primero y ante todo como
una ciencia práctica, y hasta empírica, indispensable para la agricultura, la
medición de la tierra, la tributación, la ingeniería y el arte de la guerra. La
veracidad de las proposiciones y teoremas aritméticos y geométricos era
comprobada constantemente en los tratos del hombre con la naturaleza y con su
propia estructura social.

La etapa empírico-práctica de la matemática fue complementada, casi desde sus
comienzos, por una definida tendencia opuesta, una tendencia hacia la
abstracción. La creación de los mismos conceptos de número y figura fue un hecho
de abstracción tal, que probablemente debe asociarse con un nivel relativamente
elevado de la vida neolítica. Pero tan pronto como se crean estos conceptos,
adquieren vida propia. La suma, la resta, la multiplicación, empiezan a ser
entendidas como operaciones abstractas, ejecutadas primeramente con objetos,
pero después también con los símbolos mismos. Se descubrió que mediante la suma
el sistema de números puede extenderse indefinidamente y es posible tener un
atisbo de la impresión que esto debió ejercer si se advierte el interés que
pusieron los pueblos agrícolas primitivos en la manipulación de números muy
grandes. Este hecho es importante para comprender cómo pueden crearse las
concepciones matemáticas generalizadas en el mismo proceso social, porque la
experiencia directa no conduce a una noción definida de cualesquiera números,
sino de los pequeños. También los conceptos de línea, plano y círculo empezaron
a adquirir vida propia.

La primitiva historia de las matemáticas descubre, pues, lo que Stuart Mill
observó hace cien años: que las premisas originales de esta ciencia de las
cuales se deducen las verdades permanentes, pese a todas las apariencias en
contra, son un resultado de la observación y la experiencia, y se fundan, en
suma, en la evidencia de los sentidos.

Ahora sabemos que la concepción de Mill de estas premisas originales pecaba de
exceso de simplificación, que los axiomas de la aritmética y la geometría
entrañan algo más que lo que él creía. Pero es cierto que las matemáticas, como
ciencia, se desarrollaron mediante la elaboración de conceptos abstractos
basándose en el material empírico y dejando luego que estos conceptos vivieran
su vida propia. Lo general se desarrolló partiendo de lo particular; el método
del ejemplo. Este desarrollo de una ciencia abstracta no condujo de hechos
objetivos a hechos imaginativos, sino de la verdad empírica directa a una verdad
situada en un plano más elevado de generalidad. En lugar de contar series de
pequeña extensión, tales como los dedos, pudieron enumerarse rebaños o
ejércitos. La suma, la resta y los principios de la multiplicación se
extendieron de números pequeños a otros mayores. Con el descubrimiento de que
podían escribirse todos los números dentro de la estructura de un sistema
(generalmente, el decimal), es decir, en la forma

\[ a_0b^n + a_1b^{n-1} + \dots + a_{n-1}b + a_n  \]

\noindent ($a_0, \dots,a_n$ menores que $b$, $b$ generalmente = 10), se encontró
ya una importante ley aritmética, aunque, por supuesto, no en esta forma
explícita. En este interjuego de inducción y deducción las matemáticas mostraron
ya desde su mismo nacimiento algunos de los aspectos de una verdadera ciencia.

E. T. Bell hace notar que los intuicionistas modernos, que aceptan la formación
de la secuencia inacabable de los números naturales como «intuición original»,
puede que cuenten con un punto de partida inatacable para una construcción
abstracta de las matemáticas, pero su posición es claramente anti-histórica. No
cabe duda que la intuición «es punto menos que universal entre los primitivos,
que, presumiblemente, son seres humanos». La concepción de los números naturales
fue el resultado de un proceso de generalización aplicado, en milenios de
práctica humana, a los datos empíricos, proceso sometido constantemente a prueba
a la luz de la nueva experiencia. Esta experiencia se amplió con la llegada de
las sociedades orientales que practicaban la irrigación, pero el proceso de
abstracción continuó también a ritmo acelerado.

Con el desarrollo de la antigua civilización oriental, advino otro progreso que
hizo que las matemáticas se divorciaran considerablemente del mundo de la
experiencia directa: la enseñanza de las matemáticas en las escuelas de
escribas, en los templos y edificios administrativos de Menfis, Babilonia y
otros centros orientales.

La preparación de nuevos administradores condujo al planteamiento de problemas
abstractos con fines de enseñanza, y esto empezó a adquirir ya la forma de un
cultivo de las matemáticas por su propio valor. Se llegó así a una nueva manera
abstracta de abordar los problemas relativos a número y espacio, en la que
surgieron algoritmos, teoremas y teorías. Los babilonios del segundo milenio
antes de J. C. conocían las relaciones numéricas entre los lados de un triángulo
rectángulo, conocidas ahora como el teorema de Pitágoras, y resolvían ecuaciones
cuadráticas y algunas cúbicas y bicuadráticas. Parte de estas actividades, más
complicadas, principalmente numéricas y algebraicas, puede explicarse por las
exigencias de la astronomía, pero existía claramente un cultivo de las
matemáticas por sí mismas. Las relaciones matemáticas, que se descubrieron y
estudiaron independientemente de su aplicabilidad directa, fueron el resultado
de la lógica intrínseca y del poder creador de las matemáticas mismas. La
relación de estas matemáticas orientales con la práctica siguió siendo siempre
harto notoria, y la estructura lógica se desarrolló muy poco.

Sin embargo, la comprobación de la solución numérica de una ecuación de grado
superior no se encuentra ya en su aplicación a la tributación, a la geodesia e
incluso a la astronomía, sino en la estructura matemática de la ecuación misma.
Este desarrollo, iniciado así, fructificó plenamente en las matemáticas griegas.

Un ejemplo instructivo de la manera en que los nuevos conceptos matemáticos
surgieron de la lógica del simbolismo mismo, aunque siguieron expresando al
mismo tiempo relaciones en el mundo objetivo, se encuentra en la introducción
del cero. Los babilonios usaban un sistema sexagesimal de numeración (que
facilita la computación fraccional) superpuesto a un sistema decimal (que
presenta en su base diez el vestigio constante de que el concepto de número
cristalizó en la práctica comercial de contar con los dedos). Este sistema
sexagesimal se expresó en un sistema de «valor del lugar», que fue a su vez un
gran descubrimiento en el formalismo matemático. El uso de un símbolo social
para el cero llegó como complemento necesario del mencionado sistema, para
señalar la diferencia entre la notación de números tales como $2 \cdot 60^2 + 5$
y $2\cdot 60 + 5$. Aquí, el desarrollo del simbolismo matemático mismo condujo,
no sólo a una extensión del sistema numérico, sino también a la introducción del
símbolo cero en el razonamiento matemático.

La historia de este símbolo, especialmente en la matemática india, lo conecta
con la concepción del vacío, que desempeña también un papel en la física
aristotélica. No sólo el ser, sino también el no ser, se convirtió en tema de la
operación matemática, como preliminar de la matemática del devenir, de la
matemática del cambio. Hay que insistir en que esta introducción del cero no
significaba que la matemática cayera en ociosas especulaciones, sino que no hizo
más que simplificar su procedimiento de ocuparse de los objetos, caracterizados
por los números naturales así como por las fracciones.

Con la clase de los mercaderes jónicos del siglo VI antes de J. C., dio comienzo
el desarrollo de la estructura lógica de las matemáticas. Esto constituyó un
aspecto del racionalismo jónico en su conjunto, que se desarrolló en una
atmósfera de actividades comerciales a las que el despotismo oriental no puso
demasiados obstáculos. Sabemos muy poco acerca del desarrollo de esta nueva
manera de enfocar las matemáticas, aunque parece bastante seguro que algunos de
sus descubrimientos ---lo irracional y las paradojas del infinito--- se
desarrollaron como parte de las luchas ideológicas entre la aristocracia y
democracia en la Edad de Oro de Grecia. Se reveló claramente la estructura
nórdica de la matemática, la posibilidad de un fundamento axiomático de esta
ciencia. La elaboración posterior de estos fundamentos tuvo lugar, primero en la
península griega en la escuela de Platón, y después por conducto principalmente
de los sabios profesionales en la corte de los Ptolomeos en Egipto y de algunos
otros monarcas helenísticos. La totalidad del período creador de la matemática
«griega»  duró solamente unos cuantos siglos y tuvo lugar en un escenario social
verdaderamente excepcional cuyo estudio requiere todavía un cuidadoso análisis.
El período comenzó con la primera aparición de una democracia en la historia
---aunque una democracia coexistente con la esclavitud--- y presenció el
desarrollo de una aristocracia dueña de esclavos, de la que Platón era un
representante, que luchaba los restos de democracia, y llegó a su fin en los
primeros siglos del helenismo, cuando Grecia y Oriente se reunieron bajo la
hegemonía griega. Durante el imperio romano, cuando la esclavitud alcanzó mayor
importancia y se convirtió cada vez más en la base de la propiedad, a la vez que
Grecia y el Oriente quedaban reducidos a la categoría de colonias, las clases
gobernantes volvieron a caer en una cómoda mediocridad en la que no podía
sobrevivir la originalidad.

Se abandonó entonces el contacto inmediato con la práctica en la forma típica de
la matemática «griega». Los hombres que crearon esta matemática estaban en
efecto distanciados del comercio y del trabajo manual: se les había enseñado a
despreciarlos. Así pudieron consagrar todo su tiempo al esfuerzo creador de una
naturaleza abstracta. En los \textit{Elementos} de Euclides se situaba a la
matemática dentro de un riguroso esquema deductivo. Los teoremas se derivaban de
un pequeño conjunto de axiomas. No se formulaban preguntas respecto al origen de
los axiomas ni se daban respuestas. Aunque la elección de los axiomas y los
postulados, así como el lenguaje en el que son expresados, indica aún su
relación con el mundo objetivo, los teoremas son demostrados reduciéndolos a los
axiomas y postulados. Se echa de menos toda tentativa de probar la verdad de los
teoremas con la experiencia, si bien subsiste el hecho significativo de que
muchos de los teoremas de Euclides pueden probarse sobre figuras materiales y
resultar exactos dentro de los límites de la observación. Sólo hay, por ejemplo,
cinco poliedros regulares diferentes, como prueban ampliamente los experimentos
con modelos de cartón o yeso y la investigación sobre los cristales. Sin
embargo, entre los conceptos materiales de Euclides hay algunos que ya no pueden
ser sometidos a experimento, y aquí es donde resulta más evidente el progreso
del razonamiento matemático más allá de la experiencia inmediata. La prueba
entre la conmensurabilidad e inconmensurabilidad de los segmentos lineales
radica en la razón, no en el experimento, y aunque todavía podemos demostrar con
la comprobación efectiva que el volumen de la pirámide es el tercio del de un
prisma de base y altura iguales, la prueba requiere el método exhaustivo, es
decir, la estructuración de la pirámide en tan gran número de pequeños prismas
que nos parece preferible encontrar la demostración sólo por el razonamiento. El
razonamiento mismo ha creado sus propias leyes. La prueba de la verdad en las
matemáticas se ha convertido ahora en la ausencia de contradicciones. Un teorema
es verdadero cuando puede reducirse gradualmente, mediante ciertas operaciones
lógicas aceptadas, a una serie de axiomas, cuya verdad es simplemente postulada.
La matemática moderna ha aceptado sin reservas este criterio de verdad: la
prueba de verdad de una teoría matemática es la ausencia de contradicciones
lógicas. En los tiempos modernos, las pruebas se han trasladado de la geometría
a la aritmética, de la aritmética a la lógica. Las diferentes teorías de nuestro
tiempo respecto a la fundamentación de estos campos se ocupan todas de métodos
destinados a probar las contradicciones posibles de un sistema matemático, o
mejor, las contradicciones y saltos (consecuencia e integridad). Cuando
prevalece tal punto de vista, la cuestión de probar la verdad de la matemática
como una imagen del mundo objetivo se convierte en algo remoto, y muchos
matemáticos «puros» ni siquiera le prestan gran atención.

\section*{La matemática «capitalista»}\addcontentsline{toc}{section}{La
matemática «capitalista»}

La matemática moderna es el producto de la aparición del capitalismo. Las
ciudades comerciales de la Alta Edad Media europea se hallaban gobernadas por
una clase de mercaderes, políticamente conscientes, clase que gradualmente se
emancipó de los pequeños señores feudales y estableció su propio imperio
comercial. La gran diferencia entre estos mercaderes y los jónicos de dos
milenios antes, consistía en la ausencia de esclavitud. La esclavitud, a la
larga, frustra el perfeccionamiento tecnológico, y con ello, ahoga la
originalidad científica. La clase comerciante de la Alta Edad Media no tropezó
con ese impedimento. Había artesanos libres y una clase obrera libre incipiente.
La posibilidad de obtener grandes beneficios en el nuevo sistema capitalista
condujo a los comerciantes, así como a algunos de los príncipes, a cultivar la
navegación y la astronomía. En las ciudades florecieron la aritmética, el
álgebra y el nuevo arte de la teneduría de libros. La posibilidad de emplear
máquinas con fines productivos estimuló la inventiva. El uso creciente de estas
máquinas condujo al estudio de la mecánica, y ésta, a su vez, a la investigación
de nuevos métodos en la matemática. Donde la \textit{Rechenhaftigkeit} de la
primitiva burguesía promovió la aritmética y el álgebra, la inventiva de esta
clase llevó eventualmente al cálculo. Las fuentes de la aritmética y del álgebra
se encontraron en Oriente, las del cálculo, entre los griegos, especialmente en
Arquímedes. Y cuando, en el primer entusiasmo del descubrimiento, se pasó a
menudo por alto la estructura lógica de la matemática, la naturaleza misma de
esta ciencia exigió eventualmente el retorno a las cuestiones fundamentales.
Mientras Kepler y Stevin fueron esencialmente experimentadores en el campo de la
matemática, algo así como si fueran una especie de físicos, los grandes
matemáticos del siglo XVII se vieron llevados a desarrollar la matemática por su
propio valor. Los grandes descubrimientos de este período ---el tratamiento
algebraico de la geometría y el cálculo--- sólo fueron posibles porque los
matemáticos no buscaban aplicaciones inmediatas, sino que seguían la lógica
dialéctica de la ciencia misma. Dialéctica, porque estableció nuevas relaciones
entre campos antes separados, entre el álgebra y la geometría, entre el problema
de la tangente y la determinación de volúmenes, entre el finito y el infinito,
entre la matemática y la lógica.

El hecho de que se desarrollaran las matemáticas por su propio valor no quiere
decir que se perdiera la conexión entre la teoría y la práctica. Las
matemáticas, a los ojos de todos los grandes matemáticos, desde Descartes hasta
Leibniz, constituían la clave para la mecánica; y la matemática, la clave para
el entendimiento de la naturaleza. La matemática no sólo llegó a ser el modelo
de toda la ciencia, sino que proporcionó también la clave de los inventos.
Característico de todo este período era la convicción, no sólo de que la
matemática enseña las propiedades de la materia, sino de que la armonía de la
matemática refleja en cierto modo la armonía del mundo. En los primeros tiempos
del Renacimiento, Aristóteles fue reemplazado, en muchas escuelas, por Platón,
con su creencia pitagórica en la universalidad del número, y este credo en los
poderes de la matemática pasó a convertirse nuevamente, en forma modificada, en
una característica típica de los grandes racionalistas del siglo XVII.

La misma relación íntima entre la matemática «pura» y la «aplicada»  prevaleció
en el siglo XVII y duró hasta bien entrado el siguiente, especialmente en
Francia. Para los Bernoulli, al igual que para Euler, Lagrange y Laplace, la
matemática era ante todo un instrumento para la comprensión del cosmos. Esto se
pone de manifiesto en las grandes obras de la época: las dos grandes obras
matemáticas en que culminó el siglo XVII fueron el \textit{Horologium
oscillatorium} de Huygens y los \textit{Principia} de Newton; las dos grandes
obras culminantes del siglo siguiente fueron la \textit{Méchanique analytique}
de Lagrange y la \textit{Mechanique celeste} de Laplace. Esta identificación
virtual de la matemática y la mecánica no fue esencialmente rota por las
notables descubrimientos de Euler y otros en la teoría de los números, ni por
Lagrange en el álgebra.

Con la Revolución Francesa se abrió un nuevo período en la historia de la
matemática. Es notable que el nuevo ímpetu científico procediera de la gran
transformación política, siendo mucho más indirecta al principio la influencia
de la Revolución Industrial subyacente sobre el desarrollo de la matemática.
Inglaterra, que fue el primer país afectado por la Revolución Industrial, dio
muestras de esterilidad durante muchas décadas en el terreno de la matemática.
En cambio, la revolución política afectó de manera mucho más directa las
opiniones y las perspectivas. Llevó a posiciones de influencia y poder a nuevas
clases interesadas en la promoción de la educación técnica, en el ensanchamiento
de las bases sociales de la ciencia y en nuevas ideas de libertad intelectual.
La École Polytechnique fundada en París en 1795 para la preparación de
ingenieros se convirtió en el prototipo de una institución dedicada no sólo a la
enseñanza científica, sino también a la investigación fundamental en el terreno
de la matemática y de las ciencias naturales. Los principios de la revolución en
la ciencia y en la educación se extendieron a otros países, incluso a aquellos
en los que el feudalismo conservaba aún su poder sobre el capitalismo naciente.
Después de Francia, Alemania surgió como un gran centro matemático, siendo
seguida por otros países. Las demandas hechas a la ciencia se multiplicaron.

En este escenario social, el progreso sólo podía realizarse por medio de la
especialización. Así, la matemática se desarrolló en todo su esplendor como un
campo de investigación dependiente, aunque no sólo la mecánica, sino también la
nueva física matemática ejerció poderosa influencia. Estas dos formas de abordar
la matemática, la «pura» y la «aplicada», nunca estuvieron realmente separadas,
aun cuando los matemáticos mismos se especializaron cada vez más. Eran muchos
los casos en que la matemática «pura» influía sobre la rama «aplicada»  y
viceversa. Esto demuestra que a pesar de la incrementada abstracción, ha seguido
presentando un cuadro de relaciones en el mundo objetivo. Las historias de la
matemática reconocen este hecho hasta cierto punto, si bien la interacción de la
matemática, la física y la ingeniería es mucho más íntima de lo que suele
admitirse. La geometría diferencial de las curvas se desarrolló bajo la
influencia de las investigaciones de Barré de Saint-Venant en la elasticidad, la
de las superficies como resultado del interés de Monge en la ingeniería y del de
Gauss en la geodesia. El estudio que hizo Hamilton de los rayos de la luz
condujo a su teoría de las ecuaciones diferenciales parciales. Hasta la
tendencia al rigor en la matemática moderna recibió un poderoso impulso del
estudio de las series trigonométricas, introducido por Fourier en la teoría del
calor; la rigurosa definición de la integral de Riemann se derivaba en su
trabajo de series trigonométricas. Durante la segunda mitad del siglo XIX, tanto
Klein como Poincaré, los principales matemáticos de esa época, recalcaron en su
obra la íntima relación que existe entre la matemática y sus aplicaciones, a la
vez que revelaron algunos de sus más sutiles pensamientos. Klein subrayó la
importancia de la teoría de grupos en la cristalografía y la mecánica y, en la
última parte de su vida, vio en la teoría de la relatividad otra aplicación de
sus ideas. En cuanto a la obra de Poincaré, probablemente se comprenderá mejor
si tomamos como punto de partida su obra sobre la mecánica celeste. Klein fue
quien promovió la fundación del Instituto de Matemáticas Aplicadas en Gotinga.
Esta actitud se refleja también en las conferencias de Klein sobre la historia
de la matemática en el siglo {XIX, conferencias que en muchos lugares ofrecen un
punto de vista materialista. El reconocimiento de esta estrecha relación entre
la teoría y la práctica, entre los símbolos y el mundo que representan, se
advierte también en las tentativas de Hilbert por rescatar los resultados de la
matemática moderna de la arremetida de los intuicionistas.  Sólo podrá llegarse
a una comprensión mas profunda de las fuerzas sociales que intervienen en el
desarrollo de la matemática cuando los historiadores de la materia empiecen a
percatarse del vasto territorio no descubierto aún por los historiadores
económicos. Con ello debe combinarse el estudio de la historia de la tecnología,
que es un campo relativamente inexplorado todavía.

Los matemáticos de los siglos XVII y XVIII abrigaban pocas dudas en cuanto a la
objetividad de los resultados de sus especulaciones. Como no existía una
separación entre la matemática pura y la aplicada, la matemática era para ellos,
por encima de todo, una manera de resolver ecuaciones, de encontrar áreas y
volúmenes, de resolver problemas de mecánica e ingeniería y de descubrir las
leyes de un universo mecanicista. Galileo decía que solo puede comprender la
naturaleza quien ha aprendido su lenguaje y las señales con las que nos habla:
este lenguaje es la matemática y sus señales son las figuras matemáticas. Hasta
el propio Leibniz, sin dejar de creer en el carácter «innato» de las verdades
necesarias y eternas, las entendía como «los principios mismos de nuestra
naturaleza al igual que del universo», y veía en las matemáticas una de esas
verdades necesarias que servía como la exposición universal de formas posibles
de conexiones en general y de dependencia mutua. La teoría del número, por su
carácter abstracto y su evidente «utilidad» pudo ofrecer una oportunidad para la
especulación idealista, pero no la encontramos claramente en esos autores.

La reacción idealista no procedió de los matemáticos, sino del obispo Berkeley,
cuyo \textit{esse est percipi} fue un ataque directo contra las ciencias
naturales, inclusive contra el carácter objetivo de las concepciones
matemáticas.

Berkeley se expresó con absoluta claridad:

\begin{quote}\small Pero es evidente, por lo que ya hemos demostrado, que la
extensión, la figura y el movimiento son sólo ideas que existen en el
espíritu... Que el número es por completo una criatura del espíritu, aun cuando
se permita a las otras cualidades existir sin él, será evidente para todo el que
considere que la misma cosa tiene una denominación diferente de número cuando el
espíritu la considera con diferentes aspectos. Así, la misma extensión es uno, o
tres, o treinta y seis, según que el espíritu la considere con referencia a una
yarda, un pie o una pulgada. \end{quote}

La reacción de los matemáticos puede juzgarse por Leibniz, quien desechó
despectivamente la sabiduría de Berkeley:

\begin{quote}\small El hombre de Irlanda que impugna la realidad de los cuerpos
parece que ni da razones adecuadas ni se explica suficientemente. Sospecho que
pertenece a esa clase de hombres que desean ser conocidos por sus paradojas.
\end{quote}

La crítica que Berkeley hizo de los fundamentos del cálculo siguió siendo, sin
embargo, durante mucho tiempo un reto para los matemáticos.

Durante la primera parte del siglo XIX, la actitud de la mayoría de los
matemáticos no varió gran cosa, aun cuando Kant les causó alguna impresión en lo
relativo a su manera de pensar. Gauss no creía en el carácter apriorístico de
nuestras nociones de la geometría proclamado por Kant. La euclidicidad del
espacio era para Gauss una proposición empírica:

\begin{quote}\small He llegado cada vez más a la convicción de que la necesidad
de nuestra geometría no puede ser demostrada. Hasta que se haga esto, debemos
equiparar a la geometría, no con la aritmética, que conserva su carácter
puramente apriorístico, sino más bien con la mecánica. \end{quote}

Gauss tomó incluso las mediciones geodésicas de un triángulo (formado por las
cumbres de las montañas Brocken, Hohenhagen e Inselsberg) para averiguar si la
suma de los ángulos es de 180 grados, y comprobó que, dentro de los límites de
la observación, esta suma es efectivamente de 180 grados.

Riemann, la figura central de la matemática del siglo {XIX, manifestó
claramente:

\begin{quote}\small ¿Cuándo es verdadero nuestro entendimiento del mundo? Cuando
la interconexión  de nuestras concepciones  corresponde a las interconexiones de
las cosas. \end{quote}

Y Weierstrass, en su discurso inaugural de 1857, después de señalar que la
matemática griega había establecido las propiedades de las secciones cónicas
mucho antes de que se descubriera que eran también las órbitas de los planetas,
agregó que abrigaba la esperanza de que habría más funciones con propiedades
tales como las que Jacobi atribuye con encomio a su función \textit{tetha}, la
cual nos enseña en cuantos cuadrados puede descomponerse todo número, cómo puede
rectificarse el arco de la elipse, y la cual es capaz \mbox{---siendo} la única
función que puede hacerlo--- de representar la verdadera ley de la oscilación
del péndulo.

\section*{El esclarecimiento de los fundamentos del
cálculo}\addcontentsline{toc}{section}{El esclarecimiento de los fundamentos del
cálculo}

Cuando en el transcurso del siglo XIX la matemática llegó a niveles cada vez más
profundos de generalización y abstracción, y estableció una técnica más y más
complicada, empezó a presentarse a los ojos de muchas personas como una
estructura independiente, como una creación exclusiva de la mente humana. La
ciencia solía desarrollarse en el salón académico, separada del proceso social.
Se descuidaba así su función social y la matemática participó más que las otras
ciencias en este divorcio de la vida. La creciente especialización del
matemático individual tendió a reforzar este aislamiento. La reconciliación del
materialismo con el idealismo en Kant encontró partidarios entre los
matemáticos, y la afirmación de Kant respecto al carácter apriorístico de la
geometría euclidiana fue durante muchos años un poderoso obstáculo para la
aceptación de la geometría no euclidiana. En la introducción de Hermann
Grassmann a su obra \textit{Ausdehnungslehre}, publicada en 1844, encontramos
cierto matiz kantiano en su creencia de que la matemática es solamente una
cuestión de pensamiento:

\begin{quote}\small El pensar existe solo en relación con un ser que se alza
frente a él y del que es una imagen, pero este ser es para las ciencias reales
una entidad independiente que existe por sí misma fuera del pensar, aunque para
las ciencias formales es planteado por el pensar mismo, que nuevamente se
contrapone como un segundo acto del pensar. Puesto que la verdad en general
consiste en la conformidad del pensar con el ser, en las ciencias formales
estriba especialmente en la conformidad del segundo acto del pensar con el ser
planteado por el primer acto, y por ende, en la conformidad de ambos actos del
pensar. Por consiguiente, en las ciencias formales, la prueba no trasciende del
pensar mismo a otra esfera, sino que radica puramente en la combinación de los
diferentes actos del pensamiento. Las ciencias formales, por lo tanto, no pueden
apartarse de los principios, como las ciencias reales, sino que sus fundamentos
son formados por las definiciones. Las ciencias formales estudian, sea las leyes
generales del pensamiento, sea la entidad especial planteada por el pensar. Las
primeras corresponden a la dialéctica (a la lógica) y la segunda a la matemática
pura. \end{quote}

A este aislamiento de la matemática con respecto a la vida podemos atribuir
también lo que Klein ha llamado el «nuevo humanismo»  de Jacobi, que veía el
único fin de la ciencia en el «honor del espíritu humano», sin reclamar para
ella ninguna función social:

\begin{quote}\small Cierto es que Monsieur Fourier tenía la opinión de que el
principal objetivo de la matemática consistía en la utilidad pública y en la
explicación de los fenómenos naturales, pero un filósofo como él debería haber
sabido que el único objetivo de la ciencia es el honor del espíritu humano, y
que, desde este punto de vista, una cuestión relativa a los números vale tanto
como una cuestión relativa al sistema del mundo. \end{quote}

En este período comenzó una nueva y minuciosa crítica de los fundamentos del
cálculo. Se comprendió la vaguedad que rodeaba a concepciones tales como límite,
continuidad, convergencia e infinitesimal, vaguedad que no había sido eliminada
ni por la posición «mística»  de Newton y Leibniz, ni por la «racional»  de
D'Alembert, ni por la «algebraica» de Lagrange. Esto condujo realmente a
teoremas falsos, los cuales a su vez, podían haber llevado a erróneas
aplicaciones en la mecánica y la física. El programa de esclarecimiento de los
fundamentos del cálculo fue llevado a cabo con éxito por Cauchy, Abel, Riemann,
Weierstrass y otros muchos matemáticos, con el resultado de que a partir de
entonces pudiera usarse el cálculo con confianza. Fácil es ver cómo los
matemáticos, profundamente consagrados a la realización de este programa de
sutil razonamiento, pudieron tener la impresión de que se entregaban libremente
a un juego de la mente humana. Fácil es ver también, retrospectivamente, que
estaban edificando el cálculo como un instrumento más perfecto para establecer
relaciones objetivas, y gracias a su obra el cálculo puede aplicarse ahora con
gran confianza a problemas extraordinariamente sutiles de la mecánica celeste,
de la física matemática y de la teoría de probabilidades.

Esta labor sobre los fundamentos del cálculo constituyó sólo un aspecto, aunque
muy importante, de todo el replanteamiento de los principios de la matemática
que se inició en el siglo XIX y ha continuado hasta nuestros días. Ello condujo,
en la segunda mitad del siglo XIX, a una verdadera lucha entre diferentes
opiniones concernientes a la naturaleza de la matemática. «¿Qué son los
números?», preguntaba Dedekind. «¿Qué es un límite?», preguntaba Du
Bois-Reymond. «¿Cual es la relación de la matemática con la lógica?», preguntaba
Frege. Tales preguntas atrajeron una atención renovada sobre las antiguas
dificultades de Zenón y Eudoxio, así como en cuanto a las de Newton y Lagrange
con respecto a la naturaleza del cálculo. La introducción de nuevas y sutiles
concepciones matemáticas ---las cortaduras  de Dedekind, la lógica matemática,
la teoría de conjuntos--- pareció colocar a la matemática a ritmo acelerado como
una ciencia de pensamiento puro, relacionada sólo incidentalmente con la
naturaleza. La matemática, para muchos de sus investigadores, parecía no ser
otra cosa que un esquema formal, y de aquí sólo había un paso a la creencia de
que la matemática no era nada más que un juego. Ya en 1882, Du Bois-Reymond
lanzó una advertencia contra esta posición:

\begin{quote}\small Un esqueleto de análisis puramente formalista y literal, al
que conduciría la separación del número y los signos analíticos de la cantidad,
degradaría esta ciencia ---que, en verdad, es una ciencia natural--- hasta
reducirla a un simple juego de signos, en el que podrían asignarse significados
arbitrarios a los signos escritos, como a las figuras del ajedrez o a los naipes
de la baraja. \end{quote}

Hacia esta misma época, Benjamín Peirce había definido la matemática como «la
ciencia que extrae conclusiones necesarias», recalcando el razonamiento formal a
expensas del contenido. Su hijo, C. S. Peirce, que fue uno de los hombres que
establecieron la lógica matemática, llegó más lejos. Para él, la matemática era
«el estudio de construcciones ideales»:

\begin{quote}\small Puesto que las observaciones recaen simplemente sobre
objetos de la imaginación, los descubrimientos de la matemática son susceptibles
de alcanzar total certidumbre. \end{quote}

Esto constituía el reverso directo del antiguo sentimiento materialista de los
grandes matemáticos de los pasados siglos, puesto que basaba la verdad de la
matemática en su ausencia de correspondencia con la realidad objetiva. Era otra
versión de la interpretación de «juego de ajedrez» contra la cual había
prevenido Du Bois-Reymond. Bajo tal interpretación, la inmensa aplicabilidad de
la matemática a la mecánica y a la física se convirtió solamente en una notable
coincidencia. Posteriormente, Bertrand Russell, el jefe de la escuela logística
que llevó a cabo el programa de Peirce, definió la «matemática pura»  como «la
clase de todas las proposiciones de forma $p$ implica $q$», reduciendo todas las
proposiciones de la matemática a proposiciones de lógica. La escuela formalista,
que se concentró en la tarea de demostrar que la estructura matemática puede
obtenerse mediante una serie de deducciones hipotéticas tomadas de axiomas no
interpretados, no vio a menudo en la matemática nada más que eso, aunque
Hilbert, su fundador, vio claramente el contenido real en la forma abstracta. Y
el jefe de la escuela intuicionista, Brouwer, interpretó también la matemática
como un producto exclusivamente del cerebro:

\begin{quote}\small La matemática es una creación libre, independiente de la
experiencia: emana de una sola intuición original apriorística, que puede
llamarse, o constancia en cambio, o unidad en la realidad. \end{quote}

Esta «intuición original» corresponde, como la base de un método abstracto de
construcción matemática, a una dialéctica de la realidad, y en la obra de
Brouwer condujo a descubrimientos matemáticos. Sin embargo, negó significación
matemática a considerables secciones de la matemática, y en realidad, no sólo
mutiló su aplicabilidad, sino que las privó también de algunos de sus mejores
resultados. Ello suscitó la ira de Hilbert: «Nadie nos arrojará del paraíso que
Cantor creó para nosotros». Hilbert fue especialmente quien, en la labor que
realizó en busca de un fundamento impecable de la totalidad de la matemática,
desarrolló un aparato puramente formal de signos y reglas de operación, pero
recalcó siempre la armonía entre forma y contenido, entre teoría y práctica,
entre signo humano y mundo objetivo.

La propensión de ciertos matemáticos a sacrificar el contenido a la forma o a
considerar su ciencia como separada de la naturaleza, fue ---y sigue siendo---
parte integrante de un movimiento académico general que encuentra en las
concesiones al idealismo, la solución, no sólo de sus inquietudes científicas y
filosóficas, sino también de algunas de sus preocupaciones sociales. El hecho de
que en ciertos círculos de matemáticos ---es decir, aquellos que se interesaban
por las cuestiones de fundamentación--- existiera una tendencia a cultivar la
forma más bien que el contenido y a reducir las matemáticas a un mero esqueleto
de proposiciones, fue recibido como una nueva liberación. El propio Russell
fundó toda una filosofía en su «atomismo lógico», que con toda modestia le
parecía «la misma clase de progreso que el que introdujo en la ciencia Galileo».
Esta tentativa de Russell y otros por entender el universo partiendo de un
esquema formal de lógica sirvió para que, por conducto de Mach y Avenarius, se
recayera eventualmente en el idealismo subjetivo de Berkeley, y llevó a Moritz
Schlick a la convicción de que el espacio y el tiempo son subjetivos, a Rudolf
Carnap a la creencia de que muchas enunciaciones, tales como «el tiempo es
infinito», no son enunciaciones acerca del mundo, sino sólo acerca de la manera
en que empleamos el lenguaje, y finalmente a L. Wittgenstein al descubrimiento
de que la matemática sólo es una tautología. Esto significa, según las palabras
del profesor Mannoury, que es muy sencillo escribir un libro matemático
perfecto, como por ejemplo:

\[ a = a = a = a^3 \]

Vemos todas estas concesiones al formalismo y al idealismo en el último ejemplo
como el resultado de la separación de la escuela y la vida, común bajo el
capitalismo moderno. Pero tendencias similares existen, no sólo en el terreno
filosófico, sino también en el histórico. La matemática es descrita a menudo
como una sucesión pura de ideas y desarrollada sin ninguna referencia al
escenario social en el que se manifestó. Como resultado, no se hace el menor
esfuerzo por dar respuesta a algunas de las más importantes funciones. ¿Por qué
se interrumpió el elemento creador en la matemática griega con el crecimiento
del imperio romano? ¿Por qué se originó el cálculo en el siglo XVII en Europa, y
no en Bagdad bajo los califas? Montucla, en el siglo XVIII, intentó todavía
situar la historia de la matemática en un escenario más amplio, mostrando su
relación con la mecánica y la astronomía, pero gran parte de esta actitud
desapareció en el siglo XIX, aunque dejó algunos rastros en Cantor. Uno de los
métodos de las conferencias de Klein sobre la historia de la matemática en el
siglo XIX consiste en que rara vez omite señalar las relaciones entre la
matemática y las ciencias aplicadas e incluso presta atención a las tendencias
filosóficas y sociológicas.

Frente al método idealista de abordar el estudio de la matemática, la concepción
materialista recalca:

\begin{enumerate}

\item  La influencia decisiva de los factores económico-sociales en el
moldeamiento del carácter general de la productividad matemática (o de su falta
de productividad) en un escenario histórico dado. Ya hemos esbozado cómo puede
conducirse ese estudio.

\item  La imposibilidad de separar la forma y el contenido, entrelazados en
todos los tiempos, ya que el contenido dirige igualmente el progreso de la forma
y decide su importancia y supervivencia. Esta relación de forma y contenido
impregna la totalidad de la matemática.

\end{enumerate}

Cierto es que el matemático crea sus símbolos y las leyes de operación a los que
aquéllos están sujetos. Parece ser un agente libre, un Dios que crea su propio
mundo. Sus únicas leyes son las de la consecuencia: su estructura debe estar
libre de contradicciones internas y nunca puede conducir a \mbox{$1 \neq 1$.}
Muchos matemáticos, como hemos visto, se detienen aquí y sacan la conclusión de
que están dedicados a un juego de signos desprovistos de sentido o incluso a un
juego de ficciones (como ha pretendido el filósofo Vaihinger).

Otros, llevados por la necesidad inherente a la estructura, pretenden que ésta
representa alguna característica de la mente humana o piensan que están
consagrados a una tautología. Pero la forma y el contenido son inseparables y
están entrelazados incesantemente. Hay una matemática «con sentido», que se
diferencia de un juego. Es aplicable a los procesos de la naturaleza, no de una
manera, sino de muchas. Conduce de un aspecto general a otro y establece
conexiones entre diferentes actitudes o diferentes campos. Puede ocurrir que el
lado formal triunfe durante algún tiempo y entonces la construcción libre parece
prevalecer, hasta que el contenido se empareja con la forma y dicta a su vez la
trayectoria del progreso. La historia de la matemática muestra cómo el contenido
triunfa a la larga. Nuestra matemática actual, con todas sus abstracciones y
estructuras aparentemente libres, tiene una ramificación y una interconexión
internas, y una aplicabilidad a los procesos de la naturaleza como jamás ha
tenido. La busca de la consecuencia debe reflejar las leyes del mundo externo.

El criterio final de la verdad debe ser siempre su correspondencia con la
realidad. Una teoría es cierta cuando sus teoremas pueden describir o predecir
acontecimientos del mundo real. Esto se comprende fácilmente por lo que se
refiere a los comienzos de la matemática, pero ¿cuál es la situación tratándose
de la matemática moderna, que acepta como criterio de la verdad la ausencia de
contradicciones? ¿Existe alguna relación entre el criterio objetivo (el
«filosófico») de la verdad y el criterio matemático?

No es este el lugar de examinar las diferentes formulaciones que se han dado a
la lógica matemática. Sin embargo, debemos indicar que la lógica no consiste en
el libre juego de símbolos, en una serie de convenciones de creación humana,
sino que es de por sí fundamental aspecto del mundo real tal como se refleja en
la mente humana. Nos encontramos nuevamente con la cuestión de forma y
contenido. Tómese, por ejemplo, el concepto de identidad, $x$ equivale a $y$,
con sus leyes de reflexividad $x = x$, de simetría (si $x = y$, entonces $y =
x$), y de transitividad ($x = y$ e  $y = z$, luego $x = z$). Esto significa
sobre el papel la suposición de que podemos reconocer un signo siempre que se
presenta. Al mismo tiempo, expresa una cualidad esencial del mundo, a saber, la
invariación de un objeto. El mundo fluye eternamente, de modo que dos cosas
nunca son iguales y ni siquiera una sola cosa sigue siendo la misma. Con todo,
es posible separar aspectos especiales de este mundo y tratarlos como si no
cambiaran. El profesor Levy ha llamado a estos aspectos «aisados»  y los
«aislados» pueden permanecer invariantes y convertirse en objeto de discurso.
Una línea recta conserva su identidad para todos nosotros y para todas las
generaciones, lo mismo que ocurre con el número 4, la esfera, o una integral. La
pelota, que me ha dado la concepción de una esfera, puede moverse, mojarse o
pudrirse, pero la esfera misma es una concepción invariante. Las leyes de
identidad expresan el hecho de que yo puedo llegar a conclusiones exactas
conservando sin modificaciones a través del discurso el elemento «aislado». Es
evidente que estas leyes de lógica formal crean la posibilidad del entendimiento
y de la actividad humanas, pero no establecen en modo alguno un divorcio entre
la matemática y el mundo real. Antes al contrario, hacen posible que la
matemática presente una imagen de la realidad.

La lógica formal nos permite establecer relaciones entre objetos («ais\-lados»)
matemáticos y no matemáticos; pero después de todo se trata de relaciones
trilladas (lo que no quiere decir que su estudio sea trivial) y, aun cuando su
uso pueda permitir establecer proposiciones cada vez más complicadas,
difícilmente harán de la matemática una ciencia creadora. La antigua ciencia
oriental, con su estrecha conexión entre la matemática y la realidad, era ya
infinitamente más rica. ¿En qué consiste, pues, el secreto de la matemática
creadora? En otras palabras, ¿cuál es su contenido? ¿Por qué no es la matemática
simplemente una tautología impresionante como ha preguntado y negado Poincaré y
como ha preguntado y confirmado Wittgenstein?

\section*{La matemática, creación incesante}\addcontentsline{toc}{section}{La
matemática, creación incesante}

La matemática, como aspecto del mundo real, participa de su dialéctica. La
dialéctica implica creación incesante. Por su misma naturaleza, la matemática
es, pues, creadora, trascendiendo constantemente las tautologías que pueden
surgir en su estructura. Podemos expresar esto diciendo que la matemática se
basa en algo más que la lógica formal. La lógica de la matemática, y
especialmente de la matemática en su desarrollo, es una lógica dialéctica. Gran
parte de la labor realizada en los últimos 30 años para la fundamentación de la
matemática ha sido estimulada por la necesidad de expresar en formulación exacta
el significado de este elemento creador de la matemática. Los intuicionistas han
recalcado el principio de la inducción completa; los formalistas han introducido
los «axiomas trascendentales»  y se han opuesto enérgicamente a las tentativas
de los logísticos por reducir la matemática a una tautología. La lógica
matemática moderna, en comparación con la antigua lógica aristotélica, se ha
enriquecido mucho más en el proceso. Sin embargo, es dudoso que pueda
encontrarse un sistema de verdades axiomáticas que permita explicar todos los
aspectos de la matemática: la realidad tiene más facetas que las que un Horacio
axiomaticista sueña en su filosofía. El formalismo de los axiomas se ha
sublevado contra las tentativas por reducir la matemática a una tautología, como
lo prueban las investigaciones de K. Godel y otros. Aunque todas las escuelas de
matemática se han anotado resultados, aún no se han agotado todas las
posibilidades de la matemática, la cual se ha negado hasta ahora y probablemente
se negará siempre a quedar encerrada dentro de una serie de axiomas y de reglas
interdependientes.

Engels hizo notar que los juicios de la lógica dialéctica de Hegel, a pesar de
la arbitrariedad de su clasificación, expresan modos universales de obte- ner
nueva información, apareciendo una vez más el desarrollo del pensamiento como el
desarrollo de la obtención de información objetiva.  Estos juicios pueden
considerarse como determinaciones de singularidad, particularidad y
universalidad.

Esto puede proporcionarnos, en efecto, un cuadro de la forma en que se han
desarrollado las concepciones matemáticas. Por ejemplo, nuestra moderna noción
de geometría, nacida, primeramente, de la observación de que las rotaciones
conservan las propiedades métricas de una figura (preeuclidiana), lo cual es una
determinación de singularidad. En segundo lugar, se descubrió que ciertas
propiedades se mantienen invariantes bajo diferentes transformaciones, tales
como las proyectivas, afines y rotacionales (principios del siglo XIX), lo cual
es una determinación de particularidad, y, en tercer lugar, toda geometría es la
teoría de los invariantes de cierto grupo (programa de Erlanger, 1872), lo cual
puede considerarse como una determinación de universalidad. Tales
consideraciones pueden captarse en lenguaje formal, como lo prueba una secuencia
como la siguiente: 1) siete es primo; 2) hay muchos primos; 3) el número de
primos es infinito.

El progreso de la matemática se lleva a cabo a lo largo de muchos senderos,
todos los cuales pueden considerarse como juicios dialécticos  y algunos de los
más fértiles pueden describirse así:

\begin{enumerate}

\item  Pasando de lo específico a lo general v. g., de la teoría de las
ecuaciones cuadradas, cúbicas, etc., a la teoría de Galois:

\begin{quote}\small El mismo tema de la matemática ya no puede describirse como
si se tratara simplemente de forma y número. Estas concepciones constituyen aún
el fundamento de toda la matemática, pero temas tales como la geometría
proyectiva, la topología, la teoría de grupos, rebasan esta clasificación.
Asimismo han desaparecido los límites entre la matemática y la lógica.
\end{quote}

\item Pasando de lo discontinuo a lo continuo, de lo finito a lo infinito, v.
g., de sumas a integrales, de los números racionales a los reales, o al uso del
principio de la inducción matemática:

\begin{quote}\small La trascendencia de lo finito a lo infinito ha sido uno de
los pasos más creadores que se han dado en el desarrollo, no sólo de la
matemática, sino del entendimiento en general. «Lo infinito ha removido más
profundamente que cualquier otra cuestión la mentalidad humana, ha operado de
manera más estimulante que cualquier otra idea sobre el espíritu del hombre, y,
sin embargo, necesita ser dilucidado más que cualquier otra concepción»,
escribió Hilbert. La tentativa de Hilbert por demostrar que lo infinito puede
ser eliminado, no sólo de la matemática ---tentativa hecha ya por Eudoxio---
sino también de las ciencias naturales ---«lo infinito no se realiza en ninguna
parte»--- no parece convincente: lo infinito tiene una manera propia de derribar
todos los límites establecidos por el hombre, en un período histórico dado, como
base de su entendimiento. \end{quote}

\item Pasando de la cantidad a la cualidad, v. g., de la geometría métrica a la
proyectiva, y, por ende, a la topología. Al pasar de los números positivos a los
negativos, introducimos en la aritmética la noción de dirección unidimensional,
y al pasar de los números reales a los complejos, introducimos un simbolismo
para denotar un cambio de dirección bidimensional:

\begin{quote}\small La matemática, aunque nacida del estudio del número y de la
forma, se ha emancipado de tal manera de sus orígenes, que pocos matemáticos se
muestran ya inclinados a definir su ciencia como la del número y la forma. La
misma lógica del \textit{quantum} ha conducido a la cualidad de muchas maneras.
Leibniz fue quizás el primero que se percató de este aspecto del desarrollo de
la matemática. «Creo que necesitamos otro análisis, propiamente geométrico o
lineal, que nos exprese directamente el sitio, como el álgebra expresa la
magnitud». Sin embargo, se ha manifestado una tendencia a demostrar que la
matemática puede construirse solamente sobre los números naturales, curiosa
vindicación del punto de vista de Hegel de que el fin y la noción de la
matemática es la cantidad, junto con una repudiación de su criterio de que en la
matemática no hay desarrollo y relación de ideas. \end{quote}

\item  El principio de la permanencia de las leyes formales, v. g., cuando
pasamos de la aritmética al álgebra de los números reales, y luego al álgebra de
los sistemas de números complejos e hipercomplejos, y del dominio «real» al
«ideal».

\begin{quote}\small Este principio fue formulado por Hermann Hankel en estas
palabras: «Cuando dos formas, expresadas en símbolos generales de la
\textit{arithmetica universalis}, son iguales entre sí, lo serán también cuando
los símbolos dejen de denotar cantidades simples y, por ende, las operaciones
reciban también otro contenido.» \end{quote}

\item  El principio de la permanencia de las relaciones matemáticas, llamado así
por V. Poncelet, quien se inspiró en el descubrimiento de la geometría
proyectiva (el «principio de continuidad», por ejemplo, permitió a Poncelet
pasar de los teoremas geométricos conocidos a otros nuevos o relacionar entre sí
los teoremas conocidos):

\begin{quote}\small Una aplicación especial de este principio fue la labor
realizada en geometría con elementos imaginarios, que condujo a Cayley y Klein
al descubrimiento de que la geometría métrica forma parte de la geometría
proyectiva, después de que Poncelet había derivado la segunda de la primera.

También podemos situar bajo este principio la concepción fundamental del
isomorfismo, que nos permite interpretar de diferentes maneras el mismo sistema
de relaciones formales entre elementos abstractos. De ello se tienen ejemplos en
la polaridad de la geometría proyectiva y en la prueba aportada por Hilbert
sobre el carácter no contradictorio de la geometría euclidiana aplicando ésta a
un álgebra lineal por medio de coordenadas. Aquí, la forma de un sistema
corresponde a una multiformidad de contenidos, y sólo recibe su vida del
contenido. \end{quote}

\item  El principio de conservación del lenguaje matemático, v. g., cuando
pasamos de tres sistemas paramétricos aplicables al espacio a cuatro sistemas
paramétricos, utilizando nuevamente términos tales como «punto» y «línea» para
un «espacio»  de cuatro dimensiones.

\end{enumerate}

Se ha escrito mucho acerca de la confusión que existe en el uso del lenguaje
también en la matemática. La lógica matemática ha introducido símbolos para
eliminar palabras tales como «o », «y », «todos »y «hay». Esta tendencia ha
oscurecido a veces el carácter creador del uso de las palabras. Atribuyendo
concepciones más amplias a palabras tales como «punto», «espacio»  y «vector»
se ha facilitado considerablemente la investigación en nuevos campos. Un buen
ejemplo se encuentra también en el uso de expresiones tales como «espacio
funcional», «funciones ortogonales» y «densidad de probabilidad».

\begin{quote}\small La negativa de los griegos a extender a lo irracional el
término «número» tuvo una influencia perjudicial sobre el desarrollo de su
matemática. Esto no debe considerarse simplemente como un accidente
desafortunado. Más bien fue el resultado de la importancia que los griegos
concedieron a la geometría abstracta, resultado, a su vez, de la aparente
renuencia de algunos de sus prominentes matemáticos a aceptar la concepción
oriental de la aritmética y del álgebra, lo cual sólo puede explicarse por la
división social entre griegos y orientales. \end{quote}

Sin embargo, el hecho de que se empleen símbolos y palabras para expresar
concepciones matemáticas no debe llevar a una asimilación como la de que «la
matemática es el lenguaje del tamaño» la cual sobrestima el aspecto formal de la
matemática y trata como incidental su contenido, su aspecto creador como imagen
del mundo real.

Podría continuarse esta enumeración de algunos de los principios creadores de la
matemática y analizarse aquellos casos en que se superponen. Cuando estos
principios creadores se aplican inconscientemente o semiconscientemente, podemos
hablar de intuición matemática. El estudio de esta intuición e inventiva
matemática ha sido iniciado recientemente por J. Hadamard. El sentimiento de
armonía con el universo que experimenta el matemático creador expresa la
concordancia de sus pensamientos con el mundo. Aquí, pisa un terreno común al
del artista y el místico.

Por consiguiente, el carácter deductivo con el que se presenta ante el público
la matemática bien organizada no debe impedirnos ver el carácter esencialmente
inductivo en el que se han obtenido sus resultados. Esto equivale, en última
instancia, a considerarla como un estudio de la naturaleza ---aun cuando sólo se
trate de algunas de sus características más abstractas, menos susceptibles de
experimentación fuera de los niveles elementales y ajenas a la transformación
histórica---, que proporciona la respuesta básica a la cuestión tantas veces
citada de Poincaré:

\begin{quote}\small La posibilidad misma de la ciencia matemática parece una
contradicción insoluble. Si esta ciencia solo es deductiva en apariencia, ¿de
dónde le viene ese rigor absoluto que a nadie se le ocurre discutir? Si por el
contrario todas las proposiciones que establece pueden obtenerse unas de otras
mediante las reglas de la lógica formal, ¿por qué no se reduce la matemática a
una inmensa tautología? \end{quote}

Sólo añadiremos aquí que el «rigor absoluto»  no existe. Los criterios del rigor
en la matemática son de carácter histórico: lo que parecía riguroso a Euclides o
a Gauss no nos lo parece a nosotros. Y cabe esperar ---los axiomáticos modernos
parecen confirmarlo--- que, aun en el caso de que la matemática pudiera
reducirse a una tautología a los ojos de una generación, volvería a evadirse de
los confines de su prisión en una fecha ulterior.

\section*{La experimentación directa y los descubrimientos
matemáticos}\addcontentsline{toc}{section}{La experimentación directa y los
descubrimientos matemáticos}

La prueba de la verdad en la matemática ---su carácter no contradicto\-rio--- es
la prueba de su aplicabilidad al mundo real. Esta no es una afirmación que puede
probarse como un teorema: la prueba entre idealismo y materialismo no puede
decidirse sobre el papel o en teoría, sino que tiene que decidirse también en la
práctica de la vida. Los hechos de que la matemática es posible, de que es
aplicable a la naturaleza y de que tiene una función social están todos ellos
entrelazados.

Esto no constituye un punto de vista utilitario, porque la matemática sólo puede
desarrollarse cuando puede ser explorada sin perseguir conscientemente
aplicaciones inmediatas, ni significa tampoco que la matemática sea una ciencia
empírica, como pretendía Mill. En tanto que la matemática sea empírica, no puede
atribuírsele la categoría de ciencia.

Esto no excluye el hecho de que los descubrimientos matemáticos han sido
obtenidos por medio del experimento directo. En primer lugar, cada generación
tiene que aprender por sí misma y sometiéndolos a la prueba directa de la
realidad los conceptos fundamentales de la matemática. Nuestra convicción de la
validez eterna de los teoremas de Pitágoras o del hecho de que $2 \cdot  2 = 4$,
no se basa en alguna concepción a priori ni puede ser quebrantada por cualquier
matemático habilidoso que en un gran libro repleto de fórmulas saque la
conclusión de que estos teoremas son simples convencionalismos. Nuestra
convicción se basa en el hecho de que los teoremas corresponden a propiedades
del mundo real ajenas a nuestra conciencia, las cuales pueden ser comprobadas y
cuya comprobación es accesible a todas las personas desde su temprana juventud.
La demostración de los teoremas de Pitágoras y las tentativas por probar que $2
\cdot  2 = 4$ que han existido desde los días de Leibniz sirven para conectar
estas verdades con otras todavía más elementales y todavía más fáciles de
probar. La creación de una estructura matemática es necesaria e indispensable
para el descubrimiento de otras verdades menos obvias y más difíciles de probar.

En los tiempos modernos se han hecho relativamente pocos descubrimientos
matemáticos por medio del experimento y esos descubrimientos se han logrado en
gran parte indirectamente al intentar explicar los fenómenos de la física. Un
ejemplo lo constituye la superficie ondular de Fresnel, descubierta al estudiar
la doble refracción en medios cristalinos. Otros resultados indirectos de la
experimentación son las numerosas soluciones de las ecuaciones diferenciales
obtenidas en el estudio de los problemas eléctricos o mecánicos. Ciertas
propiedades de la factorización de grandes números han sido descubiertas por
procedimientos mecánicos y el reciente desarrollo de las máquinas calculadoras
puede conducir a más descubrimientos. Algunos teoremas geométricos han sido
descubiertos por medio de modelos de yeso y alambre (v. g., el teorema de
Henrici, de 1874, sobre la movilidad de las líneas rectas en un hiperboloide).
Tales casos, en la matemática moderna, son raros y hasta ahora nunca han sido de
importancia fundamental.

El hecho de que relativamente pocos descubrimientos matemáticos modernos son el
resultado de la experimentación directa no puede utilizarse para aducir que,
cualquiera que haya sido el origen de la matemática antigua, la moderna es en
todo caso un producto puro de la mente humana. No significa que la matemática
haya llegado a una etapa en que puede desarrollarse en gran medida dentro del
marco de su propia estructura sin esperar su comprobación por medio de la
experimentación. La situación puede compararse con la de la construcción de
nuevos instrumentos (lentes, filtros de ondas, etc.), que pueden diseñarse sobre
el papel con la confianza absoluta de que ejecutarán la tarea que de ellos se
espera, debido a nuestro exacto conocimiento de las leyes de la física.

Este criterio matemático específico de la verdad nos permite descubrir sobre el
papel, con lápiz y pluma, y por medio del razonamiento, relaciones que son una
imagen de las relaciones objetivas si el conjunto de axiomas de que partimos es
esa imagen. Para el razonamiento matemático no es necesario, como ya sabemos,
tomar en consideración en la investigación futura el significado objetivo de los
axiomas, aun cuando esto pueda ejercer influencia sobre la selección de los
asuntos a investigar. Puede tomarse libremente un punto como «elemento A» y una
línea como «elemento B». La elección de los axiomas en casi todos los tipos de
matemáticas actualmente estudiados estará de acuerdo con ciertas reacciones
objetivas. A este respecto, estamos de acuerdo con A. Tarski, quien es, por su
parte, un lógico:

\begin{quote}\small Así, se oye y hasta se lee de vez en cuando que no puede
atribuirse ningún contenido definido a los conceptos matemáticos, que en la
matemática no sabemos realmente acerca de qué estamos hablando y que no nos
interesa si nuestras afirmaciones son o no verdaderas. Tales juicios deben ser
vistos con desconfianza. Si en la construcción de una teoría nos conducimos como
si no entendiéramos el significado de los términos de esta disciplina, eso no
quiere decir en modo alguno que esos términos carezcan de significación... Es de
suponer que a nadie le interesaría un sistema formal respecto del cual no seamos
capaces de dar una sola interpretación. \end{quote}

Podemos añadir que hasta la misma subsistencia de los matemáticos como grupo (no
necesariamente como individuos) depende del hecho de que la matemática tenga
sentido. En caso contrario, en las nóminas de las universidades y escuelas
superiores dejaría de haber sitio para ellos. Puede que algunos ricos estuvieran
dispuestos a contratar a matemáticos como contratan ahora jugadores de ajedrez o
expertos en bridge, y tal vez otros se interesaran por ayudar a los matemáticos
como artistas. Sin embargo, sabido es que los artistas encuentran pocos
protectores bajo el capitalismo, muchos menos que los hombres de ciencia. El
papel social constructivo que la matemática desempeñó en la edificación del
capitalismo comercial e industrial fue esencialmente lo que hizo que se
fomentara su estudio, aun cuando tuviera que adoptar formas cada vez más
abstractas para llegar a planos más profundos de la realidad.

\section*{La aplicabilidad universal}\addcontentsline{toc}{section}{La
aplicabilidad universal}

Muchos matemáticos del siglo XIX creían en una matemática «pura»\ cultivada
solamente por el «honor del espíritu humano» y en un campo «aplicado» más o
menos distinto, que solía considerarse como menos digno de esfuerzo. El cuadro
del Tiziano «El amor sagrado y el profano», de la galería Borghese de Roma,
parecíales a algunos de ellos que representaba «la matemática pura y la
aplicada». Esta distinción va perdiendo rápidamente sentido como forma
fundamental de clasificación, aunque siga siendo un modo conveniente de denotar
tipos de especialización. Toda teoría matemática «pura»  ha encontrado, o puede
encontrar pronto, su campo de aplicabilidad. El matemático que sigue sobre el
papel el razonamiento estrictamente teórico de su ciencia puede descubrir en
cualquier momento que en realidad ha estudiado las leyes de algún campo de la
física, la química o la estadística. El ejemplo clásico lo constituye la teoría
de las secciones cónicas, creada por los griegos probablemente con el resultado
de su investigación de la trisección del ángulo o la duplicación del cubo,
investigación que, dos mil años después, resultó ser el estudio de las órbitas
de los planetas. Un ejemplo moderno de índole similar es el cálculo tensorial,
descubierto como un estudio de la invariación de las formas diferenciales
cuadráticas, que varias décadas después contribuyó a proporcionar una
descripción del universo con arreglo a la teoría de la relatividad de Einstein.
Ahora sabemos también que los tensores expresan nociones fundamentales en la
dinámica, la elasticidad y la hidrodinámica. El cálculo tensorial puede
aplicarse asimismo a las máquinas eléctricas giratorias y a la comparación de
los colores, y algunas de sus concepciones más elementales han encontrado ya su
camino en la psicología.

Tales ejemplos pueden multiplicarse centenares de veces. Una de las concepciones
más abstractas de la matemática moderna es la medida de Lebesgue, que a algunos
matemáticos les pareció la misma culminación de su ciencia como juego puro del
espíritu. Sin embargo, ahora sabemos que constituye la clave para la comprensión
de la teoría de la probabilidad y de la estadística, y esto a su vez es
indispensable para la explicación de fenómenos tan distintos como el movimiento
browniano de las partículas en suspensión líquida y la conducta de largo alcance
de los sistemas mecánicos,  que muestra la conexión entre la obra de Gibbs sobre
la mecánica estática y la de Lebesgue sobre integración.

Jacobi observó ya cómo las funciones \textit{theta} describen relaciones
profundamente ocultas en la teoría de números con los movimientos del péndulo.
Ahora, la teoría de números, uno de los campos más abstractos de la matemática,
basada en la llamada función \textit{theta} de Riemann, encuentra su camino en
problemas de ingeniería eléctrica así como en algunas partes de la física. El
álgebra matricial proporciona la base de la mecánica del \textit{quantum} y de
la conducta de los átomos en acción. Y Hilbert ha indicado que ciertas leyes de
la genética, en el caso de la pequeña mosca \textit{drosophila}, obedecen a los
axiomas concernientes a la noción de «entre», así como a los axiomas euclidianos
lineales de congruencia.

Por lo tanto, es posible sostener, como a veces se cree (y la concepción
idealista subjetiva de la matemática debe conducir necesariamente a ello) que el
amplio radio de aplicabilidad de la matemática a la naturaleza y a la sociedad
es simplemente una coincidencia. La aplicabilidad de la matemática es demasiado
general, demasiado universal, para tomar en serio semejante alegato. Cuando
aumenta en tales proporciones el número de esos «accidentes», «coincidencias»  y
«golpes de suerte»  debemos buscar una explicación más racional. Todos los
hombres de ciencia están consagrados a la tarea cooperativa de descubrir el
comportamiento del mundo, y aun cuando a veces puedan divergir considerablemente
los caminos que siguen y los resultados que obtienen, la unidad fundamental del
universo hace posible la armonía eventual de todos los resultados. El autor está
convencido de que todos los hombres de ciencia serios comparten la creencia en
esta unidad fundamental de todas las cosas que existen y de que esa creencia les
comunica su profunda convicción en que sus investigaciones están repletas de
sentido.

Tanto la práctica como la teoría han justificado así ampliamente nuestra
creencia de que toda la actividad matemática es valiosa en el sentido de que
proporciona una imagen de algún aspecto del mundo real. Ello parece coincidir
con la opinión de Leibniz, según la expresó Schrecker:

\begin{quote}\small Mientras que según los utilitarios, el conocimiento es
verdadero porque es útil, sería más adecuado, a juicio de Leibniz, decir que si
el conocimiento es verdadero tiene que ser útil y que, por consiguiente, la
utilidad solo es una confirmación, pero no un elemento constitutivo de la
verdad. El conocimiento verdadero, en efecto, representa el orden real y
objetivo del universo. \end{quote}

Debido a esta íntima relación entre pensamiento y naturaleza, entre experimento
y teoría, revelada por la matemática moderna, Hilbert se mostró dispuesto a
aceptar la noción de la «armonía preestablecida»\ formulada por Leibniz; pero en
la relación entre la matemática teórica y sus aplicaciones vio todavía algo más
que esta «armonía preestablecida». Sin la matemática, es imposible la astronomía
y la física de nuestro tiempo: estas ciencias, en sus partes teóricas, se
disuelven, por así decir, en la matemática.

A esta unidad del pensamiento y de la naturaleza es a lo que ha conducido el
desarrollo de la matemática moderna.

\end{document}
