\documentclass[a4paper, 12pt]{article}

%%%%%%%%%%%%%%%%%%%%%%Paquetes
\usepackage[spanish]{babel}  
\usepackage[utf8]{inputenc}
\usepackage{tcolorbox}
\usepackage{cmbright}  %%%%%%% El tipo de letra
\usepackage{setspace}
\onehalfspacing  %%%%%%%%%%% Espacio y medio de interlineado
\parskip=1em  %%%%%%%%%%%% Separacion entre parrafos
%%%%%%%%%%%%%%%%%%%%%%



%%%%%%%%%%%%%%%%%
\title{El Sentido Histórico de la Teoría de Einstein}
\author{J. Ortega y Gaset}
\date{}
%%%%%%%%%%%%%%%%%

\begin{document}

\begin{tcolorbox}[colback=blue!5!white,colframe=blue!75!black]

\vspace{-1.8cm}
\textbf \maketitle

\end{tcolorbox}

\bigskip







La teoría de la relatividad, el hecho intelectual de más rango que el
presente puede ostentar, es una teoría, y, por tanto, cabe discutir si es
verdadera o errónea. Pero, aparte de su verdad o su error, una teoría es
un cuerpo de pensamientos que nace en un alma, en un espíritu, en una
conciencia, lo mismo que el fruto en el árbol. Ahora bien, un fruto nuevo
indica una especie vegetal nueva que aparece en la flora. Podemos, pues,
estudiar aquella teoría con la misma intención que el botánico cuando
describe una planta: prescindiendo de sí el fruto es saludable o nocivo,
verdadero o erróneo, atentos exclusivamente a filiar la nueva especie, el
nuevo tipo de ser viviente que en él sorprendemos. Este análisis nos
descubrirá el sentido histórico de la teoría de la relatividad, lo que
ésta es como fenómeno histórico.

Sus peculiaridades acusan ciertas tendencias específicas en el alma que la
ha creado. Y como un edificio científico de esta importancia no es obra de
un solo hombre, sino resultado de la colaboración indeliberada de muchos,
precisamente de los mejores, la orientación que revelen esas tendencias
marcará el rumbo de la historia occidental.

No quiero decir con esto que el triunfo de esta teoría influirá sobre los
espíritus, imponiéndoles determinada ruta. Esto es evidente y banal. Lo
interesante es lo inverso: porque los espíritus han tomado espontáneamente
determinada ruta, ha podido nacer y triunfar la teoría de la relatividad.
Las ideas, cuanto más sutiles y técnicas, cuanto más remotas parezcan de
los afectos humanos, son síntomas más auténticos de las variaciones
profundas que le producen en el alma histórica.

Basta con subrayar un poco las tendencias generales que han actuado en la
invención de esta teoría, basta con prolongar brevemente sus líneas más
allá del recinto de la física, para que aparezca a nuestros ojos el dibujo
de una sensibilidad nueva, antagónica de la reinante en los últimos
siglos.

\subsubsection*{1.- Absolutismo}

El nervio de todo el sistema está en la idea de la relatividad. Todo
depende, pues, de que se entienda bien la fisonomía que este pensamiento
tiene en la obra genial de Einstein. No sería falto de toda mesura afirmar
que éste es el punto en que la genialidad ha insertado su divina fuerza,
su aventurero empujón, su audacia sublime de arcángel. Dado este punto, el
resto de la teoría podía haberse encargado a la mera discreción.

La mecánica clásica reconoce igualmente la relatividad de todas nuestras
determinaciones sobre el movimiento, por tanto de toda posición en el
espacio y en el tiempo que sea observable por nosotros. ¿Cómo la teoría
de Einstein, que, según oímos, trastorna todo el clásico edificio de la
mecánica, destaca en su nombre propio, como su mayor característica, la
relatividad? Este es el multiforme equívoco que conviene ante todo
deshacer. El relativismo de Einstein es estrictamente inverso al de
Galileo y Newton. Para éstos las determinaciones empíricas de duración,
colocación y movimiento son relativas porque creen en la existencia de un
espacio, un tiempo y un movimiento absolutos. Nosotros no podemos llegar a
éstos; a lo sumo, tenemos de ellos noticias indirectas (por ejemplo, las
fuerzas centrífugas). Pero sí se cree en su existencia, todas las
determinaciones que efectivamente poseemos quedarán descalificadas como
meras apariencias, como valores relativos al punto de comparación que el
observador ocupa. Relativismo aquí significa, en consecuencia, un defecto.
La física de Galileo y Newton, diremos, es relativa.

Supongamos que, por unas u otras razones, alguien cree forzoso negar la
existencia de esos inasequibles absolutos en el espacio, el tiempo y la
transferencia. En el mismo instante, las determinaciones concretas, que
antes parecían relativas en el mal sentido de la palabra, libres de la
comparación con lo absoluto, se convierten en las únicas que expresan la
realidad. No habrá ya una realidad absoluta (inasequible) y otra relativa
en comparación con aquélla. Habrá una sola realidad, y ésta será la que la
física positiva aproximadamente describe. Ahora bien, esta realidad es la
que el observador percibe desde el lugar que ocupa; por tanto, una
realidad relativa. Pero como esta realidad relativa, en el supuesto, que
hemos tomado, es la única que hay, resultará, a la vez que relativa, la
realidad verdadera, o, lo que es igual, la realidad, absoluta. Relativismo
aquí no se opone a absolutismo; al contrario, se funde con éste, y lejos
de sugerir un defecto de nuestro conocimiento, le otorga una validez
absoluta.

Tal es el caso de la mecánica de Einstein. Su física no es relativa, sino
relativista, y merced a su relativismo consigue una significación
absoluta.

La más trivial tergiversación que puede sufrir la nueva mecánica es que se
la interprete como un engendro más del viejo relativismo filosófico que
precisamente viene ella a decapitar. Para el viejo relativismo, nuestro
conocimiento es relativo, porque lo que aspiramos a conocer (la realidad
tempo-espacial) es absoluto y no lo conseguimos. Para la física de
Einstein nuestro conocimiento es absoluto; la realidad es la relativa.
Por consiguiente, conviene ante todo destacar como una de las facciones
más genuinas de la nueva teoría su tendencia absolutista en el orden del
conocimiento. Es inconcebible que esto no haya sido desde luego subrayado
por los que interpretan la significación filosófica de esta genial
innovación. Y, sin embargo, está bien clara esa tendencia en la fórmula
capital de toda la teoría: las leyes físicas son verdaderas, cualquiera
que sea el sistema de referencia usado, es decir, cualquiera que sea el
lugar de la observación. Hace cincuenta años preocupaba a los pensadores
si, ``desde  el punto de vista de Sirio'', las verdades humanas lo serían.


Esto equivalía a degradar la ciencia que el hombre hace, atribuyéndole un
valor meramente doméstico. La mecánica de Einstein permite a nuestras
leyes físicas armonizar con las que acaso circulan en las mentes de Sirio.
Pero este nuevo absolutismo se diferencia radicalmente del que animó a los
espíritus racionalistas en las postreras centurias. Creían éstos que al
hombre era dado sorprender el secreto de las cosas, sin más que buscar en
el seno del propio espíritu las verdaderas eternas de que está henchido.
Así, Descartes crea la física sacándola, no de la experiencia, sino de lo
que él llama el {\it trésor de mon esprit}. Estas verdades, que no proceden de
la observación, sino de la pura razón, tienen un valor universal, y en vez
de aprenderlas nosotros de las cosas, en cierto modo las imponemos a
ellas: son verdades a priori. En el propio Newton se encuentran frases
reveladoras de ese espíritu racionalista. ``En la filosofía de la
naturaleza, dice, hay que hacer abstracción de los sentidos''. Dicho en
otras palabras: para averiguar lo que una cosa es, hay que volverse de
espaldas a ella. Un ejemplo de estas mágicas verdades es la ley de
inercia; según ella, un cuerpo libre de todo influjo, sí se mueve, se
moverá indefinidamente en sentido rectilíneo y uniforme. Ahora bien: ese
cuerpo exento de todo influjo nos es desconocido. ¿Por qué tal afirmación?
Sencillamente porque el espacio tiene una estructura rectilínea,
euclidiana, y, en consecuencia, todo movimiento ``espontáneo'' que no esté
desviado por alguna fuerza se acomodará a la ley del espacio.

Pero esta índole euclidiana del espacio, ¿quién la garantiza? ¿La
experiencia? En modo alguno; la pura razón es la que, previamente a toda
experiencia, resuelve sobre la absoluta necesidad de que el espacio en que
se mueven los cuerpos físicos sea euclidiano. El hombre no puede ver sino
en el espacio euclidiano. Esta peculiaridad del habitante de la tierra es
elevada por el racionalismo a ley de todo el cosmos. Los viejos
absolutistas cometieron en todos los órdenes la misma ingenuidad. Parten
de una excesiva estimación del hombre. Hacen de él un centro del universo,
cuando es sólo un rincón. Y éste es el error más grave que la teoría de
Einstein viene a corregir.

\subsubsection*{2.- Perspectivismo}

El espíritu provinciano ha sido siempre, y con plena razón, considerado
como una torpeza. Consiste en un error de óptica. El provinciano no cae en
la cuenta de que mira el mundo desde una posición excéntrica. Supone, por
el contrario, que está en el centro del orbe, y juzga de todo como sí su
visión fuese central. De aquí una deplorable suficiencia que produce
efectos tan cómicos. Todas sus opiniones nacen falsificadas, porque parten
de un pseudocentro. En cambio, el hombre de la capital sabe que su ciudad,
por grande que sea, es sólo un punto del cosmos, un rincón excéntrico.
Sabe, además, que en el mundo no hay centro y que es, por tanto, necesario
descontar en todos nuestros juicios la peculiar perspectiva que la
realidad ofrece mirada desde nuestro punto de vista. Por este motivo, al
provinciano el vecino de la gran ciudad parece siempre escéptico, cuando
sólo es más avisado.

La teoría de Einstein ha venido a revelar que la ciencia moderna, en su
disciplina ejemplar ---la nuova scienza de Galileo, la gloriosa física de
Occidente---, padecía un agudo provincianismo. La geometría euclidiana, que
sólo es aplicable a lo cercano, era proyectada sobre el universo. Hoy se
empieza en Alemania a llamar al sistema de Euclides ``geometría de lo
próximo'', en oposición a otros cuerpos de axiomas que, como el de Riemann,
son geometrías de largo alcance.

Como todo provincianismo, esta geometría provincial ha sido superada
merced a una aparente limitación, a un ejercicio de modestia. Einstein se
ha convencido de que hablar del espacio es una megalomanía que lleva
inexorablemente al error. No conocemos más extensiones que las que
medimos, y no podemos medir más que con nuestros instrumentos. Estos son
nuestros órganos de visión científica; ellos determinan la estructura
especial del mundo que conocemos. Pero, como lo mismo acontece a todo otro
ser que desde otro lugar del orbe quiera construir una física, resulta que
esa limitación no lo es en verdad.

No se trata, pues, de reincidir en una interpretación subjetivista del
conocimiento, según la cual la verdad sólo es verdad para un determinado
sujeto. Según la teoría de la relatividad, el suceso $A$, que desde el punto
de vista terráqueo precede en el tiempo al suceso $B$, desde otro lugar del
universo, Sirio por ejemplo, aparecerá sucediendo a $B$. No cabe inversión
más completa de la realidad. ¿Quiere esto decir que o nuestra imaginación
es falsa o la del avecindado en Sirio? De ninguna manera. Ni el sujeto
humano ni el de Sirio deforman lo real. Lo que ocurre es que una de las
cualidades propias a la realidad consiste en tener una perspectiva, esto
es, en organizarse de diverso modo para ser vista desde uno u otro lugar.
Espacio y tiempo son los ingredientes objetivos de la perspectiva física,
y es natural que varíen según el punto de vista.

En la introducción al primer Espectador, aparecido en enero de 1916,
cuando aún no se había publicado nada sobre la teoría general de la
relatividad\footnote{La primera publicación de Einstein sobre su reciente descubrimiento,
Die Grundlagen der allgemeinen Retativitätstheorie, se publicó dentro de
ese año.}, exponía yo brevemente esta doctrina perspectivista,
dándole una amplitud que trasciende de la física y abarca toda realidad.
Hago esta advertencia para mostrar hasta qué punto es un signo de los
tiempos pareja manera de pensar.
Y lo que más me sorprende es que no haya reparado nadie todavía en este
rasgo capital de la obra de Einstein. Sin una sola excepción ---que yo
sepa---, cuanto se ha escrito sobre ella interpreta el gran descubrimiento
como un paso más en el camino del subjetivismo\footnote{Bastante tiempo después de publicado esto, se me ha hecho notar que
simultáneamente había aparecido una conferencia del filósofo Geiger, donde
se habla también del sentido absoluto que va anejo a la teoría de
Einstein. Pero el caso es que la tesis de Geiger apenas tiene algún punto
común con la sostenida en este ensayo.}. En todas las lenguas y
en todos los giros se ha repetido que Einstein viene a confirmar la
doctrina kantiana, por lo menos en un punto: la subjetividad de espacio y
tiempo. Me importa declarar taxativamente que esta creencia me parece la
más cabal incomprensión del sentido que la teoría de la relatividad
encierra.


Precisemos la cuestión en pocas palabras, pero del modo más claro posible.
La perspectiva es el orden y forma que la realidad toma para el que la
contempla. Sí varía el lugar que el contemplador ocupa, varía también la
perspectiva. En cambio, si el contemplador es sustituido por otro en el
mismo lugar, la perspectiva permanece idéntica. Ciertamente, si no hay un
sujeto que contemple, a quien la realidad aparezca, no hay perspectiva.
¿Quiere esto decir que sea subjetiva? Aquí está el equívoco que durante
dos siglos, cuando menos, ha desviado toda la filosofía, y con ella la
actitud del hombre ante el universo. Para evitarlo basta con hacer una
sencilla distinción.

Cuando vemos quieta y solitaria una bola de billar, sólo percibimos sus
cualidades de color y forma. Mas he aquí que otra bola de billar choca con
la primera. Esta es despedida con una velocidad proporcionada al choque.
Entonces notamos una nueva cualidad de la bola que antes permanecía
oculta: su elasticidad. Pero alguien podría decirnos que la elasticidad no
es una cualidad de la bola primera, puesto que sólo se presenta cuando
otra choca con ella. Nosotros contestaríamos prontamente que no hay tal.
La elasticidad es una cualidad de la bola primera, no menos que su color y
su forma; pero es una cualidad reactiva o de respuesta a la acción de otro
objeto. Así, en el hombre lo que solemos llamar su carácter es su manera
de reaccionar ante lo exterior ---cosas, personas, sucesos.

Pues bien, cuando una realidad entra en choque con ese otro objeto que
denominamos ``sujeto consciente'', la realidad responde apareciéndole. La
apariencia es una cualidad objetiva de lo real, es su respuesta a un
sujeto. Esta respuesta es, además, diferente según la condición del
contemplador; por ejemplo, según sea el lugar desde que mira. Véase cómo
la perspectiva, el punto de vista, adquieren un valor objetivo, mientras
hasta ahora se los consideraba como deformaciones que el sujeto imponía a
la realidad. Tiempo y espacio vuelven, contra la tesis kantiana, a ser
formas de lo real.


%\end{document}

Si hubiese entre los infinitos puntos de vista uno excepcional, al que
cupiese atribuir una congruencia superior con las cosas, cabría considerar
los demás como deformadores o ``meramente subjetivos''. Esto creían Galileo
y Newton cuando hablaban del espacio absoluto, es decir, de un espacio
contemplado desde un punto de vista que no es ninguno concreto. Newton
llama al espacio absoluto {\it sensorium Dei}, el órgano visual de Dios;
podríamos decir la perspectiva divina. Pero apenas se piensa hasta el
final esta idea de una perspectiva que no está tomada desde ningún lugar
determinado y exclusivo, se descubre su índole contradictoria y absurda.
No hay un espacio absoluto porque no hay una perspectiva absoluta. Para
ser absoluto, el espacio tiene que dejar de ser real ---espacio lleno de
cosas--- y convertirse en una abstracción.

La teoría de Einstein es una maravillosa justificación de la multiplicidad
armónica de todos los puntos de vista. Amplíese esta idea a lo moral y a
lo estético y se tendrá una nueva manera de sentir la historia y la vida.
El individuo, para conquistar el máximum posible de verdad, no deberá,
como durante centurias se le ha predicado, suplantar su espontáneo punto
de vista por otro ejemplar y normativo, que solía llamarse ``visión de las
cosas sub specie aeternitatis''. El punto de vista de la eternidad es
ciego, no ve nada, no existe. En vez de esto, procurará ser fiel al
imperativo unipersonal que representa su individualidad.

Lo propio acontece con los pueblos. En lugar de tener por bárbaras las
culturas no europeas, empezaremos a respetarlas como estilos de
enfrentamiento con el cosmos equivalentes al nuestro. Hay una perspectiva
china tan justificada como la perspectiva occidental.

\subsubsection*{3.- Antiutopismo o antirracionalismo}

La misma tendencia que en su forma positiva conduce al perspectivismo, en
su forma negativa significa hostilidad al utopismo.
La concepción utópica es la que se crea desde ``ningún sitio''  y que, sin
embargo, pretende valer para todos. A una sensibilidad como ésta que
transluce en la teoría de la relatividad, semejante indocilidad a la
localización tiene que parecerle una avilantez. En el espectáculo cósmico
no hay espectador sin localidad determinada. Querer ver algo y no querer
verlo desde un preciso lugar es un absurdo. Esta pueril insumisión a las
condiciones que la realidad nos impone; esa incapacidad de aceptar
alegremente el destino; esa pretensión ingenua de creer que es fácil
suplantarlo por nuestros estériles deseos, son rasgos de un espíritu que
ahora fenece, dejando su puesto a otro completamente antagónico.

La propensión utópica ha dominado en la mente europea durante toda la
época moderna: en ciencia, en moral, en religión, en arte. Ha sido
menester de todo el contrapeso que el enorme afán de dominar lo real,
específico del europeo, oponía para que la civilización occidental no haya
concluido en un gigantesco fracaso. Porque lo más grave del utopismo no es
que dé soluciones falsas a los problemas ---científicos o políticos---, sino
algo peor: es que no acepta el problema ---lo real--- según se presenta; antes
bien, desde luego a priori, le impone una caprichosa forma.

Si se compara la vida de Occidente con la de Asia ---indos, chinos---,
sorprende al punto la inestabilidad espiritual del europeo frente al
profundo equilibrio del alma oriental. Este equilibrio revela que, al
menos en los máximos problemas de la vida, el hombre de Oriente ha
encontrado fórmulas de más perfecto ajuste con la realidad. En cambio, el
europeo ha sido frívolo en la apreciación de los factores elementales de
la vida, se ha fraguado de ellos interpretaciones caprichosas que es
forzoso periódicamente sustituir.

La desviación utopista de la inteligencia humana comienza en Grecia y se
produce dondequiera llegue a exacerbación el racionalismo. La razón pura
construye un mundo ejemplar ---cosmos físico o cosmos político--- con la
creencia de que él es la verdadera realidad y, por tanto, debe suplantar a
la efectiva. La divergencia entre las cosas y las ideas puras es tal, que
no puede evitarse el conflicto. Pero el racionalista no duda de que en él
corresponde ceder a lo real. Esta convicción es la característica del
temperamento racionalista.

Claro es que la realidad posee dureza sobrada para resistir los embates de
las ideas. Entonces el racionalismo busca una salida: reconoce que, por el
momento, la idea no se puede realizar, pero que lo logrará en ``un proceso
infinito''  (Leibniz, Kant). El utopismo toma la forma de ucronismo. Durante
los dos siglos y medio últimos todo se arreglaba recurriendo al infinito,
o por lo menos a períodos de una longitud indeterminada. (En el darwinismo
una especie nace de otra, sin más que intercalar entre ambas algunos
milenios). Como si el tiempo, espectral fluencia, simplemente corriendo,
pudiese ser causa de nada y hacer verosímil lo que es en la actualidad
inconcebible.

No se comprende que la ciencia, cuyo único placer es conseguir una imagen
certera de las cosas, pueda alimentarse de ilusiones. Recuerdo que sobre
mí pensamiento ejerció suma influencia un detalle. Hace muchos años leía
yo una conferencia del fisiólogo Loeb sobre los tropismos. Es el tropismo
un concepto con que se ha intentado describir y aclarar la ley que rige
los movimientos elementales de los infusorios. Mal que bien, con
correcciones y añadidos, este concepto sirve para comprender algunos de
estos fenómenos. Pero al final de su conferencia, Loeb agrega: ``Llegará el
tiempo en que lo que hoy llamamos actos morales del hombre se expliquen
sencillamente como tropismos''. Esta audacia me inquietó sobremanera, porque
me abrió los ojos sobre otros muchos juicios de la ciencia moderna, que,
menos ostentosamente, cometen la misma falta. ¡De modo ---pensaba yo--- que un
concepto como el tropismo, capaz apenas de penetrar el secreto de
fenómenos tan sencillos como los brincos de los infusorios, puede bastar
en un vago futuro para explicar cosa tan misteriosa y compleja como los
actos éticos del hombre! ¿Qué sentido tiene esto? La ciencia ha de
resolver hoy sus problemas, no transferimos a las calendas griegas. Si sus
métodos actuales no bastan para dominar hoy los enigmas del universo, lo
discreto es sustituirlos por otros más eficaces. Pero la ciencia usada
está llena de problemas que se dejan intactos por ser incompatibles con
los métodos. ¡Como sí fuesen aquéllos los obligados a supeditarse a éstos,
y no al revés! La ciencia está repleta de ucronismos, de calendas griegas.

Cuando salimos de esta beatería científica que rinde idolátrico culto a
los métodos preestablecidos y nos asomamos al pensamiento de Einstein,
llega a nosotros como un fresco viento de mañana. La actitud de Einstein
es completamente distinta de la tradicional. Con ademán de joven atleta le
vemos avanzar recto a los problemas y, usando del medio más a mano,
cogerlos por los cuernos. De lo que parecía defecto y limitación en la
ciencia, hace él una virtud y una táctica eficaz. Un breve rodeo nos aclarará la cuestión.

De la obra de Kant quedará imperecedero un gran descubrimiento: que la
experiencia no es sólo el montón de datos transmitidos por los sentidos,
sino un producto de dos factores. El dato sensible tiene que ser recogido,
filiado, organizado en un sistema de ordenación. Este orden es aportado
por el sujeto, es a priori. Dicho en otra forma: la experiencia física es
un compuesto de observación y geometría. La geometría es una cuadrícula
elaborada por la razón pura: la observación es faena de los sentidos. Toda
ciencia explicativa de los fenómenos materiales ha contenido, contiene y
contendrá estos dos ingredientes.

Esta identidad de composición que a lo largo de su historia ha manifestado
siempre la física moderna, no excluye, empero, las más profundas
variaciones dentro de su espíritu. En efecto: la relación que guarden
entre sí sus dos ingredientes da lugar a interpretaciones muy dispares. De
ambos, ¿cuál ha de supeditarse al otro? ¿Debe ceder la observación a las
exigencias de la geometría o la geometría a la observación? Decidirse por
lo uno o lo otro significa pertenecer a dos tipos antagónicos de tendencia
intelectual. Dentro de la misma y única física caben dos castas de hombres
contrapuestas.

Sabido es que el experimento de Michelson tiene el rango de una
experiencia crucial: en él se pone entre la espada y la pared al
pensamiento del físico. La ley geométrica que proclama la homogeneidad
inalterable del espacio, cualesquiera sean los procesos que en él se
producen, entra en conflicto rigoroso con la observación, con el hecho,
con la materia. Una de dos: o la materia cede a la geometría o ésta a
aquélla.

En este agudo dilema sorprendemos a dos temperamentos intelectuales y
asistimos a su reacción. Lorentz y Einstein, situados ante el mismo
experimento, toman resoluciones opuestas. Lorentz, representando en este
punto el viejo racionalismo, cree forzoso admitir que es la materia quien
cede y se contrae. La famosa ``contracción de Lorentz''  es un ejemplo
admirable de utopismo. Es el juramento del Juego de Pelota transplantado a
la física. Einstein adopta la solución contraría. La geometría debe ceder;
el espacio puro tiene que inclinarse ante la observación, tiene que
encorvarse.

Suponiendo una perfecta congruencia en el carácter, llevado Lorentz a la
política, diría: perezcan las naciones y que se salven los principios.
Einstein en cambio, sostendría: es preciso buscar principios para que se
salven las naciones, porque para eso están los principios.

No es fácil exagerar la importancia de este viraje a que Einstein somete
la ciencia física. Hasta ahora, el papel de la geometría, de la pura
razón, era ejercer una indiscutida dictadura. En el lenguaje vulgar queda
la huella del sublime oficio que a la razón se atribuía: el vulgo habla de
los  ``dictados de la razón''. Para Einstein el papel de la razón es mucho
más modesto: de dictadora pasa a ser humilde instrumento que ha de
confirmar en cada caso su eficacia.

Galileo y Newton hicieron euclidiano al universo simplemente porque la
razón lo dictaba así. Pero la razón pura no puede hacer otra cosa que
inventar sistemas de ordenación. Estos pueden ser muy numerosos y
diferentes. La geometría euclidiana es uno; otro, la de Riemann, la de
Lobatchewski, etc. Más claro está que no son ellos, que no es la razón
pura quien resuelve cómo es lo real. Por el contrario, la realidad
selecciona entre esos órdenes posibles, entre esos esquemas, el que le es
más afín. Esto es lo que significa la teoría de la relatividad. Frente al
pasado racionalista de cuatro siglos se opone genialmente Einstein e
invierte la relación inveterada que existía entre razón y observación. La
razón deja de ser norma imperativa y se convierte en arsenal de
instrumentos; la observación prueba éstos y decide sobre cuál es el
oportuno. Resulta, pues, la ciencia de una mutua selección entre las ideas
puras y los puros hechos.

Este es uno de los rasgos que más importa subrayar en el pensamiento de
Einstein, porque en él se inicia toda una nueva actitud ante la vida. Deja
la cultura de ser, como hasta aquí, una norma imperativa, a que nuestra
existencia ha de amoldarse. Ahora entrevemos una relación entre ambas, más
delicada y más justa. De entre las cosas de la vida son seleccionadas
algunas como posibles formas de cultura; pero de entre estas posibles
formas de cultura, selecciona a su vez la vida las únicas que deberán
realizarse.

\subsubsection*{4.- Finitismo}

No quiero terminar esta filiación de las tendencias profundas que afloran
en la teoría de la relatividad sin aludir a la más clara y patente.
Mientras el pasado utopista lo arreglaba todo recurriendo al infinito en
el espacio y en el tiempo, la física de Einstein ---y la matemática reciente
de Brouwer y Weyl lo mismo--- acota el universo. El mundo de Einstein tiene
curvatura, y, por tanto, es cerrado y finito\footnote{Por todas partes, en el sistema de Einstein se persigue al infinito.
Así, por ejemplo, queda suprimida la posibilidad de velocidades infinitas.}.

Para quien crea que las doctrinas científicas nacen por generación
espontánea, sin más que abrir los ojos y la mente sobre los hechos, esta
innovación carece de importancia. Se reduce a una modificación de la forma
que solía atribuirse al mundo. Pero el supuesto es falso: una doctrina
científica no nace, por obvios que parezcan los hechos donde se funda, sin
una clara predisposición del espíritu hacia ella. Es preciso entender la
génesis de nuestros pensamientos con toda su delicada duplicidad. No se
descubren más verdades que las que de antemano se buscan. Las demás, por
muy evidentes que sean, encuentran ciego al espíritu.

Esto da un enorme alcance al hecho de que súbitamente, en la física y en
la matemática, empiece una marcada preferencia por lo finito y un gran
desamor a lo infinito. ¿Cabe diferencia más radical entre dos almas que
propender una a la idea de que el universo es ilimitado y la otra a sentir
en su derredor un mundo confinado? La infinitud del cosmos fue una de las
grandes ideas excitantes que produjo el Renacimiento. Levantaba en los
corazones patéticas marcas, y Giordano Bruno sufrió por ella muerte cruel.
Durante toda la época moderna, bajo los afanes del hombre occidental, ha
latido como un fondo mágico esa infinitud del paisaje cósmico.

Ahora, de pronto, el mundo se limita, es un huerto con muros confinantes,
es un aposento, un interior. ¿No sugiere este nuevo escenario todo un
estilo de vida opuesto al usado? Nuestros nietos entrarán en la existencia
con esta noción, y sus gestos hacia el espacio tendrán un sentido
contrarío a los nuestros. Hay evidentemente en esta propensión al
finitismo una clara voluntad de limitación, de pulcritud serena, de
antipatía a los vagos superlativos, de antirromanticismo. El hombre
griego, el ``clásico'', vivía también en un universo limitado. Toda la
cultura griega palpita de horror al infinito y busca el metron, la mesura.
Fuera, sin embargo, superficial creer que el alma humana se dirige hacia
un nueva clasicismo. No ha habido jamás neoclasicismo que no fuese una
frivolidad. El clásico busca el límite, pero es porque no ha vivido nunca
la ilimitación. Nuestro caso es inverso: el límite significa para nosotros
una amputación, y el mundo cerrado y finito en que ahora vamos a respirar
será irremediablemente un muñón de universo\footnote{Otros dos puntos fuera necesario tocar para que las líneas generales
de la mente que ha creado la teoría de la relatividad quedasen completas.
Uno de ellos es el cuidado con que se subrayan las discontinuidades en lo
real, frente al prurito de lo continuo que domina el pensamiento de los
últimos siglos. Este discontinuismo triunfa a la par en biología y en
historia. El otro punto, tal vez el más grave de todos, es la tendencia a
suprimir la causalidad que opera en forma latente dentro de la teoría de
Einstein. La física, que comenzó por ser mecánica y luego fue dinámica,
tiende en Einstein a convertirse en mera cinemática. Sobre ambos puntos
sólo puede hablarse recurriendo a difíciles cuestiones técnicas que en el
texto he procurado eliminar.}.

\bigskip


(1924, Se incluye en el volumen III de las Obras completas y como apéndice
en El tema de nuestro tiempo)

\end{document}