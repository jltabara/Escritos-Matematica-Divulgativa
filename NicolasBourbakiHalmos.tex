\documentclass[a4paper, 12pt]{article}

%%%%%%%%%%%%%%%%%%%%%%Paquetes
\usepackage[spanish]{babel}  
\usepackage[utf8]{inputenc}
\usepackage{tcolorbox}
\usepackage{cmbright}  %%%%%%% El tipo de letra
\usepackage{setspace}
\onehalfspacing  %%%%%%%%%%% Espacio y medio de interlineado
\parskip=1em  %%%%%%%%%%%% Separacion entre parrafos
%%%%%%%%%%%%%%%%%%%%%%



%%%%%%%%%%%%%%%%%
\title{Nicolás Bourbaki}
\author{P. Halmos}
\date{}
%%%%%%%%%%%%%%%%%

\begin{document}

\begin{tcolorbox}[colback=blue!5!white,colframe=blue!75!black]

\vspace{-1.8cm}
\textbf \maketitle

\end{tcolorbox}

\bigskip


Su nombre es griego, su nacionalidad francesa y su historia es curiosa. 
Es uno de los matemáticos más influyentes del siglo XX. Existen 
muchas leyendas acerca de él, y cada día va habiendo más. Casi 
cada uno de los matemáticos conoce unas pocas historias acerca de 
él y probablemente ha inventado también un par de ellas más. Sus 
trabajos se leen y se citan extensamente en todo el mundo. Existen 
jóvenes en Río de Janeiro cuya educación matemática ha sido 
basada casi enteramente en sus trabajos y existen famosos matemáticos en 
Berkeley y en Gotinga que piensan que su influjo es pernicioso. Tiene 
partidarios fervientes y detractores vociferantes en cualquier grupo de 
matemáticos que se reúna. El hecho más extraño sobre él, 
sin embargo, es que no existe. 

Este francés no existente con nombre griego es Nicolás 
Bourbaki. El hecho es que Nicolás Bourbaki es un seudónimo colectivo 
utilizado por una corporación informal de matemáticos (la 
interesante denominación francesa equivalente al inglés 
corporación, «sociedad anónima», está totalmente adaptada a este 
contexto). El grupo seudónimo está escribiendo un tratado muy 
extenso en matemáticas, partiendo de los principios básicos más 
generales y concluyendo, es de suponer, con las aplicaciones más 
especializadas. El proyecto fue comenzado en 1939, y hasta ahora han 
aparecido 20 volúmenes (casi 3000 páginas) de este trabajo 
monumental. 

La razón por la que los autores eligieron llamarle Bourbaki 
está envuelta en el misterio. Existen razones para pensar que su 
elección estuvo inspirada en un personaje, oficial del ejército de 
cierta importancia en la guerra franco-prusiana. El general Charles Denis 
Sauter Bourbaki fue una figura llena de colorido. En 1862, a la edad de 
cuarenta y seis años, se le ofreció una oportunidad para llegar a 
ser rey de Grecia, pero declinó. Actualmente se le recuerda 
principalmente por la forma cruel como fue tratado por la fortuna en los 
azares de la guerra. En 1871, después de huir de Francia a Suiza con un 
resto pequeño de su ejército, fue prisionero allí y trató 
de suicidarse. Al parecer fracasó, ya que llegó a vivir hasta la 
venerable edad de ochenta y tres años. Se dice que hay una estatua de 
él en Nancy. Esto puede establecer cierta conexión entre él y 
los matemáticos que usan su nombre, ya que algunos de entre ellos 
estuvieron en tiempos diferentes asociados a la Universidad de Nancy. 

Una de las leyendas que rodean su nombre es que hace unos veinticinco o 
treinta años, algunos estudiantes de primer año de la Escuela Normal 
Superior (donde la mayor parte de los matemáticos franceses reciben su 
educación) tenían anualmente la visita de un personaje distinguido 
llamado Nicolás Bourbaki, quien les daba una lección y que, en 
realidad, era un actor aficionado y disfrazado con una barba patriarcal y 
cuya lección era una pieza maestra de trabalenguas matemático. 

Es preciso decir una palabra de precaución acerca de la poca 
garantía que ofrecen las historias sobre Bourbaki. Si bien los miembros 
de esta organización secreta no han realizado ningún juramento de 
sangre sobre su secreto, la mayor parte de entre ellos encuentra su propia 
broma tan divertida que sus historias acerca de ellos mismos son 
apócrifas y mutuamente conflictivas. Por otra parte, los extraños al 
grupo no sabrán probablemente de qué están hablando y solamente 
pueden transmitir una leyenda que ya ha sido iluminada varias veces. La 
finalidad de este artículo es describir las realidades científicas 
de Bourbaki y narrar unas pocas historias de muestra que se cuentan acerca 
de él (ellos). Algunas de estas historias no son verificables, por no 
decir otra cosa, pero esto no las hace menos entretenidas. 

La publicación científica bajo un seudónimo no es algo 
totalmente original, por supuesto. El estadístico inglés William 
Sealy Gosset publicó un trabajo vanguardista sobre la teoría de 
muestras pequeñas con el sobrenombre de «Student», probablemente para 
evitar situaciones embarazosas con sus empresarios (los fabricantes de 
cervezas  Guinness). Aproximadamente al mismo tiempo que Bourbaki 
comenzaba, otro grupo de bromistas inventó la figura de E. S. 
Pondiczery, un supuesto miembro del Instituto Real de Poldavia. Las 
iniciales (E.S.P., R.I.P.) fueron inspiradas por un artículo 
proyectado, pero nunca escrito, sobre percepción extrasensorial. El 
trabajo más importante de Pondiczery fue sobre curiosidades 
matemáticas. Su realización más importante fue el único uso 
conocido de un seudónimo de segundo orden. Al presentar para su 
publicación un artículo sobre la teoría matemática de la 
caza mayor a {\it The American Mathematical Monthly}, Pondiczery pedía en una 
carta que se le permitiese usar un seudónimo, a causa de la naturaleza 
obviamente jocosa del tema. El editor estuvo de acuerdo y el artículo 
apareció (en 1938) bajo el nombre de H. Pétard. 

Las tribus primitivas, y ocasionalmente también los 
científicos, encuentran cierta magia en su nombre. Esto dio origen a la 
publicación de una obra que nunca hubiera sido concebida si los autores 
hubieran tenido nombres diferentes. George Gamow y su amigo Hans Bethe 
vieron, y utilizaron, una magnífica oportunidad cuando apareció en 
escena un brillante y joven científico con un nombre extraño. El 1 
de abril de 1948 publicaron en {\it The Physical Review} un artículo 
totalmente serio sobre el origen de los elementos químicos cuyo 
único rasgo extraordinario fue la segunda línea. Por supuesto, esta 
decía, Alpher, Bethe y Gamow. 

A propósito del tema de artículos que aparecen bajo nombres 
extraños, parece apropiado mencionar el caso de Maurice de Duffahel. 
Este caballero adquirió inmortalidad matemática mediante el simple 
truco de publicar bajo su propio nombre algunos de los artículos 
clásicos de los grandes maestros. Hacía lo menos posible para 
disfrazar sus actividades. En 1936 republicó como propio un 
artículo que había sido publicado solamente veinticuatro años 
antes por Charles Emile Picard. La versión de Duffahel fue idéntica 
a la de Picard, palabra por palabra, símbolo por símbolo, excepto 
una omisión. Por razones fáciles de comprender, omitió una nota 
a pie de página en la cual Picard se había referido a uno de sus 
propios artículos anteriores. Finalmente, el mundo científico dio 
con el truco de Duffahel. Uno puede engañar a algunos editores algunas 
veces, pero no se puede engañar a todos los recensores todas las veces. 
Un recensor del artículo de Duffahel resultó conocer las obras de 
Picard lo suficientemente bien como para reconocer la repetición, y la 
carrera de publicaciones de Duffahel llegó a su término 
abruptamente. 

Las obras de Bourbaki no tiene necesidad de quedar ocultas a los 
empresarios de una cervecería, no son meras e inocentes diversiones, 
sino matemática seria, y ciertamente no son plagios de ningún otro. 
El grupo adoptó originariamente el seudónimo, parte en broma y parte 
para evitar una larga y aburrida lista de autores en su título. 
Continúan usándolo más como un nombre de corporación que 
como un disfraz. El nombre de los miembros son un secreto abierto a la mayor 
parte de los matemáticos. Los miembros del grupo Bourbaki, como los de 
la mayor parte de las corporaciones, cambian de vez en cuando, pero el 
estilo y el espíritu de la obra permanecen en los mismos. Es cómodo 
poder describir un cierto estilo mediante un adjetivo (el término 
aceptado es {\it Bourbachique}) más bien que mediante una referencia a la 
«joven escuela francesa» o usar una circunlocución similar. 

La primera aparición de Bourbaki en escena fue a mitad de los 
años 30, cuando comenzaron a publicar notas, recensiones y otros 
artículos en los {\it Comptes Rendus} de la Academia de Ciencias Francesa y 
en otras partes. La obra de importancia, en la que después se embarcaron 
quedó explicada en un artículo traducido al inglés e impreso 
(en 1950) en {\it The American Mathematical Monthly}, bajo el título «La 
Arquitectura de la Matemáticas». Una nota a pie de página dice: 
«El profesor N. Boubarki, antes de la Real Academia de Poldavia (!`restos 
de Pondiczery!), que reside ahora en Nancy, Francia, es autor de un tratado 
extenso de matemáticas modernas, de próxima publicación, bajo el 
título {\it Eléments de Mathématique} (Hermann et Cie., Paris, 
1939), del cual han aparecido ya diez volúmenes.» El artículo 
constituye una declaración interesante de la opinión de Bourbaki 
sobre el concepto de «estructura» en matemáticas. Es un 
descripción maestra del espíritu de Bourbaki. Otro artículo 
que apareció en {\it The Journal of Simbolic Logic}, de 1949, tiene el 
título ambicioso «Fundamentos de matemáticas para el 
matemático profesional». Es completamente técnico, pero la 
personalidad de los autores aparece a través del simbolismo. Concluye 
así: «Afirmo que puedo construir sobre estos fundamentos el cuerpo 
entero de las matemáticas de hoy día y que si hay algo original en 
mis procedimientos estriba solamente en el hecho de que, en lugar de 
contentarme con dicha afirmación, procedo a demostrarla del mismo modo 
que Diógenes probó la existencia del movimiento, y mi 
demostración llegará a ser más y más compleja a medida que 
mi tratado vaya creciendo.» 

Este artículo da como institución de pertenencia del autor a la 
«Univer\-sidad de Nan\-cago» (Nancy más Chicago). La razón principal 
para esta combinación de nombres es que uno de los patriarcas fundadores 
pertenece ahora al cuerpo docente de la Universidad de Chicago. Su nombre es 
André Weil (por más señas, hermano de la conocida mística 
religiosa Simone Weil). Si bien André Weil no es conocido del 
público en general, muchos de sus colegas están dispuestos a afirmar 
que es el matemático en vida más grande del mundo. Su trabajo sobre 
teoría algebraica de números y sobre geometría algebraica es 
profundo e importante. Su influencia sobre el desarrollo de las 
matemáticas del siglo XX es grande, e incluso algunas de sus 
contribuciones en campos un tanto alejados del suyo propio (por ejemplo, 
sobre estructuras uniformes y análisis armónico en grupos 
topológicos) han abierto direcciones nuevas e inspirado ulteriores 
investigaciones. Nancago, dicho sea de paso, aparece asimismo en una serie 
reciente de libros matemáticos avanzados que va siendo publicado bajo el 
encabezamiento impresionante {\it Publications de l'Institut Mathématique de 
l'Université de Nancago.} 

Según una de las leyendas sobre Bourbaki, su obra más 
importante, cuyo título general es {\it Elementos de Matemáticas}, debe 
su origen a una conversación entre Weil y Jean Delsarte acerca de la 
forma en que se debía enseñar el cálculo. Cualquiera que sea la 
motivación de la obra originariamente, su finalidad presente no es 
ciertamente la pedagogía elemental. Es como si una discusión sobre 
el modo mejor de enseñar a comprender la música popular hubiera dado 
lugar a un tratado completo de armonía y musicología. Los 
matemáticos consideran que el cálculo es tan «trivial» como los 
músicos consideran la música de Víctor Herbert. El tratado de 
Bourbaki (escrito en francés) es un panorama de todas las 
matemáticas desde un sofisticado punto de vista. 

El cuerpo entero del tratado comprenderá probablemente varias 
partes, pero los veinte volúmenes que han aparecido hasta ahora no 
llegan a completar la primera parte, titulada {\it Las Estructuras Fundamentales 
del Análisis.} Los nombres de unas seis subdivisiones de la primera parte 
constituyen una leve sorpresa para el profano (o para el matemático 
clásico) que piensa en términos de aritmética, geometría y 
ciertas otras palabras pasadas de moda. Estas subdivisiones son: (1) 
Teoría de conjuntos; (2) Álgebra; (3) Topología General; (4) 
Funciones de una Variable Real; (5) Espacios Vectoriales Topológicos, y 
(6) Integración. 

Cada volumen va provisto de un folleto suelto de cuatro páginas que 
vienen a ser un conjunto de indicaciones sobre la utilización adecuada 
del tratado. Detallan los requisitos necesarios para leer el tratado 
(aproximadamente dos años de matemáticas universitarias), describen 
la organización del trabajo y especifican «el orden lógico, 
rigurosamente fijo», en el cual los capítulos, libros y partes han de 
ser leídos. Las indicaciones explican también los trucos 
pedagógicos de los autores, y algunos de ellos son ciertamente muy 
buenos. Uno, de los que muchos autores podrían copiar con provecho, es 
advertir al lector en el momento en el que el tema resulta especialmente 
resbaladizo, es decir, cuando el lector puede caer probablemente en un 
error. Los pasajes resbaladizos están marcados con unas curvas muy 
visibles en forma de ese mayúscula («curva peligrosa») situada al 
margen. 

Un truco bourbakista menos admirable es la actitud ligeramente 
despectiva hacia la sustitución de lo que ellos llaman «abusos del 
lenguaje» por términos técnicos. Es generalmente admitido el hecho 
de que la adherencia estricta a una terminología rigurosamente correcta 
ha de conducir probablemente a pedantería e ilegibilidad. Esto es 
especialmente cierto en Bourbaki porque su terminología y simbolismo 
están frecuentemente en desacuerdo con el uso comúnmente aceptado. 
Lo curioso es que a menudo el «abuso de lenguaje» que ellos emplean como 
un reemplazamiento «informal» por un término técnico es, de hecho, 
el término convencional. Cansados de tratar de hacer recordar su propia 
innovación, los autores se introducen confortablemente en la 
terminología del resto del mundo matemático. 

Casi todos los volúmenes de Bourbaki contienen un excelente 
conjunto de ejercicios. Las matemáticas no pueden ser aprendidas 
pasivamente y los ejercicios de Bourbaki son un desafío a la actividad. 
Los autores han utilizado una gran cantidad de ingenio en inventar nuevos 
ejercicios y en volver a expresar y reordenar los antiguos. Por sistema, de 
ordinario no hacen referencia de los autores originales de los ejercicios 
que ellos han revisado, pero a nadie parece que le importe. Un 
matemático considerará incluso un honor probablemente el que uno de 
sus artículos haya sido «robado» por Bourbaki y utilizado como un 
ejercicio. 




Otro artificio de Bourbaki consiste en las hojas desplegables que 
recapitulan definiciones e hipótesis importantes, un diccionario para 
cada libro que sirve también como un índice extenso y una guía 
para la terminología no bourbakista y para el lenguaje bourbakista 
clásico. 

La única cosa importante que se echa de menos es una guía 
bibliográfica adecuada. La presentación de Bourbaki de cada uno de 
los temas es sistemática y profunda y a menudo incluye una visión 
histórica brillante del tema. Los ensayos históricos suelen hacer 
tan sólo unas pocas referencias como a regañadientes a los 
clásicos y omiten casi enteramente al mencionar las fuentes de 
contribuciones modernas. No se intenta engañar a nadie (Bourbaki no 
afirma haber descubierto todo el conjunto de las matemáticas modernas), 
pero en la práctica puede tener el efecto de confundir al historiador 
matemático futuro. 

Este es el ropaje externo de Bourbaki. El estilo de Bourbaki y su 
espíritu, las cualidades que le atraen amigos y repelen a sus enemigos 
son más difíciles de describir. En la misma forma que las 
cualidades de la música, han de ser sentidas más bien que 
entendidas. 

 Una de las cosas que atrajeron estudiantes a Bourbaki desde el comienzo 
fue que Bourbaki presentó el primer tratamiento sistemático de 
algunos temas (por ejemplo, topología general y álgebra 
multilineal) que no estaban a disposición del público en ninguna 
otra parte en forma de libro. Bourbaki fue pionero en la obra de reducir de 
forma ordenada una gran masa de artículos que habían aparecido a 
lo largo de varias décadas en muchas revistas y en idiomas diferentes. 
Los principales rasgos del tratamiento de Bourbaki son una actitud radical 
acerca del orden correcto para hacerlo todo, una insistencia dogmática 
en una terminología particular, una organización clara y 
económica de las ideas y un estilo de presentación que es tan 
inclinado a decirlo todo que no deja nada a la imaginación y que tiene, 
por consiguiente, un efecto diluyente y tibio. 

 Una muestra típica de minuciosidad y del ritmo lento del 
tratamiento de Bourbaki es su definición del número «1». Dedican 
casi doscientas páginas a la preparación de la definición misma. 
Entonces definen el número 1 en términos de símbolos 
extraordinariamente abreviados y condensados, explicando con una nota a pie 
de página que la forma sin abreviar de la definición en su sistema 
de notación requeriría varias decenas de millares de símbolos. 
Si se ha de hacer justicia a Bourbaki, ha de decirse también que los 
lógicos matemáticos modernos han conocido ya por algún tiempo 
que conceptos tales como el número 1 no son tan elementales como 
parecen. 

¿Cómo llega a ser escrito un trabajo en cooperación de esta 
magnitud? En gran parte esto es debido a Jean Dieudonné (originariamente 
de Nancy, y ahora en la Northwestern University), que ha sido el escriba 
jefe de Bourbaki casi desde el comienzo. Como Dieudonné es un escritor~ 
prolífico en matemáticas bajo su propio nombre, existe una cierta 
dificultad en distinguir su trabajo particular de su propia obra para 
Bourbaki. Según se cuenta, logra mantener una separación exactamente 
de una manera notable. La historia afirma que Dieudonné publicó una 
vez, bajo el nombre de Bourbaki, una nota que más tarde resultó 
contener un error. El error fue corregido en un artículo titulado 
«Sobre un error de Bourbaki», firmado por Jean Dieudonné. 

El número de miembros de Bourbaki parece variar entre diez y 
veinte. Con una excepción notable, todos los miembros han sido 
franceses. La excepción es Samuel Eilenberg (originariamente de 
Varsovia, ahora en la Universidad de Columbia). Conocido por los amigos de 
su juventud como S2P2 (que viene de «smart Sammy, the Polish Prodigy»), 
Eilenberg es un simpático extrovertido que aprendió más acerca 
de los Estados Unidos a los seis meses de su llegada que la mayoría de 
los americanos aprenden en su vida (una de las primeras cosas que hizo fue 
lanzarse a una prolongada gira en autostop). Puesto que habla francés 
como un nativo y conoce más sobre topología algebraica que 
cualquier francés, la regla implícita de restringir Bourbaki a los 
franceses admitió una excepción para incluirle a él. 

La orientación francesa de Bourbaki no es mero chauvinismo, sino 
una necesidad lingüística (puesto que fueron franceses los que lo 
comenzaron). Cuando una colección de {\it prima donas} tales como Weil, 
Dieudonné, Claude Chevallier y Henri Cartan se reúnen con sus 
colegas, la velocidad y el volumen de flujo del francés es 
impresionante. A fin de seguir y tomar parte en la conversación en tales 
circunstancias, no solamente es necesario hablar francés rápido y 
alto, sino que es preciso conocer el argot más reciente de los 
estudiantes parisienses. Incluso en el caso de que todos en la sala 
satisfagan estas condiciones, es difícil de todas maneras comprender 
como puede ser realizado trabajo alguno en los famosos congresos Bourbaki. 
Pero se hace. Los miembros se reúnen cada año, de ordinario en 
algún lugar francés de vacaciones, para determinar las decisiones 
estratégicas de mayor importancia. Puesto que su tratado ha resultado 
ser un éxito comercial (con la considerable sorpresa de Bourbaki mismo), 
hay dinero en abundancia para pagar los gastos de viaje y para proporcionar 
la comida y los vinos franceses que prestan fluidez a las reuniones. (El 
éxito comercial, dicho sea de paso, se debe principalmente al mercado 
americano. Cuatro de los cinco miembros más antiguos de Bourbaki son 
ahora residentes en los Estados Unidos.) 

La preparación de un volumen de Bourbaki lleva consigo una gran 
cantidad de trabajo. Una vez que se ha decidido sobre un proyecto 
particular, alguno de los miembros acepta escribir la primera redacción. 
Al hacerlo sabe que se le avecina una experiencia de cuidado. Cuando su 
redacción ha sido acabada, se sacan copias y se envían a todos los 
otros miembros. En el próximo congreso, la redacción se critica sin 
misericordia alguna y muy posiblemente será completamente rechazada. La 
primera redacción del libro de Bourbaki sobre Integración, por 
ejemplo, fue escrita por Dieudonné y vino a ser conocida como «el 
monstruo de Dieudonné». Corren rumores de que en espíritu y en 
contenido  «el monstruo de Dieudonné» era muy semejante a un libro 
muy conocido americano sobre esta materia, escrito por un autor cuyo nombre 
se indicará aquí simplemente por Zutano. El monstruo de 
Dieudonné nunca fue publicado. Sus colegas lo echaron abajo. Quien le 
dio la puntilla fue la observación de Weil: «Si es que nos vamos a 
dedicar a hacer algo así, podemos traducir el libro de Zutano al 
francés y hemos terminado.» 

Después de que se ha visto la primera redacción se empieza con 
la segunda, posiblemente por un miembro diferente. El proceso sigue y sigue. 
Seis o siete redacciones han sido hechas en casos particulares. El resultado 
de este esfuerzo penoso no es un libro de texto que se pueda poner 
sensatamente en las manos de un principiante (incluso Bourbaki lo admite 
así), pero es un libro de referencia, casi un enciclopedia, sin el cual 
las matemáticas del siglo XX serían, para su ventaja o desventaja, 
completamente diferentes de lo que son. 

La exuberante juventud de Bourbaki es un buen augurio para el futuro de 
sus trabajos, pero es al mismo tiempo uno de sus principales fastidios para 
sus enemigos. Los administradores de la Sociedad Matemática Americana no 
encontraron divertido el recibir una solicitud de admisión firmada por 
N. Bourbaki. Consideraron el chiste como infantil y rechazaron la solicitud. 
El secretario de la Sociedad sugirió fríamente que Bourbaki 
podría solicitar un nombramiento como miembro institucional. Como la 
cuota para un miembro institucional es bastante más elevada que la 
individual y como Bourbaki no deseaba admitir que no existía, nunca 
más se oyó del asunto. 

Es verdad que el chiste puede resultar inmaduro, como inmaduros son los 
jóvenes, pero la matemática es una profesión de jóvenes. La 
insistencia de Bourbaki en la juventud es algo laudatorio. Después de 
haber alcanzado los cincuenta años recientemente, Diedounné y Weil, 
aun siendo los patriarcas fundadores de Bourbaki, anunciaron su 
jubilación del grupo. Habían expresado antes su intención de 
retirarse a los cincuenta años y mantuvieron su promesa. 

Parece apropiado concluir advirtiendo al lector que debe estar 
prevenido sobre posibles rumores inspirados por Bourbaki acerca del autor de 
este artículo y a estar preparado a tomar tales rumores con un buen 
grado de parsimonia. La corporación no gusta de que sus secretos sean 
contados al público y ha demostrado ya su habilidad para tomar medidas 
oportunas respecto de los informadores. A buen seguro, la ficción 
bourbakista ha sido ya relatada en prensa con anterioridad a este 
artículo. En 1949, André Dechalet, en su pequeño libro sobre 
análisis matemático, se refería al «matemático 
policéfalo» N. Bourbaki, y fue tan lejos que llegó a mencionar 
algunas de las cabezas del grupo por su nombre. Un año o dos antes de 
eso, el libro anual de la Enciclopedia Británica contenía un breve 
párrafo acerca de Bourbaki como grupo. El autor de este párrafo era 
Ralph P. Boas, al tiempo editor ejecutivo de la revista {\it Mathematical 
Reviews,} y ahora colega de Dieudonné en Northwestern University. Poco 
tiempo después los editores de la Británica recibieron una sentida 
carta firmada por N. Bourbaki protestando contra la acusación de no 
existencia de Bourbaki por Boas. La confusión de los editores y el 
embarazo de Boas no disminuyeron cuando un miembro del Departamento de 
Matemáticas de la Universidad de Chicago escribió una carta 
verdadera, pero astutamente formulada, implicando, pero no diciendo, que 
Bourbaki existía en realidad. La situación fue aclarada a los 
editores por una carta del secretario de la Sociedad Matemática 
Americana (el mismo secretario que había rechazado aprobar la solicitud 
de Bourbaki como miembro). 

Bourbaki logró llevar a cabo su venganza: reuniendo todas sus 
fuerzas policefálicas e internacionales, la corporación hizo 
circular un rumor de que Boas no existía. Boas, decía Bourbaki, es 
el seudónimo colectivo de un grupo de jóvenes matemáticos 
norteamericanos que actúan en corporación como los editores de 
Mathematical Reviews. 

\end{document}
