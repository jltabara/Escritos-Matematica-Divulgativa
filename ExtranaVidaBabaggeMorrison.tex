\documentclass[a4paper, 12pt]{article}

%%%%%%%%%%%%%%%%%%%%%%Paquetes
\usepackage[spanish]{babel}  
\usepackage[utf8]{inputenc}
\usepackage{tcolorbox}
\usepackage{cmbright}  %%%%%%% El tipo de letra
\usepackage{setspace}
\onehalfspacing  %%%%%%%%%%% Espacio y medio de interlineado
\parskip=1em  %%%%%%%%%%%% Separacion entre parrafos
%%%%%%%%%%%%%%%%%%%%%%



%%%%%%%%%%%%%%%%%
\title{La Extraña Vida de Charles Babagge}
\author{P. y E. Morrison}
\date{}
%%%%%%%%%%%%%%%%%

\begin{document}

\begin{tcolorbox}[colback=blue!5!white,colframe=blue!75!black]

\vspace{-1.8cm}
\textbf \maketitle

\end{tcolorbox}

\bigskip






Durante el festival de Gran Bretaña del pasado verano, el centro del escenario en una sección de la Exhibición de Ciencia en el Museo de Ciencias de South Kensigton era ocupado por un computador reluciente de línea aerodinámica, llamado Nimrod. Un visitante que hubiera merodeado lejos de las atracciones principales hubiera podido encontrar, perdido en una remota galería, un antecesor de Nimrod cubierto de polvo. Es una colección complicada de ruedas y manivelas llamada máquina diferencial de Babbage. Hecha en 1833 fue el trabajo de un diseñador que consumió sus años y su fortuna en el intento de construir máquinas matemáticas para las cuales su época no estaba preparada, pero que ahora han llegado a ser realizadas.

Charles Babbage es un nombre conocido hoy día por algunos  matemáticos. Pocos entre sus propios contemporáneos reconocieron el valor de su trabajo, y sus vecinos londinenses le tuvieron por un chiflado, conocido principalmente como un contumaz cruzado en contra de los organistas callejeros. Así, cuando murió, el Times, de Londres, le identificó en el primer párrafo de su necrologica como un hombre que había vivido hasta casi ochenta años, «a pesar de sus persecuciones contra los organistas». Hoy, los matemáticos reconocen en él al hombre que estuvo muy por delante de su propia época. Así, en un artículo sobre uno de los modernos computadores de Estados Unidos, en la revista Británica {\it Nature} apareció el título «El  sueño de Babbage llega a ser realidad».

Babbage fue un tipo versátil. Escribió un libro sobre la economía de manufacturas y de maquinaria en el que barruntó lo que ahora se llama la investigación operativa. Llevó a cabo con tesón una campaña en favor del apoyo económico gubernamental de la investigación científica en un tiempo en el que la investigación era todavia en alto grado un «hobby» de señores. Publicó una tabla de logaritmos extensamente utilizada de 1 a 108.000. Elaboró unas tablas de mortandad e hizo un intento vanguardista a fin de popularizar el seguro de vida. Diseñó máquinas herramientas. Propuso un gran número de inventos, desde ciertos artificios para evitar las catástrofes ferroviarias hasta un sistema de iluminación de faros costeros. Escribió artículos sobre física, geología, astronomía y arquelogía. Pero la gran pasión de su vida fueron las máquinas matemáticas.

Babbage nació en 1792, en Devonshire, siendo hijo de un banquero, del cual posteriormente heredó una fortuna considerable. A causa de su salud fue educado por maestros privados hasta que ingresó en el Trinity College, en Cambridge, en 1810. Al hacerlo era ya entusiásticamente apasionado por las matemáticas, pero se desanimó al encontrar que sabía más que su tutor. Sus amigos más íntimos en la Universidad fueron John Herschel, hijo del ilustre astrónomo William Herschel, y George Peacock. Los tres estudiantes llegaron a formar un compacto grupo para «hacer todo lo que pudiesen para dejar el mundo mejor de lo que ellos lo habían encontrado». En 1812, como primer paso en esta dirección fundaron la Sociedad Analítica, destinada primordialmente a animar a los matemáticos ingleses para que reemplazasen las notaciones matemáticas newtonianas por el sistema de Leibniz, usado en el continente. Newton había denotado una derivada poniendo un punto sobre el símbolo en cuestión. Liebniz lo había hecho situando una $d$ delante de él. Babbage fundó la sociedad, hizo notar más tarde, a fin de preparar los «principios del $d$-ismo puro en oposición a la era del punto de la Universidad». A pesar de considerable oposición, la sociedad tuvo un efecto profundo en el futuro desarrollo de las matemáticas en Inglaterra.

Babbage, creyendo cierto que había de ser derrotado en los {\it tripos} (exámenes) por Herschl y Peacock, se trasladó de Trinity College a Peterhouse al tercer año, prefiriendo ser el primero en Peterhouse más bien que el tercero en Trinity. Efectivamente, se graduó en el primer puesto en Peterhouse y continuó en 1817 para conseguir el título de Master of Arts M.A. Babbage, Herschel y Peacock continuaron siendo amigos después de haber dejado la Universidad. Cada uno a su modo vivieron de acuerdo con su lema, si bien sus trayectorias fueron muy diferentes. Peacok se hizo clérigo y pronto llegó a ser decano de Ely. Herschel, después de un breve aprendizaje en leyes decidió seguir a su padre en la astronomía. No solamente se distinguió en la astronomía, sino que fue hecho caballero por el rey; llegó a ser Master of the Mint (Señor de la Moneda), evitó todas las luchas científicas y sus biógrafos refieren que su vida estuvo llena de serenidad e inocencia.

En contraste, Babbage había de pasar una vida de amarga frustración causada por sus máquinas matemáticas. Hacia el fin de su vida observó una vez ante sus amigos que nunca había tenido un día feliz en su vida, y habló «como si odiara a la humanidad en general, a los ingleses en particular, y al Gobierno inglés y a los organistas más que a todos». En realidad no era tan malo como para eso, ya que durante una gran parte de su vida se conservó como un hombre sociable, amistoso y con un buen sentido del humor. Una vez, en una visita a Francia con Herschel, Babbage pidió para desayuno dos huevos para cada uno de ellos, diciendo al camarero «pour chacun deux». El camarero transmitió así el mensaje a la cocina, «Il faut faire bouillir cinquante deux oeufs pour Messieurs les Anglais». Estos lograron parar a tiempo al cocinero, pero la historia les precedió hasta París y rápidamente pasó por varias ediciones. En una comida uno de los comensales le preguntó poco después a Babbage si pensaba que la historia acerca de dos jóvenes ingleses que se habían comido cincuenta y dos huevos y una tarta para desayunar era probable, y éste respondió serenamente que «no había nada, por absurdo que fuese, que un joven inglés no fuera capaz de realizar». Un profesor de Edimburgo, que  fue invitado una vez a cenar por Babbage, contó después que había logrado escaparse con gran dificultad a las dos de la mañana después de una velada de lo más deliciosa». En sus viajes frecuentes al continente, Babbage trató de buscar constantemente la compañía de toda clase de gentes: miembros de la aristocracia, matemáticos, hábiles mecánicos.

Sin embargo, la obsesión de Babbage por su máquina le transformó de un joven animado que era en un viejo amargo. Fue perseguido por esta obsesión, según la más verosímil de sus propias versiones diferentes, como resultado de una conversación casual con su amigo Herschel. Este último había traído consigo algunos cálculos hechos para la sociedad astronómica. En su aburrida tarea de comprobar las cifras, Herschel y Babbage encontraron un buen número de errores, y en cierto momento Babbage dijo: «Ojalá estos cálculos hubiesen sido hechos a vapor». Herschel entonces observó, «es completamente posible». Cuanto más pensaba acerca de ello, más se convencia Babbage de que era posible el hacer que las máquinas calculasen e imprimiesen tablas matemáticas. Ideó un tosco diseño de esta primera idea e hizo un pequeño modelo que consistía en 96 ruedas y 24 ejes que más tarde redujo a 18 ruedas y 3 ejes. En 1822 escribió una carta acerca de esta idea a Sir Humphry Davy, presidente de la Sociedad Real, señalando las ventajas de su «Máquina Diferencial» y proponiendo construir una para uso del Gobierno. La Sociedad Real informó favorablemente sobre este proyecto, y el canciller de Finanzas hizo un vago acuerdo verbal a fin de apoyar la empresa con fondos gubernamentales.



Babbage había esperado que el proyecto tardaría tres años en ser realizado, pero constantemente tenía nuevas ideas acerca de la máquina e iba echando por tierra todo lo que había sido hecho antes, y al final de los cuatro años todavía no se hallaba a la vista de su objetivo. El Gobierno le construyó un edificio a prueba de fuego y talleres cerca de su casa. Tras una visita hecha por el duque de Wellington mismo para inspeccionar los talleres, el Gobierno le concedió otra generosa ayuda  para continuar su trabajo. Tras cierto tiempo, Babbage y su extraordinario mecánico Joseph Clement tuvieron un «desacuerdo» acerca de los sueldos. Clement desmontó abruptamente el taller, despidió a sus empleados y partió con todos los dibujos.

En esta coyuntura crítica, Babbage tuvo una idea totalmente nueva. Una máquina analítica, que sería más simple de construir, actuaría más rápidamente y tendría una potencia mucho más extensa que la máquina diferencial. Propuso el esquema entusiásticamente al Gobierno, preguntando si debía continuar con la máquina diferencial o continuar con la nueva idea. Durante ocho años ejerció presión para obtener una decisión definitiva y al final se le comunicó que el Gobierno, sintiéndolo mucho, tenía que abandonar el proyecto. El Gobierno había gastado ya 17000 libras en este proyecto. Babbage había gastado también una suma comparable de su propio bolsillo. La máquina diferencial inacabada, en la cual Babbage había perdido interés, fue depositada en el Museo de King's College, en Londres. Finalmente, este esqueleto de sus sueños fue a parar al Museo de South Kensington, donde se encuentra ahora.

Durante varios años Babbage siguió trabajando sobre su máquina analítica, utilizando para ello sus propios fondos. Abandonó esta idea y decidió diseñar una segunda máquina diferencial, que incluía todas las mejoras y simplificaciones sugeridas por su trabajo sobre la máquina analítica. De nuevo pidió apoyo al Gobierno, pero el canciller de Finanzas lo rehusó. Babbage se quejó amargamente llamándole «Heróstrato de la ciencia, que si logra escapar al olvido será comparado al destructor del templo de Efeso». A fin de cuentas Babbage nunca llegó a completar una máquina que funcionase. Su proyecto era más grande que los medios de que entonces disponía para alcanzarlo. Babbage iba tras algo  superior a una mera calculadora de mesa. Sus planes eran construir una máquina que pudiera calcular tablas matemáticas extensas y disponerlas directamente en imprenta. Como él hizo notar: «Una maquinaria que realice... las operaciones de la aritmética común... nunca será de la misma utilidad que puede ser una máquina que calcula tablas». Su máquina diferencial estaba basada sobre el principio de las diferencias constantes. A fin de ilustrar este principio consideremos un problema para cuya solución la máquina estaba diseñada, el de calcular los cuadrados de los números sucesivos: $1^2,2^2, 3^2,4^2,\dots$, etc. Los cuadrados de todos los números enteros, si tenemos la paciencia de proceder así, pueden ser obtenidos por el simple proceso de adición, con el uso del número 2 como diferencia constante. Ponemos tres columnas. En la primera escribimos siempre el número constante 2 (que representa la segunda potencia); la segunda columna comienza con 1 y añade la constante 2 en cada paso sucesivo; esta suma se introduce en la tercera columna que comienza con 1 y da el resultado. Por ejemplo, $1+2+1$  da 4, el cuadrado de 2;
	$3+2+4$ da 9, el cuadrado de 3; $5+2+9$ da 16, el cuadrado de 4, etc.



Ahora bien, estas operaciones tan simples pueden ser realizadas fácilmente por una máquina, de forma muy parecida a la forma en que cuenta el cuentakilómetros de un automóvil, que suma mediante el giro de ciertas ruedas con números en ellas. El primer modelo preliminar de Babbage de la máquina diferencial, hecho con ruedas dentadas sobre ejes que giraban por medio de una palanca, podía dar una tabla de cuadrados hasta cinco cifras; pero la máquina que él se proponía construir había de trabajar en una escala muy superior. Los planes de Babbage apuntaban a una máquina de no menos de 20 cifras de capacidad y con diferencias del sexto orden en lugar del segundo solamente. Además, a medida que cada número aparecía en la columna respuesta había de ser transmitido a través de un conjunto de palancas a una colección de agujas de acero que habían de estampar el número sobre una placa de imprimir de cobre.

Mecánicamente todo esto era una empresa enorme. Imagínese la variedad y el número de tornillos, roscas, clavijas, eslabones, ejes y ruedas que habían de ser necesitados y recuérdese que las partes de las máquinas construidas en serie, que no requieren ajustamiento a mano, eran prácticamente  inexistentes. Babbage atacó el problema con gran habilidad. El y sus asistentes diseñaban cada parte con gran cuidado, añadiendo mecanismos suplementarios para minimizar el gasto de las partes de la máquina. Babbage se convirtió en un técnico experto, desarrollando herramientas que eran de gran calidad para su tiempo y métodos precursores de algunas de las prácticas modernas en el diseño de instrumentos. Pero tal vez el mismo cuidado y la minuciosidad del diseño fueron su debilidad más grande. Si la máquina hubiera sido alguna vez concluida hubiera consistido en cerca de dos toneladas de maquinaria de relojería de latón y acero, sin precedente alguno en lo que se refiere a standards de medición.

Lo que a Babbage se le ocurrió cuando dejó la máquina diferencial y se dedicó a trabajar en la máquina analítica fue una idea realmente profunda. Había concebido antes una idea que describía él pintorescamente como «la máquina que se muerde su propia cola». Quería decir con esto que los resultados que aparecían en la columna respuesta pudieran afectar las columnas anteriores y así cambiar las instrucciones introducidas en la máquina. La máquina analítica había de ser capaz de desarrollar cualquier operación matemática. Las instrucciones introducidas en ella informarían a la máquina sobre las operaciones que había de llevar a cabo y el orden en que las había de realizar. De esta forma había de ser capaz de sumar, restar, multiplicar y dividir. Tendría una memoria con una capacidad de 1.000 números de 50 cifras. Podría usar funciones auxiliares tales como tablas logarítmicas, de las que esta máquina había de poseer su propia bilioteca. Podría comparar números y actuar de acuerdo con estas comparaciones, procediendo así según normas que no estuviesen especificadas de modo unívoco de antemano en las instrucciones de la máquina.

Todo esto, o mucho de ello, por supuesto, es lo que se ha hecho por medio de los modernos computadores, pero Babbage tenía la limitación de tener que intentarlo mecánicamente. Su diseño no incluía el uso de circuitos eléctricos y no digamos nada de tubos electrónicos. El se propuso hacerlo todo mediante tarjetas perforadas; no las tarjetas de barajamiento rápido que tenemos hoy y que se mueven a través de rápidos y manejables circuitos sensitivos eléctricos, sino tarjetas cuyo modelo eran las utilizadas en el taller de Jacquard. Las instrucciones y las constantes numéricas habían de ser perforadas en estas tarjetas como columnas de agujeros de acuerdo con un código. Cuando las tarjetas eran introducidas en la máquina, ciertos alambres sensitivos eran pasados sobre ellas. Cuando los agujeros se encontraban en el esquema adecuado, los alambres pasaban a través de ellos y ponían en movimiento juntamente las «cadenas» de columnas y demás engranajes. De esta manera la máquina había de llevar a cabo todas sus operaciones. La gran complejidad del sistema no desanimó a Babbage, ya que poseía un retrato en color de Joseph Jacquard, tejido en seda, en cuya realización cerca de 20000 tarjetas perforadas habían sido empleadas. Esto es un esquema muy rudo de la máquina. Charles Babbage estaría orgulloso de ver hoy día cómo la estructura lógica de su máquina analítica ha sido adoptada completamente en los grandes computadores electrónicos actuales.

Además de la idea misma, Babbage realizó muchos aparatos mecánicos de uso  práctico. De la misma forma que hoy día un equipo diseñador de máquinas matemáticas pronto se ve implicado en un mundo de problemas acerca de las propiedades de los tubos de vacío y de los circuitos electrónicos, Babbage se vio envuelto profundamente en los problemas del taller de máquinas y de diseños. El y su grupo inventaron un número de nuevas herramientas complementarias del torno. Entre los trabajadores enormemente hábiles que trabajaban en su taller, uno, J. Whitworth, más tarde Sir Joseph Whitworth, llegó a ser el fabricante más
importante de máquinas de precisión en Inglaterra. Los diseños de Babbage para sus máquinas diversas, que en total constituyen más de 400 pies cuadrados de papel, fueron descritos por expertos contemporáneos como tal vez los mejores ejemplos de dibujos mecánicos jamás realizados.

El libro de investigación operativa de Babbage {\it Sobre la economia de manufacturas y de maquinaria} ha tenido varias ediciones, fue reimpreso en los Estados Unidos y traducido al alemán, francés, italiano y español. En él consideró detenidamente la manufactura de alfileres, las operaciones implicadas, los tipos de habilidad requerida, el gasto de cada proceso y sugirió mejoras en las prácticas de su tiempo. Propuso algunos métodos generales para analizar las fábricas y procesos y para encontrar el tamaño y la situación adecuada de las fábricas. Babbage apreciaba como uno de los mejores elogios que había recibido jamás una observación de un obrero inglés que le dijo: «Ese libro me ha hecho pensar».

Cuando tenía más de sesenta años, Babbage escribió una autobiografia que tituló {\it Relatos de la vida de un filósofo}. Un libro amargo, pero no sin humor, que lleva en su primera página una lista impresionante de sociedades académicas (principalmente extranjeras) tras el nombre del autor. Su autobiografía contiene tanto informe de sus fracasos como de sus logros. La escribió, dijo él, «para hacer... menos inaguantable» la historia de sus máquinas calculadoras.

Pero en realidad no había necesidad de apologética ninguna. La idea de las máquinas fue genial. La historia entera de Babbage es testimonio de la fuerte interacción entre innovación puramente científica, por una parte, y el esquema social de tecnología contemporánea, comprensión y apoyo públicos, por la otra. Sus grandes máquinas nunca llegaron a elaborar respuestas, ya que el ingenio puede trascender, pero no ser ajeno a su propio contexto. Su monumento no es la empolvada controversia de los libros, ni la prioridad en una floreciente rama de la ciencia, ni las pocas ruedas que se encuentran en un museo. El monumento de Babbage, de ninguna manera enteramente bello, pero bien elocuente, es el tipo de investigación que está hoy día recapitulada en los grandes calculadores digitales.




\end{document}