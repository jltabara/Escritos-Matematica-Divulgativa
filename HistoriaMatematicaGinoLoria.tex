\documentclass[a4paper, 12pt, draft]{article}

%%%%%%%%%%%%%%%%%%%%%%Paquetes
\usepackage[spanish]{babel}  
\usepackage[utf8]{inputenc}
\usepackage{tcolorbox}
\usepackage{cmbright}  %%%%%%% El tipo de letra
\usepackage{setspace}
\onehalfspacing  %%%%%%%%%%% Espacio y medio de interlineado
\parskip=1em  %%%%%%%%%%%% Separacion entre parrafos
%%%%%%%%%%%%%%%%%%%%%%



%%%%%%%%%%%%%%%%%
\title{Historia Sucinta de las Matemáticas}
\author{G. Loria}
\date{}
%%%%%%%%%%%%%%%%%

\begin{document}

\begin{tcolorbox}[colback=blue!5!white,colframe=blue!75!black]

\vspace{-1.8cm}
\textbf \maketitle

\end{tcolorbox}

\bigskip

\tableofcontents

\newpage

De la escuela de geómetras que desde fines de siglo hora a Italia, la venerable figura del eximio profesor de Génova, sa ha destacado por su consagración a la historia de las Matemáticas. Su incansable labor de medio siglo ha producido obras utilísimas a todos los cultores de esta ciencia. Tales son: \textit{ Il passato e il presente delle principali teorie geometriche}, que ha alcanzado varias ediciones; \textit{ Le scienze esatte nalla antica Grecia} (5 vol.); \textit{ Pagina de Storia della Scienza}; \textit{ Storia della Matematiche} (2 vol.); etc.

En todas las revistas internacionales de historia de la ciencia ha colaborado asiduamente, y en su patria fundo el \textit{ Bolletino de Biliografía e Storia delle Scienze Matematiche}.
\thispagestyle{empty}
\newpage

\setcounter{page}{1}

\section*{La aurora de la investigación matemática}\addcontentsline{toc}{section}{La aurora de la investigación matemática}



Muchos de mis lectores habrán recibido de algún amigo, apasionado por los viajes, una de esas tarjetas postales ilustradas, en que la Esfinge, grabada en la estampilla, forma contraste con la reproducción de una de las colosales Pirámides que animan el asoleado paisaje cercano al Nilo. Y es seguro que al medir mentalmente las dimensiones de esa estatua, de la que sólo es visible su cabeza, y admirando el ardor genial de que dieran prueba los constructores de aquellos imponentes edificios, uno de los cuales exigió cincuenta años de penoso trabajo a cerca de cien mil esclavos, muchos se habrán preguntado cuáles eran las normas adoptadas para transportar aquellas enormes piedras y para orientar, exactamente, a aquellos edificios, dignos de sostener comparación con muchos «rascacielos» americanos. ¿Conocían, los súbditos de los Faraones, la ciencia de la construcción? En caso afirmativo, ¿cuáles eran sus fundamentos aritméticos, geométricos, mecánicos y astronómicos? He aquí preguntas atormentadoras que, hace siglos, esperan una respuesta satisfactoria. No lo es la respuesta que nos da el padre de la Historia, cuando asegura Heródoto,
que la Geometría nace en el Egipto, ante la 
necesidad de dividir la tierra cultivable, para entregarla a los agricultores, y que el Nilo en sus anuales inundaciones cancelaba. Ni agrega mayores luces la altanera declaración del filósofo Demócrito, quien decía que para construir líneas no lo superaban ni los «arpedonaptas egipcios», ya que tal nombre
---que literalmente significa «tendedor de cuerdas»--- era asignado antiguamente, con toda probabilidad, a modestos funcionarios del gobierno, semejantes, por las tareas que desempeñaban, a los actuales agrimensores catastrales.



Los papiros encontrados en las tumbas de los reyes y príncipes egipcios, como también las paredes de los más importantes edificios, están cubiertos ---como desde hacía tiempo se había hecho notar--- con signos regulares, a los que, muy pronto, se les asignó un sentido literal o numérico. Pero nadie podía dar de ellos su significado, aun cuando el descifrarlos fué, durante siglos, el anhelo permanente de historiadores y arqueólogos.

La luz que había de aclarar tan apasionante misterio fué dada, providencial e inesperada, por un hallazgo realizado durante la expedición que al Egipto llevara a cabo Napoleón a fines del siglo XVIII. Consistió en el hallazgo de un fragmento de piedra, encontrado en la villa Roseta, situada en las bocas del Nilo, en el que estaba grabado un decreto redactado en tres lenguas: un texto era griego y los otros dos estaban escritos en las lenguas usadas por los sacerdotes y por las gentes del lugar. En el texto griego se leyeron los nombres de Ptolomeo, de Berenice y de Cleopatra; nombres que también se habían encontrado grabados en un antiguo obelisco. La comparación, letra a letra, reveló el significado de algunos símbolos que, hasta ese día, habían estado ocultos tras el más impenetrable misterio. Fué, éste, el primer paso hacia la interpretación de los jeroglíficos, gloria que le cupo ganar a Juan Francisco Champollion. Gracias a él puede hoy leerse, con toda exactitud, un escrito egipcio, con la misma precisión con que se leen los escritos latinos o griegos. Como consecuencia de este descubrimiento, la cronología, la historia, las leyes y 
costumbres de los habitantes del antiguo Egipto, no tardaron en ser conocidas en sus más mínimos detalles. Pero en lo que respecta a la Ciencia en general, y a la Matemática en particular, las tinieblas que las ocultaban no pudieron, por mucho tiempo, ser vencidas. Recién en el año 1877 se logró, por fin, descubrir la clave del secreto ---y el hecho fué saludado con grandes muestras de alegría--, cuando un benemérito orientalista de Heidelberg publicó, acompañado con agudos comentarios, ilustraciones y la correspondiente traducción al alemán, un papiro matemático que un señor inglés había adquirido y luego cedido al Museo Británico. Ese papiro
es célebre en todo el mundo, conociéndoselo por el
nombre de Papiro Rhind. Se trata, no de un Manual
de Aritmética ---como lo designara el docto editor---, sino de una recopilación de problemas con datos numéricos, que han sido resueltos aplicando reglas que no han sido ni demostradas ni enunciadas. Más que un cuaderno escolástico, se trata de un modesto «vade-mecum» para uso de un agricultor. En él hay algunas frases que permitieron, a los técnicos en esta materia, determinar la época en que fuera escrito. Y sin que cause al lector una incrédula admiración, este precioso documento fué escrito en el XXXIII año del reinado del soberano que gobernaba el Egipto, reinado que debió haberse desarrollado en el período comprendido entre los años que van del 1788 al 1580 a. de C. Pero este documento no es más que una copia de otro cuyo origen de redacción se remonta al período 1849-1801 antes de la Era vulgar. De manera que la literatura matemática tiene una antigüedad y duración que nos hace recordar los cuarenta siglos que mencionara Napoleón 
en su histórica 
 arenga. ¿Qué ciencia puede proclamar una nobleza de tan antigua data?


%\asteriscos


Toda edad venerable induce al respeto, pero no está, por esto, dispensada de que formulemos sobre ella un desapasionado juicio. Por lo tanto, cabe preguntar qué valor intrínseco posee la obra que lleva el Nº 1 en el Catálogo de la literatura matemática mundial. Un estudio detenido de ella, además de probar que, en el Egipto, se usaba un sistema numeral a base de 10, revela dos peculiares características de la antigua técnica aritmética que, también, ---he aquí un hecho interesante--- se han encontrado en algunos escritos del Medio Evo latino. Una de esas peculiaridades se refiere a los números enteros, y la otra a las fracciones.

Para realizar una multiplicación de dos números enteros, los egipcios imaginaban, como es lícito, al multiplicador escrito como suma de una serie de potencia 2, esto es, bajo la forma $2^\alpha+2^\beta + \dots$, en la que $\alpha,\beta,\dots$ son números enteros decrecientes, el último de los cuales vale 0 si el número considerado es impar. Por consiguiente, toda multiplicación puede ser substituida por una serie de duplicaciones, seguida de adiciones. En último análisis, para los egipcios, la multiplicación, como operación fundamental, no existía; sólo conocían la duplicación.

No menos importante es el uso exclusivo de fracciones con numerador unitario. ¿Provenía tal costumbre de la falta del concepto general de fracción o de la ausencia de nombres y símbolos apropiados para designar a ese ente matemático? Ambas respuestas fueron dadas sin que, pese a ello, se adujeran razones definitivas que hicieran decidir en la elección de una u otra. No voy a insistir sobre los argumentos ofrecidos para sostener cada una
de las tesis en conflicto, porque este problema tiene,
en mi opinión, una importancia secundaria. Nos
interesa más hacer notar que el uso exclusivo de
fracciones fundamentales ---como se llama a las
que tienen \mbox{numerador 1---,} lleva en sí la necesidad
de transformar cualquier cociente de dos números
enteros en una suma de fracciones fundamentales.
Ahora bien, es éste un bello problema aritmético
que fué afrontado, con todo coraje, por los antiguos
egipcios. Y para su solución, aun cuando la intentaran los menos expertos calculadores, el papiro Rhind se abre como una Tabla, que puede ser considerada como el prototipo de los libros de cuentas
efectuadas, porque enseña la descomposición en dos,
tres o cuatro fracciones fundamentales de las cincuenta que responden a la forma $2/(2n + 1)$, en la que $n = 1, 2, \dots , 50$. Esto prueba que desde la época del llamado «Reino Medieval», se distinguía a los números de la serie natural en «pares» e «impares»; y que, además, se había observado que no era necesario ocuparse «ex-profeso» del problema análogo que se plantea para las fracciones de la forma $2/2n$, ya que éstas se transforman, «ipso-facto», en fracciones fundamentales. Y así nos encontramos ante el siguiente problema: ¿Habían notado, los antiguos egipcios, que era suficiente ocuparse directamente de aquellos casos en que el número $2n + 1$ es primo, puesto que de la descomposición
$$
\frac{2}{m}=\frac{1}{a}+\frac{1}{b}+ \cdots
$$
se deduce la descomposición
$$
\frac{2}{mn}=\frac{1}{na}+\frac{1}{nb}+\cdots?
$$



No me atrevo a afirmarlo. Sólo hago observar que sea probable, aplicando aquella observación que se da en la fórmula $2/3 = 1/2 + 1/6$; la descomposición de todas las fracciones contenidas en el papiro Rhind cuyos numeradores son múltiplos de 3, y análoga génesis puede asignarse para aquellas cuyos denominadores son múltiplos de 5, $7$, 9, 11, que están, también, contenidas en el mismo documento. Queda, ahora, por ver cómo han sido obtenidas, o cómo se pueden obtener las soluciones de aquellas fracciones con denominador primo. Un ilustre matemático inglés, Sylvester, advierte que para desmenuzar una fracción propia cualquiera $m/n$ se puede proceder así: llámese $q$ al máximo entero contenido en la fracción inversa $n/m$; la diferencia
$$
\frac{m}{n}-\frac{1}{q+1}
$$
será otra fracción propia, menor que la dada, a la cual se podrá aplicar la misma consideración, y así sucesivamente, hasta un número finito de operaciones, obteniéndose, por último, el resultado buscado. Y bien, si se aplica este concepto a la fracción
$2/(2n + 1)$, se ve que es menester restar de $1/(n + 1)$; y, cuando de tal modo, se obtiene
$$
\frac{2}{2n+1}=\frac{1}{n+1}+\frac{1}{(n+1)(n+2)}
$$
el fin está cumplido. Muchas de las descomposiciones registradas en el papiro Rhind entran, como casos especiales en la fórmula precedente; muchas, 
pero no todas, porque por razones no declaradas e imposibles de adivinar, algunas de las fracciones consideradas son descompuestas, no en dos, sino en tres y hasta en cuatro fracciones fundamentales. Obtener esas descomposiciones constituye un problema del cual, en vano, se busca la solución, ya en el papiro Rhind, ya en la literatura matemática posterior. Adivinar por qué vías llegaron los egipcios a obtener esa solución, en los casos por ellos considerados, es un esfuerzo que intentaron cumplir muchos estudiosos de nuestros días, sin que lograran resultados definitivos. Un íntimo amigo mío, cuando aún era joven, se propuso establecer un algoritmo capaz, por lo menos, de demostrar la exactitud de las descomposiciones efectuadas por los antiguos; permitidme que, al indicar el punto de partida de este intento, haga notar que él no se dió cuenta de haber reconstruido un edificio del que hoy sólo se conocen algunos fragmentos.

Para obtener una descomposición trinomia de la fracción $2/(2n + 1)$, calculamos la diferencia
$$
\frac{2}{2n+1}-\frac{1}{x(2n+1)}-\frac{1}{y(2n+1)}
$$
en la que $x$, $y$ son números enteros a determinarse, de manera que dicha diferencia resulte, ella también, una fracción fundamental. Notando que ella vale
$$
\frac{2xy-x-y}{(2n+1)xy}
$$
o sea
$$
\frac{1}{(2n+1)\displaystyle \frac{xy}{2xy-x-y}}
$$
se nota que llegará a ser fundamental toda vez que el número
$$
\frac{xy}{2xy-x-y}
$$
tenga la forma $p/(2n + 1)$. Ahora bien, para hacer
esta selección del modo más cómodo, debemos construir una tabla de doble entrada que contenga los
valores adoptados por la fracción para valores enteros de $x$, $y$; y bastará extenderla a pocas parejas
de valores para obtener la descomposición trinomia
registrada en el papiro Rhind. Siguiendo una vía
semejante se llega a la descomposición cuatrinomia.


%\asteriscos





Ruego a mis lectores me disculpen por haberlos entretenido con estas particularidades técnicas; pero juzgué necesario hacerlo, ya que sino se podría haber supuesto que un entusiasmo injustificado me conducía a asegurar, como lo hago, que hace un millón de días la aritmética práctica había llegado a un tan alto grado de perfección, que podía afrontar con éxito problemas que, aún hoy, despiertan nuestro interés invitándonos a la realización de importantes estudios. Si, como cuenta Aristóteles, de la Ciencia de aquellos tiempos eran guardianes celosos, en el Egipto, los sacerdotes de Isis y Osiris, la Tabla, sobre la cual estamos tratando, debió haber sido calculada en una época que se pierde en la noche de los tiempos y comunicada, sin ninguna demostración, a los agrimensores y comerciantes contemporáneos, como si fuera un don de los dioses y, unos y otros, la habrán usado sin discutirla, como una revelación celeste. Esta reconstrucción, si arroja alguna luz sobre la psicología de aquellos que habitaban el valle del Nilo, deja, sin embargo, sin solución el problema histórico antes mencionado; ni informa sobre qué servicios puede prestar a quienes pretenden resolver problemas relacionados con la historia de la Geometría de los Egipcios, sobre la cual debo decir algunas palabras.

El papiro Rhind contiene algunos simplísimos problemas que se refieren a la valoración del	
área; en lo que respecta a las que se refieren a objetos cuadrados y rectangulares, nada hay que criticar;
pero otros ---los relativos al cálculo de la superficie de un triángulo y de un trapecio--- dejan al lector titubeando e incierto frente a la interpretación de las figuras ilustrativas, ya que algunos las interpretan como relativas a triángulos y trapecios rectángulos, y otros a triángulos y trapecios isósceles. Pero más interesantes, aún, son aquellos problemas que demuestran que el famoso problema de la cuadratura del círculo, fué estudiado hace cerca de cuatro mil años; y no solamente esto, sino que los egipcios lo habían resuelto con una gran aproximación, ya que la regla aplicada por ellos equivale a adoptar $\pi=(16/9)^2$, esto es, 3.16; valor notablemente calculado, ya que en la Biblia $\pi$ es solamente igual a 3.

Pese a esto, durante medio siglo se sostuvo que los egipcios, en Geometría, habían quedado en un nivel mucho más bajo que el que habían conseguido en Aritmética. Pero en 1931 ha sido publicado otro papiro, del que ya se tenía la seguridad de su  existencia, ignorándose empero su contenido, al cual, los técnicos en estas cosas le dan la misma antigüedad que al papiro Rhind. Este nuevo documento está conservado en el Museo de Bellas Artes de Moscú, y se le conoce, entre los estudiosos, con el nombre del Papiro de Moscú. Dejando de lado otros aspectos contenidos en este papiro, en él encontramos la solución exacta de dos difíciles problemas estereométricos: el cálculo del volumen de
un tronco de pirámide de base cuadrada, y el cálculo de la superficie y volumen de un hemisferio.

No intentaré describir la conmoción que esta publicación produjo en el ambiente culto, conmoción muy justificada, ya que para algunos  parecía que
la gloria de Arquímedes estaba amenazada de ser oscurecida, mientras que otros se alegraban ante la posibilidad de emanciparse del culto que, desde siglos, se venía profesando al pueblo griego.

Tampoco me he de detener mucho en referir las ingeniosas conjeturas que se formularon para reconstruir la vía que condujo a aquellos importantísimos resultados y sobre la cual, el documento en cuestión, no ofrece ninguna información; sólo haré notar que, para darse cuenta de cómo fué posible calcular el volumen de un tronco de pirámide, se supone que el problema análogo para la pirámide se hubiera resuelto experimentalmente, esto es, trasvasando agua, contenida en un recipiente piramidal, a otro de igual base de forma paralelepípeda, y que después ---pasando al tronco de pirámide---, se aplicara la teoría de la similitud, cuyos rudimentos eran ya conocidos,  ciertamente, por los egipcios.

No habiendo sido aceptada por todos como
satisfactoria esta reconstrucción, se sugirieron otras; pero ellas hacen honor a quienes las postularon, sin que se las pueda considerar como elementos suficientes para delinear un cuadro más o menos perfecto de la ciencia egipcia. Mas ha sido bien
justificada la esperanza de que este humillante estado de cosas no fuera eterno. Nótese que los papiros Rhind y de Moscú documentan una ciencia bien desarrollada; de lo cual deducimos que éstos deben ser la expresión de conocimientos de una época muy remota; por lo tanto: ¿no queda perfectamente justificada la esperanza de que sean descubiertos documentos de épocas más antiguas ---que, por lo tanto, aclaren los estadios anteriores---, y otros de fechas más recientes, que permitan el conocimiento de las fases posteriores? Además, se trata de escritos muy humildes, cuyo nivel los hace comparables a cuadernos escolares; de donde, ¿por qué prohibir que se piense que nuevas búsquedas, realizadas en antiguos templos, no puedan llevarnos a documentos con caracteres verdaderamente científicos, compendio de la ciencia de la cual eran celosos depositarios los religiosos del lugar, documentos éstos mucho más preciosos que las joyas descubiertas en la tumba de Tutankamon? Cuando se observan los progresos cumplidos en este campo durante la última mitad del siglo pasado, y se tiene presente el ardor con que actualmente se procede a la búsqueda de documentos en el subsuelo egipcio, tal fe no puede ser equiparada a un sueño de una mente enferma.

Si, estimado lector, abandonamos al Egipto y, atravesando el Mar Rojo, tomamos la ruta de Jerusalén y Damasco, encontraremos otras gentes que, habiendo llegado a un alto grado de civilidad, ilustraron esta región, bañada por el Tigris y el Eufrates, región que, según se lee en la Biblia, fué la 
sede del Paraíso terrestre. Mientras que los egipcios
tuvieron en la antigüedad fama de expertos medidores de la tierra, los asirios-babiloneses llegaron a obtener gran renombre como observadores de los fenómenos celestes y como autores de los primeros catálogos estelares; fueron impulsados a estos estudios, aparentemente estériles, no con el fin de descubrir las leyes que gobiernan el curso de los astros, sino por el inagotable deseo de encontrar signos premonitores del futuro de la vida de los hombres. En esa feliz ingenuidad, esos hombres primitivos, llegaron a observar que las estrellas sonríen complacientes a un coloquio amoroso, así como también escuchan, con olímpica indiferencia, los gritos de dolor y las violentas imprecasiones que se elevan de  los campos de batalla.

La célebre civilización del pueblo babilónico hace siglos que ha desaparecido, y los restos que aun se encuentran, aparecen al viajero como trozos que han quedado de un inmenso naufragio. En lo que respecta a sus estudios de astronomía y aritmética, dejaron pruebas interesantes, no en hojas de papiro o pergamino, sino en tablas de arcilla, en las que aparecieron impresos numerosos signos ---análogos a las incisiones rupestres, tan comunes en nuestra Liguria--- cuya regularidad demostró, de inmediato, que se trataba de letras y números; estas tabulas tenían signos marginales que indicaban que ellas se reunían en series que constituían pseudovolúmenes.

Las excavaciones realizadas en el curso de muchos años, pusieron, a los historiadores, en posesión de millares de estos documentos y despertaron el deseo de conocer el significado de esos que, por su forma, se los llama «cuneiformes». Su secreto significado recién fué descubierto en 1802 por un estudioso alemán, F. G. Grotenfeld, quien, sobre este tema, presentó una memoria a la Sociedad de Ciencias de Götingen; pero esta famosa Academia científico-literaria fué atacada de ese terror abúlico, tan frecuente en los grandes cuerpos culturales, cuando se les pide que se pronuncien sobre el valor de un descubrimiento espantosamente original. Y así es que se le negó hospitalidad en sus Actas, para luego, en un tardío arrepentimiento, concedérsela un siglo más tarde cuando las explicaciones hechas por Grotenfeld, obtenidas por otra vía, entraron a formar parte de la ciencia oficial.


Una vez que fueron descifrados los caracteres cuneiformes, fué posible reconstruir la historia de aquel pueblo que los utilizó durante tres milenios de vida, anteriores a nuestra era. En el aspecto que a nosotros nos interesa, se supo que ellos se servían de un sistema aritmético, no sólo en base 10, como ya hemos visto que lo usaban los egipcios, sino hasta de base 60 y que, con su sola ayuda era posible expresar, con la palabra y las manos, números enormemente grandes, como por ejemplo el siguiente: 195955500000000. Mientras que la explicación de la presencia de la base 10 es dada en forma general, y quizá por vez primera, por Aristóteles, quien la descubre en el uso, común, de calcular utilizando los dedos de las manos; la intervención del 60 es justificada de varias maneras. No podemos detenernos a exponer las varias explicaciones propuestas, limitándonos a hacer notar el hecho de que cada 360 días los fenómenos celestes se reproducen con gran aproximación, debiéndose, sin embargo, no tomarse esto al pie de la letra, por dos razones principales: una, que la aritmética precede, no sigue, a la astronomía de la que es la más potente y fiel auxiliar; otra, que los babilónicos, expertos observados de los hechos que tienen por teatro el cielo, no podían contentarse, como hacen los banqueros, con la grosera aproximación que hace que el año dure 360 días.

Lo que debe hacerse notar es que, gracias al pueblo que nos ocupa, los números 60 y 360 consiguieron en la ciencia un puesto estable y de importancia; todos saben que en 360 grados se divide, aún hoy, a la circunferencia, pese a los esfuerzos que en contra realizaron, hacia fines del siglo XVIII, los creadores del sistema decimal de pesos y medidas; que en 60 minutos se divide el grado y en 60 segundos a éste.

Los documentos que iluminan sobre la estructura de la aritmética asirio-babilónica, son de dos especies. Algunos son simples elencos numéricos; así las célebres tablas descubiertas en Senkreh que corresponden al período 2300-1600 a. de C., en las que se encuentran los elencos de los cuadrados de los números de 1 a 60 y de los cubos de 1 a 30. Sobre su interpretación no cabe ninguna duda, ya que ninguna discusión se ha planteado al respecto. Otros contienen la solución de problemas geométricos, y que hoy día son objeto de profundos estudios. Entre
los problemas tratados encontramos la determinación de la diagonal de un rectángulo de lados conocidos, lo cual prueba que las propiedades características del triángulo rectángulo eran conocidas, no sólo para el triángulo de lados 3, 4, 5, conocido por los egipcios, sino que, también, para otros, y esto cerca de un milenio antes de nacer Pitágoras, a quien, por consenso general, se le atribuye el descubrimiento de esas propiedades. Es ésta una constatación basada sobre hechos, que podrá doler a los modernos neopitagóricos, pero que no debe admirar a quien estudie a este pueblo, que sabía edificar edificios que, por su solidez y grandiosidad, podían sostener comparación con las pirámides egipcias; de un pueblo que ---entendedlo bien, señores ingenieros--- estaba en situación ---como lo prueba un mármol recién descubierto--- de delinear con satisfactoria precisión la planta de un edificio. ¿Quién pudo haber supuesto que la primera raíz de la geometría descriptiva se encontraba cerca de la Torre de Babel?




Otras sorprendentes revelaciones son de esperar, ya que el asiduo pico del explorador, con su brillosa punta, hace saltar, diariamente, del granítico suelo de la Arabia, luminosas chispas que rasgan las tinieblas, varias veces seculares, de la historia de este pueblo.

%%\asteriscos

A las pruebas expuestas de las actividades matemáticas cumplidas en dos grandes regiones africanas, otras vienen a nuestro encuentro desde el Oriente, que merecen nuestro interés por las importantes repercusiones que los resultados obtenidos tuvieron sobre la constitución de esa ciencia, de la que los europeos nos sentimos orgullosos. Que alguno de los documentos a que aludo provengan de la India, no debe admirar a quien recuerde la frase, de hiperbólica admiración, que Plutarco, en su «Vida de Alejandro Magno», dedicó a los gimnosofistas, depositarios de toda la sabiduría del pueblo hindú. ¿Quién no ha leído las descripciones?, ¿quién no conoce las fotografías de los espléndidos edificios sagrados existentes en las tierras que bañan el Indo y el Ganges, y que prueban la existencia de grandes artistas en la técnica de las construcciones civiles? Finalmente, debemos recordar que el sánscrito ---idioma clásico de la asoleada península---, por perfección, nobleza y riqueza, ha sido juzgado capaz de sostener comparación con el griego y con el latín; y que por su antigüedad supera a ambas lenguas, habiéndosele asignado más de veintinueve siglos de vida.

Saliendo de estas generalidades y volviendo a lo que directamente nos interesa, diré que sus escritos matemáticos comenzaron a ser conocidos en Europa hacia la segunda década del siglo pasado [el siglo {XIX], gracias a las publicaciones de dos distinguidos funcionarios del gobierno inglés, los cuales, habiendo vivido mucho tiempo en Calcuta, llegaron a familiarizarse con el idioma local. Las obras que ellos tradujeron al idioma inglés ---y lo mismo podemos decir de otra tercera traducción publicada casi cien años después, por obra de otro miembro del Gobierno Británico---, presentan algunas analogías con las descubiertas en el Egipto, por tratarse, en gran parte, de colecciones de problemas numéricos, que son resueltos aplicando reglas no demostradas. Pero aquí termina la analogía.



No se trata ciertamente de apuntes tomados por algún mediocre escolar o humilde agricultor, sino de escritos elaborados por una o varias personas de alta cultura. Para demostrarlo, baste decir que están redactados en verso, circunstancia que cito no como objeto de simple curiosidad, sino porque la prosodia sánscrita está regulada por normas rigurosísimas; estas obras se sustrajeron a las posibilidades de ser mancilladas, como lo fueron los manuscritos europeos por parte de amanuenses infieles y audaces. Además, la forma poética logró dar a los enunciados de los problemas tratados un
contenido más atrayente del que hoy usamos. He
aquí un ejemplo interesante: una de las más célebres obras que estamos tratando tiene este título:
\textit{ Livalati}, nombre de una mujer a la cual el autor
se dirige, y que contiene numerosos problemas, como, por ejemplo,
el del siguiente tipo: «Preciosa 
y querida Livalati, de ojos semejantes a los de un joven gamo, dime cuál es el número que se obtiene al multiplicar 135 por 12». ¡La conmoción que produce esta cumplida invocación, no impide a Livalati contestar exactamente! Otro ejemplo de enunciado alado y amoroso, pero de mayor dificultad, es el siguiente: «Con flores de loto que constituían un bello ramo, un tercio, un cuarto y un sexto fueron ofrecidos a los dioses Silva, Visnú y al Sol; un cuarto fué ofrecido a Bhavani; seis fueron dados a un personaje venerable. Dime, pronto, ¿cuál era el número de estas flores?» Este es un problema que hoy se resuelve mediante una ecuación de primer grado, pero también es fácil responder por el método de la falsa posición.

A la más antigua obra hindú sobreviviente, algún historiador indígena le atribuye la sorprendente edad de un millón de años; dato cronológico éste que no debemos tomar muy en serio, puesto que, sin duda, está envuelto en la atmósfera de la fantasía oriental, ya que la redacción de la obra que hoy tenemos a nuestro alcance no se remonta más allá del III o IV siglo de nuestra era. Las reglas que en ella se aplican han sido tomadas, sin lugar a dudas, de otras obras, de las que nada sabemos, por lo que es imposible decir si son indígenas o exóticas. Es de hacer notar la presencia de las propiedades características del triángulo rectángulo, expuestas por medio de triángulos especiales y aplicada a la construcción de ángulos rectos mediante cuerdas ---cosa que ya habían hecho, pero con menor amplitud, los egipcios---. Otras obras que los eruditos en filología sánscrita datan de los
siglos V y VI de nuestra era, contienen la primera exposición de aquel sistema decimal de posición que ninguno ignora, ya que es el que todos aplicamos hoy, con el uso metódico del cero y con nombres especiales para las sucesivas potencias de 10 hasta la XVII inclusive. Tenemos, así, que el nombre de «cifras arábigas» comúnmente usado, debe entenderse, no como una declaración de propiedad, sino como un homenaje a los que las enseñaron a los europeos. Y frente a la enormidad del número 10 elevado a la XVII potencia, es ella síntoma de una característica mental del pueblo que nos ocupa, que debe destacarse en abierto contraste con otra que es característica de los griegos: mientras que éstos mostraban una invencible repugnancia para concebir números muy grandes, los hindúes se complacían afirmando que la casa de Brahma estaba alegrada por la presencia de \textit{ cien millones} de hijos y que el cielo estaba  habitado por 24 \textit{ millones de millones} de dioses.

Introducido el cero en la aritmética, mudo protagonista de nuestro sistema aritmético, fueron los hindúes los primeros que investigaron sus propiedades, demostrando que cuando se lo agrega a cualquier número no lo perturba, mientras que si se lo combina por medio de la multiplicación, su resultado será nulo; luego intentaron usarlo como divisor y se admiraron ante la aparición, en la aritmética, de un ente nuevo que poseía la sorprendente cualidad de mantenerse inmutable si se lo adicionaba a cualquier otro número: así hizo su entrada, en nuestra ciencia, el \textit{ número actualmente infinito}, y el autor, de quien tomo la noticia de este importante progreso, entusiasmado dice que el nuevo número «es semejante al Infinito, al Dios Invariable que no sufre ningún cambio, mientras los viejos mundos desaparecen y nuevos mundos son creados, mientras innumerables y especiales criaturas nacen y mueren».


Otras interesantes innovaciones que se leen en las obras escritas en sánscrito son las resoluciones, en números enteros, de las ecuaciones indeterminadas de 1º y 2º grado; la expresión del área de un triángulo, en función de la longitud de los lados y la análoga para un cuadrilátero inscripto en un círculo, pero sin la cláusula que es aplicable, exclusivamente, cuando el cuadrángulo sea inscriptible. Esta omisión esencial modera el entusiasmo que nos dominaría frente a un resultado tan bello; y este sentimiento de celo se acrecienta cuando se lee, más adelante, después de la expresión del área de un triángulo como producto de la base por la mitad de su altura, la afirmación de que el volumen de una pirámide es igual al producto de su base por la mitad de su altura; y la forma métrica ---ya lo hemos dicho antes--- excluye la posibilidad de que se trate de un casual error de transcripción. Piadosa excusa a la que no se puede recurrir cuando se quiera obtener la absolución de los hindúes por haber calculado el volumen de una esfera con un procedimiento inaceptable. Pero éstos no son los únicos errores que se encuentran en las obras provenientes del sagrado Ganges; porque la aplicación imperfecta de reglas exactas induce, naturalmente, a pensar que el calculador seguía, ciegamente, una vía que otros habían previamente arreglado, que había aplicado, casi automáticamente, reglas de las que ignoraba su génesis y justificación; lo que nos lleva a preguntarnos si sus presuntos maestros eran o no gentes de su misma estirpe.

%\asteriscos




Es necesario tener bien presente que los autores de los escritos hindúes, hasta ahora descubiertos, descifrados y traducidos, pertenecen a una época un poco posterior a aquella en que brillaron los más grandes matemáticos griegos. No debemos olvidar que de muchos de los escritos de Arquímedes ---para no hablar de otros geómetras---, han llegado hasta nosotros sólo sus títulos ---uno de ellos se refiere, precisamente, al cuadrilátero inscriptible en un círculo---, y solamente, de algunos, se descubrieron trazas seguras en la literatura de los árabes; además, de los trece libros de Diofanto, el príncipe de los aritméticos griegos, seis han llegado hasta nosotros, y nada autoriza a decir que en los restantes ---los cuales trataban los temas más interesantes\mbox{---,} fuese excluída la exposición, por citar un ejemplo, de la resolución, en número entero, de las ecuaciones indeterminadas de 1º y 2º grado, de la cual los hindúes celebraban ser el descubrimiento más importante que hubieron obtenido.

Es indispensable liberarse del prejuicio de que, cuando no existían la imprenta y la telegrafía, con o sin hilos, cuando ni se suponía la posibilidad de comunicaciones rápidas como las que permite el ferrocarril, automóviles o aeroplanos, no existiesen comunicaciones entre países lejanos. Viajeros curiosos y audaces, comerciantes impacientes por ampliar la búsqueda de nuevos mercados, siempre han existido; unos y otros, cuando llegaban a la meta de sus peregrinaciones, se ofrecían gustosos a comunicar a sus huéspedes todo lo que sabían; deseosos de novedades, escuchaban sus palabras y, cuando regresaban a sus patrias, contaban cuanto habían oído y visto, enseñaban todo lo que habían aprendido; y así fueron posibles aquellos providenciales fenómenos de exósmosis y endósmosis en todos esos campos en que el inquieto espíritu humano busca el por qué de todas las cosas, y no solamente aquello que le interesa. Y es evidente que si Alejandro de Macedonia concibió el proyecto de llegar hasta las riberas del Indo, vestido con su armadura de conquistador, ello se debe a que su codicia fué despertada por los cuentos que describían aquellas tierras como aptas, por la intensa fecundidad del suelo y por la refinada civilización adquirida, de compensar los riesgos de tamaña empresa, cuya audacia se nos hace hoy casi demente.

Si se admite la existencia, en antiquísimos tiempos, de variadas relaciones entre la Grecia y la India, nace el problema de saber si la influencia se ejercitó, de preferencia, del Occidente hacia el Oriente, o bien, en sentido inverso. Las obras puestas hoy a disposición de los historiadores no bastan para resolver esta grave cuestión, aunque el número de esas obras haya aumentado recientemente. Para concluir, se oponen a la solución de este problema las condiciones políticas de la península ---¿lo crees, lector?---, en otras palabras, el Ghandismo. Mientras que los documentos de que disponemos, en estos últimos tiempos, provenientes de funcionarios enviados por Inglaterra a la India, estaban impregnados del concepto de que los hindúes eran deudores de la Grecia de todo  su saber matemático; desde hace algún tiempo se hace sentir la voz de doctos hindúes que, en trabajos de rigurosas consideraciones científicas, llevan un vivaz y poderoso espíritu de nacionalismo. La tierra que se alegrara viendo que dos de sus hijos, a breve distancia de
tiempo, eran honrados con el Premio Nobel, se industria hoy para demostrar que era grande aún antes de absorber la ciencia europea a través de la dominadora Inglaterra; sentimiento altamente honorable ---y que a nosotros los italianos nos es profundamente simpático, en razón de la evidente analogía con nuestro reciente pasado---, pero que obliga a una extrema y oculta prudencia antes de aceptar las conclusiones en que se inspira, porque nada es más peligroso, para un historiador, que estar animado ---aún cuando sea inconscientemente--- de la aspiración de establecer una tesis
preconcebida.

%\asteriscos

Esta actitud reservada que debe adoptar quien investigue acerca de la antigua ciencia hindú, no debe, ya sea por razones de otras especies, ser abandonado por quien desea dar un resumen exacto de las contribuciones aportadas a nuestra ciencia por otra inmensa aglomeración humana que, desde hace un tiempo, se halla convulsionada por las discordias intestinas y por la guerra que la atormenta, pese a lo cual atrae la atención del mundo entero. La China ---nación a la que estamos aludiendo--- cerca del V siglo antes de nuestra era, vió nacer a uno de los más famosos moralistas, Confucio, y se enorgullece de haber inventado, mucho antes que nosotros, la pólvora del fusil, la hélice como motor y los caracteres móviles de la imprenta; además, para
observar el curso de los fenómenos celestes, instituyó, desde la antigüedad, una repartición oficial que denominó, pomposamente, Tribunal de las Matemáticas. Que estos maravillosos hechos estuvieran en desacuerdo con cuanto referían los europeos vueltos del Extremo Oriente, lo hacía destacar, a fines del siglo XVIII, un ilustre historiador de las matemáticas, Montucla, el cual, con mucha finura decía: «Es necesario decir que el pueblo chino ha sido un pueblo privilegiado; porque el progreso de las ciencias y de todas las buenas instituciones se produjo de una manera diametralmente opuesta a la verificada en todas las otras naciones por nosotros conocidas. En éstos, antes de que se les pueda suponer la existencia de una ciencia, han transcurrido muchos siglos de barbarie. Entre los Chinos, en cambio, se nota que sus esfuerzos son dirigidos a dar vida a un pueblo y dirigir los primeros pasos de una ciencia especulativa como es la astronomía.»

Es indispensable hacer observar que las más antiguas noticias, relativas al saber de los chinos, llegaron a Europa por intermedio de los misioneros que fueron a la China hacia la segunda mitad del siglo XVI; sucedió entonces un hecho admirable: el gobierno de Pekín se apresuró a ofrecer a los \textit{ Gesniti}, sus huéspedes, los puestos de efectivo comando en el Tribunal de las Matemáticas; así, todos supieron que el principal trabajo de éste era la redacción anual del Calendario civil y que, en el desempeño de este modesto oficio, los Mandarines de las largas colas no se sentían muy expertos.¿En que consistía su tan mentada familiaridad con los fenómenos celestes? Las dudas al respecto se vieron 
reforzadas cuando, hace unos años, dos hábiles astrónomos japoneses demostraron que un eclipse, que
los chinos se jactaban de haber previsto y observado,
\textit{ no era visible en Pekín}.

Estos hechos, y muchos otros semejantes, hicieron nacer la duda de que los maravillosos cuentos, referidos por los crédulos misioneros, no estaban de acuerdo con la verdad. A reforzarlo vino la observación de que,
en la psicología del pueblo que nos 
ocupa, existen dos peligrosas características: una consiste en un desconfiado orgullo nacional; la otra por la tendencia a darle un impulso que llega a alturas hiperbólicas y que atribuye a su pensamiento métodos y descubrimientos a los cuales ellos no habían pensado jamás dirigir la atención de su espíritu. Agréguese a esto que, si algún europeo, para juzgar sobre los fundamentos de esta duda probó de preguntar a algún habitante del ex-celeste imperio las pruebas sobre las que afirmaba el valor de la antigua ciencia, se oía, como respuesta, que todas ellas se encontraban en una obra que un desgraciado emperador, que vivió durante el III siglo de nuestra era, condenó al fuego por razones desconocidas; incendio verdaderamente providencial, ya que bastaba para tapar la boca al indiscreto preguntón.

Es necesario ver ahora que las relaciones comerciales de la China con el Imperio Romano pueden ser históricamente probadas hacia fines del siglo II de la era vulgar; que relaciones más íntimas se establecen con la India, como lo demuestra la introducción del Budismo en la China hacia fines del siglo I d. de C.; y que, por último, durante el siglo XIII, Marco Polo, con su viaje al Extremo Oriente, dió un ejemplo que fué luego muy seguido. De todo esto sacamos como consecuencia que nada nos autoriza a dudar, más bien todo nos lleva a creer, que influencias de otras estirpes se hicieron sentir en toda el Asia, sin excluir la extremidad oriental.

Para iluminar los numerosos interrogantes que nuestra mente formula, el mejor método consistiría ---y lo diría también Monsieur de la Palisse--- en consultar las obras chinas que escaparon a la manía destructora de aquel legendario emperador. Pero ---aparte de las dificultades de una lengua que es monopolio de unos pocos electos--- muchas de aquellas sobre las que hay información fidedigna, y a las que se atribuye antiquísimo origen, se presentan, después de haber sufrido muchos arreglos, aun en tiempos más cercanos a nosotros, como entidades inciertas, de manera que ¿quién es el que garantizará que ellas no tengan sus raíces en obras extranjeras? Son obras del género de las que hemos encontrado en Egipto y en la India; esto es, colecciones de problemas con datos numéricos resueltos ---no siempre exactamente--- en base a reglas no enunciadas, y menos demostradas, que ofrecen desemejanzas contradictorias tan graves e impresionantes como las que notáramos en las obras escritas en las riberas del Ganges. Para darle una idea al lector, sin abusar de su paciencia, me limitaré a dar un sólo ejemplo.

En la teoría de los números, se encuentra bajo el nombre de «problema de los restos», el de determinar un número, el cual, dividido por números dados, conduzca a restos fijados. Gauss fué el primero que, en 1801, enseñó a los europeos a resolverlo en una célebre obra que de inmediato le aseguró los fastos de la gloria. Y bien, el método inventado por Gauss se encuentra en obras chinas
anteriores en muchos siglos a la \textit{ Disquisiciones arimeticae}, sin ser demostrado, ni aplicado siempre con
exactitud, pero designado con un nombre especial para informar sobre su importancia y su originalidad.

En la misma obra se encuentra enunciado este otro problema: una mujer embarazada de 29 años de edad, espera el nacimiento de su hijo para el noveno mes del corriente año; ¿cuál será el sexo del niño? No quiero defraudar al lector sobre la respuesta que espera: tomad 49, agregad el mes de la concepción, sacad la edad de la madre, luego el cielo 1, la tierra 2, el hombre 3, las estaciones 4, los elementos 5, las leyes 6, las estrellas 7, los vientos 8 y las provincias 9; si el resto es impar, nacerá un varón, si es par, una mujer. Se nota que, realizado el cálculo, toda esta confusa computación equivale agregar 4 entre el mes de la concepción y la edad de la madre, operación insignificante para determinar la paridad de un número; de ahí  que se concluya simplemente que nacerá  un varón
o una mujer, según que resulte impar o par la diferencia entre la edad de la madre y el mes de la concepción. Claro está que dejo a los obstétricos de profesión el que pronuncien su juicio sobre el valor de esta pretendida regla adivinatoria; pero me creo obligado a llamar la atención del lector sobre la inútil complicación de la referida respuesta, porque ella manifiesta una «forma mentis» del pueblo chino, edición pálida de numerosos ejemplos. Y para no hacer más largo esto, me limitaré a citar un ejemplo que ilustrará magníficamente la observación hecha.

En una de las colecciones, de la que he podido conocer con exactitud su contenido, hay un problema cuyo fin es la determinación del diámetro inscripto en un triángulo, en base a un cierto sistema de datos numéricos. La solución no está totalmente expuesta; solamente se asegura que el problema se resuelve por medio de una ecuación de 10º grado, que el autor escribe completa, asegurando que su raíz es 3, lo cual es falso, porque la mencionada ecuación \textit{ no tiene por raíz} 3. Ahora bien, el problema propuesto es tan simple que extraña la aplicación de una ecuación de grado tan elevado; y si realizamos una simple consideración elemental sobre la figura, vemos que en realidad depende de una ecuación de tercer grado, \textit{ y ésta es la que admite, efectivamente, la raíz 3.}

Frente a tal flagrante contradicción, y para ser extraño a toda maligna sospecha, ¿cómo alejar la suposición, que de inmediato llega a nuestra mente, de que ese problema no haya sido transcripto de una obra desconocida para nosotros, y que el transcriptor, para destacarse frente al lector, haya substituído la verdadera ecuación del problema, que le pareciera muy humilde, por otra de mayor dificultad, sin llegar a la alteración del resultado final?

No es éste el único ejemplo que podemos ofrecer de hechos que, si no queremos llamarlos con el nombre poco parlamentario de «trampas», pueden ser calificados con elegancia moderna de «bluffs»; y estos deplorables hechos crecen desmesuradamente a la par del crecimiento de las obras chinas, que los chinólogos ponen a disposición de los matemáticos que no pueden leerlas. Esto hace penoso y espinoso el camino que el historiador debe recorrer, ya que se le ha asignado el oficio ingrato de agente de investigación, con la perspectiva de ser luego llevado al cargo del juez que condena. Por eso es que tenemos el derecho, y hasta el deber, de mantener en cuarentena las magnilocuentes afirmaciones de los representantes del pueblo chino.

Así que, mientras que no negamos admitir que algunos elementos legados por los egipcios, babilónicos e hindúes hayan llegado a ser sangre y médula de nuestra ciencia, debemos creer que hubiéramos sido más bien maestros que discípulos de un pueblo, en el que el orgullo de raza hizo olvidar el respeto que se le debe a la más indiscutible de las propiedades: la propiedad sobre la obra del
ingenio.

Estando por dar término a esta rápida excursión, de la cual soy un modesto cicerone, me asalta la duda de que para muchos de mis lectores haya sido amarga exclusión el hecho de que, más que dar resultados, hayamos enunciado problemas graves, cuya solución aún hoy se espera. Es un hecho que podrá ser mediocremente agradable, pero no causa de maravilla, para quien sepa que la historia de la ciencia en general y de las matemáticas en especial, cuando se entiende la palabra historia en su significado más elevado, es una disciplina muy reciente. En el siglo XVIII se creía que el fin del historiador era el de juntar, sin discutir, datos biográficos relativos a personajes representativos en una especialidad y de un cierto tiempo, agregando noticias, no muy profundas, sobre la obra que lo hiciera ilustre. Pero la influencia del libre examen, que induce a la crítica desapasionada de todo el pasado, y la creación del método histórico, distinguido honor que le corresponde al siglo XIX, tuvo por consecuencia la no aceptación de toda aquella noticia que no estuviera documentada y controlada, y el rechazo de cualquier proposición histórica de la que no se conociese una convincente demostración. De ahí la necesidad de que los más
representativos conocedores de la filología
clásica
y aquellos que estaban familiarizados con las lenguas orientales, vinieran en socorro de los cultores
de la ciencia, que titubeaban ante la interpretación
de antiguos textos, aportando ediciones definitivas acompañadas de ilustraciones y versiones, de la más reputada producción científica. Y como esta colaboración fué ampliamente acordada ---especialmente del otro lado de los Alpes---, fué posible iniciar una reconstrucción «ab imis fundamentis» de todo el edificio anteriormente construido.

Es éste un trabajo al cual, hace cerca de un siglo, se le está dedicando las mejores fatigas; trátase de una tarea que en algunas de sus partes está avanzada, pero que jamás será agotada, ya que la obra del historiador es semejante  a la fatiga de
Sísifo: nuevos documentos hacen 
que se precipiten 
al valle las rocas que creíamos haber juntado en la cima del monte. Esta característica hace que a la gesta de los pensadores de épocas lejanas se la juzgue como poseyendo la misma dignidad de la verdadera ciencia. Que cualquier disciplina que no atormente, al que la cultiva, con incesantes interrogaciones, no es un ser vivo, sino un gélido cadáver. Mas quiero imaginar que la Historia de las Matemáticas no sea esto, buscando apoyo para esta opinión en todo lo que venimos exponiendo.



\newpage

\section*{Los grandes geómetras de la antigua Grecia}\addcontentsline{toc}{section}{Los grandes geómetras de la antigua Grecia}



A las ciencias naturales se las designa bajo el nombre común de \textit{ ciencias experimentales}, siendo conocida la matemática, por antonomasia, como las \textit{ ciencias del razonamiento}, pudiéndose afirmar que es, en definitiva, lógica en acción. Con esto no se quiere decir que las observaciones sean extrañas a la búsqueda matemática, puesto que, por el contrario, la historia muestra que, en muchas ocasiones, una figura construida con cuidado, ha llevado a la formulación de proposiciones geométricas; y que se ha llegado a un buen número de teoremas aritméticos inspeccionando tablas numéricas exactamente calculadas. Lo que queremos afirmar es que ninguna proposición matemática tiene derecho a ocupar un puesto en el campo de nuestra ciencia, si no se presenta como el último eslabón de una sólida cadena de razonamientos. Ahora bien, ningún documento, ni egipcio, ni babilónico, ni chino, ofrece trazas de tan preciosa distinción; aunque no se quiera dudar, ni negar, que alguna argumentación, ya sea simple e inconsciente, sea la base de las soluciones de los problemas resueltos por personas pertenecientes a aquellos antiguos pueblos. Pero
para encontrar pruebas, indiscutibles e indiscutidas, 
de la presencia de verdaderas y propias demostraciones, basta con recurrir a la literatura del pueblo privilegiado por la naturaleza, y del cuál vamos a ocuparnos ahora.

%\asteriscos

Desgraciadamente las informaciones que existen sobre la primera época del desarrollo de las matemáticas, son muy raras; provienen, en general, de personas que carecían de competencia 
en estas disciplinas, faltándoles, por ello, la imprescindible precisión; sus autores son historiadores y filósofos, de los que es vano esperar enunciados del tipo a que estamos habituados, tanto más, que algunos de ellos están animados de un, no siempre, justificado entusiasmo, mientras que en otros son visibles la hostilidad de los preconceptos de escuela.

Vemos así que se nos informará que Tales, el primero de los siete sabios de Grecia, creó un sistema filosófico basado en el concepto de que el agua es el elemento primordial del cosmos; que predijo el eclipse que tuvo lugar en el año 585 a. de C. Pero estas informaciones nada nos dicen sobre el origen y esencia de sus conocimientos científicos. No son más precisas las informaciones que, en torno al saber matemático de Anaximandro y Anaxímenes, que
llegaron a la jefatura de la escuela de los Físicos
Jónicos ---creada por Tales---, poseemos.

En condiciones menos desfavorables se encontrará aquel que intente estudiar la obra científica de Pitágoras. Sólo que chocará contra un obstáculo gravísimo: el secreto que rodea las disciplinas profesadas por este filósofo y que él impuso a sus discípulos. Estas disciplinas pudieron llegar al dominio general, y difundiéronse entre los estudiosos, después de la disolución de la comunidad pitagórica y después de haber sufrido muchas deformaciones, ya sea de parte de ardientes admiradores, como también de parte de irreductibles opositores. Todavía la prudente crítica histórica moderna debe concluir de fijar las características de la fisonomía científica de Pitágoras, de la que voy a ocuparme brevemente.

Lo que distingue un conjunto de proposiciones matemáticas de un sistema verdaderamente científico, es la presencia de las definiciones de los entes considerados. Pitágoras, es digno de ser considerado el jefe de la gran familia de los matemáticos, por haber dado definiciones, que aun hoy son dignas de consideración ---por ejemplo: haber dicho que el punto es la unidad que tiene posición---, y por haber dividido a toda la matemática en aritmética, música, geometría y astronomía, elementos del \textit{ quadrivio}, que fuera el plan de estudios para la instrucción de la juventud de toda la Edad Media. Y si ciertos acercamientos entre números y cosas pueden ser considerados como signo de una ardiente fantasía, no se puede negar que ellos ejercitaron una
gran influencia sobre mentes selectas, como, por
ejemplo, sobre Dante, cuando delineó el plano general de la \textit{ Comedia}; y sobre su maestro Virgilio,
cuando escribía «Numerus Deus impar gaudet». Pitágoras, al colocar el número como fundamento de
todo su sistema filosófico, ha conquistado el gran
mérito de haber impreso a la investigación científica
ese ordenamiento que permite a las disciplinas naturales celebrar los más clamorosos triunfos, ordenamiento que preveía Leonardo da Vinci cuando decía que las ciencias son cada vez más verdaderas
cuanto más se informan en los métodos matemáticos.

Sea, que Pitágoras fué conducido a la máxima general «las cosas son números» por el descubrimiento de que los fenómenos acústicos están gobernados por una ley numérica; o que esta ley, haya sido descubierta aplicando aquel principio general; todos deben reconocer que, en aquella oportunidad, se dió el más importante pasó para la explicación de los fenómenos naturales.

No se limitó, el genial filósofo de Samos, a determinar esta generalidad, ya que ha legado su nombre a algunos importantes descubrimientos. Así, por ejemplo, cuando los constructores de pavimentos, queriendo usar, exclusivamente, lajas regulares, todas de la misma especie, dicen que éstas sólo pueden tener la forma de triángulo, cuadrados o hexágonos; no hacen más que repetir una observación de la escuela de Pitágoras. Pero no se limitó éste a estos especiales polígonos; sino que los consideró en general, no sólo suponiéndolos convexos, sino también estrellados, y fijándose, con particular atención, sobre el pentágono regular de la misma especie. Y sus discípulos se enorgullecieron del descubrimiento que esta figura produjo, a la que llamaron \textit{ pentalfa}, que la adoptaron como signo de reconocimiento de los miembros de la escuela. El vínculo establecido era tan firme que se cuenta ---no se si es histórico o legendario---, que habiendo caído enfermo un pitagórico, durante un viaje, y sintiéndose morir, sin haber podido compensar las atenciones de su huesped, sugirió a éste poner sobre la puerta de la casa una muestra en la que se dibujara un pentágono estrellado, asegurándole que, tarde o temprano, sería pagado. Y así sucedió; tiempo más tarde, un cofrade del muerto, que accidentalmente pasaba frente a este albergue y puesto al tanto de los hechos, se apresuró a gratificar al propietario:

\begin{quote}

¡Oh, gran bondad la de los antiguos caballeros!

\end{quote}

%\asteriscos

Puestos en la tarea de estudiar a las figuras dotadas de completa regularidad, los pitagóricos extendieron al espacio las consideraciones geométricas
obtenidas, logrando, así, el descubrimiento de cinco
poliedros regulares. Ahora bien, si el concebir al
tetraedro regular y al cubo, no presenta ninguna dificultad; si al octaedro podemos llegar superponiendo
dos pirámides cuyas bases son dos cuadrados iguales entre sí; concebir al dodecaedro y al icosaedro
regulares, exige tal y tanta familiaridad con las
cuestiones estereométricas, que nos maravillamos y
admiramos al constatar que se haya podido llegar
a tanto, cinco siglos antes de nuestra era. La alegría
que les produjo a nuestros lejanos progenitores ---todos ellos plenos de misticismo matemático---, el
poseer formas geométricas tan bellas, los llevó a
confiar en un oficio extramatemático que les hizo
hacer representar los cuatro elementos, el aire, el
agua, la tierra y el fuego, y todo lo creado; de donde
aquella frase de «figuras cósmicas», tan conocida
por los lectores de Platón. El nombre de Pitágoras está unido ---y nada nuevo es esto para el lector--- al descubrimiento de las propiedades características del triángulo rectángulo. Mucho antes que por él, estas propiedades eran conocidas para triángulos especiales, pero ---al respecto están de acuerdo todos los testimonios dignos de fe---, en general, es Pitágoras, o algún discípulo suyo, los que las determinaron. No se sabe, con certeza, por qué medios se ha llegado a la más famosa proposición de la geometría elemental. Para el caso de un triángulo isósceles la verdad de esa proposición es perceptible «ad-oculo», pudiendo demostrarla, en los casos generales con el uso de figuras análogas; pero son construcciones tan artificiosas que, solamente, pudieron idearlas personas que ya conocían el teorema a demostrar. Más probable es que  los pitagóricos, que conocían 
 los teoremas fundamentales de la teoría de la similitud, hayan observado que, si transportamos la altura $AD$ del vértice del ángulo recto sobre la hipotenusa del triángulo rectángulo $ABC$, éste resultará semejante a los dos triángulos parciales, obteniéndose, así, las relaciones:
 $$
 {AB}^2= BC \cdot BD, \quad {AC}^2= BC \cdot CD
 $$
 las que, sumadas, dan:
$$
{BC}^2 = {AB}^2 + {AC}^2
$$

Tendremos la oportunidad dentro de poco, de volver sobre esta presunta argumentación pitagórica. Destacamos aquí, que si se considera un cuadrado con sus diagonales, se ve, de inmediato, que el cuadrado de una de ellas es el doble del cuadrado de un lado. De donde, la sorprendente consecuencia que de esto se obtiene: que la relación de la diagonal con el lado de un cuadrado no se puede expresar mediante números enteros, ni, tampoco, por medio de fracciones. Indescriptible es la turbación que, en el alma simple de los pitagóricos, debió producir la constatación de este hecho, el cual estaba en abierto
contraste con la máxima enseñanza por el Maestro:
«las cosas son números». Se acaba de constatar un
hecho \textit{ escandaloso}, sobre el cual debía guardarse el más absoluto secreto; y, según una difundida leyenda, el pitagórico que violó esta decisión, murió miserablemente en un naufragio, víctima de la cólera de los dioses. Nosotros carecemos de tantos prejuicios y somos más libres que los antiguos; no nos alarman fenómenos de la especie del descripto; al contrario, nos felicitamos de que la escuela pitagórica haya ampliado el dominio de la aritmética agregándole el nuevo y amplio campo de los números irracionales.


%\asteriscos

Mientras que a un filósofo se debe que la matemática adoptara los lineamientos de una verdadera ciencia, a otros dos pensadores de la misma característica mental se debe a que ella manifestara, en Grecia, una floración tan maravillosa: son, estos, Platón y Aristóteles. El primero no contribuyó directamente al progreso de las ciencias exactas, sino que atrajo sobre ellas la atención de los estudiosos, adornando sus escritos con consideraciones tomadas de la aritmética y de la geometría. «Dios mismo, geometriza», llegó a sentenciar; «Nadie entre en esta casa si ignora la geometría», hizo grabar en el arco de la entrada a su escuela. Estas frases manifiestan, del modo más claro, la convicción del divino filósofo de que «la geometría es el medio por el cual
podemos dirigir el alma hacia el ser eterno; es una escuela preparatoria para una mente científica, capaz de impulsar la actividad del alma hacia las cosas sobrehumanas». Por eso es que opinaba: «Si destruimos la ciencia de los números, el Arte y la Ciencia no podrán reinar más». Se debe a él la aguda observación de que, para la comprensión de las cosas, «existe una celestial diferencia entre quien se ocupe de geometría y aquel que la ignore». Fiel a estos conceptos, sostuvo, antes que Napoleón, que era deber del Estado promover, por todos los medios a su alcance, el estudio de las matemáticas. Claro que hoy, estas apreciaciones las comparten la mayoría de los más selectos pensadores; pero, ¿cómo no hemos de admirarnos leyéndolas escritas en una época en que nuestra ciencia balbuceaba aún; cómo no ver, en éste, una prueba de que para el genio de amplia visión, el tiempo carece de límites?


Por su parte, Aristóteles, al reunir en un cuerpo de doctrina los conceptos y los procedimientos fundamentales de la lógica deductiva, puso a disposición de los investigadores, una guía segura que, aún hoy, tiene plena eficiencia... al menos en la mayor parte del mundo, porque una reciente obra rusa, cuyo título es «\textit{ Lógica antiaristotélica}», induce a que pensemos que el bolcheviquismo haya logrado hasta subvertir la lógica fundamental del silogismo.

La influencia ejercida por las exhortaciones platónicas y las enseñanzas aristotélicas, no tardó en hacerse sentir. De inmediato comenzaron a efectuarse
estudios sobre tres problemas geométricos, los cuales fueron, durante siglos, estímulo para los investigación  geométrica y tormento para los investigadores. Esos problemas son: la cuadratura del círculo, la trisección del ángulo
y la duplicación delcubo. El  primero fué, sin duda, sugerido por la práctica, como lo prueba el hecho de encontrarlo entre los  documentos de pueblos salidos de la barbarie; el 
segundo es consecuencia natural de la bisecección del ángulo. Finalmente el tercero, según una difundida leyenda, debió haber sido impuesto por	voluntad
de los dioses. Cuéntase, en efecto, que habiéndose desencadenado sobre la isla de Delos una terrible epidemia, fueron interrogados los sacerdotes del templo, que en honor de Apolo existía en esa isla, cómo se podría aplacar la ira celeste; éstos respondieron que se debía duplicar un altar de forma cúbica, que ya existía. Los habitantes del lugar, se dieron a la tarea de substituirlo por otro, cuyos lados eran dobles del primero; pero, con dolorosa admiración, se constató que el flagelo, lejos de terminar, aumentaba en intensidad. Esto nos hace ver que Apolo era un buen geoómetra, ya que se dió cuenta de que el altar no era el doble, sino	el octuplo del antiguo. La historia no nos dice como fué  
aplacada su cólera, antes de que hubiera muerto toda la población de la isla; ella sólo se limita a darnos esa explicación del «problema de Delos», formulado para la situación indicada. Ese problema quedó, durante mucho tiempo sin solución, puesto que no se puede resolver usando, solamente, la regla y el compás, que eran los únicos instrumentos que conocían los geómetras. La dificultad  fue vencida recurriendo a otros medios, y los esfuerzos para resolver éste y los otros dos problemas enunciados, dieron origen a muchas curvas notables, como, por ejemplo, las secciones	cónicas. La ineluctable necesidad de  volver a servirse de él, queda demostrado, en nuestros días, por sutiles argumentaciones algébricas. Aún hoy	 la serie de aquellos que tienen la
ilusión y	el  anhelo de alcanzar su fin, parece ser
que no ha terminado, porque el problema de convencer a un pretendido cuadrador, duplicador o trisector, es un problema, mucho, pero mucho más arduo que el de resolver aquellos famosos problemas.


%\asteriscos




 Las noticias relativas a la primera época del desarrollo de la matemática griega, difícilmente se
pueden agrupar consultando obras de carácter general y escritos de filósofos e historiadores; porque
los trabajos puramente matemáticos, anteriores al III siglo a. de C., no se han sustraído a las injurias del tiempo y a la furia destructora de los bárbaros. La literatura de la matemática de Grecia, como de toda la Europa, comienza con Euclides. Los \textit{ Elementos}, su obra capital, aparece en la época en que terminado el imperio de Alejandro Magno, encontrará, en Alejandría ---Egipto---, un nuevo hogar el afán de estudio. No es éste el primer tratado de Geometría
que se haya escrito; pero el hecho de  que de todas las obras 	anteriores hayan desaparecido hasta sus menores rastros, da fe de su indiscutida superioridad; no se trata de una obra completamente original, porque de fuente muy segura sabemos que Euclides respetó con devoción casi religiosa, todo lo que habían pensado y escrito sus predecesores. El éxito de los \textit{ Elementos} no fué efímero, ya que los doctos en bibliografía aseguran que esta obra, por el número de ediciones y traducciones, sólo es superada por la Biblia y lucha valientemente con la \textit{ Divina Comedia}.

Este éxito de librería está justificado por el hecho de que, quien estudie los posteriores tratados de geometría, verá que toman, más o menos, como modelo la euclídea. Estas difieren de la euclídea por el cambio y retoque de las primeras páginas ---agregando aquellas que han llegado a nosotros
evidentemente desfiguradas por audaces copistas---,
por algunos cambios ---no siempre en beneficio---
de algunas demostraciones y por agregados provenientes del progreso que más tarde lograra nuestra
ciencia. Si durante el siglo pasado se creó una geometría noeuclídea, se trata aquí de un edificio surgido lateralmente pero no en sustitución del que creara el gran alejandrino; los \textit{ Elementos} tienen la solidez de las pirámides de Egipto, pese al ataque de que fué víctima de parte de personas me creyeron poseer una competencia de la que, en realidad, carecían.

Entre estos detractores, duele encontrar a un filósofo alemán que, durante la mitad del siglo pasado [el siglo XIX], gozó de un envidiable renombre: me refiero a Arturo  Schopenhauer. Después de dirigir invectivas contra la aridez de la forma euclídea, llego a declarar que la demostración del teorema de Pitágoras que se lee en los \textit{ Elementos}, es artificiosa porque se rige por incertidumbres, siendo una verdadera trampa para los pobres escolares. Pero todo esto demuestra, y nada más, que a Schopenhauer se le escapaba la esencia y la belleza del razonamiento euclídeo, la génesis del cual resulta evidente para quien tiene presente el razonamiento que, según mi opinión, condujo al descubrimiento de aquella famosa proposición. Se observa, en efecto, que aquel razonamiento no podía, sin duda, introducir al Libro Ide los \textit{ Elementos}, ya que se basa en la teoría de la similitud. Pero pone en claro el hecho de que los cuadrados $ACDE$, $ABFG$, descriptos sobre los dos catetos del triángulo $ABC$, son iguales, respectivamente, a los dos rectángulos en los que el cuadrado de la hipotenusa $BCHK$ es dividido por la correspondiente altura prolongada. Por eso es que Euclides se habrá preguntado cómo demostrar esto sin recurrir a triángulos semejantes; ya que tirando las rectas $FC$ y $AK$ nacen dos triángulos iguales entre sí, de los cuales, uno, es la mitad del cuadrado de un cateto, y el otro, la mitad de uno de los rectángulos en que se ha descompuesto el cuadrado de la hipotenusa; por lo que estas dos áreas son iguales entre sí. De la misma manera se demuestra la igualdad del otro cuadrado a la segunda porción del cuadrado de la hipotenusa; con lo que se está en situación de concluir \textit{ quod erat demonstrandum}.

Espero haber demostrado al lector que si se debe hablar de trampas, a propósito de esta parte de los \textit{ Elementos}, es solamente para recordar la voltereta dada por aquel pensador ultramontano. Claro está que no vamos a sostener que los \textit{ Elementos} de Euclides sea una obra que pueda leerse como una novela o un artículo periodístico; al contrario exige una lectura seriamente meditada. Esto era, además, lo que pensaba el mismo Euclides, el cual interrogado un día por el soberano si no había un medio más fácil para aprender la geometría, respondió agudamente: «Majestad, sobre la tierra existen caminos para el pueblo y caminos para el rey; en geometría, por el contrario, sólo existe un camino real».

%\asteriscos

Mientras que Euclides se presenta como el verdadero tipo de aquellos que, sistematizando y completando los resultados obtenidos, y exponiéndolos en forma inteligible, preparan la ruta por la cual ha de avanzar la juventud presente y futura; Arquímedes, el matemático que cronológicamente le sigue, es el modelo más genuino y puro del audaz investigador que descubre nuevos territorios, donde plantar el estandarte triunfal de la ciencia.

Desarrollando y aplicando conceptos y métodos que otro eminente pensador griego, Eudoxio de Cnido, había ya estructurado, Arquímedes ha resuelto, genialmente, problemas que hoy se considera que pertenecen al análisis infinitesimal y, así, ha legado su nombre a descubrimientos de gran valor. Afrontó, también, el problema de la cuadratura del círculo, y reconociendo la imposibilidad de resolverlo con la regla y el compás, aplicó consideraciones puramente aritméticas, llegando así a resolverlo aproximadamente. Así es que se ofreció a la crítica de los sacerdotes de la pureza geométrica, dando, pese a ello, un esquema de cálculo para la relación de la circunferencia con el diámetro, que permitió más tarde, determinar un valor con 707 cifras decimales, aproximación inaudita y ---digamos la verdad--- superflua, ya que en ninguna búsqueda en el macrocosmos ni en el microcosmos ---astronomía y física molecular---, fué necesario utilizarla. El valor 3.14 que Arquímedes da para expresar la mencionada relación, no exige ningún esfuerzo para ser recordado; no pudiendo decirse lo mismo para aquellos valores más aproximados que, para ayudar a los calculadores, fueron imaginados contenidos en frases, en que se debía substituir para cada palabra el número de letras que la formaba, obteniéndose, así, el valor buscado. Se podrá creer que estos ejercicios mnemónicos fueron abandonados durante la agitada época moderna; pero, desengáñese el lector, la única cosa de la que hay actualmente gran abundancia en el mundo es el Tiempo y, naturalmente, la desocupación que perturba en los Estados Unidos de América, ha originado esta frase del tipo indicado «How I want a drink, alcoholic of course, after the heavy sections involving quantum mechanics», la que traducida, dice más o menos esto: «Después de haberse estudiado la teoría de los \textit{ Quanta}, qué placer reconfortarse el estómago con una buena bebida, naturalmente alcohólica». Las palabras, «naturalmente alcohólica», no son un agregado mío, y muestran que el autor, no es, ciertamente, un partidario del régimen seco.


%\asteriscos

Arquímedes ha sido un gigante, no sólo en ciencia pura, sino también, en la ciencia aplicada. Una piedra, recientemente descubierta, ha demostrado que le corresponde a él el concepto fundamental de la reforma del Calendario decretada por Julio César, y que sirvió durante quince siglos, hasta el día en que, el Papa Gregorio XIII, consideró necesario una ulterior reforma y perfeccionamiento. El gran siracusano ha puesto las bases definitivas de la estática de los sistemas rígidos y, con el descubrimiento del «Principio» que lleva su nombre,
ha escrito las primeras páginas de la hidrostática; como también, de la mecánica de los aeriformes; prueba de ello, es el aparato con que Picard pudo realizar su ascensión a la estratósfera, el cual fué concebido como una genial extensión de las ideas de Arquímedes.

Otro punto de vista que hace aparecer a Arquímedes distinto a Euclides es que, mientras que de éste no se conoce ninguna particularidad biográfica, de aquél la historia se complace en detenerse sobre él, y la leyenda, en torno suyo, ha florecido lujuriante. Es que mientras Euclides vivía tranquilamente a la sombra del trono de los Faraones, Arquímedes actuó como figura importante durante la segunda guerra de Roma contra Cartago; y como Siracusa era aliada de la gran metrópoli africana, Arquímedes se encontró frente a las huestes latinas y ofreció el ejemplo, único en los anales de la guerra, de un hombre combatiendo contra todo un ejército. Las gestas, por él cumplidas, fueron celebradas por Plutarco, mientras escribía la \textit{ Vida de Marcelo}, haciéndole, así, adquirir notoriedad, cosa que está vedada a los que se consagran a la ciencia pura. Pero al mismo tiempo la fisonomía de Arquímedes sufrió tales deformaciones que la historia imparcial se ve obligada a rectificarlas.

Así, por ejemplo, mientras no es inverosímil que Arquímedes conociera el funcionamiento de la palanca, ya que un día afirmara que si pudiera disponer de un punto de apoyo podría levantar al
mundo ---«da mihi ubi consistam et terra movebo»---, no podemos dejar de sonreirnos con incredulidad al mirar un dibujo, en el que se lo presenta tranquilamente sentado en la costa del mar, mientras que con un pequeño aparato entre las manos ---sería, quizá, una palanca--- traía, hacia sí, una gran nave de carga llena de soldados.

Menos verosímil es, todavía, lo que cuenta el historiador Tzetze, en una página digna de figurar en una biografía novelada del gran investigador. «Cuan\-do, escribe el citado historiador, las naves de Marcelo estuvieron al alcance del arco, el viejo ---esto es Arquímedes--- hizo que llevaran un espejo hexagonal de su invención. Colocó, luego, otros espejos análogos, pero más pequeños, a una distancia conveniente del primero y movibles a su alrededor, en una cremallera, por medio de palancas metálicas cuadradas. Expuso su espejo a los rayos del sol meridional, y reflejándose, estos, en el espejo, consiguió provocar un terrible incendio entre las naves enemigas que fueron, así, reducidas a cenizas».

Ahora bien, para juzgar sobre la posibilidad de tal hecho ---sobre el que Polibio, Tito Livio y Plutarco observan el más riguroso silencio---, es necesario tener presente que la óptica, entre los griegos, estaba en sus primeros estadios, conociéndose, solamente, la ley de la reflexión de la luz; y en el mismo estado se encontraba la delicada técnica de la construcción de los espejos; de donde cabe preguntarse ¿cómo es posible que Arquímedes haya polido obtener del sol calor suficiente para poder incendiar toda una flota? Agréguese, a esto, que un intento, realizado a fines del siglo XVIII con un aparato formado por 168 espejuelos planos, ideados por el	naturalista Buffon, para reproducir el efecto obtenido por Arquímedes, condujo a resultados negativos; y que otro tanto puede decirse del realizado por Peyrard, en una época más próxima a nosotros, pese a los vastos conocimientos que se tienen sobre la teoría de la luz y la perfección obtenida en la construcción de toda clase de espejos, anteojos, telescopios, etc. Además, ¿cómo pensar que la humanidad, que necesita descubrir nuevas fuentes de energía para compensar las reservas naturales de combustibles, que día a día van mermando con alarmante rapidez, no haya pensado en utilizar la idea de Arquímedes, recurriendo al sol? Y finalmente, ¿cómo podemos explicar que el hombre, en su belicosa búsqueda de medios para destruir a sus semejantes, no haya pensado reproducir el aparato arquimediano, transformando al sol, eterno creador de vida, en un nefasto sembrador de la muerte?

La explicación de todo es, y no puede ser más que una sola: que los espejos ustorios son una pura y simple leyenda, generada por el terror que Arquímedes imponía a los soldados de Marcelo.

Muy diverso es el juicio que merece un conjunto ideado por Arquímedes, el «caracol» o «rosca», que lleva su nombre glorioso, y que Galileo no tuvo reparo en proclamar «no sólo maravillosa, sino milagrosa porque ...  el agua asciende en la rosca descendiendo». Se trata, todos lo saben, de un tubo
en espiral que se enrosca sobre un cilindro que gira en torno a su propio eje. Es Vitrubio, el gran arquitecto romano, quien nos da noticia de ella en una parte de su gran obra sobre la ciencia y el arte de la construcción; el texto que se refería a esta cuestión debía estar ilustrado con una figura, pero como no pudiera salvarse de la destrucción, durante mucho tiempo fué un problema que atormentó a los historiadores y técnicos el saber cómo se obtenía la rotación del aparato; numerosos fueron los expedientes sugeridos para tal fin, pero todos eran tan complicados que la invención arquimediana terminó por perder su elegancia y utilidad, la solución fué obtenida por un camino inesperado, esto es, al descubrirse un fresco en las excavaciones de Pompeya, y que fuera reproducido en las \textit{ Notizie degli scavi} que se publica bajo los auspicios de la \textit{ Accademia dei Lincei}. Observada la fotografía de aquel fresco se notó que la rotación de dicho cilindro se obtenía, pura y simplemente, con el auxilio de los pies de un esclavo, víctima inocente de las elucubraciones arquimedianas.


%\asteriscos

Si gloriosa fué su existencia, la muerte de Arquimedes fué cruel. Porque, pese a las severas órdenes dictadas por Marcelo, quien dispuso que al ser conquistada Siracusa debía respetarse la vida de todos los ciudadanos, un rudo legionario, al no recibir respuesta alguna a sus preguntas dirigidas a un viejo absorto en sus propias meditaciones --Arquímedes---, le dió muerte bárbaramente. Episodio doloroso que debió haber tenido, de inmediato, resonancia en todo el mundo, ya que en las excavaciones recientes de Herculano se ha encontrado un mosaico que se interpretó como representando la muerte de Arquímedes.


En cuanto  Marcelo tuvo noticia de aquel delito contra la ciencia y contra la humanidad, dominando su ira y su dolor, dió orden de que al ilustre muerto se le diera digna sepultura y que ---en homenaje a un voto realizado por el propio Arquímedes--- sobre la piedra sepulcral que cubriría sus despojos se grabara una esfera inserta en un cilindro, en memoria de uno de sus descubrimientos, al cual, con razón, él atribuía la mayor importancia. Desgraciadamente los siracusanos demostraron no rendir mucho culto a la ciencia y tener poco amor por su patria, ya que debieron haber transformado a aquella tumba en un venerado altar. Cicerón, al llegar a Sicilia como cuestor, en vano inquirió noticias del lugar en que se hallase el sepulcro, ordenando, entonces, búsquedas detenidas para encontrarla, siendo aquella figura geométrica grabada en la Piedra la que permitió su reconocimiento. Y la hallaron en un lugar desierto, por lo que Cicerón dispuso que su tumba se transportara a un lugar más en evidencia y más honorable. «Por consiguiente ---dice Cicerón inmodestamente---, la más noble y docta ciudad de Sicilia ignoraría, todavía, el lugar donde yace el más ilustre de sus conciudadanos, si no se los hubiera revelado un pequeño hombre de los Arpinos».

%\asteriscos




Pese a estos amplios datos sobre la obra matemática de Arquímedes, están muy lejos de detallarla en modo completo, aun cuando yo agregase a ella otros datos no llegaría a describirla totalmente, porque no es conocida todavía en toda su totalidad, como lo demuestran, de cuando en cuando, las revelaciones que obtienen los cultores de la filología clásica y de la literatura árabe, como consecuencia de afortunadas exploraciones en bibliotecas inaccesibles u olvidadas. Creo, si no me equivoco, que todo lo que he expuesto bastará para mostrar lo justificada que es su altísima fama y explicar cómo y por qué, pese al gran número de grandes pensadores que después de él se presentaron sobre la escena del mundo, su figura aparece gigantesca y, como es ya opinión general, ni Galileo, ni Newton, ni Lagrange, ni Gauss pudieron sobrepasarlo. Sólo él bastaría para establecer, inconmovible, la gloria eterna de la Grecia antigua.

Pero esta tierra, de maravillosa fecundidad, no se agotó produciéndolo.

Hacía	que Arquímedes había descendido  a su tumba ensangrentada, cuando veía la luz
matemático que, pese a la comparación a que se viera expuesto con el gran siracusano, fue designado con el  nombre de «gran geómetra». Es Apolonio de Perga. Recordando  aquí la excelencia de su famosa obra sobre las \textit{ Secciones cónicas}, temo que los lectores ingenieros no me den su asentimiento,
ya que con estas curvas les han... o mejor, les hemos amargado tanto, que exponiendo los múltiples teoremas descubiertos por Apolonio sobre el centro, los diámetros, los ejes, los focos, etc., sólo lograré, como resultado, la evocación de penosos recuerdos. Pero pasando a consideraciones más elevadas, haremos notar que con la recopilación, coordinación, generalización y terminación de las propiedades destacadas de esas curvas y que, componiendo, con ello, un todo orgánico, Apolonio ha dado a Kepler los medios para descubrir las leyes reguladoras del movimiento de los astros; por que si, por un momento, pensamos retirar el nombre de Apolonio de la historia del pensamiento humano, desaparecería también el de Kepler, a quien Alemania, en el tercer centenario de su muerte, hubo de recordar orgullosamente. Uniendo al gran geómetra griego el más grande astrónomo alemán, se destaca una de las más brillantes características de nuestra ciencia: la de suministrar auxiliares, puramente doctrinales, y nuevos algoritmos, destinados a servir, en el porvenir, como provindenciales ayudas a los investigadores posteriores; característica que hace resaltar, ante los ojos de lo adoradores de lo útil inmediato ---entre los que duele encontrar a un León Tolstoy---, las más abstractas búsquedas analíticas y geométricas; característica que se confirma cuando se llega a la Relatividad general, aplicando consideraciones y fórmulas exclusivamente teóricas, como son las que constituyen el cálculo diferencial absoluto.


%\asteriscos

Es hoy difícil medir, con exactitud la influencia ejercitada sobre los contemporáneos y los herederos inmediatos por la brillante tríada Euclides - Arquímedes - Apolonio, porque la determinación de los geómetras menores de la Grecia constituye, todavía, el problema central de la historia de la antigua geometría, siendo muy escasos los documentos que, para resolverlo, poseemos. Pero compulsando, con sumo cuidado, las obras existentes, alguna luz puede lograrse al respecto. Así es como, estudiando con detenimiento los innumerables pergaminos, aparece el nombre de Herón de Alejandría, viéndose, en él a un profundo conocedor y genial aplicador de todo aquello que Arquímedes pensó y escribió acerca de la mecánica de los sólidos y de los fluidos; que todos estos principios sirvieron para explicar los ingeniosos aparatos creados por Herón para levantar pesos y transformar movimientos, para hacer andar sus maravillosas fuentes de agua, sus sorprendentes autómatas y originales relojes de agua. Es éste el resultado de un perfecto acuerdo entre la ciencia y el arte, exponente de dotes mentales de quien no se sabe  si admirar más su actitud científica o su inagotable fantasía. Y la \textit{ eolipila}, tan citada, prueba que con Herón comienza la historia de la máquina a vapor, mostrando que, con razón, aseguraba estar en condiciones de servirse de los elementos: agua, aire, tierra, fuego.
 
Claudio Ptolomeo, para calcular una tabla de cuerdas de los arcos de un círculo, auxiliar indispensable para un astrónomo en una época en la que los senos no habían, aun, ingresado en la ciencia y los logaritmos eran, todavía, desconocidos, se inspiró en escritos de Arquímedes que, desgraciadamente, son poco conocidos; y en la exposición de la teoría de los excéntricos y de los epiciclos, otra teoría fundamental de la antigua astronomía, se manifestó como discípulo de Apolonio. Y por si alguien llegara a creer que el \textit{ Almagesto} la «opus magnus» de Ptolomeo, después de la adopción general del sistema copernicano, deba ser colocado en un museo de instrumentos fuera de uso, revelaré, que durante nuestros días, un editor de París, que ciertamente conoce su oficio,
juzgó oportuno reimprimirla en su texto original acompañada de su versión francesa
lo que demuestra que los astrónomos tendrían interés en ella. A los que se interesan por la evolución histórica del pensamiento humano, les será grato saber que la obra de Ptolomeo ofrece hechos extraños y ciertamente característicos y sorprendentes. Tal es, ante todo, el desarrollo completo que en ella encontramos de la trigonometría esférica, mientras que sobre la trigonometría plana no se lee más que algunas líneas; lo que nos lleva a concluir que algunos de nuestros tradicionales esquemas escolásticos tienen un valor muy relativo. Por otra parte el hecho de que encontremos en el \textit{ Almagesto}, ampliamente aplicada la determinación de los puntos de la esfera celeste, mediante la latitud y la longitud, demuestra que las coordenadas esféricas fueron concebidas y aplicadas 1.500 años antes que Descartes sugiriese el uso de las coordenadas como auxiliar poderoso para el estudio de la geometría plana; y ésto nos lleva a determinar que el espíritu humano no procede siempre, en el descubrimiento de la verdad, en la clásica dirección de lo simple a lo compuesto.

Estas noticias sobre los epígonos de los más grandes geómetras de Grecia, presentaría una imperdonable laguna si no se dijera algo acerca de la \textit{ Colección Matemática} de Pappo, otro geómetra de Alejandría. Aunque se trate de una obra de comentarios, contiene, también, proposiciones nuevas y puntos de vista originales; siendo ello, una prueba más del hecho indiscutible de que el genio geométrico de los griegos, cual una lámpara próxima a
extinguirse, ha sabido lanzar, en su periodo de decadencia, un rayo de luz que ha llegado a iluminarnos, a	nosotros, sus lejanos parientes.

%\asteriscos


Toda la antigua matemática griega, posee un carácter profundamente geométrico; hasta los libros de los \textit{ Elementos}, en que Euclides expone los fundamentos de la aritmética racional, tienen como base representaciones geométricas. Mas no debemos creer que aquellos hayan descuidado la ciencia de los números. Es verdad que su sistema de numeración, en el que las letras son las representantes de los números, ofrece, al que no está familiarizado con ellas, dificultades de manejo; pero el ilustre historiador Tannery asegura que, habiéndolas usado, durante mucho tiempo, exclusivamente, llegó a servirse de ellas con tanta seguridad como nosotros cuando usamos las cifras indoarábigas. Es necesario hacer notar que el pueblo griego, consideró a la aritmética, práctica, como un arte menor respecto a la teoría de los números y de la geometría, por lo cual no ha dejado ninguna exposición completa. Por eso, aquel que quiera saber cómo calculaban los griegos, deberá espiar y sorprender a Herón, a Ptolomeo y a sus comentadores, mientras desarrollan las operaciones aritméticas.

Una más amplia información al respecto la hallamos en el estudio de los libros existentes de la obra de otro eminente científico alejandrino, Diofanto, porque están constituidos por una interesante colección de problemas numéricos completamente resueltos, los cuales son tan instructivos, que es razonable la desesperación que nos embarga ante el hecho de que, de trece libros, en que la obra estaba dividida, solamente seis hayan llegado hasta nosotros. Algunos historiadores consideran a Diofanto como el primer algebrista que haya existido sobre la tierra; ahora bien, para decidir si compartimos o no este modo de ver, es necesario entenderse primero sobre el significado que debe atribuirse a la palabra «álgebra». Si éste nombre es el que usa el arte de resolver problemas con datos numéricos, no dudamos que la obra de Diofanto deberá llamarse algebraica; pero es que entonces, por coherencia, deberán tratarse como algebraicas las mejores exposiciones de la aritmética que encontramos en nuestras escuelas. Si, en vez, consideramos como característica del álgebra el uso de una simbólica especial, será muy dudoso que la obra de Diofanto merezca el nombre de algebraica, desde el momento que el método de exposición usado nada tiene de común con el que estamos familiarizados. Si el designar como algebraico al volumen diofánteo es, en último análisis, una cuestión de palabras, su valor eminente es por todos reconocido y proclamado. Y con plena razón, ya que los artificios que él emplea para resolver los problemas numéricos, tienen una genialidad y un valor que se notan observando que muchos han quedado, aún, en la ciencia, y otros han sido estímulo y guía para los que se ocuparan de cuestiones análogas.  Se ha visto, además, que si buscamos los fundamentos doctrinales de muchos de sus procedimientos de cálculo, se los encuentra en importantes proposiciones que, descubiertas luego por otros, han pasado a ser de dominio universal.

%\asteriscos






Las dotes eminentes de la gran obra diofantea fueron apreciadas por los modernos, cuando ella fué sacada de los anaqueles en que yaciera olvidada durante la obscura noche medieval. Pero los camaradas e inmediatos sucesores del gran aritmético grigo no habían, por cierto, reconocido sus dotes eminientes. Prueba de ello son los comentarios con que ha sido honrada. De uno de ellos se ha guardado especial memoria, aunque todas las búsquedas hechas para encontrarlo llevan a la conclusión de que aquel no se ha substraído a la furia destructiva de los bárbaros. Es la obra de la primer mujer cuyo nombre se encuentra registrado en los fastos de las matemáticas: Hypathia, la hija de Theón de Alejandría. El ingenio natural y la amplia cultura que adquirió estudiando en Atenas, la hicieron llegar al grado de jefe de la secta neopitagórica, que florecía en la tierra de los Faraones durante la mitad del IV siglo de nuestra era. Su elocuencia, su virtud, su belleza
más al mismo tiempo la hicieron adversaria 
en la lucha desencadenada por entonces, entre el paganismo agonizante y el cristianismo triunfante. Y fué durante uno de los tantos episodios sangrientos de esta áspera cotienda que de civil no tenía más que el nombre, que Hypathia encontró la muerte y, como muchos siglos después le sucediera a la princesa Lainballe, su cuerpo destrozado fué arrastrado por las calles de la ciudad para su ludibrio.
 La desaparición de la eminente pensadora es un
hecho que conmueve desde el punto de vista humano, y también es de gran importancia en la historia de la cultura, porque marca el fin de la gloriosa Escuela de Alejandría. Mientras que la pérdida de todos los escritos de Hypathia impiden usarlos, como documentos, para resolver el eterno problema de la aptitud de la mujer en la investigación científica, la virgen pagana dió material para obras de fantasía por parte de quienes vieron que el ejemplo por ella dado, de saber morir para el triunfo de una idea, era un hecho capaz de elevar el espíritu y con el corazón, mucho más que las detalladas descripciones de los amores, más o menos irregulares, de Franeesca, de Parisina, de Julieta y de Manón.


%\asteriscos

De todo lo que he expuesto, pese a las innumerables cosas que he debido omitir para no abusar de la benevolente atención del lector, se deduce que las rosadas esperanza de un brillante porvenir que acompañaran la constitución de las matemáticas griegas, no fueron ilusorias. La antorcha de la investigación matemática encendida por Tales pasó, a su muerte, de Mileto a la Magna Grecia, donde quedó temporalmente oculta para ser, luego, vigorosamente alimentada en los concilios pitagóricos; resurge luego, con libre y esplendente luz, en Atenas, gracias a las enseñanzas de Platón, el más eminente discípulo de Pitágoras en el campo matemático,
adquiriendo el máximo esplendor, durante la época greco-alejandrina, que delimitará la desaparición de Alejandro Magno. Los esfuerzos combinados de Euclides, de Arquímedes y de Apolonio, y de sus discípulos inmediatos, aseguraron, sobre base granítica, al edificio geométrico produciendo un gran desarrollo de la ciencia de la extensión figurada y, al mostrar la existencia de campos ilimitados, la cultura prometía a todos gloria, ejercitando una influencia que se puede intuir pero que, por la falta de documentos sincrónicos, es difícil que midamos exactamente. Desaparecidos aquellos valerosos campeones, la matemática griega pierde la primitiva florescencia, no encontrando más cultores que entre una
turba de comentadores agotados y anémicos que,
finalmente, la llevaron a una irreparable ruina.

Es vano querer, opino, encontrar una causa única de este deplorable y deplorado fenómeno. Esa
explicación no se encontrará en el hecho de que los
grandes geómetras de la antigüedad hayan agotado todo el material al cual habían dedicado sus estudios; en realidad la geometría elemental presentaba lagunas que era sumamente necesario colmar, y la teoría de las secciones cónicas, la cual con Apolonio había logrado llegar a una tan alta perfección, debía llevarlos al estudio de otras líneas, tanto más que, puestos sobre este camino, se podía estar seguro del conocimiento que ya se tenía de otras líneas particulares, como, para no citar más que algún ejemplo, la espiral de Arquímedes, la conchoide de Nicomedes y la cixoides de Díocles.

No se vaya a creer que los métodos de investigación, entonces en uso, no permitieran llegar a tal fin, ya que cuando después de muchos siglos de oprobioso letargo, resurgió el deseo de hacer progresar la geometría, a ellos se recurrió, y no en vano. A este propósito deberíamos repetir con un viejo poeta ---Chaucer--- «como de los viejos campos nacen, cada año, nuevos trigales, así de estos viejos papeles surgen, de continuo, nuevas verdades».

Finalmente las miserables condiciones políticas en que se encontró la Grecia en la época en que Roma era la Señora de todo el mundo conocido, no son suficientes para explicar la razón de la declinación de las matemáticas griegas; que la historia enseña ---y muy bien lo sabemos nosotros los italianos--- como pueblos esclavos, oprimidos y dispersos, hayan manifestado, aún en las más diversas y abstractas elucubraciones, su perenne vitalidad y la indestructible unidad de su estirpe.


Pero si, en el estudio de la evolución del pensamiento matemático helénico, prescindimos de consideraciones de tiempo y raza, no tardaremos en reconocer que el pensamiento matemático helénico no ha sufrido insanables obscurecimientos o definitivas desapariciones. Su presencia está siempre viva, aunque transfigurada, en toda la matemática que se constituye en Europa a partir del Renacimiento y está operando siempre, con la fuerza y el empuje de una eterna juventud. Un análisis minucioso y profundo de todo nuestro patrimonio geométrico y algebraico muestra estar constituido, en su mejor parte, por los frutos que dieron y dan, todavía, los métodos y las teorías que hicieron inmortales a los grandes matemáticos antiguos.

Desde este punto de vista, nuestra ciencia sufre una impresionante, aunque parcial, analogía con otro ramo de la sabiduría: la arquitectura, la cual, durante un gran número de siglos se ha desarrollado uniformándose a las normas estéticas que se obtienen con la contemplación del Partenón y el Propileo. Pero, mientras artistas actuales, animados de espíritu revolucionario, han alzado la enseña de la revuelta contra una dominación que dura hace centenares de años, un semejante y triste día, aún no ha aparecido para la geometría. Los dardos del futurismo no han sido, hasta ahora, dirigidos contra nuestra ciencia, aunque los más ardientes y valientes
sostenedores de este nuevo camino del pensamiento se han dado cuenta que no podrían rasguñar
la majestuosa y granítica construcción creada bajo el cielo de la Hélade, que a medida que el tiempo pasa es más sólida, más amplia y más bella por la acción de una ininterrumpida sucesión de hombres que constituyen la gloria del género humano.



\newpage



\section*{Siglos de lucha; siglos de gloria}\addcontentsline{toc}{section}{Siglos de lucha; siglos de gloria}

Al pueblo griego, que en el portentoso drama de la historia, desempeñó gloriosamente, durante muchos siglos, la parte de protagonista, le sigue la gran Roma. La nueva estirpe que, en la última faz de su desarrollo, dominó sobre todo el mundo conocido, mostróse rica en preciosas dotes, tanto en el arte de la guerra como en la ciencia del derecho, en grado de celebrar la gloria de los héroes; historiadores hábiles que eternizaron, de inmediato, la gesta de los más valerosos conductores; pero en la filosofía y en la ciencia, ya sea en la del razonamiento o de la observación, mostró un estado negativo que, honestamente, reconocieron personajes representativos, como Cicerón y Vitrubio, entendiendo la palabra «negativo», no solamente como una ineptitud para el descubrimiento de nuevas verdades, sino también incapacidad para comprender y apreciar las obras de los otros. Esto es el origen del deplorable hecho de que las grandes obras, incontestables documentos del genio matemático griego, cayeron en un completo olvido y muchas sufrieron
la desgracia de su irreparable pérdida. Las condiciones no mejoraron, cuando con la proclamación de «rex gentium» de Odoacro, tuvo su término el imperio romano; porque se inició, entonces, la edad de las incursiones bárbaras, durante la cual la civilización latina debió sostener luchas trágicas, y de desgraciada fortuna, contra la ignorancia de gente privadas de inteligencia y cultura; prueba de ello la suerte corrida por Severino Boecio, torturado espantosamente y luego condenado a muerte por Teodoríco, hecho consumado en Pavía, recordando que este rey habiendo, en un principio, apreciado su valor, le quiso hasta como ministro.

Con la consolidación y difusión del Cristianismo, las condiciones naturales de la cultura sufrieron en Europa una indiscutible mejora, ya que, en la paz de los claustros, buen número de pensadores encontró ambiente propicio para la serena contemplación de la Verdad, aunque fuera, para el libre curso del pensamiento, grave obstáculo conciliar la palabra divina con las reglas de Aristóteles. Pero si las consecuencias a que se llegaban se encontraban obstaculizadas, no existían, en cambio, trabas para el trabajo espontáneo que realizaban espíritus originales.


Es un ejemplo, grandemente impresionante, el de  Rogerio Bacon. El fervor religioso del que estaba, animado, desde su juventud, lo llevó a vestirse el hábito monacal, ingresando en la orden franciscana. Tal elección debía serle fatal, ya que, de todas las congregaciones religiosas de su tiempo, ninguna era menos propensa al estudio que la que tomara su nombre del pobrecito de Asis. Es norma de esa congregación, que sus miembros hagan voto de humildad, pobreza, ruegos, ayunos, labores manuales pero nada de ejercitar las facultades mentales. Muy distinta habría sido la suerte de Rogerio Bacon si hubiera elegido la orden de los predicadores, que tomó su nombre del de S. Domingo, que contara en las filas de sus miembros a Alberto Magno y a Santo Tomás de Aquino. El profundo conocimiento que él tenía de la lengua griega, lo condujo a descubrir los errores que pululaban en las versiones latinas de las obras de Aristóteles y ---con un rasgo de genialidad que constituye uno de sus mayores títulos de gloria--- vió que el medio más seguro para desarraigar ideas falsas, que eran entonces enseñadas en todas las escuelas, era el de recurrir a la observación y a la experiencia. Pero la palabra independiente con que criticó los sistemas pedagógicos de su tiempo, hicieron surgir, contra él, una legión de opositores, que bien pronto se transformaron en implacables enemigos, los que, en varias oportunidades obtuvieron que fuera recluido en duras cárceles y privado de todo medio de estudio y comunicación con el mundo exterior. Un Papa, amigo suyo, Clemente IV, logró hacer suspender la primera condena; otro que era su enemigo, Nicolás IV, nada hizo para que obtuviese la modificación de la segunda. Es que Bacon era un pensador de ideas profundamente ortodoxas, proclamando que «solamente la ciencia asegura alas al alma, la prepara para el conocimiento del mundo celeste, haciéndola digna de asociarse a la existencia divina. La ciencia es el fin, el destino supremo de la condición del hombre; ella destruye al mal se eleva a las esferas superiores, penetra en el seno de la naturaleza y se dirige hacia los astros». Las persecuciones no cesaron con su muerte; presentado como brujo, alquimista, astrólogo, sus obras fueron recogidas y destruidas por orden expresa de los altos dignatarios de su orden; salvándose, apenas, aquellas que pusiera bajo la protección de las altas autoridades eclesiásticas. Tres siglos más tarde, las obras de Rogerio cayeron en las manos de otro Bacon, en quien la amplitud del ingenio no formaba nivel con la integridad de su carácter. No tardó en ver el enorme valor de las ideas que aquél expusiera y, olvidando voluntariamente citar las fuentes que ampliamente usara, las difundió en el mundo, contribuyendo, así, a la constitución del método experimental. Pero este período de larga espera fué fatal para el progreso, porque, no faltamos a la verdad cuando aseguramos que el desconocimiento del valor de los altos pensamientos de Rogerio Bacon retardó, por varios centenares de años, el renacimiento de la búsqueda científica en el campo experimental.


%\asteriscos

Las tinieblas que, durante un largo curso de siglos, cayeron sobre Europa, fueron desgarradas, no por el propio impulso de sus habitantes, sino gracias
a la influencia, verdaderamente providencial, del pueblo árabe, el cual, primero que aquél, supo medir y apreciar la imponente belleza de los trabajos de los 
los	matemáticos de la Grecia antigua.
Entre los fenómenos ofrecidos por la historia universal, ninguno es más maravilloso que aquel que presenta a un pueblo que, sin haber ejercido ni antes ni después ninguna influencia sobre el desarrollo de los sucesos humanos, bajo la voz potente de Mahoma, descubrió en sí mismo admirables energías, gracias a las cuales, durante muchos siglos, pudo dominar cuerpos y espíritus; fenómeno, éste, que sólo podrá ser explicado el día en que poseamos las leyes que gobiernan la mecánica de las fuerzas morales.

Bajo la guía de los sucesores del Profeta, los creyentes del Corán, ocuparon, sucesivamente, la Persia, la Siria, el Egipto, toda la costa septentrional de Africa y, finalmente, la España. No pagándose por la gloria de sus armas, se hicieron autores de la cultura y difundieron en todos los países, por ellos ocupados, las grandes obras de pensamiento filosófico y científico antiguo, mostrándose capaces de incorporar notables agregados. Buen número de escritos originales fueron, recientemente, puestos a disposición de los no-orientalistas, gracias a las meritorias fatigas cumplidas por un manípulo de eruditos; así fueron puestas a la luz dos preciosas características del pueblo que estamos estudiando. Ante todo su perfecta honestidad, consistente en atribuirse solamente lo que es de ellos. Por consiguiente, el estudio de las obras árabes, no impone aquellos penosos problemas que encontráramos al estudiar la historia de la matemática entre los hindúes y chinos, problemas que consistían en trazar una neta línea de demarcación entre lo que es original y lo que es importado. En segundo lugar, los árabes dieron prueba de una indiscutible aptitud para la investigación matemática, documentada por los importantes agregados que estos hicieron a los escritos anteriores que trataban de geometría y álgebra.

%\asteriscos

La influencia de los árabes sobre nosotros, se advirtió de manera indiscutible, un siglo antes que Dante ---mediante su Comedia--- determinase la constitución del idioma característico de nuestra estirpe. En efecto, en 1202, aparece el \textit{ Liber Abaci} de Leonardo Fibonacci. Era éste, no un matemático de profesión, sino un simple contador del municipio de Pisa, que habiendo, cuando joven, viajado por las principales ciudades de la cuenca del Mediterráneo, se impregnó de los procedimientos aritméticos y algebráicos allí en uso, proponiéndose hacerlos conocer por sus conciudadanos. Traduciendo en acto este noble propósito, adquirió el mérito sumo
de haber determinado el renacimiento de la
investigación  matemática, no solamente en Italia, sino en toda la Europa. Poca fué la calma que recogió vida, pero el tiempo le ha hecho plena justicia, ya que es opinión general que, si Dante hubiera podido medir su valor, hubiera frenado el desdén que le hiciera llamar a la patria «baldón de las gentes».

Entre las innovaciones que se deben al \textit{ Liber Abaci}, la de mayor valor es la que está representada por nuestro sistema de numeración. Pero no hay que creer que a ese todos le recibieran bien. Se mostró hostil la clase de los comerciantes, porque temían que las nuevas cifras, cuya forma no estaba determinada en modo inequivocable, como lo eran las lapidarias cifras romanas, se prestaran al fraude; tampoco le fué favorable la Iglesia, para quien un método creado por un pueblo de infieles, y apoyado por otro, tenía olor a herejía. Por ello es que el sistema indoarábigo debió luchar largamente antes de ser adaptado por todos. Las particularidades de este debate no se conocen completamente y con precisión, pero una célebre obra del sacerdote Luca Pacioli sirve para probar, que a fines del siglo XV, el nuevo sistema era, ya, definitivamente victorioso.


Otra innovación que Italia difundió por toda Europa, gracias a Fibonacci, es el álgebra, llamada «arte de las cosas», siendo \textit{ Cosa} el vocablo que se usaba entonces para designar a la incógnita; porque, en efecto, la principal obra alemana sobre álgebra de aquel tiempo tenía como título \textit{ Die Coss},
lo que probaría que nuestra patria era reconocida como maestra en tal campo.

Debemos recordar que las exhortaciones para el estudio de las obras clásicas hechas por Petrarca y Bocaccio, con la palabra y con el ejemplo, terminaron por ejercer una benéfica influencia hasta sobre nuestros estudios, ya que por entonces las más ricas bibliotecas italianas fueron detenidamente exploradas por doctos matemáticos; siendo, así, como se pusieron en su lugar de honor las obras de Arquímedes, Apolonio, Ptolomeo, etc., que desde hacía siglos yacían en un vergonzoso olvido. Y así, gérmenes que estaban profundamente sepultados desde hacía siglos, se mostraron ricos de perenne vitalidad, no tardando en dar espléndidos frutos; ya que el ejemplo dado por nuestros eruditos no tardó en ser seguido por los habitantes del otro lado de los Alpes, aunque en este campo nuestra patria se mostró y fué reverenciada como portaestandarte y maestra.


%\asteriscos

Si observamos lo que el álgebra era a fines del siglo XV, no se tardará en reconocer que ella enseñaba, nada más, que a resolver ecuaciones determinadas de 1º y 2º grado, resultados a los que, en el fondo, habían llegado también los griegos recurriendo a oportunas figuras. Los estériles esfuerzos realizados para sobrepasar estos límites, sirviéndose exclusivamente del cálculo, habían inducido a los
matemáticos árabes a sentenciar que el álgebra no estaba en situación de ir más allá, y Luca Pacioli, que no era un advenedizo, mostró que compartía esta escéptica apreciación. Le estaba reservado a un hombre eminente, que aquel tuvo quizá como colega en la época en que enseñaba en la Universidad de Bologna, a extirpar este perjuicio, descubriendo el modo de resolver las ecuaciones de tercer grado. Por qué caminos Scipione del Ferro ---éste es el nombre del valeroso hombre--- haya llegado a este memorable resultado, no se conoce; porque el método de resolución, por él ideado, por razones desconocidas y que nos es difícil adivinar, lo mantuvo siempre en secreto; cuando murió legó a Annibale della Nave, su yerno y sucesor, un cuadernillo en el que aquel estaba expuesto. Desgraciadamente este precioso documento se extravió y, pese a las más asiduas búsquedas, no ha sido posible encontrarlo.
Es probable que la noticia de este importantísimo descubrimiento debió haberse filtrado, porque en aquel tiempo las ecuaciones cúbicas se encuentran enunciadas en los problemas propuestos como desafíos, según un sistema muy en boga durante el siglo XVI. Ahora bien ---y es éste un fenómeno del que la historia de la ciencia ofrece muchos ejemplos--- saber que un problema estaba ya resuelto, estimulaba a los estudiosos a adivinar el método seguido por el que había obtenido la solución, terminando siempre por llevarlos a una nueva solución del problema.

Sin hacerme garante que a un fenómeno de tal naturaleza se deban los sucesivos descubrimientos de las resoluciones de las ecuaciones cúbicas, es necesario destacar que ésta se logró en Venecia, por mérito de Nicolás Tartaglia, en 1534. Pero también este gran matemático, queriendo egoístamente conservar el monopolio de su descubrimiento, mantuvo secreto su método, limitándose a aplicarlo para resolver algunas cuestiones que un adversario suyo le propusiera públicamente, con el deshonesto propósito de confundirlo. De esto tuvo noticias otro eminente algebrista de ese tiempo, Girolano Cardano, quien, estando por publicar un metódico tratado de las ecuaciones algebraicas, pidió aclaraciones al colega. Tartaglia, que no sin razón desconfiaba de Cardano, se negó, por mucho tiempo, a satisfacer a su colega; pero, en una visita que le hiciera en Milán terminó por confiarle su método, mediante algunos groseros tercetos, a los que, aun cuando no hayan encontrado un puesto en alguna crestomanía poética, sin embargo se los considera como un fragmento clásico de la literatura matemática. Tartaglia le impuso, a su huésped, mantener secreta la comunicación que le hiciera; pero Cardano, hombre de moralidad muy escasa, no tuvo escrúpulo alguno en intervenir en el desafío. Este hecho dió motivo a una violenta polémica, cuya primera parte se destaca en un interesantísimo volumen de Tartaglia, cuya lectura recomendamos a quien desee formarse una idea del modo altamente descortés con que, en aquel
tiempo, se desarrollaban las polémicas sobre temas científicos. Pero el contenido de este volumen es un simple tiroteo de avanzada. La lucha estalló en toda su violencia, cuando un discípulo de Cardano, Ludovico Ferrari ---célebre por haber sido el primero en enseñar a resolver las ecuaciones de 4º grado---, que derribaba a todo aquel que hubiera osado atacar a su venerado maestro, lo invitó a un concurso público, del cual se declararía vencedor a aquel que resolviera todos los problemas algebráicos y geométricos propuestos por su adversario. Esto motivó un cambio de carteles de desafío, que constituyen uno de los más curiosos e instructivos de la historia de las matemáticas del siglo que nos ocupa.

Este trabajo no permite que hagamos conocer todas las gamas que ellas poseían, haciendo solamente mención a la forma áspera y violenta con que estaban redactados. Así, mientras Tartaglia, para hacer enojar a su adversario, lo trataba siempre como un monigote, como un simple instrumento utilizado por Cardano; Ferrari, usaba en sus comunicaciones el latín con el único fin de poner en aprietos a su competidor, que notoriamente era un hombre de cultura limitadísima. Tartaglia vacilaba en aceptar la propuesta de ir a Milán para participar en el proyectado duelo, intuyendo que en la gran metrópoli lombarda, encontraría un ambiente que le sería hostil, de manera que habría terminado por encontrarse en una situación muy semejante a la de Valentino, quien al batirse con Fausto, sabía que ése tenía ayuda de Mefistófeles. Después de largas discusiones sobre las condiciones en que se libraría la gran justa, y aún cuando sus amigos le aconsejaran lo contrario, terminó por aceptar, de manera que el torneo se realizó en una iglesia de Milán el 19 de Agosto del año 1584. Desgraciadamente no se redactó ningún expediente de esa histórica jornada,
aunque, por encargo del príncipe Boneompagni, se  ha minuciosamente investigado las crónicas milanesas de la época, éstas se mostraron mudas al respecto. Se conoce, solamente, una relación escrita por el mismo Tartaglia, en la que, pese a la explicable suavización del asunto, se ve que a él le toca peor parte; y en cuanto fueron depuestas las no corteses armas se apresuró a retomar a Brescia, donde vivía. Que fuera realmente vencido se deduce del hecho que, mientras a él le quitaron el cargo de enseñanza que le habían confiado sus conciudadanos, a su rival le llovieron ofertas de cátedras de la más importantes ciudades italianas.

Pero este infortunio, que sobre su trabajo le cayera a Tartaglia, no debe hacernos olvidar que a él se debe el haber podido traspasar las columnas de Hércules que se interponían al libre paso del álgebra. Ni tampoco debemos creer que la herida, grave, que le infiriera Ludovico Ferrari, sea la única que recibiera en su atribulada existencia. En efecto siendo aún niño, la ciudad de Brecia fué sitiada y tomada por un ejército francés al mando de Gastón de Foix, y habiéndose refugiado con los suyos en una iglesia, fué herido de un sablazo, en la cara, por un vil mercenario ultramontano, que le dejó tartamudo para toda su vida. No menos grave fué para su fama y funesto para la ciencia, el hecho de que la muerte le alcanzó mientras redactaba un enciclopedia matemática y, casi, había llegado a la solución de las ecuaciones superiores al 2º grado, lo que le impidió lograr la ambicionada reinvidicación de la derrota que sufriera en Milán.


%\asteriscos

La pluma que cayó de la mano del eminente matemático bresciano, fué tomada por otro italiano, Rafael Bombelli, el cual en una obra, que con todo derecho merece los fastos de la fama, expuso, no solamente una completa teoría de las ecuaciones de los cuatro primeros grados, sino que, venciendo la repugnancia que, generalmente, sentían los matemáticos de su tiempo por los números imaginarios, emitió una teoría plenamente satisfactoria de esos entes misteriosos y extraños.

Pero pese a estos espléndidos descubrimientos que ningún extranjero tuvo la audacia de discutirle a la Italia, el cetro algebráico, a fines del siglo XVI, le fué quitado a nuestra patria por Francia, gracias a los méritos de Francisco Viéte, que fué prontamente reconocido como hombre de ciencia de igual mérito que los más grandes. No voy a dar una explicación, atendible, de ese deplorable abandono que, de una floreciente provincia, hicieran los matemáticos italianos. Es que, quizá, a la producción de ese hecho no debió ser extraña la aparición, bajo el cielo de nuestra patria, de otro grande hombre que, con sus palabras y sus escritos, determinó una nueva y felicísima orientación del pensamiento científico italiano: me refiero a Galileo Galilei.








Educado por su padre en el ejercicio de la libertad del pensamiento, cuando llegó a la cátedra universitaria ---apenas contaba veinticuatro años de edad---, se sintió molesto en ella, por cuanto la Universidad de Pisa, en la que profesaba, era un bastión inexpugnable del más supino aristotelismo. Por eso recibió con grande alegría la noticia de que el iluminado Gobierno de la República de Venecia, le confiaba, en su Ateneo, un cargo semejante al que estaba desempeñando. Se iniciaron, desde ese momento, para él los años de la más fecunda labor científica y de intensa acción didáctica. Para contemplarlo en sus tareas y para escucharlo en sus elocuentes lecciones, llegaron de todas partes de Europa, numerosos jóvenes; y, con el transcurso de los años, recordaba que la época más feliz de su vida fué la que pasara en la ciudad de San Antonio.

Su renombre, ya grande, se acrecentó, aún más, al tener noticias de que en Holanda se había construido un maravilloso aparato con el que, gracias a una inteligente combinación de lentes, se podían ver los objetos lejanos como si estuvieran al alcance de las manos; Galileo adivinó su estructura y construyó uno semejante, que ha recibido el nombre del «anteojo de Galileo». Dirigió este aparato al cielo, y en breve tiempo reveló más fenómenos de la física celeste que los que habían hecho conocer treinta siglos de observación. En efecto ---y para sólo citar las más importantes observaciones astronómicas---, señaló los montes de la Luna y las faces de Venus, descubrió tres de los cuatro satélites de Júpiter; cuando afirmó que, también, el sol presenta manchas, los adoradores de Aristóteles gritaron escandalosamente, puesto que ello demostraba, en consecuencia, que era ilusoria la máxima que sostenía la incorruptibilidad del cielo, que había sido afirmada por el «maestro di color che sanno». El gran renombre que obtuviera hizo surgir numerosos detractores, algunos de los cuales llegaron a negar la validez de la verdad de sus afirmaciones, otros alegaron pretendidos derechos de propiedad; pero su mayor desgracia fué que sus principales adversarios pertenecieran a la Compañía de Jesús. Otra consecuencia de los descubrimientos astronómicos de Galileo fué, que el Gran Duque de Toscana, insistiera en invitarle a volver a sus estados, ofreciéndole condiciones muy favorables, no ya dándole más honores, sino que lo dispensaba de la obligación de dictar clases. Y la perspectiva de poder dedicar todas sus fuerzas y todo su tiempo al estudio tan amado, fué lo que indujo a Galileo a abandonar Padua por la ciudad de Florencia, decisión fatal de la que habrá, más tarde, que arrepentirse amargamente.

En Roma, a la que fuera poco después de su traslado, recibió la acogida que merecía, ya sea de parte de los estudiosos que en ella vivían ---prueba de ello es su ingreso a la Academia de los Linceo---, como también de las más distinguidas autoridades eclesiásticas, incluso el Papa. Una suerte contraria le llevó, poco después, a exponer algunas ideas ---no muy ortodoxas, por cierto--- sobre el modo científico de interpretación de hechos y palabras que se leen en las Sagradas Escrituras; ideas que ofrecen una profunda semejanza con las directivas de los modernistas, que tanto escándalo hicieron producir bajo el pontificado de León XIII. Estas no podían pasar desapercibidas en una época en que ---yo no hago más que repetir una observación de quien fuera eminente hombre de ciencia y príncipe de la Iglesia, esto es el Cardenal Maffi---, las más altas autoridades religiosas estaban preocupadas ante la gran difusión que asumían, en toda Europa, los principios de reforma sostenidos por Lutero y Calvino. En consecuencia el mundo eclesiástico, que hasta ese día era un tibio admirador del gran científico, no tardó en transformarse en hostil, tanto más que en dos importantes cartas por él firmadas, una a Benito Caetelli y la otra a Cristina de Lorena, Gran Duquesa de Toscana, hizo explícita adhesión al sistema astronómico conocido por el nombre de Copérnico. Estos hechos no fueron extraños para que el Pontífice de aquel tiempo: Pablo V, tomara una gravísima decisión. Con fecha 16 de Febrero de 1616, los teólogos del Santo Oficio fueron informados, por la Santa Sede, que debían censurarse las dos proposiciones siguientes:

\begin{itemize}

\item Que el sol sea el centro del mundo y que, por consiguiente, esté in\-móvil.

\item Que la tierra no está inmóvil, sino que goza de un movimiento diurno.

\end{itemize}

La censura fué inmediatamente decretada y el cardenal Bellarmino, designado delegado para el caso, el día 26 de febrero del mismo año la puso en conocimiento de Galileo, que por aquel entonces se hallaba en Roma, a quien se le notificó que se le prohibía, para el futuro, enseñar y defender la doctrina copernicana.

Este hecho se conoce bajo el nombre del «primer proceso de Galileo»; pero erróneamente, porque a las mencionadas deliberaciones, el gran científico no fué invitado a participar, con su presencia, en el debate de su causa. Es de hacer notar que fué, ésta, una de las más graves deliberaciones adoptadas, ya que, por primera vez, la Iglesia sentenciaba en una materia puramente científica. El inmediato corolario de esa sentencia fué, que poco después, se censuró la famosa obra de Copérnico, obra compuesta por el canónico de una conspicua catedral y que el Papa de ese tiempo ---Pablo III--- había aceptado la dedicatoria que su autor le ofreciera. Inmediatamente
a tal censura, aquel precioso volumen fué quitado de las manos de la juventud y ---sea esta una observación incidental---, es en vano solicitar ese volumen en nuestra Biblioteca universitaria, tan rica en obras científicas y que la Universidad ha heredado de un Colegio de Jesuitas, lo cual se explica por lo dicho.

Otra consecuencia de esa deplorable sentencia fué que las Academias científicas italianas de aquellos tiempos, la del Linceo y la de Cimento, llevaron una vida triste y terminaron por morir de inanicción. Y para darse cuenta de las condiciones en que ellas llegaron a encontrarse, se puede evocar un suceso que hace poco tiempo ha sido puesto a luz. Se ha descubierto en el registro de la Academia de Cimento, que se había realizado, allí, la célebre experiencia que, más tarde, guió a Foucault a la demostración de la rotación de la tierra, lo que permite que se postule el derecho de prioridad de esa demostración para Italia. Ahora bien, si no me equivoco, haciendo eso, se había dado un paso demasiado peligroso; ya que para deducir de aquella experiencia la mencionada demostración, era necesario haber dado ese paso importante del que no se ha encontrado ninguna prueba en los documentos existentes. Pero es indudable que, si alguno de aquellos geniales académicos hubiera pensado en darlo, habrá juzgado prudente el detenerse para no exponerse a los rayos fulmineos de una autoridad a la que nadie podía resistir.


Que como inmenso se juzgara, entonces, el poder de la Iglesia, lo demuestra otro hecho, estrechamente ligado a los que estoy ilustrando: Renato Descartes, ciudadano francés, que podía contar con una segura protección de Luis XIV, y que vivía en Holanda, país protestante, siguiendo los dictados del espontáneo impulso de sus propios pensamientos, compuso una obra cosmológica, en la que el sistema copernicano formaba parte importante. Ahora bien, cuando años más tarde tuvo noticias de la sentencia dictada en Roma, no solamente renunció a publicar su obra, sino que, prudentemente, la tiró a las llamas. «El coraje», dice don Abbondio, «no se lo puede dar uno mismo»; y la naturaleza, evidentemente, había sido avara con el famoso filósofo.

Galileo se inclinó ante un poder, contra el cual era inútil y vano el querer resistir, retomando, serenamente, sus estudios. Pero fué arrastrado de nuevo a recomenzar la lucha contra enemigos constantemente en acecho, por las críticas, vivaces y apasionadas, que el padre Grassi hiciera al \textit{ Discurso sobre los cometas} que Mario Guidici, devoto discípulo de Galileo, leyera en la Academia Florentina para hacer conocer algunas ideas de su Maestro. Para rebatir las insustanciales objecciones dirigidas contra Giudicci apareció el volumen galileiano titulado \textit{ Il Saggiatore}, modelo de polémica científica, el que también, por el espléndido estilo en que está escrito, fué ávidamente buscado para leerlo, por todas las personas cultas, no tardando en ser catalogado entre las obras clásicas de nuestra literatura.

Hacia ese tiempo, Galileo, dirigía toda su atención para realizar una metó\-dica exposición de sus ideas sobre el flujo y reflujo del mar, problema que, según él, era suficiente para dar argumentos que decidirían si el sol estaba fijo o era la tierra la que se encontraba inmóvil. Galileo ---que en ésta, como en muchas otras ocasiones, dió prueba de una ingenuidad casi infantil--- creyó que evitaría los peligros que ofrecía la navegación en un mar lleno de escollos agudísimos, presentando su obra bajo la forma de diálogo, en el que las opiniones en conflicto fueran expuestas por diversos personajes. Así nació el famoso \textit{ Diálogo sobre dos sistemas máximos}, el cual debía ser la causa de su ruina. Sus adversarios, siempre armados para perderlo, para conseguir su finalidad, imaginaron un medio verdaderamente infernal: hicieron creer a Urbano VIII, que entonces ocupaba la cátedra de San Pedro, que Simplicio, el personaje que en el diálogo galileiano representa la parte del inepto partidario del sistema ptolemáico, era la caricatura que Galileo había hecho del Papa. Gracias a esta astucia lograron transformar en implacable enemigo del gran astrónomo, a quien, como el cardenal Barberini, había sido su sincero amigo y ardiente admirador. Como consecuencia de esto, hacia fines de 1632, Urbano VIII impuso al autor del diálogo incriminando que se presentara, sin retardo, ante el Santo Oficio, en calidad de acusado.

El viaje de Florencia a Roma era, en aquel momento, terriblemente peligroso por la furia de la peste, aquella terrible epidemia cuya historia todos conocen a través de la conmovedora descripción que ha hecho Alejandro Manzoni. Era ésta, una amenaza más de peligros para Galileo, cuya salud, por esos días, estaba muy resentida. A tal efecto intercedió el Gran Duque de Toscana, quien puso en acción todo su ascendiente para obtener, por lo menos, una dilación del amenazador juicio. Pero fué en vano. Y así fué como Galileo durante los primeros días del mes de Enero de 1633, pese a la inclemencia de un rigurosísimo invierno, se encaminaba hacia la Ciudad Eterna. La extensión de este trabajo, me impide detenerme en los detalles del más famoso proceso que registra la historia de la ciencia, que duró no menos de cinco largos meses. Haré notar que, ante la actitud extremadamente sumisa del acusado, no fué necesario llevarlo a la tortura como los jueces deseaban hacerlo. Por su parte, era ya costumbre, desde hacía tiempo, que aquel bárbaro procedimiento no lo aplicara la Inquisición a las personas que habían llegado a la edad de 70 años. La sentencia se dictó el día 19 de Junio de 1622 y fué, no muy severa, quizá muy benigna, ante lo que razonablemente se podía esperar, ya que al tribunal que lo juzgaba se le había impuesto, taxativamente, aplicarle la censura anticopernicana; y en muchas páginas del \textit{ Diálogo} incriminado ---como lo confesara cándidamente el mismo inculpado--- se leían frases que ---contrariamente a una solemne promesa que Galileo hiciera ante el Cardenal Bellarmino--- constituían una patente violación a lo establecido. Por consiguiente el Santo Oficio declaró a Galileo «vehementemente sospechoso de herejía», pero de tal culpa se declaró dispuesto a absolverlo siempre que él formulara una abjuración solemne. Habiendo sido ésta pronunciada en una sesión, bajo forma explícita y solemne, la pena de Galileo fué reducida a «carcel formal» sujeta al arbitrio del Santo Oficio, y a recitar todas las semanas, durante tres años consecutivos, los siete salmos penitenciales. Es una simple leyenda la que dice que Galileo al oir la condena contra él dictada, haya exclamado: «Y con todo, se mueve» («Eppur si muove»); fábula, cuya génesis y desarrollo, la ha determinado la erudición alemana, fábula que manifiesta la imposibilidad de que una frase tan audaz fuera pronunciada por un hombre en las desgraciadas condiciones de espíritu y de cuerpo en que se encontraba Galileo, frente al tremendo tribunal que lo había juzgado. Se trata, sin embargo, de una leyenda destinada a quedar arraigada en el alma popular como una afirmación rotunda de la futilidad de todo esfuerzo que pretenda violentar las leyes eternas de la naturaleza; inanidad que Voltaire, con serena bonomía, afirmaba al escribir, comentando la famosa sentencia.


\begin{verse}

Y sin embargo la Tierra, en su carrera fiel\\
 lleva, consigo, a Galileo y a sus jueces 

\end{verse}


%\asteriscos

Gracias a los buenos oficios del Gran Duque de Toscana, se le concedió a Galileo permiso para dejar la prisión del Santo Oficio y trasladarse a Siena, como huésped de este arzobispo. Este iluminado pastor se había propuesto hacer que Galileo sufriera menos por la lejanía de su familia, pero sus implacables enemigos lograron que se le impusiera trasladarse a Arcetri, para vivir en un domicilio previamente fijado. Y basta sus solicitudes de permiso para trasladarse a Florencia, con el fin de atender su delicada salud, fueron siempre rechazadas.

Concédame el lector el poder pasar por alto otros episodios de este largo proceso, porque deseo hacer resaltar, ante vuestra atención, el raro fenómeno que ofrecía ese viejo, desecho por los sufrimientos físicos y morales, que parecía un cadáver ambulante, y que, sin embargo, mantenía su inteligencia esplendente y poderosa como para poder dar fin a aquellos \textit{ Diálogos sobre dos nuevas ciencias}, en la que se deben buscar los fundamentos de la moderna dinámica. Lagrange a propósito de esto, observa
agudamente: «El descubrimiento de los satélites de Júpiter, de las faces de Venus, de las manchas del sol, etc., etc., sólo exigen telescopio y asiduidad en la observación; pero era necesario poseer un genio extraordinario para descubrir leyes naturales de fenómenos que siempre habían estado bajo nuestra observación, pero que su explicación, no obstante, siempre se les había escapado a los filósofos».

¡Fué éste el último guiño de una lámpara próxima a extinguirse! Completamente ciego y consternado por el descenso a la tumba de su hija predilecta, la dulce Sor María Celeste, la muerte fué para el gran investigador una verdadera liberación. ¡Pero sus enemigos no se aplacaron ni en presencia de la majestad de la muerte! El Gran Duque de Toscana quería levantarle un digno monumento, pero fueron tales los obstáculos que pusieron ante su idea, que renunció a ella; y miles de dificultades encontraron aquellos que pretendían llevar a la práctica el más ardiente deseo del gran pensador: ver reunidos, en un todo orgánico, sus admirables escritos. Pero con el tiempo se hizo justicia sobre todo esto; los decretos que obstaculizaban el libre desarrollo del pensamiento científico, fueron derogados por la misma autoridad que los había dictado; Galileo ha obtenido en Santa Croce el recuerdo a que tenía sobrados derechos, ello fué en la época en que Hugo Foscolo escribía su \textit{ Sepulcros}; y centenares de ediciones, traducciones y comentarios han difundido y continúan difundiendo, en todo el mundo, el pensamiento galileiano, y, por consiguiente, la benéfica influencia por él ejercitada es visible hasta en campos de la actividad humana, que él había preparado
sin profundizar su cultivo.

Como prueba de esto, están las ideas, por él manifestadas, sobre el modo de utilizar, en el estudio de la geometría, el concepto de infinito, ideas que habían estallado en su mente, cuando, en los años juveniles, meditaba sobre la obra de Arquímedes; esas fueron recogidas por Buenaventura Cavalieri, quien dió a luz su célebre \textit{ Geometría de los indivisibles}, cuyo método, original e importantísimo, fué confirmado en toda su potencia por Evangelista Torricelli. Pero todo esto no basta: los gentiles hombres ingleses que escucharon las lecciones de Galileo, vueltos a su patria llenos de entusiasmo por su gran maestro, difundieron sus ideas del otro lado de la Mancha, actuando de manera muy semejante a las aves que transportan en sus alas los gérmenes de nuevas plantas; y éstas germinaron visiblemente en las obras de Wallis, de Barrow, los cuales, preparando el camino a Newton, determinaron el período más glorioso de la historia de las matemáticas en Inglaterra.

%\asteriscos

Con el descubrimiento de la gravitación universal, y la invención del método de las fluxiones, se abre una nueva y gloriosa era para las ciencias físicomatemáticas. Aun cuando las nuevas ramas nacidas sobre el vetusto tronco parecieran a algunos no lo suficientemente vigorosas, sus brillantes aplicaciones llevaron al convencimiento de estar en posesión de principios capaces de desatar los más complicados nudos, de inundar de luz las cuestiones más profundas. El alborozo que acompañara a las nuevas conquistas, llevó a hacer creer que los nuevos cálculos podían extender su poder sobre todo lo creado, no solamente sobre el mundo físico, sino que también sobre el mundo moral. ¿Quiere el lector una prueba? Se llegó a soñar que la más caprichosa de las pasiones humanas, el amor, estaba sujeta a leyes numéricas; y Algarotti, conocido popularizador de la filosofía newtoniana, enunció, con gran valentía, la siguiente ley: «El amor de un amante decrece en razón inversa al cubo de la distancia que lo separa de su amada y del cuadrado del tiempo que dure su
ausencia».



Es necesario que, con justicia, se advierta como a la nueva orientación del pensamiento matemático, producido a fines del siglo XVII, contribuyó, poderosamente, el más grande matemático que Alemania haya dado antes que Gauss, esto es Leibnitz. Se debe en efecto, reconocer que si los \textit{ Philosophie naturalis principia mathematica} de Newton, la obra que algunos juzgan como la más grande producción científica que haya visto, hasta ahora, la luz, dió un potente impulso a los estudios de mecánica y de astronomía, los matemáticos puros le deben a Leibnitz
enorme gratitud, ya que él les aseguró una agil simbólica, modelada sobre la del álgebra, gracias a la
cual, el poner en ecuación las más elevadas y delicadas cuestiones geométricas no presenta dificultad substancial, mientras que a tanto no permiten llegar las argumentaciones que han tornado como modelo las usadas por Newton. Esta es la razón por la cual, mientras que los discípulos del gran investigador inglés son pocos en número y de escaso valor, los partidarios de su émulo alemán, muy pronto formaron legión, y el valor de los que la componían queda probado, recordando que entre ellos formaban los dos hermanos Jaime y Juan Bernouilli. En esta diferencia, a la cual no podía permanecer indiferente Newton, se debe, si no me engaño, buscar la profunda raíz de la gran polémica planteada en torno a la prioridad de la invención del nuevo cálculo y que se mantuvo hasta el comienzo del siglo XVIII, y que prosiguió, con grandes intervalos y en forma descortés, durante los primeros treinta años de ese siglo.



%\asteriscos

El que desee conocer los orígenes de esta polémica, verá que en la época en que estos dos grandes hombres de ciencia elaboraban los métodos destinados a conducirlos a la inmortalidad, sus relaciones, aún cuando se desarrollaban a través de comunes amigos, fueron tan cordiales que, en 1687, Newton insertó, en el II Libro de sus \textit{ Principia}, un escolio en el que declaraba que entre ellos había existido un cambio de ideas en torno al modo de usar el concepto de «infinito» en el estudio de los fenómenos que ofrecían las extensiones figuradas. Por el contrario, Leibnitz, al publicar tres años antes la célebre memoria que contiene la exposición de los conceptos fundamentales del cálculo diferencial, no hizo mención alguna a sus relaciones con Newton; más aún, tres años después de la publicación de los \textit{ Principia}, dió a luz una infeliz memoria acerca del movimiento de los proyectiles en un medio resistente, en la que entre algunas inexactitudes originales, se leen, sin citar las fuentes, varias proposiciones tomadas de los \textit{ Principia} newtonianos. Este procedimiento incorrecto indujo a Wallis, sin duda alguna de acuerdo con Newton, a incluir en la colección de sus \textit{ Obras Completas} una memoria de éste sobre el cálculo fluxional, que lleva la fecha de 1666, pero que hasta entonces, se había mantenido inédita.

Estos episodios pasaron casi desapercibidos y asumieron el grado de escaramuzas, sólo cuando la guerra entre los dos pensadores, fué oficialmente declarada.

Esto se produjo el día ---estamos en 1699--- en que Faccio de Duillier ---mediocre matemático, de origen italiano, emigrado en Suiza y más tarde en Londres---, desahogó su rencor contra Leibnitz, por una pretendida ofensa recibida de aquel, y quizá más bien como acto de cortesanía hacia Newton, cuando al dar a luz una publicación de carácter doctrinal, insertó la declaración de que Newton, era, indiscutiblemente, el primero en inventar los nuevos cálculos, pudiéndose considerar a Leibnitz, como el \textit{ segundo}. Leibnitz contestó, citando el ya mencionado escolio de los \textit{ Principia}, que él consideraba como explícito reconocimiento, por parte de Newton, de sus derechos de \textit{ contemporaneidad} si no de prioridad.

Un nuevo choque se produjo, cuando, en 1704, en el apéndice al \textit{ Tratado de óptica} de Newton fué incluída una vieja memoria suya sobre la cuadratura de las curvas. Las \textit{ Actas eruditorum}, el gran periódico de Leipzig en cuya redacción intervenía Leibnitz con aportes de importancia, publicó una resensión anónima, que se cree haya sido escrita por el mismo Leibnitz, en la que se insinúa, delicadamente, que las «fluxiones» newtonianas no son otra cosa que una metamórfosis de las «diferenciales» leibnitzianas. Ante esta afirmación el gran inglés se sintió profundamente ofendido. Pasaron tres años ---nótese la lentitud con que, entonces, sucedían las cosas--- antes que apareciera una réplica a la audaz afirmación de Leibnitz. Esa se encuentra en una memoria, escrita por Keill, fiel portavoz de Newton, incluida en el \textit{ Philosophical Transactions}, de la Sociedad Real de Londres, de la que Newton era entonces presidente. En ella se afirma explícitamente que Newton es, \textit{ indiscutiblemente}, el primero en inventar los nuevos cálculos, habiéndolos él usado antes que Leibnitz diera a luz la ya mencionada memoria del año 1684.

El filósofo del optimismo al leer este trabajo debió reconocer que las cosas de este mundo no se realizaban del mejor modo posible y, aprovechándose de su calidad de miembro de la Sociedad Real de Londres pidió, a ésta, satisfacciones por la ofensa que se le había inferido, mediante una explícita negación de lo afirmado por Keill. Grande fué el embarazo en que se encontró la excelsa corporación científica inglesa, de la que, recordémoslo era, siempre, Newton su presidente.

En vez de acceder al pedido recibido, decidió encargar, al mismo Keill, la exposición completa de las razones en base a las cuales debíase mantener como indiscutible la prioridad de Newton. Este documento fué redactado ---con el asesoramiento del mismo Newton--- en muy breve tiempo e inmediatamente comunicado al principal interesado. Como podía preveerse, Leibnitz no quedó satisfecho y envió una nueva nota, de un tenor de más acentuada protesta, a la más alta institución cultural de Inglaterra. Esta encargó, como consecuencia, a una Comisión, el recoger y publicar todos los documentos relativos a la espinosa cuestión. Los componentes de esa Comisión eran hombres de ciencia y políticos ingleses, Newton no figuraba oficialmente, pero fué inspirador y guía constante de sus trabajos, mientras que a Leibnitz no se le invitó a participar de sus reuniones, ni tan siquiera designando un representante. Gracias a la activa y oculta intervención de Newton, la Comisión pudo despacharse con sorprendente rapidez, tanto que la obra que compendia el trabajo ---y que se la designa comúnmente con el
nombre de \textit{ Comercium episitolicum}---
pudo ser presentada, impresa, a la Sociedad Real en su primera sesión del año 1713.

Para darse cuenta de la naturaleza e importancia de esta publicación, es menester tener presente que en la época que estamos estudiando el periodismo científico estaba en sus comienzos y era muy poco difundido. Por eso es que los descubrimientos se comunicaban, por medio de cartas, a los amigos, los cuales, a su vez, los comunicaban a los que se interesaban en la materia. A pesar de que hubieran aparecido verdaderas agencias de información científica, eran enormes las dificultades que encontraban aquellos que, usando una frase moderna, «querían ponerse al día». ¿Así que quién no ve la imposibilidad de poseer un catálogo completo y exacto de las personas que tuvieran conocimiento de determinados descubrimientos y del momento en que estos eran realizados por alguien? Y, ¿quién puede determinar el número y valor de los documentos dispersos, entonces, y más tarde irremisiblemente perdidos? Ahora bien, el \textit{ Comercium epistolicum} sirvió para impedir la pérdida de tesoros de inestimable valor, pero, como documento histórico, sufría de insanable vicio de origen: no era una exposición cuyo fin debía ser iluminar ampliamente la cuestión debatida, sino que es una recopilación de escritos aptos para establecer los derechos de prioridad de Newton respecto a la invención de los nuevos cálculos; por eso es que, más que una relación redactada sobre la causa, por un sereno magistrado, es semejante a una arenga de un experto, por no decir astuto, abogado.

Al recibir la copia que le destinaron, Leibnitz, antes de decidirse como debía contestar, deseó conocer la opinión de una persona, amigo suya y que gozaba de indiscutible autoridad: Juan Bernouilli. Este le escribió una carta reservada en la que manifestaba, sin reticencias, una opinión completamente favorable a su ilustre corresponsal. Leibnitz, al leer, gozoso, el reconocimiento que se hacía de sus derechos, no pudo resistir la tentación de publicarla, y aún cuando en su impresión se colocara que ella provenía de un anónimo y «eminente matemático», todo el mundo se dió cuenta de quién la había escrito. Bernouilli, que tenía interés en mantener amistosas relaciones con los hombres de ciencia ingleses desmintió esa atribución, llegando hasta un falso juramento; pero pese a todo ello la paternidad de esa carta nadie se la negó.

Como se ve, la polémica se agriaba y ampliaba, como las ondas concéntricas de las aguas quietas de un estanque turbadas por la caída de una piedra. En ella muchos personajes intentaron interponerse como pacificadores, pero, como le sucede comúnmente al que asume tal actitud, el único resultado que obtenían era la enemistad de una de las partes. Durante este proceso, Leibnitz pensó defender sus intereses componiendo un \textit{ Commercium epistolicum} que se opusiera al publicado en Inglaterra. Pero cuando estaba dedicado a este trabajo fué interrumpido por la muerte ---14 de Noviembre de 1716---.

Para dar una idea de la morbosa pasión que él ponía en este trabajo, hay un hecho poco honorable para él, que fué revelado por la indiscreta curiosidad de sus herederos: entre los documentos con los que iba a probar sus derechos había una carta con fecha de 1675. Y bien, ¡él no tuvo escrúpulos en cambiar la fecha por la de 1673! Frente al espectáculo ofrecido por esta falsedad, cometido por un hombre de limpia moral, como no recordar al Príncipe de Macchiavelli, que, siguió siendo honesto hasta el día que se le presentó la oportunidad de ganar un reino. Ni el alejamiento de uno de los contendores detuvo la larga disputa, la cual había terminado por salir del campo científico para asumir el carácter de libelo polémico entre dos pueblos, raramente amigos, pero que en aquella época habían terminado por ser hostiles ante los manejos que llevaron al trono de Inglaterra a un príncipe de aquella Casa de Hannover, de la que Leibnitz era, no sólo un historiador, sino su consejero áulico escuchadísimo. Y Newton, habiendo constatado que el \textit{ Commercium epistolicum} tenía poca difusión en el mundo científico, preparó y dirigió una reedición. Además, y para peor, al preparar la tercera edición
de su \textit{ Principia}, alteró totalmente el escolio relativo a sus relaciones con Leibnitz, del que ya he hecho mención varias veces, y con indiscutible perfidia dejó intactas las primeras y últimas palabras, de manera que la metamórfosis pasara inobservada, suprimiendo todo aquello que había sido interpretado como un reconocimiento de los derechos de Leibnitz a la invención del cálculo infinitesimal.


Poco después ---18 de Marzo de 1727--- también el grande hombre ciencia inglés descendía a la tumba. La discusión prosiguió igualmente, pero perdiendo gradualmente su aspereza. Una verdadera legión de estudiosos, de todos los tipos y de todas las nacionalidades, se ocupó de investigar en esta polémica, y el fruto de ello es una imponente literatura, rica en documentos que, en gran parte, están dotados de alto interés. Con todo esto, ¿hay ya elementos para pronunciar, en última instancia, el juicio definitivo sobre la más importante cuestión de prioridad registrada en los anales de la ciencia?

Para poder responder a esta pregunta, es necesario hacer, previamente, una distinción.

Si se quiere decidir, tomando como base documentos inexpugnables, conviene reconocer que el tiempo, aun, no ha pasado para poder tentar una respuesta definitiva. A tal fin, es necesario esperar que la edición de la \textit{ Obras} de Leibnitz esté completa;
edición que, antes de la guerra, había empezado
la Asociación Internacional de las Academias y que, terminada la guerra, Alemania, en un gesto patriótico, ha resuelto terminar por si sola. Pero es necesario esperar que Inglaterra, decida, por fin, dar a publicidad la totalidad de los manuscritos de aquel que le aseguró, durante un siglo, una indiscutible superioridad en el campo de las ciencias físico-matemáticas.

Si en vez de ello nos proponemos resolver la cuestión sobre la base de argumentos intrínsecos, se llega a sostener que la gran polémica, fomentada por bajos sentimientos de hombres pequeños y alimentada por ardientes pasiones políticas, si hoy fuera sometida a un juez agudo y desapasionado como una querella de plagio, conduciría a una absolución general de los contendores por no haber cometido el delito. Porque ni Leibnitz, ni Newton, enriquecieron a la humanidad con una nueva construcción, pero se propusieron llegar, y llegaron, a la cúspide de un edificio que, lentamente y constantemente, se estaba construyendo. Sus cimientos habían sido puestos por Eudocio de Cnido y Arquímedes; los distintos pisos eran obra de geómetras de distintas estirpes: de los italianos Galileo, Cavalieri, Torricelli; de los franceses Pascal, Fermat, Roberval; de los ingleses Wallis, Mercatore, Barrow. Estos expertos constructores ---aún cuando no se pueda negar que en su fatigoso trabajo hayan sufrido recíprocas influencias---, llegaron a la anhelada meta 
usando conceptos diametralmente diferentes. Mientras que Newton recurría a razonamientos geometricos y mecánicos, los cuales presentan una indiscutible semejanza con otros de origen griego; Leibnitz, cediendo a la invencible tendencia al simbolismo, que caracteriza toda su producción científica y filosófica, enriqueció la ciencia con un algoritmo del todo nuevo, gracias al cual el progreso del análisis fué de tal vuelo
que no hay lengua ni pluma capaz de describirlo.

%\asteriscos

Por eso es que, si las particularidades de la larga polémica interesan a quien estudia la psicología de los grandes hombres o de quien, malignamente, se complace en constatar que el hombre, por más alto que lo haya puesto la naturaleza, no llega siempre a dominar las pasiones y sentimientos, que son resortes poderosísimos para los sucesos que, diariamente, se desarrollan en nuestra presencia; quien sólo intente seguir la evolución del pensamiento científico, buscará de ser informado acerca de la orientación que ese asume por obra de los dos más grandes representantes que son la gloria de Inglaterra y Alemania.

Esto es lo que me propongo exponer en la última parte de este trabajo, admitiendo, siempre, que encuentre \textit{ una} persona, por lo menos, que tenga la cortesía de seguirme.



\newpage

\section*{Ayer y hoy}\addcontentsline{toc}{section}{Ayer y hoy}




Pocas épocas han sido tan severamente juzgadas como lo fué el siglo XVIII. Damas con el guardainfante, recubiertas con polvo de arroz, de afeites y de lunares; hombres en empalagosas posturas galantes, y de complacientes maridos, hicieron que se le denominase con el poco envidiable epíteto de «galante». Pero en este, como en todos los juicios muy sintéticos, inspirados en una visión unilateral, se anida una causa de error. Es verdad que los ociosos de los más altos estratos sociales, contemplados por el vulgo, llevaban una vida que José Parini, con su pluma inflexible, ha descripto vivazmente. Pero el movimiento del pensamiento humano, ni entonces ha sufrido deplorables detenciones; esa época tan depravada posee, para el historiador de las ciencias exactas, una extraordinaria importancia, ya que en ella se produce el primer paso del desarrollo de la matemática moderna, hoy instrumento habitual de trabajo para quienes cultivan las ciencias exactas o aplicadas.

A la aparición de este fenómeno contribuyeron, no solamente los creadores del método infinitesimal, que ya conocemos, sino que también todos los valerosos pioneros que ilustraron el fecundo siglo XVII.

En los primeros años de ese siglo, un barón escocés, Neper, enseñó dos expedientes aptos para aliviar las fatigas de los calculadores: con sus famosos «bastoncillos» ofreció un medio instrumental modesto, pero útil, inspirado por Pascal y Leibnitz, cuando imaginaron las máquinas de calcular, que llevan sus nombres, y que son las progenitoras de esas que tanta difusión han adquirido en la acelerada época actual; y con la invención de los logaritmos, Neper conquistó la gratitud de todos aquellos calculadores que pasan su vida manejando grandes números. Cuando se tiene presente que hacía poco tiempo se había logrado dar buen término a la gigantesca empresa de construir tablas de las líneas trigonométricas con quince cifras decimales, se está tentado de encontrar, en la creación del nuevo expediente de cálculo, la confirmación de la célebre máxima biológica: la función crea al órgano.

No había aun descendido Neper a su tumba, que ya Descartes iniciaba los estudios que debían culminar con la creación de un medio de investigación geométrica, capaz de satisfacer la viva aspiración de los modernos de proceder, moviéndose sobre vías bien balaustradas, con rapidez constante: esa nueva creación era la Geometría Analítica. Es útil observar que los que consultan, hoy, la célebre obra del filósofo francés, son golpeados por una gran desilusión, porque en ella encontrarán muy poco de lo que hoy constituye el moderno método de las coordenadas. Eso explica por qué, cuando esa obra apareciera, fué grandemente admirada y comentada, más por la contribución que ella presta a la teoría de las ecuaciones algébricas, que como reveladora de un nuevo instrumento para la investigación de las cualidades de las extensiones figuradas. Es verdad que, con el orgullo que distingue a cada página por él escrita, Descartes enunció una serie de problemas a los que, según él lo afirma, son aplicables sus consideraciones; pero se trata más bien de un programa para labores futuras, que un catálogo de obras ya terminadas. Así que, teniendo en cuenta los esfuerzos que fueron necesarios realizar para el desarrollo de este programa, se ve, que sin equivocarse, él escribiera en el epílogo de su obra, que a sus herederos les hubiera sido grato, no sólo por las cosas que él les enseñara, sino también por aquellas que omitiera a propósito para dejarles el placer de hallarlas.

La indiferencia de los contemporáneos por la nueva rama, que nacía sobre el vetusto tronco, es tanto más maravillosa cuando se piensa que a la misma concepción había llegado, por esa época, otro gran matemático francés, Fermat, quien, por el contrario, expuso sus ideas sobre ese argumento en una forma mucho más próxima a la que estamos acostumbrados; eso puede ser explicado teniendo en cuenta que, en aquella época, todos los matemáticos
de renombre querían acelerar el parto de otra criatura que ya había llegado a su madurez: el Análisis infinitesimal.

Tiene Fermat el gran mérito de haber enseñado al mundo que la serie natural de los números ofrece un admirable campo para el desarrollo de geniales investigaciones. Con los importantes problemas que él resolviera y con los originales teoremas por él descubiertos, ha creado la Teoría de los números, y demostrado que su serie es completamente heterogénea, en la que cada uno de sus elementos goza de prerrogativas diferentes de las que poseen los otros, siendo semejante, no a uno de los puntos de una recta, sino mejor a uno de los elementos simples de la química clásica. Al resolver algunos de los problemas del juego de cartas o dados, propuestos por Pascal, escribió las primeras páginas de otra nueva rama de las matemáticas: la Teoría de la Probabilidad, disciplina diversamente juzgada pero que ha permitido al método experimental el de ser aplicado a la investigación de los fenómenos que se producen en la sociedad civil.

Pero con esto no he agotado la lista de los beneficios matemáticos de aquel gran siglo; falta en ella el concepto de proyección que, al asumir en la Geometría un puesto de comando, le hizo seguir las evoluciones que el siglo XIX debía llevar a su total realización, creando la moderna Geometría Proyectiva. El principal mérito de estos resultados se debe a Desargues y Pascal, dos matemáticos que legaron sus nombres a fundamentales proposiciones que llegaron a ser, justamente, famosas y por todos conocidas.

%\asteriscos


No debe creerse que todos estos resultados, en una época tan belicosa como fuera el siglo XVII, hayan sido obtenidos sin dar lugar a discusiones, polémicas, y desafíos matemáticos. Descartes y Fermat se cambiaron, en efecto, por carta y bajo una forma muy caballeresca, rasguños y estocadas que dejaban la señal en aquel hacia el que iban dirigidos. Al mismo tiempo, Fermat dirigía a los matemáticos ingleses dos carteles de desafío, que son recordados con gratitud porque, por su intermedio, permitieron realizar notables progresos en el campo de la alta aritmética. Por su parte, Pascal proponía un concurso con premio, que dió lugar a agitadas polémicas; su gloria se acrecentó, su bolsa no sufrió daño alguno, ya que jamás fueron pagadas las sesenta «pistola» que él prometiera; pero su fama fué obscurecida por las indignas calumnias que lanzara contra un muerto en quien la grandeza de su ingenio estaba a la par de su gran honestidad; nos referimos a Evangelista Torricelli. Por último, el mismo Desargues, cansado ante la oposición inconsulta que encontraban sus procedimientos para delinear las perspectivas, de parte de los artistas, prometió un premio en dinero a aquel que fuera capaz de encontrar otros mejores. Así es que el largo periodo que se abre con las discusiones, más o menos serenas, que se produjeron en Italia entre los que solucionaron las ecuaciones cúbicas y bicuadradas, y que concluye con la gran polémica entre Leibnitz y Newton, está cubierto de episodios cuya violencia ofrece un gran contraste con los fines puramente ideales por los cuales se combatía.

%\asteriscos

Teoría de las ecuaciones algebráicas y teoría de los números; geometría, en el sentido que le daban los antiguos, y aplicación del álgebra para la investigación de las propiedades de las figuras; uso del concepto de probabilidad para el estudio de los fenómenos que ofrece la vida social; cálculo diferencial y cálculo integral; tales son los campos de nuestra ciencia que constituyen la rica herencia que el siglo XVII legó al siguiente, y que éste no tardó en demostrar su capacidad de cultivarlo de la mejor manera.

A principios de este siglo se destacan Italia y Suiza; aquélla con tres nobles familias ---los Manfredi, los Fagnano y los Riccati--- y más tarde con el gran Lagrange; ésta con la dinastía de los Bernouilli y Euler. La historia nos mostrará un fenómeno extraño, diré casi patológico, ofrecido por el análisis del infinito; y en la ebriedad producida por la gran conquista, los investigadores, en presencia de los resultados que estaban en pleno desacuerdo
con el más vulgar sentido común, no trepidaron en retenerla aunque ella fuera producto del error. Y así fué como el gran Euler se atrevió a estampar
en una de sus obras la fórmula
$$
1-2+3-4+5-6+ \cdots = \frac{1}{4}
$$
	doblemente absurda, ya sea porque combinando por adición y substracción cuantos números enteros se quieran, no se puede llegar a un número fraccionario; ya sea porque la suma $1+3+5+\cdots$ es evidentemente menor a la de $2 + 4 +6+\cdots$, de donde, restando ésta de aquélla no se puede obtener un número positivo. Por su parte Cuido Grandi, docto eclesiástico que fué uno de los primeros, en Italia, en usar el análisis leibnitziano, al obtener la fórmula
	$$
	\int_0^x x^n \, dx = \frac{x^{n+1}}{n+1}
	$$
que sirve para calcular el área de una parábola de orden superior, la aplicó\linebreak imprudentemente al caso $n = -1$, obteniendo un resultado infinito; adelantándose, luego, a considerar valores del exponente $< - 1$, para interpretar geométricamente su resultado, introdujo, juntamente con Wallis, la inconcebible noción de «espacios más que infinitos», defendiéndola con una obstinación digna de mejor causa, contra matemáticos más razonables que él. El mismo matemático, un buen día, partiendo de la conocida fórmula
$$
\frac{1}{1+x}= 1-x+x^2-x^3+\cdots
$$
que se obtiene de la suma de los términos de una progresión geométrica, haciendo $x = 1$ ---substitución de cuya legitimidad, por entonces, nadie dudaba--- obtuvo el siguiente resultado:
$$
\frac{1}{2}= 1-1+1-1+\cdots
$$

Ahora bien, según se escriba esta relación bajo una u otra de las dos formas siguientes
\begin{eqnarray}
\frac{1}{2}& = & (1-1)+(1-1)+ \cdots = 0+0+\cdots \\
\frac{1}{2}& = & 1-(1-1)-(1-1- \cdots = 1-0-0- \cdots
\end{eqnarray}
se encuentra como valor del segundo miembro 0 o bien 1, resultados éstos absurdos y contradictorios entre sí.

Y bien; el buen sacerdote no tuvo escrúpulo alguno en explicar la presencia de $1/2$ como valor medio entre 0 y 1, asegurando que la fórmula (1) era la expresión matemática del mundo de la nada.





Pero en matemáticas no pueden existir paradojas permanentes, porque la presencia de un resultado de apariencia absurda es un signo cierto de la existencia de un punto a aclarar, de un problema a resolver; y, en efecto, todos sabemos que las fórmulas que acabamos de citar durante mucho tiempo estuvieron desterradas del cuerpo de la ciencia, y lo fueron por la acción de hombres de ciencia de los que dentro de un momento nos vamos a ocupar.


Pese a estos defectos, que la probidad histórica me obliga a señalar, la obra analítica desarrollada por el siglo XVIII es verdaderamente imponente, especialmente la cumplida por las grandes obras de Euler y Lagrange, dignos ambos de dar su nombre a la época en que vivieron. Si la obra del primero produce admiración por la perfección de su estilo, por la cristalina lucidez de las argumentaciones, por la simplicidad de los cálculos que, muchas veces, enmascara la gran profundidad de su pensamiento; ¿cómo expresar adecuadamente los sentimientos que despierta el espectáculo ofrecido por unos cuarenta volúmenes en cuarto escritos por un hombre que fué ciego gran parte de su vida, gracias a los cuales todas las ramas del análisis puro y aplicado llegaron a un gran perfeccionamiento, y algunas teorías consiguieron su equilibrio definitivo? Renán escribía que la más grande recompensa a la que pueda
aspirar un pensador es que sus ideas vayan perdiendo los rasgos de su creador al ir transformándose en propiedad de todos. Si, como parece, esta afirmación está de acuerdo con la verdad, al gran hombre de ciencia suizo no le ha faltado el goce de la satisfacción de esa aspiración, ya que, recorriendo los escritos de Euler, a cada paso se encuentran ideas, métodos y resultados que todos conocen por haber sido adoptados por los posteriores cultores de la materia, y sin que en general se sintiera necesidad de declarar su origen.


%\asteriscos

Frente a los grandiosos descubrimientos de Newton, Francia se mantuvo al principio indiferente, y luego, por largo tiempo, titubeante, pareciéndole a muchos de los miembros de la Academia de París que el principio de la gravitación universal infirmase algunos de los cánones fundamentales de la filosofía cartesiana. Y así fué, que los hombres de ciencia franceses, se dividieran en dos distintas categorías: los newtonianos y los cartesianos. Entre los segundos se encontraba un eminente connacional nuestro, Juan Domingo Cassini, exilado y naturalizado francés, el que sostenía que la tierra era un esferoide alargado por los polos, mientras que, para la concepción newtoniana, debía ser achatada. Para terminar con las largas discusiones entre los sostenedores de la «sandía» y los de la «naranja», el gobierno francés dispuso subvencionar dos expediciones geodésicas: una encargada de medir un grado del meridiano cerca del ecuador, y la otra de hacer otro tanto en la Laponia. La primera expedición fué capitaneada por La Condamine, y la segunda por Maupertuis. Ahora bien, la comparación de los resultados obtenidos demostró, aún a los más incrédulos, que Newton era el que se encontraba en la razón. Desde ese momento, y gracias en parte a la acción de Voltaire, Francia, acallando sus propios sentimientos nacionalistas, deviene francamente newtoniana, quedando en esta postura para siempre. Prueba de ello es que todos los trabajos de mecánica celeste de Laplace pueden ser justamente caracterizados como variadas y geniales aplicaciones del principio de la gravitación universal.


%\asteriscos

Pero antes que Laplace recogiese los materiales para componer su \textit{ Tratado de mecánica celeste}, un poco antes, podemos decir, de haberse emancipado de la esclavitud de Descartes que ella había voluntariamente adoptado, Francia había encontrado en d'Alembert una persona digna de continuar su gloriosa tradición matemática. Enunciando el principio que aún lleva su nombre, enseñó la manera de poner en ecuación cualquier problema de dinámica, y así fué como suministró a Lagrange una de las columnas en que se apoya su \textit{ Mecánica analítica}, siendo la otra el «principio de las velocidades virtuales».

D'Alembert, nacido de los amores clandestinos de una noble dama con un oficial francés, sintió duramente las consecuencias de un nacimiento irregular y, mejor que nadie, estuvo en situación de medir y advertir las consecuencias del estado de humillante sujeción en que, por entonces, se encontraba el humilde pueblo. Mentalidad abierta a todas las más audaces corrientes del pensamiento, prestó complaciente oído a las vivas protestas que hicieran estallar en toda Francia los escándalos del gobierno de Luis {XV; y aliándose con Diderot para la compilación y publicación de la célebre \textit{ Encyclopédie méthodique}, preparó, y en parte promovió el gran movimiento político y social que debería regenerar no solamente a Francia, sino que también a la Europa entera; esa gran sacudida que determinó una nueva orientación en la vida de la humanidad.

Es extraño a mi tarea el describir de nuevo la trágica época durante la cual, sobre las riberas del Sena, con el derramamiento de torrentes de sangre, quedó inaugurada una era nueva para la historia del mundo. Pero no me alejo de mi tema al hacer notar que, al sucederse en el poder, con fantasmagórica rapidez, partidos políticos cada vez más propensos al uso de medios violentos de renovación social, no desconociera y pisoteara los sacrosantos derechos de la cultura.

También es verdad que para el partido de los Jacobinos y para el régimen del Terror son manchas indelebles no menos de tres víctimas que pertenecían a la aristocracia de la inteligencia. Pero se debe recordar que, llevando al patíbulo a Lavoisier, no se desconocían o negaban los grandes beneficios que se debían al creador de la química moderna, sino que con ello se quería golpear a aquel «fermier général» que erigiera en torno de París las insalvables vallas aduaneras que motivaron violentas protestas, las que, puestas en música, hacían que el pueblo parisiense cantara:

\begin{verse}\it

Le mur mourant Paris, jait Paris murmurant\footnote{Juego de palabras que, traducido literalmente significa: «El muro del París moribundo, hace que París murmure»}.

\end{verse}

Por otra parte, si no valieran para salvar de la horca a Bailly las consideraciones de que la astronomía esperaba de él ulteriores progresos, se debe a que fué considerada una falta imperdonable, por él cometida, el hecho de que se lo encontrara en el cargo de Jefe de la administración civil de la Capital, justamente en el momento en que se produjera la violenta represión del 17 de julio de 1793.

Si, finalmente, Condorcet recurrió al veneno para escapar al oprobio de la guillotina, se debe a que fuera perseguido, no como el Antiguo Secretario Perpetuo de la Academia de las Ciencias, sino como uno de los cuarenta y un componentes del partido de los Girondinos, de los cuales querían deshacerse los Jacobinos triunfantes. Para demostrar cómo era reconocido su gran valor, en aquella época de perturbadas pasiones, tenemos la deliberación realizada por la Convención Nacional, pocos días después de su muerte, por la cual dispuso publicar, a expensas del Estado, para difundirlo por toda Francia, aquel bellísimo \textit{ Tableau des progrés de l'esprit humain}, que él había podido milagrosamente escribir cuando, oculto en una casa amiga, tentaba de substraerse a una cierta y segura muerte.

Para demostrar de qué manera las asambleas que gobernaron a nuestra vecina de Occidente después de la caída de la monarquía, eran conscientes de la necesidad de proveer a resolver los problemas culturales de la nueva Francia ---surgida de las cenizas humeantes de la Bastilla---, basta recordar que por esa época se dictaron las bases de dos grandes institutos de instrucción: la Escuela Normal y la Escuela Politécnica. A la segunda, y con toda razón, se la puede considerar como el más famoso vivero de hombres de ciencia del mundo entero, ya que se enorgullecen de haber sido alumnos de ella los más grandes representantes del campo científico que honraron a Francia durante el último siglo. Para asegurar eficacia a la enseñanza, el
gobierno de ese entonces sacó violentamente de los gabinetes donde meditaban a los más eminentes hombres de ciencia, obligándolos a difundir,
la cátedra, la verdad por ellos descubierta. Esto puede parecer carente de significado para nosotros, los italianos, ya que en nuestro país la investigación científica fué realizada, casi en su totalidad, por profesores universitarios; en efecto, tales fueron Galileo, Cavalieri, Volta y tantos otros a quienes se debe la fama conquistada por nuestros Ateneos. En vez, en Francia los más eminentes matemáticos del siglo XVII y gran parte del XVIII, no eran catedráticos. No era, en efecto, profesor Viete, quien, abandonando la profesión de abogado, que había elegido para entrar en la magistratura, llegó, durante el reinado de Enrique IV, a ocupar un asiento en el Consejo Privado del Reino; y durante la guerra de la Liga llegó a descifrar algunos criptogramas escritos por los españoles, quienes para tomar venganza lo denunciaron como brujo ante el Sagrado Colegio, pudiéndose salvar de la condena gracias a las altas protecciones que tenía. No
eran profesores Descartes y Pascal, a quienes su renta personal les permitió dedicar todo el tiempo
a las más altas especulaciones intelectuales. Un alto
magistrado era Fermat, y Desargues era ingeniero, y de tan grande renombre que Richelieu lo llevó consigo al asedio de la Rochelle, episodio que conocen todos, aun los que no están familiarizados con la	de Francia, porque el mago que
responde al nombre de Alejandro Dumas eligió las escarpas de las murallas de aquella ciudad como teatro de una de las más impresionantes gestas realizadas por los Cuatro de su Tres Mosqueteros. No pertenecieron a la enseñanza Maupertuis y D'Alembert; y los ejemplos podrían multiplicarse, descendiendo algunos grados en la escala de los valores. Pero estos ejemplos son más que suficientes para demostrar cuán valiosa y oportuna fué la decisión adoptada por el gobierno francés, en el momento en que el Tribunal revolucionario le daba trabajo
al verdugo.

Los efectos no tardaron en manifestarse bajo una forma escueta e impresionante. Monge, Laplace, Lagrange y Lacroix, llamados a profesar en la Escuela Politécnica, debieron reelaborar y difundir entre las masas métodos y resultados conocidos con el nombre de ellos; y así la Geometría Analítica adopta la forma que hoy todos le conocemos; y la Geometría Descriptiva, que hasta ese entonces se enseñaba exclusivamente a los doctos oficiales del ejército, como un secreto de Estado, pasó al dominio universal y pudo ser cultivada no sólo en todas las tierras que se extienden entre el Océano y el Rhin, sino que también por los pueblos que, en aquel momento, luchaban contra Francia por el trono y por el altar.

%\asteriscos

El espíritu de crítica desapasionada de toda la herencia del pasado que animaba a los enciclopedistas, y que fué llevado al paroxismo en Francia en el último ventenio del siglo XVIII, ejerció una benéfica influencia sobre toda la Tierra ---podría decirse---, aun en aquellos campos que parecían al reparo de cualquier discusión ofensiva. Lo prueba el hecho que también para Euclides, secular soberano de la Geometría, aparecieron días no alegres y, a semejanza de Luis XVI, encontró en Legendre a su Danton, en Bolyai a su Marat y en Lobacefski a su Robespierre; pero a diferencia con aquel desgraciado soberano, le fué evitada la vergüenza de tener que abandonar el poder, viendo solamente surgir junto a su reino otros dominios de mayor extensión, los cuales no tardaron en prosperar, influyendo sobre el antiguo reino para mejorar su estructura y hacer que su funcionamiento fuera más racional.

La misma tendencia, radicalmente innovadora, mientras conducía a la creación de un nuevo y excelente sistema de pesas y medidas a base decimal, ejerció una benéfica influencia sobre el análisis matemático, actuando en un sentido que podríamos razonablemente equiparar al pasaje del régimen absoluto de gobierno al sistema constitucional. Y en efecto, mientras que en el descuidado siglo XVIII se creía que el poder de las fórmulas no conocía limitación alguna, Cauchy y Abel hicieron triunfar el principio de que toda fórmula posee un determinado campo de acción, que debe asignársele en cada caso. Así terminó la época del análisis libre, marcado por la presencia de relaciones privadas de sentido, como aquellas que ya he tenido ocasión de mencionar, y comenzó un gobierno regido por un estatuto que nadie tiene el derecho de violar sin que se le declare, por ello, fuera de la ley. Y es de hacer notar que éste nuevo estado de cosas, que podía suponerse deletéreo para el análisis, haciendo imposible los más audaces vuelos, resultó, en cambio, sumamente saludable, ya que los matemáticos fueron inducidos a investigaciones de gran profundidad y de inigualable finura, las cuales, por otra parte, llevaron a entes analíticos de tan difícil concepción ---aludo a las curvas continuas privadas de tangentes o capaces de llenar toda un área--- que no se podrían haber obtenido si se continuaba en considerar una función como el representante analítico de cualquier curva trazada por la mano libre del dibujante.


Además, la obligación de determinar, para cualquier serie, el campo de convergencia, impuso una metódica revisión de las que estaban en uso, y en particular de todo el aparato analítico sobre el cual se alzaba la mecánica celeste. Como consecuencia de esto, la astronomía adquirió una perfección numérica que, antes, era locura esperar. De ahí la razón por la cual, hoy, jamás una estrella del firmamento faltará al convenio fijado por los astrónomos a decenios de distancia; de ahí, también,
la explicación del hecho de que por dos veces el cálculo previó la existencia de cuerpos celestes que ningún ojo humano había hasta entonces percibido, y buscados que fueron, por atentos observadores, en la región del cielo en la que la teoría había dicho que debían encontrarse, fueron finalmente descubiertos y recibieron los nombres de Neptuno y
Plutón.

Gracias a estas bases analíticas de granítica solidez, la astronomía pudo celebrar continuos triunfos y llegar a advertir la existencia y determinar la estructura de agrupamientos estelares situados a las inconcebibles distancias de 140 millones de años luz. Y así es lector que, al poseer estos admirables resultados, después de un sentimiento de confusión, nos sentimos penetrados de un sentimiento mixto de humillación y orgullo; de humillación al reconocer que el hombre es una pequeñísima partícula de polvo en la inmensidad del universo, siendo toda la historia de la humanidad desarrollada en un instante si la comparamos al período de tiempo empleado por ciertos rayos luminosos para llegar hasta nuestros ojos; de orgullo, cuando constatamos que este átomo pensante, dispersado en la inmensidad de los cielos, ha llegado a contemplar un espectáculo que, por su grandiosidad, supera a aquellos que pudieran ofrecer los poetas de fantasía más desenfrenada y potente, espectáculo ignoto e inconcebible hasta para una mente privilegiada como la de Josué, Carducci, que aconsejaba una estéril indiferencia
frente a los grandes fenómenos de la naturaleza, escribiendo:


\begin{verse}


Es mejor olvidar, sin indagarlo,\\
 este enorme misterio del universo.

\end{verse}

%\asteriscos


Los grandes progresos realizados por la geometría y por el análisis, de los que acabamos de hacer una rápida síntesis, son tantos que bastarían para ilustrar un siglo entero; pero no son los únicos que encontramos recorriendo el afortunado período que abarca la revolución francesa, el consulado y el imperio.


En una época en que la vida de los ciudadanos estaba suspendida de un hilo, bastando para ser llevado a la guillotina la acusación de «tibio civismo», formulada por cualquier calumniador oculto entre las sombras; en una época en la que la insaciable sed de poder de Napoleón hizo derramar ríos de sangre por todos los ángulos de Europa; se podía creer que a la humanidad le fuera imposible dedicarse a la más abstracta y menos práctica de las ciencias del razonamiento. Los hechos se han encargado de desmentir estas previsiones; porque pareciera que en aquella época el fervor de innovar y progresar, partiendo de las salas gubernativas y descendiendo a las
calles y plazas, llegara a penetrar en los más remotos gabinetes de estudio, asediando e impeliendo aun a aquellos investigadores menos propensos a dejarse arrastrar por el torbellino revolucionario. A las pruebas dadas sobre este sorprendente fenómeno, podríamos agregar muchas otras, y no menos impresionantes, como las que ofrecen aquellas teorías que pretenden explicar los fenómenos físicos.

Recordemos que, a tal fin, Lagrange en su célebre
\textit{ Mécanique analytique} ha enseñado a someter bajo
la férrea ley del cálculo los más simples hechos naturales, esto es, aquellos en los que intervienen exclusivamente los movimientos abstractos y las fuerzas, entidades metafísicas que todos creemos conocer hasta el momento en que se nos pide su definición. Ahora bien, era natural que, siguiendo en esa dirección, se pretendiera hacer otro tanto con otros fenómenos.

El primero en realizarlo fué un prefecto del Imperio, que logró mantenerse a salvo aun después de la Restauración: José Fourier. Este enseñó a investigar, por medio del análisis, los fenómenos de la propagación del calor y, habiendo conseguido tal fin, enriqueció el análisis con un nuevo y potente instrumento: las series trigonométricas.

Hacia el mismo tiempo el estudio de los fenómenos luminosos es iniciado con un fervor semejante al éxito obtenido. Siendo ---al menos con una gran aproximación---, los rayos luminosos rectilineos, el estudio de los sistemas por ellos formados podía afrontarse con los medios ofrecidos por la geometría pura; y, en efecto, Malus y Dupin, trabajando desde este punto de vista, llegaron a teoremas que se refieren a la reflexión y refracción de la luz, que no tardaron en ser clásicos. Prosiguiendo en el mismo camino, un gran matemático irlandés, Hamilton, llegaba a la inesperada conclusión de que hay casos en que un rayo de luz, refractándose, da lugar a un cono entero de rayos. Frente a este resultado obtenido por el cálculo, los físicos, durante cierto tiempo, titubearon y adoptaron una postura escéptica, hasta que un hábil experimentador, demostrando su exactitud, los obligó a admitir la «refracción cónica» como uno de los fenómenos indiscutibles ofrecidos por la propagación de la luz. No es éste el único hecho que demuestra como la óptica, que hasta entonces se mantenía en pañales, hacia el primer cuarto del siglo XIX, haya llegado a ser una ciencia madura y vigorosa. Es entonces que Fresnel somete al cálculo los fenómenos de interferencia y difracción, cuya teoría presentaba, todavía, imperfecciones y lagunas. La exactitud de las conclusiones doctrinales a que había llegado tuvieron una espléndida confirmación el día en que las demostró experimentalmente, en un caso simple que se le sugiriera como de fácil control tomado de aquellos cuyo examen le había confiado la Academia de Ciencias de París.

El éxito obtenido por la aplicación de la geometría y del análisis a los fenómenos ofrecidos por la luz y por el calor, dió fuerzas para hacer otro tanto con respecto a un tercer agente físico, la electricidad, el cual, gracias a los trabajos de Volta y Coulomb, durante los primeros años del siglo pasado, comenzaba a atraer la atención de los más eminentes investigadores. El primero en probar tal aplicación fué Ampére, y su éxito está documentado por el renombre que aun lo rodea después de un siglo de su muerte.
Estos tres órdenes de estudios físico-matemáticos, contemplados en su conjunto por un profesor de gran valor, Poisson, y manejados por él, dieron lugar a una nueva y gran provincia del imperio de las ciencias exactas, la física matemática ---física como nombre, matemática como adjetivo---, en cuyo ejercicio se ilustraron, después, hombres de ciencia ingleses, franceses, italianos y alemanes.

Esta estrecha unión entre las dos nobles disciplinas fué fecunda de admirables resultados para ambos contrayentes. Si la física obtuvo sólidos gemelos para guiar su curso, la matemática enfrentó nuevos problemas para resolver los cuales fué necesario crear procedimientos algorítmicos completamente nuevos. Además, demostrando la perfecta adhesión con la realidad que tenían los métodos característicos de las matemáticas, acrecentó su crédito entre aquellos que son propensos, fácilmente, a juzgar como estériles y vanos los esfuerzos que constituyen nuestra cotidiana labor.


Séame lícito agregar que para establecer el pleno acuerdo que existe entre los resultados del puro razonamiento y el mundo en el que tenemos la discutible prerrogativa de vivir, no es necesario recurrir a experimentos de física, bastando para ello la geometría pura, y para demostrarlo me permitiré citar un ejemplo que me es particularmente caro. Se trata de un problema elemental de geometría descriptiva. Una serie de consideraciones puramente teóricas conduce a la determinación de seis puntos, los cuales ---de acuerdo a la teoría---, deben encontrarse tres a tres sobre dos rectas debiendo éstas ser, entre sí, paralelas. Y bien, esta figura, ejecutada centenares de veces, con la condición de que sea trazada con la debida precaución, conduce a un resultado que, plenamente, satisfizo a estas múltiples condiciones.

La perfecta adhesión de la matemática con la realidad, documentada por los testimonios que provee la astronomía, la física, y la geometría, muestra lo mal que cuidaba su propio renombre Poincaré, cuando aseguró que los postulados de la matemática no son ni verdaderos ni falsos, sino simplemente cómodos; adjetivo que puede satisfacer la legítima ambición de un constructor de sillas de reposo, pero que es una ofensa cuando se la aplica a una ciencia fruto de las meditaciones de los más altos intelectos, de los que se siente orgulloso el género humano, Ciertamente, si él, con esa frase poco feliz, quiso
afirmar que los postulados fundamentales de nuestra ciencia fueron dictados por la aspiración de establecer una concordancia perfecta entre los fenómenos que caen bajo el control de nuestros órganos sensoriales y las conclusiones a las cuales llega nuestro pensamiento en su libre acción, no hizo más que afirmar que la matemática es, no una consecuencia de una revelación divina, sino obra esencialmente humana. Pero debió haber agregado que ella ofrece vehículos de absoluta seguridad a los más audaces aviadores del espíritu, gracias a los cuales nuestra mente puede llegar a las más altas regiones que le sean permitidas, porque, como lo dice Pascal, «Todo aquello que está más allá de la geometría, nos sobrepasa». Por otra parte, el célebre pensador francés cuando leyó los inesperados y no deseados apoyos obtenidos de parte de sistemáticos denigradores de las ciencias exactas, se esforzó en limitar el alcance de su frase, porque advirtió que el severo juicio que había pronunciado podría conducir a una desvalorización de las obras destinadas a asegurarle la inmortalidad.

%\asteriscos

A estos sucesos, tan importantes para nuestra ciencia, que se verificaron, en su mayor parte, en Francia, se opone otro, no menos decisivo que se manifestó a principios del siglo pasado en Alemania. 
Se debe creer que, en los siglos anteriores, esta tierra haya quedado extraña e indiferente a las investigaciones matemáticas; porque, si en el siglo XVI se puede hablar de «una matemática italiana», si luego encontramos en la historia un capítulo sobre la «matemática francesa» y luego otro sobre la «inglesa», no podemos razonablemente hablar de una verdadera «matemática alemana», en una época precedente a la nuestra. Prueba de ello es que Leibnitz, el más grande de los cultores de las ciencias exactas que haya aparecido, en tiempos pasados, en la tierra de Arminius, encontró, no en su patria, sino en Suiza, gente que lo ayudara en su obra de creación de un nuevo análisis, como también, aliados para la lucha que él sostenía para defender sus derechos de paternidad.

Pero en los primeros años del siglo XIX comenzó, para Alemania, una enérgica acción de recuperación, gracias a la aparición sobre la escena de aquel que desde el comienzo fué llamado «princeps mathematicorum»: hablo de Carlos Federico Gauss. Teoría de los números, mecánica celeste, análisis, física matemática, geometría, geodesia, todo lo abrazó con su mente soberana y a todas estas disciplinas les dió perfeccionamientos de la más alta rigurosidad. Reacio a entregar a la imprenta cosa alguna que no fuera perfecta en todas sus partes, la totalidad de sus descubrimientos fué recién conocida muchos años después de su muerte, cuando sus admiradores y discípulos hurgaban entre sus papeles, continuando en esas fatigosa tarea, porque eran premiados con el hallazgo de nuevos e importantes tesoros. Hoy, en todos los campos por él cultivados, aparece como «maestro y amo» de una gran cohorte de valerosos discípulos que aseguraron a la nación alemana un puesto en la historia de las matemáticas, no inferior al ocupado por las otras grandes
naciones europeas.

Si, sin embargo, ni durante el siglo XIX se puede hablar de una «mate\-mática alemana», ello se debe a que nuestra ciencia a asumido un carácter de internacionalidad que, ni la reciente conflagración mundial, ha podido hacerle perder, porque se reconoce que a tal carácter se debe, en gran parte, su maravilloso desarrollo y continuado éxito; carácter que se ha ido acentuando continuamente cuando las ciencias exactas encontraron expertos cultores, no sólo en Europa, sino, sucesivamente, en los Estados Unidos de América, Japón y hasta en las riberas del sagrado Ganges. Gracias a esta férvida e incesante labor, el patrimonio matemático asumió proporciones tales que la hace parecer como multimillonaria a la par de las otras disciplinas.

Lo prueba un balance de todas las obras matemáticas impresas realizado por un infatigable bibliotecario alemán ---G. H. Valentín---, quien dedicó a este trabajo cuarenta y dos años de intensa labor; y aún cuando excluyera de su lista a los tratados elementales y todos los problemas propuestos y resueltos, había llegado, en el año 1926, cuando le
alcanzó la inexorable muerte, ha recoger no menos de doscientas mil cuartillas y si hubiera encontrado un editor ---y ante la mole de papeles ello determinó una fuga general de editores--- su Catálogo habría ocupado buen número de gruesos volúmenes. Si se la pusiera al día, esa colección llenaría muchos más, ya que el ritmo de la producción matemática ha ido acelerándose constantemente: para demostrarlo basta citar el hecho de que el volumen I del más reputado periódico bibliografía matemática, aparecido en 1868, consigna una lista de cerca de 1200 trabajos, mientras que, el volumen XLVIII, dedicado al período 1921-1922, contiene cerca de seis mil. Verdad es que no todos los trabajos, de los cuales se conservan noticias, merecen ocupar un puesto estable en la ciencia, ya sea porque muchos de ellos son nuevas ediciones, tratados escolásticos y ensayos para conquistar una cátedra, y que otros se podrían utilizar para demostrar la existencia del vacío torricelliano; pero, aun separando a éstos, queda una rica y brillante colección de escritos, capaz de sostener el orgullo de cualquier rama de la ciencia.


%\asteriscos

Pese a la gran variedad de los campos cultivados por los matemáticos y a la independencia de los cultivadores de una ciencia que se caracteriza ---como muy bien lo dice Jorge Cantor--- por su absoluta libertad, aquel que siga su evolución más que milenaria, no tarda en reconocer en ella una constante directiva ejercida sobre su perenne tendencia de alejarse de lo concreto para elevarse a lo abstracto. Del número, creado por el humilde pastor para proceder al balance de su rebaño, se ha pasado, gradualmente, a la idea de número entero, de este, realizando generalizaciones  sucesivas, se llega a números fraccionarios, a los negativos, a los imaginarios, hasta llegar a los números complejos a varias unidades y al transfinito de Cantor.

Por la otra parte, de las simples figuras que consideraban los agrimensores y arquitectos, se llega, paso a paso, al estudio de los entes geométricos que se definen, exclusivamente, mediante ecuaciones y que, ninguna mano de dibujante, por experto que sea, sería capaz de reproducir sobre el papel. Más aun: de los modestos polinomios, por mucho tiempo único material de trabajo y estudio para los algebristas, se ha llegado, lentamente, al concepto general de función y luego a los entes analíticos, de los cuales nuestra mente es incapaz de formarse una imagen, así sea solamente aproximada. ¿Y qué decir del gran esfuerzo realizado por los matemáticos, cuando, rebelándose del vínculo impuesto por la naturaleza de ocuparle exclusivamente de figuras situadas en el plano y en el espacio, llegaron a razonar sobre espacios de cuantas dimensiones se quiera, hasta de infinitas dimensiones.
Podría creerse que esta tendencia a actuar en ambientes cada vez más lejanos del de nuestro pequeño mundo, hiciera que la matemática fuera inepta para prestar eficaces servicios a los investigadores de los fenómenos naturales. Y al contrario, aquellos que indagan lo que sucede en el microcosmos, requieren de la matemática incesantes servicios, lamentándose, solamente, que ella no esté, todavía, en condiciones de satisfacerlos plenamente.


%\asteriscos


Todo esto demuestra que el colosal Catálogo redactado por Valentín, no representa el balance de una hacienda próxima a ser clausurada, sino un simple recuento momentáneo de un floreciente negocio que se encuentra en un maravilloso desarrollo. Ya que, y pese a que se repita con demasiada complacencia, las condiciones económicas, sociales y políticas del mundo obstaculizan, más que favorecen, la desapasionada investigación de la verdad, la producción matemática se presenta, hoy, en continuo y confortante incremento, los valiosos operarios no faltan, y las ocasiones de destacarse son innumerables. Existen, en efecto, desde hace siglos, problemas que, hasta ahora, resisten los esfuerzos de los calculistas y enunciados que, también desde hace siglos, esperan la definitiva sentencia que les permita asumir el grado de verdad definitivamente adquirida o los haga ser expulsados de nuestra ciencia.

Además, gracias a una especial prerrogativa de las matemáticas, las provincias que, desde hace siglos, constituyen el imperio geométrico, no dan ningún signo de agotamiento, produciendo sin cesar nuevas plantas de bellos y gustosos frutos; valga, para probarlo, la venerable teoría de las secciones cónicas la cual, en manos de investigadores emprendedores continúa dando pruebas de inexhauta fecundidad. También las más simples figuras geométricas, como el triángulo y el tetraedro, asiduamente estudiados, dieron como resultado una serie de trabajos que llegaron a constituir una surtida y numerosa biblioteca, la cual, día a día va ampliándose. No faltan, tampoco, territorios inexplorados y vírgenes, en las regiones más elevadas de nuestra ciencia. Así, por ejemplo, las geniales ideas de Jorge Cantor, influyendo con un sentido profundamente reformador en todas las ramas del análisis, dieron tales y tantos frutos, que un periódico fundado en Varsovia, para recoger tales escritos, llegó, en doce años, al volumen XIX, y todavía continúa publicándose. A la par de ésto, la geometría proyectiva, que alguno juzga como una disciplina que ya ha cumplido su misión histórica, refugiándose bajo las grandes alas de la geometría infinitesimal, ha impreso a esta rama de nuestra ciencia una nueva vida poderosísima. Finalmente, la crisis de desarrollo por la que actualmente pasa la Física, obligó a afinar ciertos antiguos procedimientos algorítmicos y crear otros nuevos.

Mientras que el gran edificio fundado por los más grandes geómetras de la antigua Grecia se iba, constantemente, ampliando y elevándose, no faltó quien se ocupase de examinar con rigor sus cimientos, para analizar sus elementos y asegurarse de su solidez; trabajo, éste, muy delicado en el que participaron, también filósofos, psicólogos y lógicos y que dieron como último resultado teorías tan generales y abstractas que no se sabe bien qué lugar deben ocupar en el conocimiento humano; pero, ¿qué importa?, si ellos constituyen una grada más en la ascensión de la humanidad.

Por todas estas razones nos es hoy completamente desconocido aquel sentido de desconsolada tristeza que inducía a Lagrange, ya anciano, a equiparar a las matemáticas con una mina ya agotada y que debería ser abandonada, claro está, siempre que no se descubrieran nuevos filones auríferos; y estos filones no tardaron en ser descubiertos en gran número, precisamente siguiendo las huellas trazadas por aquellos geniales descubridores y sus más hábiles herederos, resultando pródigo su rendimiento del más noble metal.




\end{document}