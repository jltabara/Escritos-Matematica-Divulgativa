\documentclass[a4paper, 12pt]{article}

%%%%%%%%%%%%%%%%%%%%%%Paquetes
\usepackage[spanish]{babel}  
\usepackage[utf8]{inputenc}
\usepackage{tcolorbox}
\usepackage{cmbright}  %%%%%%% El tipo de letra
\usepackage{setspace}
\onehalfspacing  %%%%%%%%%%% Espacio y medio de interlineado
\parskip=1em  %%%%%%%%%%%% Separacion entre parrafos
%%%%%%%%%%%%%%%%%%%%%%



%%%%%%%%%%%%%%%%%
\title{La Evolución de la Imagen del Físico de la Naturaleza}
\author{P. A. M. Dirac}
\date{}
%%%%%%%%%%%%%%%%%

\begin{document}

\begin{tcolorbox}[colback=blue!5!white,colframe=blue!75!black]

\vspace{-1.8cm}
\textbf \maketitle

\end{tcolorbox}

\bigskip


En este artículo quisiera discutir el desarrollo general de la teoría de la física. Cómo se ha desarrollado en el pasado y cómo se puede esperar que se desarrolle en el futuro. Se puede considerar este desarrollo continuo como un proceso evolutivo, un proceso que ha ido teniendo lugar durante varios siglos.

El primer paso principal en este proceso de evolución fue realizado por Newton. Antes de Newton se consideraba el mundo como esencialmente bidimensional. Las dos dimensiones en las cuales se puede andar y la dimensión de arriba abajo parecían ser  esencialmente diferentes. Newton mostró cómo se puede considerar la dirección de arriba abajo simétrica a las otras dos direcciones introduciendo las fuerzas gravitatorias y mostrando cómo ocupan su lugar en la teoría  de la física. Se puede decir que Newton nos capacitó para pasar de una imagen con simetría bidimensional a una imagen con simetría tridimensional.

Einstein dio otro paso en la misma dirección, mostrando cómo se puede pasar de una imagen con simetría tridimensional a una imagen con simetría tetradimensional. Einstein introdujo el tiempo y mostró que desempeña un papel que en muchos aspectos es simétrico al de las otras tres dimensiones espaciales. Sin embargo, esta simetría no es totalmente perfecta. Mediante la imagen de Einstein, uno es conducido a considerar el mundo desde un punto de vista tetradimensional, pero las cuatro dimensiones no son completamente simétricas. Existen algunas direcciones en la imagen tetradimensional que son diferentes de las otras, direcciones que se denominan direcciones nulas, a lo largo de las cuales se puede mover un rayo de luz. Por tanto, la imagen tetradimensional no es completamente simétrica. A pesar de  todo existe una buena porción de simetría entre las cuatro dimensiones. La única falta de simetría, en lo que concierne en las ecuaciones de la física, consiste en la aparición de un signo menos con respecto a la dimensión tiempo que no aparece en las otras tres dimensiones espaciales.

Tenemos, por tanto, el desarrollo de la imagen tridimensional del mundo a la imagen tetradimensional. El lector probablemente no se sentirá satisfecho con esta situación porque todavía el mundo aparece como tridimensional en su conciencia. ¿Cómo puede esto compaginarse con la imagen tetradimensional que Einstein requiere que el físico tenga?
Lo que aparece ante nuestra conciencia es realmente una sección tridimensional de la imagen tetradimensional. Hemos de tomar una sección tridimensional para representarnos lo que aparece ante nuestra conciencia en un tiempo dado. En otro tiempo posterior tendremos una sección tridimensional diferente. La tarea del físico consiste en gran parte en relacionar los sucesos en una de estas secciones con los sucesos de otra sección que se refiere a un tiempo posterior. Por tanto, la imagen con simetría tetradimensional no nos da la situación entera. Esto parece particularmente importante cuando se toman en consideración los desarrollos que han sido causados por la teoría cuántica. La teoría cuántica nos ha enseñado que debemos tener en cuenta el proceso de observación, y las observaciones de ordinario requieren que tengamos en cuenta las secciones tridimensionales de la imagen tetradimensional del universo.

La teoría especial de la relatividad, introducida por Einstein, requiere que pongamos todas las leyes de la física en una forma tal que resulte aparente su simetría tetradimensional. Pero cuando usamos estas leyes para obtener resultados sobre observaciones, hemos de introducir algo adicional a la simetría tetradimensional, es decir, las secciones tridimensionales que describen nuestra conciencia del universo en un cierto tiempo.

Einstein hizo otra contribución enormemente importante al desarrollo de nuestra imagen física: propuso la teoría general de la relatividad, que nos exige suponer que el espacio de la física es curvo. Antes de esto los físicos habían trabajado siempre con un espacio plano, el espacio tridimensional plano de Newton que se había extendido después al espacio tetradimensional plano de la relatividad especial. La relatividad general hizo una contribución realmente importante a la evolución de nuestra imagen física exigiéndonos llegar a un espacio curvo. Los requisitos generales de esta teoría indican que todas las leyes de la física han de ser formuladas en el espacio curvo tetradimensional y que deben exhibir simetría entre las cuatro dimensiones. Pero de nuevo, cuando intentamos introducir observaciones, como lo tenemos que hacer, y miramos la situación desde el punto de vista de la teoría cuántica, nos tenemos que referir a una sección de este espacio tetradimensional. Con el espacio tetradimensional curvo, cualquier sección que hagamos en él también es curva, porque en general no podemos dar sentido a una sección plana en un espacio curvo. Esto nos conduce a una imagen en la cual hemos de tomar secciones tridimensionales curvas en el espacio tetradimensional curvo y discutir las observaciones realizadas en estas secciones.


Durante los últimos años algunos han intentado aplicar las ideas cuánticas a la gravitación, así como a otros fenómenos de la física, y esto ha conducido a un desarrollo más bien inesperado, consistente en que cuando uno considera la teoría gravitacional desde el punto de vista de las secciones se encuentra con que existen algunos grados de libertad que desaparecen de la teoría. El campo gravitacional es un campo tensorial de diez componentes. Se encuentra entonces que seis de las componentes son ya adecuadas para describir cualquier cosa de importancia física y que las otras cuatro pueden ser eliminadas de las ecuaciones. Sin embargo, no se pueden escoger estas seis importantes componentes del conjunto completo de diez de modo que no se destruya la simetría tetradimensional. Así, si se insiste en preservar la simetría tetradimensional de las ecuaciones, no se puede adaptar la teoría de la gravitación a una discusión de mediciones en el modo que la teoría cuántica lo requiere sin verse forzado a una descripción más complicada de lo que es necesario para la situación física. Este resultado me ha conducido a dudar hasta qué punto es fundamental la exigencia tetradimensional en la física. Hace unas cuantas décadas parecía totalmente cierto que había que expresar el conjunto de la física en forma tetradimensional, pero ahora parece que la simetría tetradimensional no es  tan importante, puesto que la descripción de la naturaleza a veces resulta simplificada si uno se aparta de ella.

Quisiera ahora proceder a señalar las transformaciones que han sido ocasionadas por la teoría cuántica. La teoría cuántica es la discusión de cosas muy pequeñas y ha constituido el tema principal de la física durante los últimos sesenta años. Durante este período, los físicos han ido amontonando un gran cantidad de información experimental y desarrollado una teoría que corresponda a ella, y esta combinación de teoría y experiencia ha conducido a transformaciones importantes en la imagen del mundo del físico.

El cuanto hizo primero su aparición cuando Planck descubrió la necesidad de suponer que la energía de las ondas electromagnéticas puede existir solamente como múltiplos de una cierta 
unidad, dependiendo de la frecuencia de las ondas, a fin de explicar la ley de la radiación del cuerpo negro. Entonces Einstein descubrió la misma unidad de energía en el efecto fotoeléctrico. En este trabajo primitivo sobre la teoría cuántica se tenía simplemente que aceptar la unidad de energía sin ser capaz de incorporarla en una imagen física.

La primera imagen nueva que apareció fue la imagen de Bohr del átomo. Fue una imagen en la cual teníamos los electrones moviéndose en ciertas órbitas bien definidas y ocasionalmente haciendo un salto de una órbita a la otra. No podíamos imaginarnos cómo tendría lugar el salto. Teníamos que aceptarlo simplemente como una especie de discontinuidad. La imagen de Bohr del átomo iba bien solamente para ejemplos especiales, esencialmente cuando existía solamente un electrón, que era el importante para el problema que se consideraba. Así la imagen era una imagen primitiva e incompleta.

El gran progreso en la teoría cuántica llegó en 1925 con el descubrimiento de la mecánica cuántica. Este progreso fue realizado independientemente por dos personas, primero por Heisenberg y poco después por Schrödinger, que trabajaban desde puntos de vista diferentes. Heisenberg trabajó manteniéndose cercano a la evidencia experimental sobre los espectros que se estaba recopilando en aquel tiempo y encontró que la información experimental podría ser ajustada en un esquema que ahora se conoce como mecánica matricial. Todos los datos experimentales de la espectroscopía se ajustaban magníficamente en el esquema de la mecánica matricial, y esto condujo a una imagen completamente diferente del mundo atómico. Schrödinger trabajó desde un punto de vista más matemático, tratando de encontrar una teoría bella para describir los sucesos atómicos, y fue ayudado en esto por las ideas de De Broglie acerca de las ondas asociadas con las partículas. Fue capaz de extender las ideas de De Broglie y de lograr una ecuación bella, conocida como la ecuación de ondas de Schrödinger, a fin de describir los procesos atómicos. Schrödinger logró esta ecuación a base de  pensamiento puro, buscando alguna generalización bella de las ideas de De Broglie y no manteniéndose cercano al desarrollo experimental del tema de la forma en que lo hizo Heisenberg.

Podría contarles la historia que yo oí del mismo Schrödinger sobre cómo, cuando logró por primera vez la idea de su ecuación, inmediatamente la aplicó a la conducta del electrón del átomo de hidrógeno, y entonces obtuvo resultados que no estaban de acuerdo con los experimentos. El desacuerdo surgía porque en aquel tiempo no se conocía que el electrón tiene un spin. Eso, por supuesto, constituyó un gran disgusto para Schrödinger e hizo que abandonara el trabajo algunos meses. Entonces observó que si aplicaba la teoría en una forma más aproximada, no teniendo en cuenta los refinamientos requeridos por la relatividad, su trabajo estaba de acuerdo con la observación en este grado de aproximación tosca. Publicó su primer artículo solamente con esta aproximación tosca y de esa forma la ecuación de ondas de Schrödinger fue presentada ante el mundo. Después, por supuesto, cuando se encontró cómo tener en cuenta correctamente el spin del electrón, la discrepancia entre los resultados de la aplicación de la ecuación relativista de Schrödinger y los experimentos fue aclarada completamente.

Pienso que esta historia tiene como moraleja  que es más importante el encontrar belleza en una ecuación que el encontrarla ajustada al experimento. Si Schrödinger hubiera sido más confiado en su propio trabajo, hubiera podido publicar este trabajo algunos meses antes y hubiera podido publicar una ecuación más exacta. Esa ecuación se conoce ahora como la ecuación de Klein-Gordon, si bien fue realmente descubierta por Schrödinger, y de hecho fue descubierta por Schrödinger antes de descubrir su tratamiento no relativista del átomo de hidrógeno. Parece que si uno trabaja desde el punto de vista de obtener belleza en sus propias ecuaciones, y si se tiene realmente una intuición profunda, entonces está en una línea segura de progreso. Si no existe completo acuerdo entre sus resultados y sus experimentos no nos debemos  desanimar, porque la discrepancia puede muy bien deberse a características de menor importancia que no se toman propiamente en cuenta y que serán aclaradas con ulteriores desarrollos de la teoría.


Es así como la mecánica cuántica fue descubierta. Condujo a un cambio drástico en la imagen del mundo del físico, tal vez el más grande que nunca haya tenido lugar. Este cambio proviene de que tenemos que abandonar la imagen determinística que había sido siempre supuesta. Nos vemos conducidos a una teoría que no predice con certeza lo que va a suceder en el futuro, sino que nos proporciona información solamente acerca de la probabilidad de que ocurran diferentes sucesos. Este abandono de la determinación ha sido objeto de mucha controversia y algunos no lo consideran de su agrado. Einstein en particular nunca lo consideró así. Si bien Einstein fue uno de los grandes contribuyentes al desarrollo de la mecánica cuántica, con todo fue siempre hostil a la forma en que se desenvolvió la mecánica cuántica durante su vida y a la forma que todavía conserva.

La hostilidad que algunas personas tienen con respecto al abandono de la imagen determinística puede ser centrada en un artículo muy discutido de Einstein, Podolsky y Rosen, que trata de la dificultad que se encuentra en formar una imagen consistente que esté aún de acuerdo en sus resultados con las reglas de la mecánica cuántica. Las reglas de la mecánica cuántica están completamente determinadas. Se sabe cómo calcular resultados y cómo comparar los resultados de los cálculos con el experimento. Todo el mundo está de acuerdo con el formalismo. Funciona tan bien que nadie  puede opinar lo contrario. Pero, con todo, la imagen que nos hemos de formar detrás de este formalismo es objeto de controversia.

Me gustaría sugerir que no es necesacio preocuparse demasiado sobre esta controversia. Mi opinión  es que la etapa que la física ha alcanzado hasta el día actual no es la etapa final. Es solamente una etapa en la evolución de nuestra imagen de la naturaleza y deberíamos esperar que este  proceso de evolución continúe en el futuro, del mismo modo que la evolución biológica continúa en el futuro. La presente etapa de la teoría física es meramente una etapa de paso hacia las etapas mejores que tendremos en el futuro. Podemos estar completamente seguros de que habrá etapas mejores, simplemente por las dificultades que ocurren en la física de hoy.

Quisiera ahora dedicarme un poco a comentar las dificultades de la física del día presente. El lector que no es un experto en la materia tal vez pudiera tener la idea de que por razón de todas estas dificultades la teoría física se encuentra en bastante mal estado y que la teoría cuántica no es demasiado buena. Quisiera corregir esta impresión diciendo que la teoría cuántica es una teoría sumamente buena. Produce un acuerdo maravilloso con las observaciones en un gran campo de fenómenos. No existe ninguna duda de que es una buena teoría y la única razón de que los físicos hablen tanto acerca de las dificultades en esta teoría consiste en que estas dificultades son  sumamente interesantes. Los éxitos de la teoría son más bien considerados como algo debido. No se llega muy lejos simplemente afirmando y volviendo a afirmar los éxitos una y otra vez, mientras que hablando sobre las dificultades se puede esperar hacer algún progreso.

Las dificultades en la teoría cuántica son de dos clases. Las podría llamar dificultades de clase 1 y dificultades de clase 2. Las dificultades de clase 1 son las dificultades que ya he mencionado. Cómo se puede formar una imagen consistente con las reglas de la teoría cuántica presente. Estas dificultades de clase 1 no preocupan realmente al físico. Si el físico sabe cómo calcular resultados y compararlos con el experimento, se siente completamente satisfecho si los resultados concuerdan con sus experimentos, y eso es todo lo que necesita. Es solamente el filósofo, que desea tener una descripción satisfactoria de la naturaleza, quien se preocupa por las dificultades de clase~1.

Existen, además de las dificultades de clase 1, las dificultades de clase 2, que surgen del hecho de que las leyes presentes de la teoría cuántica no son siempre adecuadas para producir resultado alguno. Si se extrapolan estas leyes aplicándolas a condiciones extremas, los fenómenos que envuelven energías muy altas o distancias muy pequeñas, se llega a veces a resultados que son ambiguos o que no tienen sentido en absoluto. Entonces es claro que se han alcanzado los límites de aplicación de la teoría y de que se necesita un desarrollo ulterior. Las dificultades de clase 2 son importantes incluso para el físico porque imponen una limitación sobre la extensión en la que puede usar las reglas de la teoría cuántica para obtener resultados comparables con los experimentos.

Quisiera decir un poco más acerca de las dificultades de clase 1. Mi idea es que nadie se debería ocupar de ellas demasiado, porque son dificultades que se refieren a la presente etapa en el desarrollo de nuestra imagen física y casi ciertamente han de cambiar con el futuro desarrollo. Existe una fuerte razón, pienso, por la que se puede permanecer completamente confiado en que estas dificultades cambiarán. Existen algunas constantes fundamentales en la naturaleza: la carga del electrón (designada $e$) la constante de Planck dividida por $2\pi$ (designada por $\hbar$) y la velocidad de la luz ($c$). A partir de estas constantes fundamentales se construye un número que no tiene dimensiones: el número $\hbar c/e^2$. Este número se encuentra mediante experimentos que tiene el valor 137, o algo muy cercano a 137. Ahora bien, no existe ninguna razón conocida por la que debería tener este valor más bien que algún otro. Varias personas han propuesto ideas acerca de tal valor, pero no existe ninguna teoría acertada. Con todo, se puede estar bastante seguro que algún día los físicos resolverán el problema y explicarán por qué el número tiene este valor. Existirá una física en el futuro que funcionará cuando  tenga el valor 137 y que no funcionará cuando tenga otro valor.

La física del futuro, por supuesto, no puede tener las tres cantidades $\hbar$,
$e$ y $c$, todas ellas como cantidades fundamentales. Solamente dos de ellas pueden ser fundamentales y la tercera derivada de las otras  dos. Es casi cierto que $c$ será una de las dos fundamentales. La velocidad de la luz $c$ es tan importante en la imagen tetradimensional y desempeña un papel tan fundamental en la teoría especial de la relatividad, poniendo en relación nuestras unidades de espacio y tiempo, que ha de ser fundamental. Entonces nos encontramos confrontados con el hecho de que de las dos cantidades $\hbar$ y $e$ una será fundamental y una será derivada. Si $\hbar$ es fundamental, $e$ tendrá que ser explicada de alguna forma en términos de la raíz cuadrada de $\hbar$, y parece muy improbable que ninguna teoría fundamental pueda dar $e$ en términos de una raíz cuadrada, puesto que las raíces cuadradas no aparecen en las ecuaciones básicas. Es mucho más probable pensar que $e$ será la cantidad fundamental y que $\hbar$ será explicada en términos de $e^2$. Entonces no existirá ninguna raíz cuadrada en las ecuaciones básicas. Se me ocurre que se encuentra uno sobre terreno firme si conjetura que en la imagen física que nos encontraremos en algún futuro estadio $c$  y $e$ serán las cantidades fundamentales y $\hbar$ será derivada.

Si $\hbar$ es una cantidad derivada en lugar de ser una cantidad fundamental, nuestro conjunto entero de ideas acerca de la incertidumbre será cambiado: $\hbar$ es la cantidad fundamental que aparece en la relación de indeterminación de Heisenberg que relaciona la cantidad de indeterminación en una posición y en un momento. Esta relación de indeterminación no puede desempeñar un papel fundamental en una teoría en que $\hbar$ mismo no es una cantidad fundamental. Pienso que una conjetura segura puede ser que las relaciones de incertidumbre en su forma presente no sobrevivirán en la física del futuro.

Por supuesto que no habrá un retorno al determinismo de la teoría física clásica. La evolución no va hacia atrás. Ha de ir siempre hacia adelante. Existirá algún nuevo desarrollo que aparecerá como totalmente inesperado, acerca del cual no podemos hacer ninguna conjetura, que nos llevará aún más allá de las ideas clásicas, pero que alterará completamente la discusión de las relaciones de indeterminación. Y cuando este nuevo desarrollo ocurra, se encontrará más bien inútil el haber tenido una sola discusión sobre el papel de la observación en la teoría, porque se tendrá entonces un punto de vista mucho mejor desde el cual considerar las cosas. Así afirmo que si podemos encontrar alguna forma de describir las relaciones de indeterminación y la indeterminación de la mecánica cuántica actual que satisfaga  nuestras ideas filosóficas, nos podemos dar por contentos. Pero si no podemos encontrar tal modo de explicación, esto no constituye una causa para preocuparnos demasiado. Simplemente hemos de tener en cuenta que nos encontramos en un estadio de transición y que tal vez es completamente imposible en este estadio el obtener una imagen satisfactoria.




Nos hemos desecho de las dificultades de clase 1 afirmando que realmente no son tan importantes, que si se puede hacer algún progreso acerca de ellas nos podemos dar por contentos y que si no se puede no es necesario realmente preocuparse demasiado. Las dificultades de clase 2 son las realmente serias. Surgen primariamente del hecho de que cuando aplicamos nuestra teoría cuántica a ciertos campos en la forma en que tenemos que hacerlo si pretendemos concordarla con la teoría especial de la relatividad interpretándola en términos de las secciones tridimensionales que he mencionado antes, obtenemos ecuaciones que al principio parecen correctas. Pero cuando se trata de resolverlas, se encuentra que no tienen ninguna solución. En este punto deberíamos afirmar que no tenemos ninguna teoría. Pero los fisicos son muy ingeniosos acerca de esto, y han encontrado un modo de progresar a pesar de este obstáculo. Se encuentran con que cuando tratan de resolver las ecuaciones la dificultad estriba en que ciertas cantidades que deberían ser finitas son de hecho infinitas. Obtienen integrales que divergen en lugar de converger a algo definido. Los físicos han hallado que existe una forma de manejar estos infinitos según ciertas reglas que hacen posible obtener resultados definidos. Este método se denomina el método de la renormalización.

Explicaré someramente la idea por medio de palabras. Comenzamos con una teoría que concierne ciertas ecuaciones. En estas ecuaciones se tienen ciertos parámetros; la carga del electrón, $e$; la masa del electrón, $m$, y otras cosas de naturaleza semejante. Entonces se encuentra que estas cantidades que aparecen en las ecuaciones originales no son iguales a los valores medidos de la carga y de la masa del electrón. Los valores medidos difieren de éstos en ciertos términos correctores, $\Delta e$, $\Delta m$, etc., de tal forma que la carga total es $e +\Delta e$ y la masa total es $m+\Delta m$. Estos cambios de la carga y de la masa son ocasionados a través de la interacción de nuestra partícula elemental con otras cosas. Entonces se dice que $e +\Delta e$ y $m + \Delta m$, que son las cosas observadas, son lo importante. Las cantidades originales $e$ y $m$ son meramente parámetros matemáticos. Son inobservables y, por tanto, solamente herramientas que se pueden desechar cuando se ha ido bastante lejos para introducir las cosas que uno puede comparar con la observación. Esto sería una forma completamente correcta de proceder si $\Delta e$ y $\Delta m$ fueran correcciones pequeñas (o, incluso, si no fueran pequeñas, pero sí finitas). Según la teoría actual, sin embargo, $\Delta e$ y $\Delta m$ son infinitamente grandes. A~pesar de este hecho se puede todavía usar el formalismo y obtener resultados en términos de $e +\Delta e$ y $m + \Delta m$, que podemos interpretar afirmando que los originales $e$ y $m$ tienen que ser menos infinito de un orden adecuado para compensar los valores de $\Delta e$ y $\Delta m$ que son infinitamente grandes. Se puede
usar la teoría para obtener resultados que se pueden comparar con el experimento, en particular en electrodinámica. Lo sorprendente es que en el caso de la electrodinámica se obtienen resultados que están enormemente de acuerdo con los experimentos. El acuerdo se extiende a muchas cifras significativas, la clase de exactitud que previamente se tenía solamente en astronomía. Es por razón de este acuerdo tan bueno por lo que los físicos atribuyen algún valor a la teoría de la renormalización a pesar de su carácter ilógico.

Parece ser completamente imposible poner esta teoría sobre una base sólida matemática. Hubo un tiempo en que la teoría física estaba toda ella construida sobre matemáticas que eran inherentemente sólidas. No diré que los físicos han utilizado siempre matemáticas sólidas. A menudo usan pasos infundados en sus cálculos. Pero previamente cuando ellos lo hicieron así fue simplemente, se podría decir, por pereza. Deseaban obtener resultados tan rápidamente como fuese posible sin hacer  trabajos innecesarios. Era siempre posible para el matemático puro el acudir en auxilio de ellos y hacer la teoría sólida introduciendo pasos ulteriores y tal vez introduciendo una gran cantidad de notación pesada y de otras cosas que son deseables desde un punto de vista matemático, a fin de obtener todo expresado rigurosamente, pero que no contribuye a las ideas físicas. Las matemáticas primitivas han podido ser siempre hechas sólidas de esta manera, pero en la teoría de la renormalización tenemos una teoría que ha desafiado todos los intentos del matemático para hacerla sólida. Me inclino a sospechar que la teoría de la renormalización es algo que no sobrevivirá en el futuro y que el acuerdo notable entre sus resultados y el experimento debería ser considerado como una carambola.

Tal vez esto no es del todo sorprendente, porque han ocurrido carambolas similares en el pasado. De hecho, la teoría de Bohr sobre la órbita del electrón resultó dar un acuerdo muy bueno con la observación en tanto que uno se limitaba a considerar problemas con un solo electrón. Pienso que hay gente que dirá ahora que este acuerdo era una carambola, porque las ideas básicas de la teoría de la órbita de Bohr han sido superadas por algo radicalmente diferente. Yo creo que los éxitos de la teoría de la renormalización se consideran de un modo análogo a los éxitos de la teoría orbital de Bohr aplicada a los problemas de un electrón.

La teoría de la renormalización ha logrado deshacerse de algunas de las dificultades de clase 2 si es que se puede aceptar el carácter ilógico de descartar infinitos, pero no las desecha todas. Existen muchísimos problemas que quedan, referentes a partículas distintas de las que intervienen en la electrodinámica: mesones de varias clases y neutrinos. Aquí la teoría se encuentra todavía en un estado primitivo. Es bastante cierto que ha de haber cambios drásticos en nuestras ideas fundamentales antes de que estos problemas puedan ser resueltos.

Uno de los problemas es el que he mencionado ya de dar razón del número 137. Otros problemas consisten en cómo introducir la longitud fundamental en la física de algún modo natural, cómo explicar la relación de las masas de las partículas elementales y cómo explicar sus diversas propiedades. Creo que se necesitarán ideas distintas para resolver estos problemas diferentes y que serán resueltos uno a uno a través de etapas sucesivas en la futura evolución de la física. En este punto me encuentro en desacuerdo con la mayor parte de los físicos. Se inclinan a pensar que una idea magistral resolverá estos problemas a la vez. Yo pienso que es pedir demasiado el esperar que alguien sea capaz de resolver todos estos problemas juntos. Se deberían separar uno de otro tanto como sea posible y tratar de  resolverlos separadamente. Y yo creo que el futuro desarrollo de la física consistirá en resolverlos uno a uno, y que después de que cada uno de ellos haya sido resuelto será todavía un gran misterio el modo en que se deban atacar los restantes.

Pudiera tal vez discutir algunas ideas que se me han ocurrido acerca de cómo se pueden tal vez atacar algunos de estos problemas. Ninguna de estas ideas ha sido muy elaborada y no tengo mucha esperanza sobre ninguna de ellas. Pero pienso que vale la pena mencionarlas brevemente.

Una de estas ideas consiste en introducir algo correspondiente al éter portador de la luz que fue tan popular entre los físicos del siglo XIX. Decía antes que la física no marcha hacia atrás. Cuando hablo de reintroducir el éter no quiero decir que se deba volver a la imagen del éter que se tenía en el siglo XIX, sino lo que trato de indicar consiste en introducir una imagen nueva del éter que se pueda adaptar a nuestras ideas presentes sobre la teoría cuántica. La objeción a la idea antigua del éter fue que si se supone que es un fluido que llena el espacio entero, en cualquier lugar tiene una velocidad definida, lo que destruye la simetría tetradimensional requerida por el principio especial de la relatividad de Einstein. La relatividad especial de Einstein puso fin a esta idea del éter.



Pero con nuestra actual teoría cuántica no tenemos ya que señalar una velocidad definida a cada cosa física dada, porque la velocidad está sometida a relaciones de indeterminación. Cuanto más pequeña es la masa de aquello en lo que estamos interesados, más importantes son las relaciones de incertidumbre. Ahora bien, el éter tendrá ciertamente masa muy pequeña y así las relaciones de incertidumbre serán sumamente importantes para él. La velocidad del éter en algún punto particular no debería ser imaginada como definida porque estará sujeta a relaciones de indeterminación y, por tanto, puede tomar cualquier valor de entre una amplia gama de ellos. De esta forma podemos superar las dificultades existentes en reconciliar la existencia de un éter con la teoría especial de la relatividad.

Existe un cambio importante que esto causará en nuestra imagen del vacío. Tendemos a pensar en el vacío como en una región en la cual tenemos simetría completa entre las cuatro dimensiones de espacio-tiempo, como se requiere en la relatividad especial. Si existe un éter sujeto a relaciones de indeterminación, no será posible tener esta simetría exactamente. Podemos suponer que la velocidad del éter se encuentra con igual probabilidad dentro de una gama amplia de valores que darían la simetría sólo aproximadamente. No podemos llegar en ningún modo preciso hasta el límite de permitir todos los valores de la velocidad entre más y menos la velocidad de la luz, lo que deberíamos tratar de hacer, a fin de lograr una simetría exacta. Así el vacío resulta ser un estado que es inalcanzable. Yo no pienso que esto sea una objeción física a la teoría. Significaría que el vacío es un estado al que nos podemos aproximar tanto como queramos. No existe límite en lo que se refiere a la aproximación que podemos alcanzar, pero nunca lo podemos alcanzar. Pienso que esto sería totalmente satisfactorio para el físico experimental. Sin embargo, significaría una separación de la noción del vacío que tenemos en la teoría cuántica, donde comenzamos con el estado vacío considerándolo como un estado que tiene exactamente la simetría requerida por la relatividad especial.

Esto es una idea para el desarrollo de la física en el futuro que cambiaría nuestra imagen del vacío, pero que la cambiaría de un modo que no es inaceptable para el físico experimental. Ha resultado difícil llevar adelante la teoría, porque se necesitaría establecer matemáticamente las relaciones de incertidumbre para el éter y hasta ahora no ha sido descubierta ninguna teoría satisfactoria que vaya por esta línea. Si se pudiera desarrollar satisfactoriamente, daría lugar a una nueva clase de campo en la teoría física, lo que podría ayudar a explicar algunas de las partículas elementales.


Otra imagen posible que quisiera mencionar se refiere a la razón por la que todas las cargas eléctricas que son observadas en la naturaleza han de ser múltiplos de una unidad elemental, $e$. ¿Por qué no se tiene una distribución continua de carga en la naturaleza?

La imagen que propongo tiene su origen en la idea de las líneas de fuerza de Faraday e implica un desarrollo de esta idea. La líneas de fuerza de Faraday constituyen una forma de imaginar los campos eléctricos. Si tenemos un campo eléctrico en cualquier región del espacio, entonces según Faraday podemos trazar un conjunto de líneas que tiene la dirección del campo eléctrico. La proximidad de las líneas entre si da una medida de la fuerza del campo. Están próximas donde el campo es fuerte y menos próximas donde el campo es débil. Las líneas de fuerza de Faraday nos proporcionan una buena imagen del campo eléctrico en la teoría clásica.

Cuando pasamos a la teoría cuántica, introducimos una especie de carác\-ter discreto en nuestra imagen básica. Podemos suponer que la distribución continua de las líneas de fuerza de Faraday que tenemos en la imagen clásica es reemplazada por solamente unas pocas líneas de fuerza discretas con ninguna línea de fuerza entre ellas.

Ahora bien, las líneas de fuerza en la imagen de Faraday acaban donde existen cargas. Por tanto, con estas líneas de fuerza cuantizadas de Faraday sería razonable suponer que la carga asociada con cada línea, que ha de estar al final si la línea de fuerza tiene un fin, es siempre la misma (con excepción de su signo) y es siempre precisamente la carga electrónica, $- e$ o $+e$. Esto nos conduce a una imagen de líneas de fuerza discretas de Faraday, cada una asociada con una carga $- e$ o $+ e$. Existe una dirección adaptada a cada línea, de modo que los extremos de una línea que tiene dos extremos no coinciden y existe una carga $+ e$ en un extremo y una carga $- e$ en el otro. Podemos tener líneas de fuerza que se extienden hasta el infinito, por supuesto, y entonces no existe ninguna carga.
Si suponemos que estas líneas de fuerza discretas de Faraday son algo básico en física y subyacen en el fondo de nuestra imagen del campo electromagnético, tendremos una explicación de por qué las cargas aparecen siempre como múltiplos de $e$. Esto sucede porque si tenemos una partícula con algunas líneas de fuerza que terminan en ella, el número de estas líneas ha de ser un número entero. De esa forma obtenemos una imagen totalmente razonable.

Suponemos que estas líneas de fuerza pueden moverse. Algunas de ellas, formando lazos cerrados o simplemente extendiéndose desde menos infinito a más infinito, corresponderán a las ondas electromagnéticas. Otras líneas tendrán extremos y los extremos de estas líneas serán cargas. Podemos tener una línea de fuerza que se quiebra a veces. Cuando eso sucede, tenemos dos extremos que aparecen y han de existir cargas en los dos extremos. Este proceso, la ruptura de una línea de fuerza, sería la imagen para la construcción de un electrón ($e^-$) y un positrón ($e^+$). Sería una imagen completamente razonable, y si se pudiera desarrollar, proporcionaría una teoría en la cual $e$ aparece como una cantidad básica. Aún no he encontrado ningún sistema razonable de ecuaciones del movimiento para estas líneas de fuerza, y así me limito a proponer la idea como una posible imagen  física que pudiéramos tener en el futuro.

Existe un rasgo muy atractivo de esta imagen. Cambiará completamente la discusión de la renormalización. La renormalización que tenemos en nuestra electrodinámica cuántica actual procede de comenzar con lo que la gente llama un electrón desnudo, un electrón sin carga. En cierto estadio de la teoría se introduce la carga y se la pone sobre el electrón, haciendo así que el electrón interactúe con el campo electromagnético. Esto produce una perturbación en las ecuaciones y origina un cambio en la masa del electrón, el $\Delta m$, que ha de ser añadido a la masa previa del electrón. El procedimiento es más bien indirecto porque comienza con el concepto antifísico de un electrón desnudo. Probablemente en la imagen física mejorada que tengamos en el futuro el electrón desnudo no exista en absoluto.

Ahora bien, esa situación es precisamente la que tenemos con las líneas discretas de fuerza. Nos podemos imaginar las líneas de fuerza como cordeles, y entonces el electrón de la imagen es el extremo de un cordel. El cordel mismo es la fuerza de Coulomb alrededor del electrón. Un electrón desnudo significa un electrón sin fuerza de Coulomb alrededor de él. Eso resulta inconcebible con esta imagen, de la misma forma que es inconcebible
pensar en el extremo de una pieza de cordel sin pensar en el cordel mismo. Esta, pienso yo, es la clase de método con el que deberíamos ensayar y desarrollar nuestra imagen física, introducir ideas que hacen inconcebibles las cosas que no deseamos tener. Aquí de nuevo tenemos una imagen que aparece como razonable, pero yo no he encontrado las ecuaciones adecuadas para desarrollarla.

Podría mencionar una tercera imagen con la cual he venido trabajando últimamente. Implica una separación de la imagen del electrón como un punto y el pensar en él como una especie de esfera de tamaño finito. Por supuesto, es una idea totalmente antigua el imaginar al electrón como una esfera, pero previamente se tenía la dificultad de analizar una esfera que está sometida a la aceleración y al movimiento irregular. Tal esfera aparecerá distorsionada y, ¿cómo se podrán tratar estas distorsiones? Propongo que se debería permitir que el electrón tuviese, en general, una forma y tamaño arbitrarios. Existirán algunas formas y tamaños en los cuales tiene menos energía que en otros. Y tenderá a asumir una forma esférica con un cierto tamaño en el cual tiene un mínimo de energía.

Esta imagen del electrón generalizado ha sido estimulada por el descubrimiento del meson $\mu$ o muón, una de las nuevas partículas de la física. El muón tiene la propiedad sorprendente de ser casi idéntico al electrón, excepto en una cosa particular, a saber, que su masa es aproximadamente 200 veces más grande que la masa del electrón. Aparte de esta disparidad en la masa, el muón es notablemente semejante al electrón, teniendo en un grado sumo de exactitud el mismo spin y el mismo momento magnético en proporción a su masa que tiene el electrón. Esto conduce a la sugerencia de que el muón debería considerarse como un electrón excitado. Si el electrón es un punto, resulta extraordinariamente difícil el imaginarse cómo puede ser activado. Pero si el electrón es el estado más estable de un objeto de tamaño finito, el muón puede ser precisamente el estado próximo más estable en el cual ese objeto sufre una cierta clase de oscilación. Esta es una idea en la que he venido trabajando recientemente. Existen dificultades en el desarrollo de esta idea, en particular la dificultad de introducir el spin correcto.

He mencionado tres posibles modos en los cuales se pudiera conjeturar el desarrollo de nuestra imagen física. No hay duda alguna de que existirán otros que otras personas estarán ideando. Se espera que más pronto o más tarde alguien encuentre una idea que realmente se adapte y que conduzca a una evolución importante. Yo soy más bien pesimista acerca de ello y me inclino a pensar que ninguna de ellas será lo suficientemente buena. La evolución futura de la física básica, es decir, un desarrollo que resuelva realmente uno de los problemas fundamentales, tal como el introducir la longitud fundamental o calcular la razón de la masa, puede requerir algún cambio mucho más drástico en nuestra imagen física. Esto significaría que en nuestros intentos actuales por idear una nueva imagen física estamos poniendo a trabajar nuestra imaginación en términos de conceptos físicos inadecuados. Si es éste realmente el caso, ¿cómo podemos esperar hacer algún progreso en el futuro?

Existe otra línea por la cual se podría proceder por medios teóricos. Parece ser uno de los rasgos fundamentales de la naturaleza el que las leyes físicas fundamentales se describan en terminos de una teoría matemática de gran belleza y poder, necesitándose unas matemáticas enormemente elevadas para entenderla. Se puede uno preguntar: ¿por qué la naturaleza está construida a lo largo de estas líneas? Solamente se puede responder que nuestro conocimiento presente parece mostrar que la naturaleza está construida de esta forma. Lo único que podemos hacer es simplemente aceptarlo. Se puede describir tal vez la situación afirmando que Dios es un gran matemático y que usó matemáticas muy avanzadas al construir el universo. Nuestros débiles ensayos matemáticos nos capacitan para entender un poco del universo, y a medida que procedemos a desarrollar matemáticas más y más avanzadas, podemos esperar entender mejor el universo.

 Esta consideración nos proporciona otro camino por el cual podemos esperar hacer progresos en nuestras teorías. Estudiando matemáticas podemos esperar hacer conjeturas sobre la clase de matemáticas que intervendrán en la física del futuro. Una buena porción de gente se encuentra trabajando en la base matemática de la teoría cuántica tratando de entender la teoría mejor y de hacerla más potente y más bella. Si alguien puede dar con las líneas correctas a lo largo de las cuales se puede hacer este desarrollo, ello puede conducir a un progreso futuro mediante el cual la gente descubrirá primero las ecuaciones y después, examinándolas, aprenderá gradualmente cómo aplicarlas. Hasta cierto punto, eso corresponde a la línea de desarrollo que tuvo lugar con el descubrimiento de Schrödinger de su ecuación de ondas. Schrödinger descubrió la ecuación simplemente buscando una ecuación con belleza matemática. Cuando la ecuación fue descubierta por primera vez, la gente vio que se adaptaba en cierto modo, pero los principios generales según los cuales se debería aplicar fueron elaborados solamente dos o tres años más tarde. Muy bien pudiera ser que el próximo progreso de la física apareciera a lo largo de esta línea de desarrollo:
se descubrirán primero las ecuaciones y se necesitarán después unos pocos años de desarrollo, a fin de encontrar las ideas físicas tras estas ecuaciones. Mi propia opinión es que esto es una línea de progreso más probable que la de tratar de conjeturar imágenes físicas.

Por supuesto, puede suceder que, incluso esta línea de progreso falle y entonces la única línea que queda es la experimental. Los físicos experimentales continúan su trabajo de forma totalmente independiente de la teoría, coleccionando una gran cantidad de información. Más pronto o más tarde vendrá un nuevo Heisenberg que será capaz de escoger los rasgos importantes de esta información y de ver cómo usarla de una forma semejante a la forma en que Heisenberg utilizó el conocimiento experimental de los espectros para construir su mecánica matricial. Es inevitable que la física se desarrolle finalmente a lo largo de estas líneas, pero podemos tal vez tener que esperar un tiempo bastante largo si es que no se da con ideas brillantes para desarrollar el aspecto teórico.

\end{document}