\documentclass[a4paper, 12pt]{article}

%%%%%%%%%%%%%%%%%%%%%%Paquetes
\usepackage[spanish]{babel}  
\usepackage[utf8]{inputenc}
\usepackage{tcolorbox}
\usepackage{cmbright}  %%%%%%% El tipo de letra
\usepackage{setspace}
\onehalfspacing  %%%%%%%%%%% Espacio y medio de interlineado
\parskip=1em  %%%%%%%%%%%% Separacion entre parrafos
%%%%%%%%%%%%%%%%%%%%%%



%%%%%%%%%%%%%%%%%
\title{El Porvenir de las Matemáticas}
\author{H. Poincaré}
\date{}
%%%%%%%%%%%%%%%%%

\begin{document}

\begin{tcolorbox}[colback=blue!5!white,colframe=blue!75!black]

\vspace{-1.8cm}
\textbf \maketitle

\end{tcolorbox}

\bigskip

Para prever el porvenir de las matemáticas, el verdadero método es
estudiar su historia y su estado actual.

¿No es eso para nosotros, los matemáticos, un procedimiento un poco
profesional? Estamos acostumbrados a extrapolar, que no es más que un
medio de deducir el porvenir del pasado y del presente, y como sabemos lo
que vale, no corremos el peligro de ilusionarnos sobre el alcance de los
resultados que nos da.

En épocas anteriores han existido profetas de desgracias. Repetían,
convencidos, que todos los problemas susceptibles de ser resueltos lo habían sido ya, y que después de ellos lo único que haría
falta hacer es ponerse a cosechar. Felizmente el ejemplo del pasado nos
tranquiliza. Tantas veces hemos creído haber resuelto todos los
problemas, o por lo menos, haber hecho el inventario de los que implican una
solución. Más tarde el sentido de la palabra solución se ha
ensanchado, los problemas insolubles se han convertido en los más
interesantes de todos, se han planteado nuevos problemas en que ni siquiera
se había soñado. Para los griegos, una buena solución era la
que no empleaba más que la regla y el compás; en seguida ha sido la
que se obtiene por la extracción de radicales, después en la que no
figuran más que funciones algebraicas o logarítmicas. Los
pesimistas se encontraban así siempre desbordados, forzados a
retroceder, de manera que en el presente creo que no hay más.

Mi intención no es, por lo tanto, combatir los puestos que ya han dejado
de existir; sabemos que las matemáticas continuarán desenvolviéndose, pero se trata de saber en qué
sentido. Se me responderá: \guillemotleft en todos los sentidos\guillemotright, y esto es verdad en parte; pero si todo esto fuera
cierto, sería un poco espantoso. Nuestras riquezas no tardarían en
sernos embarazosas y su acumulación produciría un revoltijo tan
impenetrable que sería, como para el ignorante, la verdad desconocida.

El historiador, el físico, deben elegir entre los hechos. El cerebro
del sabio, que no es más que un rincón del Universo, no podrá
contenerlo jamás entero, de manera que en medio de los innumerables
hechos que la naturaleza nos ofrece, hay algunos de los que no nos
ocuparemos mientras que, por el contrario, estudiaremos otros. Sucede lo
mismo, {\it a fortiori}, en matemáticas; el matemático tampoco puede
conservar entremezclados todos los hechos que se le representan; tanto más cuanto que estos hechos lo representan a él mismo, perdón,
iba a decir que era su capricho el que los había creado. Es él
quien construye con todas las piezas una combinación nueva relacionando
los elementos; no es por lo general la naturaleza quien lo ha traído
todo hecho.

Ocurre a veces que el matemático se plantea un problema para satisfacer
una necesidad de la física; que el físico o el ingeniero le piden
que calcule un número a fin de aplicarlo. ¿Dirán que los geómetras debemos limitarnos a esperar los pedidos y, en lugar de cultivar
nuestra ciencia para nuestro placer, no tener otra preocupación más
que la de acomodarnos al gusto de nuestra clientela? Si las matemáticas
no tienen otro objeto más que acudir en ayuda de los que estudian la
naturaleza, es lógico que de estos últimos debemos esperar la
palabra de orden. ¿Es legítima esta manera de pensar? De ningún
modo; si no hubiéramos cultivado las ciencias exactas por ellas mismas,
no habríamos creado el instrumento matemático, y el día en que
hubiera venido la orden del físico nos hubiera encontrado desarmados.

Los físicos tampoco esperan para estudiar un fenómeno que cualquier
necesidad urgente de la vida material les haya creado una necesidad, y
tienen razón; si los sabios del siglo XVIII hubieran abandonado la
electricidad, puesto que no representaba para ellos más que una
curiosidad sin interés
práctico, no existirían en el siglo XX ni la telegrafía, ni la
electroquímica, ni la electrotécnica. Los físicos obligados a
elegir no van únicamente guiados por la utilidad. ¿Cómo hacen
entonces para escoger entre los hechos naturales? Los hechos que les interesan son los que pueden
conducir al descubrimiento de una ley; o sea los que son análogos a
muchos otros, y que no se nos aparecen aislados, sino estrechamente
agrupados con otros. El hecho aislado choca tanto al vulgo como al sabio.
Pero es que el físico solamente sabe ver el lazo que une varios hechos
en los que la analogía es profunda, pero oculta. La anécdota de la
manzana de Newton es probable que no sea verdadera, pero es simbólico;
hablemos entonces como si lo fuera. Debemos creer que antes que Newton hubo
muchos hombres que vieron caer manzanas: sin embargo, ninguno dedujo nada.
Los hechos serían estériles si no hubiera espíritus capaces de
escoger entre ellos, discerniendo aquellos detrás de cuales se oculta
alguna cosa y de reconocer lo que se oculta detrás, espíritus que
en el hecho bruto verán el alma del hecho.

En matemáticas hacemos más o menos lo mismo; de los diversos
elementos que disponemos, podemos hacer millones de combinaciones
diferentes, pero cada una, mientras esté aislada, está completamente
desprovista de valor. Nos ha costado a menudo mucho trabajo construirla,
pero esto no nos sirve absolutamente de nada, si no es para darnos un tema
de ejercicio para la enseñanza secundaria. Sucederá todo lo
contrario el día en que esta combinación entre en una clase de
combinaciones análogas y en la que habremos notado esta analogía;
no nos hallaremos más en presencia de un hecho, sino de una ley. Y ese día, el verdadero inventor no será el obrero que pacientemente haya
edificado alguna de estas combinaciones, será el que haya puesto en
evidencia su analogía. El primero no habría observado más que
el hecho, el otro habrá percibido el alma del hecho. A menudo para
afirmar esta analogía le habría bastado inventar una palabra nueva
y esta palabra habría sido creadora; la historia de la ciencia nos
provee de una cantidad de ejemplos que a todos nos son familiares.



El célebre filósofo vienés Mach dijo que el papel de la ciencia
es producir economía de pensamiento, de la misma manera que la máquina produce economía de esfuerzo. Y es muy justo. El salvaje cuenta
con los dedos, o juntando pequeñas piedras. Enseñando a los niños la tabla de multiplicar, les ahorraremos para más tarde innumerables
maniobras con piedras. Con piedras o de otra manera alguien ha comprobado
que 6 veces 7 son 42 y ha tenido la idea de anotar el resultado. Por esto es
por lo que no tenemos necesidad de volver a comenzar.

Éste no ha perdido su tiempo, aunque hubiera calculado nada más que
por placer; la operación no ha requerido más de dos minutos, pero
habría exigido dos mil, si mil hombres la hubieran vuelto a comenzar
después de él.

La importancia de un hecho se mide entonces por su rendimiento, es decir,
por la cantidad de pensamiento que nos permite economizar.

En física los hechos de gran rendimiento son los que entran en una ley
muy general, puesto que permiten prever un gran número de ellos; lo
mismo sucede en matemáticas. Me he dedicado a un cálculo complicado
y he llegado penosamente al resultado; no sería recompensado en mi
esfuerzo si no hubiera sido capaz de prever los resultados de otros
resultados análogos y dirigirlos con seguridad evitando los tanteos a
los que tuve que resignarme la primera vez. No habría perdido el tiempo
si estos mismos tanteos acabaran por revelarme la profunda analogía del
problema que he tratado con una clase mucho más extensa de otros
problemas; si me hubieran enseñado a la vez los parecidos y las
diferencias; si, en una palabra, me hubieran hecho entrever la posibilidad
de una generalización. No es un resultado nuevo el adquirido, sino una
fuerza nueva.

Una fórmula algebraica que nos da la solución de un tipo de
problemas numéricos, siempre que se reemplacen al final las letras por números, es el ejemplo simple que se presenta al espíritu. Gracias a
ella un solo cálculo algebraico nos ahorra el trabajo de recomenzar
constantemente nuevos cálculos numéricos. Pero esto no es más
que un grosero ejemplo; todo 
el mundo sabe que hay analogías que no pueden expresarse por una fórmula y que son las más valiosas.



Un resultado nuevo tiene valor cuando reúne elementos conocidos hace
mucho tiempo, pero dispersos hasta el punto de parecer extraños los unos
a los otros, e introduce de repente el orden donde reinaba el desorden. Nos
permite ver entonces en conjunto cada uno de estos elementos y el lugar que
ocupan en él. Este nuevo hecho no es solamente de gran valor por él
mismo, sino que él solo da valor a los viejos hechos que relaciona.

Nuestro espíritu es enfermizo como lo son nuestros sentidos; se perdería en la complejidad del mundo si esta complejidad no fuera armoniosa.
No vería los detalles sino lo mismo que un miope, y estaría
obligado a olvidar cada uno de ellos antes de examinar el siguiente, puesto
que sería incapaz de abarcarlos todos. Los únicos hechos dignos de
nuestra atención son los que introducen el orden en esta complejidad y
la tornan, por lo tanto, accesible.

Si los matemáticos atribuyen una gran importancia a la elegancia de sus métodos y de sus resultados, no es sólo por diletantismo. ¿Qué
es lo que nos da, en efecto, en una solución, en una demostración,
el sentimiento de la elegancia? Es la armonía de las diversas partes,
su simetría, su feliz equilibrio; es en una palabra todo lo que pone
orden, lo que da unidad y nos permite, por lo tanto, ver claro y comprender
el conjunto al mismo tiempo que los detalles. Pero precisamente esto es lo
que le da un gran rendimiento; en efecto, cuanto más claramente veamos
el conjunto, mejor nos daremos cuenta de su analogía con otros objetos
vecinos, más probabilidades tendremos, por consiguiente, de adivinar las
generalizaciones posibles. La elegancia puede provenir del sentimiento de lo
imprevisto por el encuentro inesperado de objetos que no estamos
acostumbrados a relacionar; aun ahí es fecunda, puesto que nos
descubre parentescos hasta entonces desconocidos; también es fecunda
cuando resulta del contraste entre la sencillez de los medios y la
complejidad del problema planteado; nos hace reflexionar entonces sobre la
razón de este contraste, y lo más frecuente es que nos haga ver que
esta razón no es debida al azar, que se encuentra en cualquier
 ley insospechada. En una palabra, el sentimiento de la elegancia matemática no es otra cosa que la satisfacción debida a no sé qué
adaptación entre la solución que se acaba de descubrir y las
necesidades de nuestro espíritu, y es a causa de esta adaptación cómo la solución puede ser para nosotros un instrumento. Esta
satisfacción estética está, por consiguiente, unida a la economía de pensamiento. Otra vez más la comparación del Erecteón
me viene al espíritu, pero no deseo emplearla con demasiada frecuencia.


%\end{document}

Es por la misma razón por la que, cuando un cálculo un poco largo
nos ha conducido a cualquier resultado simple y sorprendente, no estamos
satisfechos hasta tanto no hayamos mostrado que podíamos haberlo
previsto, si no todo el resultado, por lo menos sus trazos más característicos. ¿Por qué? ¿Qué es lo que nos impide contentarnos con
un cálculo que nos ha enseñado, según parece, todo lo que desearíamos saber? Es porque en casos análogos, el gran cálculo no
podría servirnos de nuevo; según parece no sucede lo mismo con el
razonamiento, a menudo medio intuitivo, que podía habernos permitido
preverlo. Como este razonamiento es cierto, vemos al mismo tiempo todas las
partes, de manera que se da uno cuenta inmediatamente de lo que hacía
falta cambiar para adaptarlo a los problemas de la misma naturaleza que
puedan presentarse. Puesto que él nos permite prever si la solución
de éstos será simple, nos muestra también si el cálculo
merece ser emprendido.

Lo que acabamos de decir basta para demostrar cuán vano sería
reemplazar por un procedimiento mecánico cualquiera la libre iniciativa
del matemático. Para obtener un resultado que tenga valor real, no es
suficiente crear continuamente cálculos y más cálculos, poseer
una máquina para ordenar las cosas; esto nos indica que no es solamente
el orden, sino el orden inesperado el que produce algo. La máquina puede
trabajar sobre el hecho bruto, pero nunca logrará aprehender el alma de 
éste.



Desde mediados del siglo pasado [siglo XIX], los matemáticos han tratado con ahínco creciente de alcanzar la certeza absoluta; tienen razón y esta
tendencia se acentuará cada vez más. En matemáticas la certeza
no es todo, mas sin ella no hay nada; una demostración que no sea rigurosa, es nada. Creo que a nadie se le ha
de ocurrir discutir esta verdad. Pero si se la tomara demasiado al pie de la
letra, se podría inferir que antes del año 1820, por ejemplo, no había matemáticas, y esto sería excesivo; los geómetras de
entonces sobrentendían voluntariamente lo que nosotros explicamos
mediante prolijos discursos; claro es que esto no quiere decir que no se
hubiesen apercibido de todo; pero pasaban por encima demasiado rápidamente, y para verlo bien hubiera sido necesario que se hubieran
molestado en decírnoslo.

¿Es acaso necesario decirlo tantas veces? Los que primero se han preocupado
del rigor, nos han dado razonamientos que debemos tratar de imitar, pero si
las demostraciones del futuro deben ser construidas sobre este modelo, los
tratados de matemáticas se tornarían muy extensos; si temo por el
tamaño no es solamente que me preocupe por el amontonamiento que se
produciría en las bibliotecas, sino porque al alargar nuestras
demostraciones perderían esa aparencia de armonía de la cual acabo
de explicar hace un momento la utilidad.

Es hacia la economía del pensamiento a la que se debe tender; no es
suficiente dar modelos a imitar. Es preciso que se pueda después de
nosotros abandonar esos modelos y en lugar de repetir un razonamiento ya
hecho, resumirlo en algunas líneas. Y es en esto en lo que se ha
logrado el éxito varias veces: por ejemplo, existía una clase de
razonamientos que se parecían y que se encontraban en todas partes;
eran perfectamente rigurosos, pero eran largos. Un día se imaginó
la palabra convergencia uniforme, y esta sola frase los ha tornado
inútiles; ya no hay más necesidad de repetirlos, puesto que se
sobrentienden. Los que dividen las dificultades en cuatro, pueden, pues,
hacernos un doble servicio, primero enseñarnos a actuar como ellos en la
necesidad, pero sobre todo permitirnos lo más a menudo posible no hacer
nada como ellos, sin sacrificar nada al rigor.

Acabamos de ver a través de un ejemplo cuál es la importancia de las
palabras en matemáticas; podría citar muchos otros. Como dice bien
Mach, no podemos darnos cuenta cabal de cuánto pensamiento puede
economizar una palabra bien elegida. No recuerdo si ya he dicho que las matemáticas son el arte de dar el mismo nombre a cosas diferentes. Es conveniente que
estas cosas diferentes por la materia, sean parecidas por la forma, que
puedan, válganos la frase, fundirse en el mismo molde. Cuando el
lenguaje ha sido bien elegido, nos sorprende ver que todas las
demostraciones hechas para un objeto conocido se aplican inmediatamente a
muchos objetos nuevos; no hay que cambiar nada, ni las palabras, puesto que
los nombres se han vuelto idénticos.

Una palabra bien elegida es suficiente muchas veces para hacer desaparecer
las excepciones que traen las reglas enunciadas en el antiguo lenguaje; es
para esto para lo que se han imaginado las cantidades negativas, las
cantidades imaginarias, los puntos del infinito y no sé cuántas más; las excepciones, no lo olvidemos, son perniciosas, puesto que
ocultan las leyes.

Ésta es una de las características en las que se reconocen los
hechos de gran rendimiento, son los que permiten esas felices innovaciones
del lenguaje. El hecho bruto se halla algunas veces desprovisto de interés, se le ha podido señalar muchas veces sin haber prestado gran servicio
a la ciencia; no adquiere valor hasta el día en que un pensador
perspicaz se da cuenta de la relación, relación que pone
inmediatamente en evidencia y que simboliza mediante una palabra.

Los físicos obran lo mismo: han inventado la palabra energía, y
esta palabra ha sido prodigiosamente fecunda, puesto que también crea la
ley eliminando las excepciones y designa con la misma palabra cosas
diferentes por la materia y parecidas por la forma.

Entre las palabras que han ejercido más influencia señalaré las
de grupo y la de invariante. Nos han hecho conocer la esencia de muchos
razonamientos matemáticos; nos han mostrado en cuántos casos los
viejos matemáticos consideraban los grupos sin saberlo y cómo, creyéndose muy alejados los unos de los otros, se encontraban de pronto
aproximados sin comprender por qué.

Hoy diríamos que habían encarado los grupos isomorfos. Sabemos
ahora que en un grupo la materia interesa poco, que es solamente la forma la
que interesa, y que cuando se conoce bien un grupo, se conocen por
consiguiente todos los grupos isomorfos; y gracias a estos nombres de
grupos e isomorfismo, que resumen en pocas sílabas esta regla sutil y la tornan en
seguida familiar a todos los espíritus, el tránsito es inmediato y
puede hacerse economizando todo esfuerzo de pensamiento. La idea de grupo se
une, por lo tanto, a la de transformación. ¿Por qué se atribuye
tanto valor a la invención de una nueva transformación? Pues porque
de un solo teorema nos permite sacar diez o veinte; tiene el mismo valor
que un cero colocado a la derecha de un número entero.

Esto es lo que ha determinado hasta ahora el sentido del movimiento de la
ciencia matemática y lo que lo ha de determinar en el futuro.

Pero a esto contribuye también la naturaleza de los problemas que se
plantean. No podemos olvidar cuál debe ser nuestro propósito; de
acuerdo con mi opinión este propósito debe ser doble; nuestra
ciencia confina a la vez con la filosofía y con la física, y es
para estos dos vecinos para quienes trabajamos. Es por eso por lo que hemos
visto y veremos aún marchar las matemáticas en dos direcciones
opuestas.

Por un lado la ciencia matemática debe reflexionar sobre ella misma, y
esto es útil, porque cavilar sobre ella misma es reflexionar sobre el espíritu humano que la ha creado, tanto más puesto que entre las múltiples creaciones del hombre es esta ciencia la que menos ha pedido al
exterior, en una palabra, la más pura y esencial. Es por esto por lo que
ciertas especulaciones matemáticas son útiles, por ejemplo, las que
encaran el estudio de los postulados, de las geometrías no
acostumbradas, de las funciones de porte extraño. Cuanto más se
aparten estas especulaciones de las concepciones más comunes y por
consiguiente de la naturaleza y de sus aplicaciones, mejor nos mostrarán
lo que el espíritu humano es capaz de hacer cuando se sustrae a la tiranía del mundo exterior; por consiguiente, nos lo harán conocer
mejor en cuanto a él mismo.

Pero es al lado opuesto, al de la naturaleza, al que hace falta dirigir el
grueso de nuestro ejército.

Allí nos encontramos al físico o al ingeniero que nos dicen: \guillemotleft ¿Podría usted integrarme esta ecuación
diferencial? La necesitaré dentro de ocho días para tal construcción que debe ser terminada para tal fecha.\guillemotright \ 
\guillemotleft Esta ecuación, responderemos,
no entra en uno de los tipos integrables, y usted bien sabe que no hay más.\guillemotright \  \guillemotleft Sí, lo sé, ¿pero entonces
para qué sirve usted?\guillemotright\   A menudo bastaría llegar a
un acuerdo: el ingeniero en realidad no necesita la integral en términos
finitos; necesita conocer el porte general de la función integral, o
simplemente una cierta cifra fácil de deducir de esta integral en caso
de conocerlo. Por regla general no lo conoce; pero podría calcular
esta cifra sin él, si supiese justamente de qué cifra tiene el
ingeniero necesidad y con qué aproximación.

Antes no se consideraba resuelta una ecuación sino cuando se había
expresado la solución con ayuda de un número finito de funciones
conocidas; pero esto apenas si es posible más que en el uno por ciento.
Lo que siempre podemos hacer es resolver el problema {\it cualitativamente}, es
decir, tratar de conocer la forma general de la curva, que representa la
función desconocida.

Nos queda luego por encontrar la solución cuantitativa del problema;
pero si lo desconocido no puede ser determinado por un cálculo finito,
lo podemos representan siempre por una serie infinita convergente que
permita calcularlo. ¿Puede esto considerarse como una solución? Se
cuenta que Newton comunicó a Leibniz un anagrama poco más o menos
como éste: aaaaabbbeeeii, etc. Leibniz, naturalmente, no comprendió
nada; pero nosotros, que tenemos la clave, sabemos que este anagrama quiere
decir, traduciéndolo al lenguaje moderno: \guillemotleft Yo sé
integrar todas las ecuaciones diferenciales.\guillemotright\  Claro es, que
una de dos, o Newton tuvo suerte, o se hacía singulares ilusiones. Quería simplemente decir que podía formar (por el método de los
coeficientes indeterminados) una serie de potencias satisfaciendo
formalmente la ecuación propuesta.

Hoy una solución parecida no nos satisfaría más, por dos
razones: porque la convergencia es demasiado lenta y porque los términos
se suceden sin obedecer a ninguna ley. Por el contrario, la serie $(x)$ nos
parece satisfactoria, primero porque converge muy rápida (esto para el
practicante que desea obtener su número lo antes posible), y segundo
porque vemos de un solo vistazo la ley de términos (esto satisface las
necesidades estéticas del teórico).



Pero entonces no hay problemas resueltos y otros que no lo están; sólo hay problemas más o menos resueltos, según lo sean por una serie
convergente más o menos rápida, o bien regidos por una ley más o
menos armoniosa. Ocurre a veces que una solución imperfecta nos conduce
hacia una solución mejor. Sucede a menudo que la serie es de una
convergencia tan lenta que el cálculo es impracticable y no se logra
demostrar más que la posibilidad del problema.

Entonces el ingeniero encuentra esto irrisorio y tiene razón, puesto que
no le ayudará a terminar la construcción para la fecha fijada. Se
preocupa poco de saber si esto será útil a los ingenieros del siglo
XXII. Nosotros pensamos de distinta manera: algunas veces somos
más felices por haber ahorrado un día de trabajo a nuestros nietos
que una hora a nuestros contemporáneos.

Alguna vez tanteando empíricamente, digámoslo así, llegamos a
una fór\-mula suficientemente convergente. ¿Qué más quiere
usted?, nos dice el ingeniero; sin embargo, a pesar de todo, no nos damos
por satisfechos: hubieramos querido prever esta contingencia. ¿Por qué?
Porque si la hubiéramos previsto una vez, sabríamos preverla otra.
Hemos tenido éxito, pero es poca cosa a nuestros ojos, si no tenemos la
firme esperanza de volver a comenzar.

A medida que la ciencia evoluciona es más difícil abarcarla; en
vista de ello se la divide a fin de estudiarla detalladamente; en una
palabra, esto es especializarse. Si se exagera en este sentido sería un
obstáculo para el progreso de la ciencia. Ya lo hemos dicho, es por
aproximaciones inesperadas entre las diversas partes como se efectúan
los progresos. Especializarse mucho sería lo mismo que prohibir estas
aproximaciones. Esperemos que en congresos como los de Heidelberg y Roma,
poniéndonos en contacto los unos con los otros, abriremos miradores
sobre el campo del vecino, obligándonos a compararlo con el nuestro, y
de esta manera saldremos un poco de nuestro pequeño villorrio; ése sería el mejor remedio para el peligro que acabo de señalar.

Pero me he detenido demasiado en estas generalidades y es necesario entrar
en el detalle.


Pasemos revista a las ciencias particulares que forman las matemáticas;
veamos lo que cada una ha hecho, hacia dónde nos conducen y qué es
lo que se puede esperar de ellas. Sí lo que precede es justo,
observaremos que los grandes progresos del pasado se han producido cuando
dos de estas ciencias se han aproximado, cuando hemos tenido conciencia de
la similitud de sus formas, a pesar de lo dispar de sus materias, cuando se
han modelado la una sobre la otra, de tal manera que una pudo aprovecharse
de las conquistas de la otra. Debemos también entrever en acercamientos
de este género los progresos del porvenir.


\section*{La aritmética}


Los progresos de la aritmética han sido mucho más lentos que los del 
álgebra y análisis; es fácil comprenderlo; el sentimiento de
la continuidad es una guía inapreciable de la cual carece el aritmético; cada número entero está separado de los otros, tiene, permítaseme expresarlo así, cada uno de ellos una especie de excepción, y es por eso por lo que los teoremas generales son más raros en
la teoría de los números, y los que existen están más
ocultos y han de escapar más tiempo a los investigadores.

Si la aritmética está retrasada con respecto al álgebra y al análisis, le conviene tratar de modelarse sobre estas ciencias para
aprovecharse de sus adelantos. El aritmético debe guiarse por las analogías con el álgebra. Estas analogías son numerosas y si, en
muchos casos, no han sido estudiadas bastante profundamente como para ser de
utilidad, son presentidas, por lo menos después de largo tiempo; el
lenguaje de estas dos ciencias indica que ya se las ha percibido. Así
se habla de números trascendentes; nos damos cuenta en seguida de que
la clasificación futura de éstos tiene ya por imagen la clasificación de las funciones trascendentales, y, sin embargo, no vemos muy bien
todavía cómo podremos pasar de una clasificación a otra; pero
si la hubiéramos visto ya, estaría hecho y no sería obra del
futuro. El primer ejemplo que se me ocurre es el de la teoría de las
congruencias, en la que se encuentra un paralelismo perfecto con la de las
ecuaciones algebraicas. Claro es que se llegará a completar este paralelismo, que
debe subsistir, por ejemplo, entre la teoría de las curvas algebraicas
y la de las congruencias de dos variables.

Cuando los problemas relativos a las congruencias con varias variables sean
resueltos, se habrá dado el primer paso hacia la solución de muchas
cuestiones de análisis indeterminados.

\section*{El álgebra}

La teoría de las ecuaciones algebraicas detendrá mucho tiempo aún la atención de los geómetras; los lados por donde se la puede
estudiar son numerosos y diversos.

No hay que creer que el álgebra está agotada porque nos provea de
las reglas necesarias para formar todas las combinaciones posibles; nos
falta buscar las combinaciones interesantes, las que satisfacen tal o cual
condición. De esta manera se constituirá una especie de análisis
indeterminado en el que las incógnitas no serán más números
enteros, sino polinomios.

Entonces esta vez es el álgebra quien se modela sobre la aritmética,
guiándose sobre la analogía del número entero, ya sea con el
polinomio entero de coeficiente cualquiera, o con el polinomio entero de
coeficiente entero.

\section*{La geometría}


Parece como si la geometría no pudiera contener nada que no se
hubiera tratado ya en el álgebra o en el análisis; que los hechos
geométricos no fueron otra cosa que los hechos algebraicos o analíticos expresados en otro lenguaje.

Se podría creer entonces que después del examen que acabamos de
hacer, no nos quedaría nada por agregar que se refiriera especialmente
a la geometría.

Esto sería desconocer la importancia de un lenguaje bien constituido,
no comprender el sentido que se les puede dar a las mismas cosas, la manera
de expresarlas y, por consiguiente, de agruparlas.


En primer lugar, las consideraciones geométricas nos conducen a
plantearnos nuevos problemas; son, si se quiere, problemas analíticos,
pero que no nos hubiéramos planteado con motivo del análisis. El análisis se aprovecha, sin embargo, de la misma manera que se aprovecha de
lo que está obligado a resolver para satisfacer las necesidades de la física.



Una gran ventaja de la geometría es que los sentidos pueden socorrer a
la inteligencia y ayudarla a entrever la ruta a seguir; muchos espíritus prefieren por eso llevar los problemas del análisis a la forma geométrica. Desgraciadamente nuestros sentidos no pueden conducirnos muy
lejos; se alejan desde el momento en que queremos salir fuera de las tres
dimensiones clásicas. ¿Quiere esto decir que fuera de este dominio
restringido donde parecen querer encerrarnos, no debemos contar más que
sobre el análisis puro? ¿Que toda geometría que tenga más de
tres dimensiones es vana y sin objeto? En la generación que nos ha
precedido, los más grandes maestros hubieran contestado: \guillemotleft 
Sí, hoy en día estamos tan familiarizados con esta noción que
podemos hablar de ella incluso en un curso universitario sin provocar
demasiada sorpresa.\guillemotright 

Pero ¿para qué puede servirnos? Es fácil saberlo: expresa en términos muy concisos lo que el lenguaje analítico ordinario diría
en frases complejas. Además, este lenguaje nos hace denominar con el
mismo nombre lo que se parece y afirma de las analogías, no dejándonoslo olvidar. Nos permite, por lo tanto, orientarnos en este espacio que
es demasiado grande para nosotros y que no podemos ver, recordándonos
sin cesar el espacio visible que no es más que una imagen imperfecta,
sin duda, pero que aún es una imagen. Aquí, como en todos los casos
anteriores, es la analogía con lo simple lo que nos permite comprender
lo complejo.

Esta geometría de más de tres dimensiones no es una simple geometría analítica, no es solamente cuantitativa, también es
cualitativa, y es por ello por lo que es más interesante. Hay una
ciencia que se llama {\it Análisis Situs} y que tiene por objeto el estudio de
las relaciones de posición de los diversos elementos de una figura,
abstrayendo sus tamaños. Esta geometría es solamente cualitativa,
sus teoremas serían verdaderos si las figuras, en vez de ser exactas,
fueran groseramente copiadas por un niño. Se puede también hacer un {\it Análisis Situs} de más de tres dimensiones. Su importancia es tan grande que nunca se podría insistir lo bastante sobre ello; el partido que ha sacado Riemann,
uno de sus principales creadores, bastaría para demostrarlo. Es
necesario que se la llegue a construir completamente en los espacios
superiores; tendremos un instrumento que nos permitirá realmente ver
en el hiperespacio, supliendo nuestros sentidos.

Los problemas del {\it Análisis Situs} no se habrían, por otra parte,
planteado si no se hubiera hablado más que en lenguaje analítico; o más bien, me equivoco: se habrían planteado seguramente, puesto que
su solución es necesaria a un gran número de cuestiones de análisis; pero se habrían planteado aisladamente, los unos después de
los otros y sin que se hubiera notado su lazo común.


\section*{El cantorismo}

Me he referido antes a la necesidad que tenemos de recordar continuamente
los primeros principios de nuestra ciencia y al provecho que se puede sacar
del estudio del espíritu humano. Es esta necesidad la que ha inspirado
dos tentativas que han ocupado un gran lugar en la historia de las matemáticas en estos últimos años. La primera es el cantorismo, cuyo
aporte a la ciencia todo el mundo conoce. Cantor ha introducido en la
ciencia una manera nueva de considerar el infinito matemático. Uno de
los rasgos más característicos del cantorismo es que en lugar de
elevarse a lo general edificando construcciones cada vez más complicadas
y de definir por construcción, parte del {\it genus supremus}, y no define
como lo hubieran hecho los escolásticos {\it per genus proximun et
diferentiam specificam}. De ahí el horror que algunas veces han
inspirado a ciertos espíritus, a Hermite, por ejemplo, cuya idea
favorita era comparar las ciencias naturales con las matemáticas. En la
mayoría de nosotros estas prevenciones se habían disipado, pero ha
ocurrido que se ha chocado con ciertas paradojas y contradicciones aparentes
que hubieran colmado de gozo a Zenón de Elea y a la escuela de Megara. Entonces todos
nos hemos puesto a buscar el remedio. Por mi parte, pienso, y no soy el único, que lo importante está en no introducir más que otros seres
que lo puedan definir completamente en un número finito de palabras.
Cualquiera que sea el remedio adoptado, nos podemos prometer el placer que
un médico experimenta al ser llamado para seguir un interesante caso
patológico.


\section*{La búsqueda de los postulados}


Nos hemos esforzado en enumerar los axiomas y postulados
más o menos disimulados que sirven de fundamento a las
diferentes teorías matemáticas. El señor Hilbert ha obtenido
brillantes resultados. Parece, al principio, que este dominio
era limitado, que no hubiera nada más que hacer cuando el
inventario estuviera terminado, lo que sucedería pronto.
Pero cuando se tenga todo enumerado aún quedarán bastantes
maneras para clasificarlo todo, pues un buen bibliotecario
encuentra siempre en qué ocuparse y cada nueva clasificación
es de  utilidad al filósofo.

Detengo este examen que no podría ni en sueños presentar
completo. Creo que estos ejemplos habrán sido suficientes
para mostrar por qué mecanismo las ciencias matemáticas han progresado
en el pasado y en qué sentido deben marchar en el futuro.


\bigskip
\begin{flushright}

Digitalización: maplewhite@gmail.com
\end{flushright}


\end{document}


%\end{document}



La génesis de la invención matemática es un problema que debe
inspirar mucho interés al psicólogo. Es el acto en el cual el espíritu humano prescinde más del mundo exterior, en el que no obra más que por él mismo y sobre él mismo, de manera que estudiando
los procesos del pensamiento geométrico, podemos tener esperanzas de
alcanzar lo más esencial de él.

Se ha comprendido desde hace mucho tiempo; no hace muchos meses una revista
titulada {\it La enseñanza matemática}, dirigida por los señores
Laisant y Fehr, emprendió una encuesta sobre las costumbres del espíritu y los métodos de trabajo de los diferentes matemáticos.

Había ya esbozado los principales trazos de este artículo cuando
los resultados de la encuesta fueron publicados; no pude por consiguiente
utilizarlos; me limitaré a decir que la mayoría de los testimonios
confirman mis conclusiones, no digo que la totalidad, puesto que cuando se
consulta el sufragio universal no puede uno vanagloriarse de reunirla.

Un primer hecho debe sorprendernos, o más bien debía sorprendernos
si no estuviésemos tan acostumbrados. ¿Cómo es que hay gentes que
no comprenden las matemáticas? Si las matemáticas no invocan más
que las leyes de la lógica aceptadas por todos los espíritu
centrados; si su evidencia está fundada sobre los principios comunes a
todos los hombres y que ninguno podría negar sin estar loco, ¿cómo
existen tantas personas refractarias a ellas?

Que no todo el mundo tenga capacidad de inventiva no tiene nada de
particular. Que no todos puedan retener una demostración que hayan aprendido anteriormente, pase aun, pero que no
todos puedan comprender un razonamiento matemático, cuando se expone, he
aquí lo que al reflexionar parece sorprendente. Sin embargo, la mayoría no puede seguir este razonamiento sino con gran trabajo; esto es
indudable y la experiencia de los maestros de enseñanza secundaria no me
contradecirá seguramente.

Hay más aún. ¿Cómo es posible el error en matemáticas? Una
inteligencia sana no debe cometer una falta de lógica y, sin embargo,
hay espíritus muy finos que no tropezarán en un razonamiento muy
corto, tal como los que deben efectuar en los actos comunes de la vida,
incapaces de seguir o de repetir sin equivocarse las demostraciones de matemáticas, que si bien son más largas, no son después de todo más que una acumulación de pequeños razonamientos, análogos a los
que se hacen con tanta facilidad todos los días. ¿Es necesario agregar
que los matemáticos tampoco son infalibles? La respuesta nos parece
necesaria. Imaginemos una larga serie de silogismos en los cuales las
conclusiones de los primeros sirvan de premisas a los siguientes: seremos
capaces de comprender algunos de estos silogismos; y no es al pasar de las
premisas a la conclusión donde correremos el riesgo de equivocarnos.
Pero entre el momento en que encontramos por primera vez una proposición
como conclusión de un silogismo y aquel en que la volvemos a encontrar
como premisa de otro silogismo, habrá transcurrido a veces mucho
tiempo, habremos desarrollado numerosos anillos de la cadena; puede ocurrir
también que se haya olvidado y lo que es peor aún, que nos hubiéramos olvidado del sentido. Puede, pues, entonces, acontecer que se la
reemplace por una proposición diferente, o que, conservando el mismo
enunciado, que se le atribuya otro sentido; así es como se está
expuesto al error.

A menudo el matemático debe servirse de una regla: naturalmente comienza
a demostrar esta regla; mientras la demostración está fresca en la
memoria, comprende perfectamente su sentido y su alcance y no corre el
riesgo de alterarla; pero en cuanto confía en su memoria y no la
aplica más que de una manera mecánica, entonces si la memoria le
llega a faltar puede aplicarla al revés. Es de esta manera,
presentando un ejemplo simple y casi vulgar, como a veces nos
equivocamos en operaciones de cálculo por haber olvidado nuestra tabla
de multiplicar.

De esta manera la aptitud especial hacia las matemáticas será debida
a una memoria muy fiel, o bien a una fuerza de atención prodigiosa. Sería una cualidad análoga a la del jugador de {\it whist}, que recuerda las
cartas tiradas, o bien, para poner otro ejemplo, a la del jugador de ajedrez
que puede encarar un número de combinaciones muy grande y conservarlas
en su memoria. Todo buen matemático debería ser al mismo tiempo
buen jugador de ajedrez y viceversa; todo buen jugador de ajedrez debería ser igualmente un buen calculador numérico. Cierto que esto
ocurre a veces: Gauss era a la vez un geómetra de genio y un calculador
muy rápido y seguro.

Pero hay excepciones, o más bien me equivoco: no puedo llamar a esto
excepciones, pues si no las excepciones serían más numerosas que
los casos ajustados a la regla. Es Gauss, por el contrario, quien era una
excepción. En cuanto a mí, estoy obligado a confesarlo, soy incapaz
de hacer una suma sin faltas. Sería igualmente muy mal jugador de
ajedrez; calcularía bien que jugando de tal manera me expongo a tal
peligro; pasaría revista a muchos otros golpes que rechazaría por
otras tantas razones, y acabaría por jugar la combinación examinada
al principio, habiendo olvidado en el intervalo el peligro previsto.

En una palabra, mi memoria no es mala, pero es insuficiente para hacer de mí un buen jugador de ajedrez. ¿Por qué no me falla en un
razonamiento matemático en el que la memoria de los jugadores de ajedrez
se perdería? Es evidente: porque está guiada por la marcha general
del razonamiento. Una demostración matemática no es una simple
yuxtaposición de silogismos; son silogismos colocados en un cierto
orden, y el orden en el cual están colocados estos elementos es mucho más importante que ellos mismos. Si tengo el sentimiento, la intuición de este orden, de manera que me pueda dar cuenta rápidamente del
conjunto del razonamiento, no debo temer más olvidarme de uno de los
elementos, pues cada uno vendrá a colocarse en el cuadro que le he
preparado, sin que haya hecho ningún esfuerzo de memoria.

Me parece entonces, repitiendo un razonamiento aprendido, que lo
hubiera podido inventar; esto no es con frecuencia más que una ilusión; pero, asimismo, aunque no soy bastante fuerte para crear por mí
mismo, lo vuelvo a inventar a medida que lo repito.

Se concibe que este sentimiento, esta intuición del orden matemático
que nos hace adivinar las armonías y las relaciones ocultas, no puede
pertenecer a todo el mundo. Los unos no poseerán ni este
sentimiento delicado difícil de definir, ni una fuerza de memoria y de
atención por encima de lo vulgar, y entonces serán incapaces de
comprender las matemáticas un poco elevadas; esto ocurre en la mayoría. Otros no tendrán este sentimiento más que en pequeño
grado, pero estarán dotados de una memoria poco común y de una gran.
capacidad de atención. Aprenderán de memoria los detalles, unos después de otros, podrán comprender las matemáticas y alguna vez
aplicarlas, pero serán incapaces de crear. Los otros, en fin, poseerán en un grado más o menos elevado la intuición especial de que acabo
de hablar, y entonces no solamente podrán comprender las matemáticas
aunque su memoria no tenga nada de extraordinario, sino que podrán
llegar a ser creadores y tratarán de inventar con más o menos éxito, según que esta intuición esté en ellos más o menos
desarrollada.

¿Qué es, en efecto, la invención matemática? No consiste en
hacer nuevas combinaciones con otros seres matemáticos ya conocidos.
Esto cualquiera podría hacerlo, pero las combinaciones que se podrían formar así serían infinitas y la mayor parte estaría
totalmente desprovista de interés.

Inventar consiste precisamente en no construir combinaciones inútiles,
sino en construir sólo las que pueden ser útiles, que no son más
que una ínfima minoría. Inventar es discernir, es elegir.

Ya expliqué antes cómo debe hacerse esta elección; los hechos
materiales dignos de ser estudiados son los que por su analogía con
otros resultan capaces de conducirnos al conocimiento de una ley matemática, de la misma manera que los hechos experimentales nos conducen al
conocimiento de una ley física. Son los que nos revelan parentescos
insospechados
entre otros hechos conocidos desde hace mucho tiempo, pero que erróneamente se creyeron extraños entre sí.

Entre las combinaciones que se escogerán, las más fecundas serán
frecuentemente las que están formadas por elementos procedentes de
sitios muy alejados; y no quiero decir que sea suficiente para inventar el
acercar los objetos más dispares posibles; la mayor parte de las
combinaciones que se formarían de este modo serían totalmente estériles, pero algunas de ellas, muy pocas veces, serían las más
fecundas de todas.

Inventar, lo he dicho en otra oportunidad, es elegir, pero la palabra no es
del todo justa, hace pensar en un comprador al que se le presentan un gran número de muestras, que examina una después de la otra a fin de hacer
su elección. En este caso las muestras serían tan numerosas que una
vida entera no bastaría para examinarlas. No es así como suceden
las cosas. Las combinaciones estériles ni siquiera se presentarán al
espíritu del inventor. En el campo de su conciencia no aparecerán más que las combinaciones realmente útiles y algunas que rechazará, pero que participan un poco de los caracteres de las combinaciones útiles. Todo sucede como si el inventor fuera un examinador de segundo grado,
que no tuviera que examinar más que los candidatos declarados admisibles
después de una primera prueba.

Lo que he dicho hasta aquí es lo que se puede observar e inferir
leyendo los escritos de los geómetras, con la única condición de
reflexionar sobre esta lectura.

Pero ya es hora de entrar más en el tema, a fin de ver qué es lo que
pasa en el alma misma del matemático. Para esto creo que lo mejor que
puedo hacer es apelar a recuerdos personales.

Voy a limitarme a contaros cómo escribí mi primer trabajo sobre las
funciones fuchsianas. Os pido perdón, voy a emplear algunas expresiones técnicas, pero no os debéis asustar, pues no hay necesidad alguna de
que las comprendáis. Diré, por ejemplo, que encontré la
demostración de tal teorema en tales circunstancias, este teorema tendrá un nombre bárbaro que muchos de entre ustedes desconocerán.
Esto no tiene imporcia; lo que le interesa al psicólogo no es el
teorema, son las circunstancias.


Desde hacía quince días me esforzaba en demostrar que no pedía existir ninguna función análoga a lo que yo más tarde llamé
funciones fuchsianas; en aquella época era muy ignorante; todos los días me sentaba en mi mesa de trabajo, pasaba una hora o dos, ensayaba
un gran número de combinaciones y no llegaba a ningún resultado. Una
noche tomé café, contrariando mis costumbres, y no me pude dormir;
las ideas surgían en masa, las sentía cómo chocaban, hasta que
dos de ellas se engarzaron, por así decir, para formar una combinación estable. A la mañana siguiente ya había establecido la
existencia de una clase de funciones fuchsianas: las que derivan de la serie
hipergeométrica; no hice más que redactar los resultados; no tardé más que algunas horas.

Quise a continuación representar estas funciones por el cociente de dos
series; esta idea fue perfectamente consciente y reflexionada: la analogía con las funciones elípticas me guiaba. Me pregunté cuáles debían ser las propiedades de estas series si existiesen, y llegué sin dificultad a formar las series que he llamado thetafuchsianas.

En ese momento me fui de Caen, donde vivía, para tomar parte en un
concurso geológico emprendido por la Escuela de Minas. Las peripecias
del viaje me hicieron olvidar mis trabajos matemáticos; al llegar a
Coutances, subimos en un ómnibus para dar no sé qué paseo;
en el momento en que ponía el pie en el estribo la idea me vino sin que
nada en mis pensamientos anteriores me hubiera podido preparar para
ella,que las transformaciones de que había hecho uso para definir 
las funciones fuchsianas eran idénticas a las de la geometría 


no-euclidiana. No hice la verificación; no hubiera tenido tiempo,
puesto que apenas sentado en el ómnibus proseguí la conversación comenzada, pero tuve en seguida la absoluta certidumbre. De regreso a Caen
verifiqué el resultado más reposadamente para tranquilidad de mi espíritu.

Me puse entonces a estudiar las cuestiones aritméticas sin gran
resultado aparente y sin sospechar que pudiera tener la más mínima
relación con mis anteriores descubrimientos. Disgustado por mi fracaso,
me fui a pasar algunos días al mar y pensé en otras cosas. Un día paseándome sobre el tajamar, la idea me vino, siempre con los mismos caracteres de brevedad, instantaneidad y certeza inmediata, que las transformaciones aritméticas de formas cuadradas ternarias indefinidas eran idénticas a las de
la geometría no-euclidiana.

Ya de vuelta, en Caen, reflexioné sobre este resultado y saqué
las consecuencias; el ejemplo de las formas cuadráticas me enseñaba
que existían más grupos fuchsianos que los que correspondían a la serie hipergeométrica; vi que podía aplicarlos a la teoría de la serie thetafuchsiana y que, por consiguiente, existían
otras funciones fuchsianas además de las que se derivaban de la serie
hipergeométrica, las únicas que conocía hasta entonces. Me
propuse, naturalmente, formar todas estas funciones; realicé un asedio
sistemático y fui sacando una tras otra todas las obras avanzadas; sin
embargo había entre ellas algunas que se me resistían y cuya caída
debía acarrear la de los cuerpos de plaza. Pero todos mis esfuerzos no
sirvieron primero nada más que para hacerme conocer mejor la dificultad,
lo cual ya era algo. Todo este trabajo fue perfectamente consciente.

Después de esto partí para Mont-Valerien, donde tenía que
hacer mi servicio militar; tuve por lo tanto preocupaciones muy diferentes.
Un día, atravesando el bulevar, la solución de la dificultad que me
había detenido se me apareció de repente. No traté de
profundizarla inmediatamente y fue sólo después de mi
servicio cuando proseguí la cuestión. Tenía todos los
elementos, no tenía más que juntarlos y ordenarlos. Redacté
entonces mi memoria definitiva de un tirón y sin ninguna dificultad.

Me limitaré a este único ejemplo, pues es inútil multiplicarlos; en lo que concierne a mis otros descubrimientos, tendría que contar
hechos análogos y las objeciones aportadas por otros matemáticos en
la encuesta de {\it La enseñanza matemática} no podrían más que
confirmarlos.

Lo que sorprenderá, primero, son estas apariencias de iluminación súbita, signo manifiesto de un largo trabajo inconsciente anterior; el
papel de ese trabajo inconsciente en la invención matemática me
parece indudable y se hallarán huellas en otros casos donde es menos
evidente. A menudo cuando se trabaja en una cuestión no se hace nada
bueno la primera vez que se pone uno a trabajar; tras esto se toma un
reposo más o menos largo y vuelve de nuevo a sentarse a trabajar delante
de su mesa. Durante la primera media hora se continúa no encontrando
nada, y después, de golpe la idea decisiva se presenta a la mente. Se
podría decir que el trabajo consciente ha sido más fructífero,
puesto que ha sido interrumpido y el reposo ha devuelto al espíritu
su fuerza y su frescor. Pero es más probable que este reposo haya sido
reemplazado por un trabajo inconsciente y que el resultado de este trabajo
se haya revelado en seguida al geómetra, lo mismo que en los casos
citados; solamente que la revelación, en vez de efectuarse en un paseo
o en un viaje, se produce durante un período de trabajo consciente, mas
con independencia de este trabajo, que desempeña además un papel
de desprendimiento, como si fuera el aguijón que hubiera excitado los resultados, ya adquiridos durante el reposo, que subsistían inconscientes, a tomar la forma consciente.

Hay otra observación que hacer respecto a las condiciones de trabajo de
la tarea inconsciente: que no es posible, o en todo caso que solamente es
fecundo, si es precedido, por una parte, y seguido por otra, de un período de trabajo consciente. Jamás (y los ejemplos que ya he citado lo
prueban bastante) estas inspiraciones repentinas se producen sino al cabo de
varios días de esfuerzos voluntarios, que han parecido absolutamente
infructuosos y donde se ha creído no hacer nada bueno, en los que da la
impresión de haber hecho una ruta totalmente falsa. Estos esfuerzos no
han sido tan estériles como se piensa, han puesto en marcha la máquina inconsciente; sin ellos no habría marchado ni, por lo tanto,
producido nada.

La necesidad del segundo período de trabajo consciente después de
la inspiración se comprende mejor aún. Hace falta poner en orden los
resultados de esta inspiración, deducir las consecuencias inmediatas,
ordenarlas, redactar las demostraciones, pero sobre todo verificarlas. Hablo
del sentimiento de certeza absoluta que acompaña a la inspiración;
en los casos citados este sentimiento no era engañoso, y frecuentemente
se presenta así: no hay que creer que esto sea una regla sin excepción; con frecuencia este sentimiento nos engaña sin que por ello sea
menos vivo, y uno no se da cuenta de ello
sino cuando se ha establecido la demostración. Observé sobre todo
este hecho, por las ideas que me han venido por la mañana o por la
noche en mi lecho, en un estado semihipnagógico.

Tales son los hechos. He aquí ahora las reflexiones que nos inspiran.
El yo inconsciente, o como hemos dicho, el yo subconsciente desempeña un
papel capital en la invención matemática. Pero se considera
vulgarmente el yo subconsciente como puramente automático. Además
hemos visto que el trabajo matemático no es más que un simple
trabajo mecánico, que no podríamos confiar a una máquina por más perfeccionada que se la supusiera. No se trata solamente de aplicar
las reglas, de fabricar todas las combinaciones posibles, de acuerdo con
ciertas leyes fijas. Las combinaciones así obtenidas serían demasiado
numerosas, inútiles y embarazosas. El verdadero trabajo del inventor
consiste en elegir entre estas combinaciones a fin de eliminar las que son inútiles, o más bien para no tomarse el trabajo de hacerlas. Las
reglas que deben guiar esta elección son tan finas y delicadas que es
poco más o menos que imposible enunciarlas en un lenguaje preciso; más bien se sienten que se formulan. ¿Cómo, en estas condiciones,
pensar en una criba capaz de aplicarlas mecánicamente?

Entonces se nos presenta una primera hipótesis: el yo inconsciente no es
inferior al yo consciente; no es sólo automático, es capaz de
discernir, tiene tacto, delicadeza, sabe elegir y adivinar. Qué digo, más aún, sabe adivinar mejor que el yo consciente, puesto que ha
tenido éxito allí donde el otro ha fracasado. En una palabra, el yo
inconsciente, ¿no es superior al yo consciente? Ustedes comprenden toda la
importancia de esta pregunta. El señor Boutroux, en una conferencia
reciente, ha mostrado cómo se ha presentado en diferentes ocasiones y qué consecuencias entrañará una respuesta afirmativa. 

¿Esta contestación afirmativa nos es impuesta por los hechos que acabo
de exponer? Confieso que por mi parte no la aceptaría sin repugnancia.
Revisemos los hechos y busquemos si no tienen otra explicación.



Es cierto que las combinaciones que se presentan al espiritu como una
especie de iluminación súbita, después de un trabajo
inconsciente un poco prolongado, son generalmente combinaciones útiles y
fecundas que parecen el resultado de la primera tría. Se deduce que el
yo subconsciente, habiendo adivinado por una intuición delicada que
estas combinaciones podrían ser inútiles, no ha formado más que 
éstas, o más bien, ha formado muchas otras que estaban desprovistas
de interés y que han permanecido en el inconsciente.

En este segundo punto de vista todas las combinaciones se formarían a
consecuencia del automatismo del yo subconsciente, pero solamente las que
fueran interesantes penetraría el campo de la consciencia. Y esto aún es muy misteriosa ¿Cuál es la causa que hace que entre los mil
productos de 
nuestra actividad inconsciente haya algunos que puedan franquear su atrio,
mientras que otros permanecen dentro? ¿Es un simple azar el que les confiere
este privilegio? Evidentemente, no; entre todas las excitaciones de
nuestros sentidos, por ejemplo, sólo las más intensas lograrán
retener nuestra atención, a menos que esta atención no haya sido atraída hacia ellas por otras causas. Pero generalmente los fenómenos
subconscientes privilegiados, aquellos susceptibles de tornarse conscientes,
son los que directa o indirectamente afectan más profundamente nuestra
sensibilidad.

Podemos sorprendernos de ver invocar la sensibilidad con motivo de
demostraciones matemáticas que aparentemente no podrían interesar más que a la inteligencia. Esto sería olvidar el sentimiento de la
belleza matemática, de la armonía de los números y de las
formas, de la elegancia geométrica. Es un auténtico sentimiento estético que todos los verdaderos matemáticos conocen. He aquí una
verdadera sensibilidad.

Según esto, ¿cuáles son los seres matemáticos a los que
atribuimos este carácter de belleza y elegancia y que son susceptibles
de desarrollar en nosotros una especie de emoción estética? Aquellos
cuyos elementos están armoniosamente dispuestos, de manera que el espíritu pueda sin esfuerzo abarcar todo el conjunto penetrando en los
detalles. Esta armonía es a la vez una satisfacción para nuestras
necesidades estéticas y una ayuda para el espíritu que sostiene y guíe..

 Al mismo tiempo poniendo ante nuestros ojos un todo bien ordenado, nos hace presentir
una ley matemática. Puesto que ya antes lo hemos dicho, los solos hechos
matemáticos dignos de retener nuestra atención y susceptibles de ser 
útiles son los que pueden hacernos conocer una ley matemática. De
tal manera que llegamos a la conclusión siguiente: las combinaciones 
útiles son precisamente las más bellas, quiero decir, las que pueden
encantar más a esa sensibilidad especial que todos los matemáticos
conocen, pero que los profanos ignoran hasta el punto de sonreírse.

¿Qué sucede entonces? Entre las numerosas combinaciones que el yo
subconsciente ciegamente ha formado, casi todas carecen de interés y de
utilidad; pero por eso mismo no excitan la sensibilidad estética; la
conciencia no las conocerá jamás; algunas solamente son armoniosas,
y por consiguiente, a la vez inútiles y bellas, serán capaces de
conmover esa sensibilidad especial del geómetra, que acabo de referirme
y que una vez excitada llamará sobre ella nuestra atención y le dará así ocasión de volverse consciente.

Esto no es más que una hipótesis. Mientras tanto he aquí una
observación que podría confirmarla: cuando una iluminación súbita invade el espíritu del matemático, sucede con frecuencia
que lo engaña; pero acaece también algunas veces, lo he dicho, que
no soporta la prueba de una verificación; y ¡bien!, se advierte casi
siempre que esta idea es falsa; si hubiera sido justa habría halagado
nuestro instinto natural de elegancia matemática.

De este modo es esta sensibilidad estética especial la que juega el
papel de la criba delicada a la que me refería antes, y esto hace
comprender, por otra parte, por qué el que esté desprovisto de ella
no será jamás un verdadero inventor.

Todas las dificultades no han desaparecido, sin embargo; el yo consciente
está estrechamente limitado; en cuanto al yo subconsciente no conocemos
sus límites, y es por eso lo mucho que nos repugna suponer que él
haya podido formar en tan poco tiempo más combinaciones que la vida
entera de un ser consciente podría abarcar. Estos límites existen,
sin embargo; ¿es verosímil que pueda formar todas las combinaciones
posibles cuyo número aterraría a la imaginación? Esto
parecerá necesario, no obstante, porque si no produce más que una
pequeña parte de estas combinaciones y si lo hace al azar, tendrá
pocas posibilidades para que la buena que deba escoger se encuentre entre
ellas.

Puede ser que sea necesario encontrar la explicación en este período de trabajo consciente preliminar que precede siempre a todo trabajo
inconsciente fructífero. Permítaseme una grosera comparación.
Representémonos los elementos futuros de nuestras combinaciones como
parecidos a los átomos ganchudos de Epicuro. Durante el reposo completo
del espíritu, estos átomos permanecen inmóviles, están,
por así decirlo, enganchados al muro; este reposo completo puede
entonces prolongarse indefinidamente sin que estos átomos se encuentren, y por consiguiente, sin que pueda producirse ninguna combinación
entre ellos.

Por el contrario, durante un período de reposo aparente y de trabajo
subconsciente, algunos se desenganchan y ponen en movimiento. Surcan en
todos los sentidos el espacio, iba a decir el reducto donde están
encerrados, como podría hacerlo, por ejemplo, una nube de moscardones
o, si prefieren una comparación más sabia, como lo hacen las moléculas gaseosas en la teoría cinética de los gases. Sus choques
mutuos, pueden entonces producir combinaciones nuevas.

¿Cuál va a ser el papel del trabajo consciente preliminar?
Evidentemente movilizar alguno de estos átomos, desengancharlos del muro
y ponerlos en movimiento. Se cree que no se ha hecho nada bueno, porque se
han movido estos elementos de mil maneras diferentes para tratar de
reunirlos y no se ha podido encontrar un conjunto satisfactorio. Pero después de esta agitación que les ha sido impuesta por nuestra voluntad,
estos átomos no vuelven a su reposo primitivo, continúan su danza
libremente.

Según esto, nuestra voluntad no los ha elegido al azar, sino que
persigue un objeto perfectamente determinado; los átomos movilizados no
son, por lo tanto, átomos cualesquiera, son aquellos en los cuales
podemos razonablemente encontrar la solución buscada. Los átomos
movilizados van entonces a sufrir choques, que los harán entrar en
combinación, sea ellos, 
o sea con otros átomos que han permanecido in-. entre




y que han chocado en su curso.

móviles



Pido perdón una vez más. Mi comparación es grosera: pero 'cómo podría hacer comprender mi pensamiento de otra no sé
forma.

Sea lo que fuere, las solas combinaciones que tienen posibilidad de
formarse, son aquellas en que uno de los elementos por lo menos, es uno
de estos átomos escogidos por nuestra voluntad. Según esto, es
evidente que entre ellos se encuentra lo que llamé antes la buena
combinación. Puede ser que allí haya un medio de atenuar lo que tenía de paradójica la hipótesis primitiva. Otra observación.
No sucede nunca que el trabajo subconsciente nos provea, todo hecho, el
resultado de un cálculo un poco largo, en el que tenemos que aplicar
reglas fijas. Se podría creer que el yo subconsciente, automático,
es particularmente apto para este género de trabajo, es en cierta manera
exclusivamente mecánico. Parece que pensando por la noche en los
factores de una multiplicación, se podría esperar el producto
hecho al despertar, o también un cálculo algebraico; una verificación, por ejemplo, podría hacerse inconscientemente. Esto es falso,
como prueba la observación. Todo lo más que se puede esperar de
estas inspiraciones, frutos del trabajo subconsciente, son los puntos de
partida para cálculos parecidos; en cuanto a los cálculos mismos
hace falta hacerlos en el segundo período de trabajo consciente, el que
sucede a la inspiración y en el que se verifican los resultados de esta
inspiración, del que se sacan las consecuencias. Las reglas de estos cálculos son estrictas y complicadas; exigen disciplina, atención,
voluntad y, por consiguiente, consciencia. En el yo subconsciente reina, por
el contrario, lo que yo llamaría la libertad, si se puede dar este
nombre a la simple ausencia de disciplina y al desorden nacido del azar.
Solamente este desorden permite los acoplamientos inesperados.

Haré una última observación. Cuando he expuesto antes algunas
observaciones personales, he hablado de una noche de excitación en que
trabajaba contra mi voluntad; los casos en que esto sucede son frecuentes y
no es necesario que la actividad cerebral anormal esté causada por un
excitante físico como el que he citado. Pues bien, me parece que en ese caso asiste uno
mismo a su propio trabajo inconsciente, que se ha vuelto perceptible a la
consciencia sobreexcitada y que no por esto cambia de naturaleza. Se da uno
cuenta entonces vagamente de lo que distingue a los dos mecanismos, o si se
quiere, los métodos de trabajo de los dos yos. Las observaciones psicológicas que he podido hacer de esta manera parecen confirmar en sus líneas generales los puntos de vista que acabo de presentar.

El interés de la cuestión es tan grande que no me arrepiento de
haberla sometido al lector.

\end{document}
