\documentclass[a4paper, 12pt]{article}

%%%%%%%%%%%%%%%%%%%%%%Paquetes
\usepackage[spanish]{babel}  
\usepackage[utf8]{inputenc}
\usepackage{tcolorbox}
\usepackage{cmbright}  %%%%%%% El tipo de letra
\usepackage{setspace}
\usepackage{amsmath}
\onehalfspacing  %%%%%%%%%%% Espacio y medio de interlineado
\parskip=1em  %%%%%%%%%%%% Separacion entre parrafos
%%%%%%%%%%%%%%%%%%%%%%



%%%%%%%%%%%%%%%%%
\title{Los Problemas Futuros de la Matemática}
\author{D. Hilbert}
\date{}
%%%%%%%%%%%%%%%%%

\begin{document}

\begin{tcolorbox}[colback=blue!5!white,colframe=blue!75!black]

\vspace{-1.8cm}
\textbf \maketitle

\end{tcolorbox}

\bigskip


¿Quién de nosotros no se alegraría de levantar el velo tras el que se oculta el futuro; de echar una mirada a los próximos avances de nuestra ciencia y a los secretos de su desarrollo durante los siglos futuros? ¿Cuáles serán los objetivos concretos por los que se esforzarán las mejores mentes matemáticas de las generaciones venideras? ¿Qué nuevos métodos y nuevos hechos descubrirán las nuevas centurias en el amplio y rico campo del pensamiento matemático? 

La historia nos enseña que hay continuidad en el desarrollo de la ciencia. Sabemos que cada época tiene sus propios problemas, que la época siguiente o bien resuelve o bien desecha por estériles y reemplaza por otros nuevos. Si tuviésemos una idea del desarrollo probable del conocimiento matemático en el futuro inmediato, deberíamos dejar pasar ante nuestras mentes las preguntas no resueltas y examinar los problemas que la ciencia de hoy plantea y cuya solución esperamos del futuro. El día de hoy, a caballo entre dos siglos, me parece muy adecuado para hacer una revisión semejante de los problemas. Pues el cierre de una gran época no sólo nos invita a mirar hacia el pasado, sino que también dirige nuestros pensamientos hacia el futuro desconocido. 

No se puede negar la profunda importancia de algunos problemas para el avance de la ciencia matemática en general y el importante papel que desempeñan en la obra del investigador individual. Siempre que una rama de la ciencia ofrece abundancia de problemas, está viva: una falta de problemas pronostica la extinción o el cese de un desarrollo independiente. De la misma forma que toda empresa humana persigue ciertos objetivos, también la investigación matemática requiere sus problemas. Mediante la solución de problemas es como se curte la fortaleza del investigador: éste encuentra nuevos métodos y nuevas perspectivas, y alcanza un horizonte más amplio y más libre. 

Es difícil, y a menudo imposible, juzgar correctamente el valor de un problema por adelantado, pues la recompensa final depende de lo que gana la ciencia a partir del problema. En cualquier caso, podemos preguntar si existen o no criterios generales que marquen un buen problema matemático. Un viejo matemático francés decía: «No puede considerarse completa una teoría matemática hasta que no se haya hecho tan clara que se pueda explicar al primer hombre que encontremos por la calle». Esta claridad y facilidad de comprensión en la que aquí se insiste para una teoría matemática, yo la pediría aún más para un problema matemático si quiere ser perfecto; pues lo que es claro y fácilmente comprensible nos atrae, y lo complicado nos repele. 

Más aún, un problema matemático debería ser difícil para que nos atraiga, aunque no completamente inaccesible, no sea que frustre nuestros esfuerzos. Debería ser para nosotros una señal indicadora en los tortuosos senderos hacia las verdades ocultas, que finalmente nos recompensan con el placer de una solución satisfactoria. 

Los matemáticos de los siglos pasados estaban acostumbrados a dedicarse a la solución de difíciles problemas individuales con un celo apasionado. Conocían el valor de los problemas difíciles. Simplemente les recuerdo el «problema de la línea de descenso más rápido», propuesto por Johann Bemoulli. La experiencia nos enseña, explicaba Bemoulli al anunciar públicamente este problema, que las mentes ilustres se ven llevadas a esforzarse por el avance de la ciencia sin más que ponerse ante problemas difíciles y al mismo tiempo útiles, con lo que esperaba ganarse el agradecimiento del mundo matemático siguiendo el ejemplo de hombres como Mersenne, Pascal, Fermat, Viviani y otros al presentar ante los distinguidos analistas de su época un problema en el cual pudieran poner a prueba, como piedra de toque, el valor de sus métodos y medir su fortaleza. El cálculo de variaciones debe su origen a este problema de Bernoulli y a otros problemas similares. 

Como es bien sabido, Fermat ha afirmado, que la ecuación diofántica 
$$
x^n+y^n=z^n
$$
($x$, $y$ y $z$ enteros) es irresoluble ---excepto en algunos casos evidentes---. El 
intento de demostrar esta imposibilidad ofrece un ejemplo sorprendente del efecto inspirador que un problema muy especial y aparentemente sin importancia puede tener en ciencia. Pues Kummer, espoleado por el problema de Fermat, se vio llevado a la introducción de los números ideales  y al descubrimiento de la ley de la descomposición unívoca de los números de un campo ciclotómico en factores primos ideales ---una ley que, hoy, en su generalización a cualquier campo algebraico por Dedekind y Kronecker, está en el centro de la moderna teoría de números y cuya importancia se extiende mucho más allá de las fronteras de la teoría de números y entra en el dominio del álgebra y la teoría de funciones. 

Para hablar de un área de investigación muy diferente, les recuerdo el problema de los tres cuerpos. Los fructíferos métodos, y los principios de largo alcance que Poincaré ha introducido en mecánica celeste y que son hoy reconocidos y aplicados en astronomía práctica, se deben al hecho de que él se propuso tratar de nuevo ese difícil problema y acercarse más a una solución.


Los dos últimos problemas mencionados ---el de Fermat y el problema de los tres cuerpos--- nos parecen casi polos opuestos: el primero, una libre invención de la pura razón, perteneciente al área de la teoría de números abstracta; el segundo, impuesto por la astronomía y necesario para una comprensión de los más simples fenómenos fundamentales de la Naturaleza.  Pero con frecuencia sucede también que el mismo problema especial  encuentra aplicación en las ramas más dispares del conocimiento matemático. Así, por ejemplo, el problema de la línea más corta juega un papel clave históricamente importante en los fundamentos de la geometría, en la teoría de líneas y superficies, en mecánica y en el cálculo de variaciones. ¡Y de qué forma tan convincente ha presentado F. Klein, en su trabajo sobre el icosaedro, la importancia del problema de los poliedros regulares en geometría elemental, en teoría de grupos, en la teoría de  ecuaciones y en la de ecuaciones diferenciales lineales! 


Para arrojar luz sobre la importancia de ciertos problemas, puedo también mencionar a Weierstrass, quien hablaba de la buena fortuna que  tuvo al encontrar al inicio de su carrera científica un problema tan importante en el que trabajar como era el problema de inversión de Jacobi. 

Habiendo recordado la importancia general de problemas en matemá\-ticas, volvamos a la cuestión de las fuentes de las que esta ciencia obtiene sus problemas. Por supuesto, los primeros y más viejos problemas  en cada rama de las matemáticas derivan de la experiencia y son sugeridos por el mundo de los fenómenos externos. Incluso las reglas del cálculo con enteros deben haber sido descubiertas de esta manera en una etapa inferior de la civilización humana, igual que los niños de hoy aprenden la aplicación de dichas leyes por métodos empíricos. Lo mismo es cierto de los primeros problemas de la geometría, los problemas que nos ha legado la Antigüedad, tales como la duplicación del cubo y la cuadratura del círculo; y también los más antiguos problemas en la teoría de la solución de ecuaciones numéricas, en la teoría de curvas y en el cálculo diferencial e integral, en el cálculo de variaciones, la teoría de las series de Fourier, y en la teoría del potencial ---por no hablar de la adicional abundancia de problemas propiamente pertenecientes a la mecánica, la astronomía y la física. 
 
 Pero en el desarrollo posterior de una rama de las matemáticas, la mente humana, animada por los éxitos de sus soluciones, se hace consciente de su independencia. Por medio de combinación lógica, generalización, especialización, separando y recogiendo ideas de manera afortunada ---a menudo sin influencia apreciable del exterior--- desarrolla nuevos y fructíferos problemas por sí misma, y entonces es ella misma la que aparece como el interrogador real. Así surgieron el problema de los números primos y los demás problemas de la teoría de números, la teoría de ecuaciones de Galois, la teoría de invariantes algebraicos, la teoría de funciones abelianas y automorfas; de hecho, casi todas las más bellas cuestiones de la aritmética y la teoría de funciones modernas aparecen de esta manera. Al mismo tiempo, mientras la potencia creativa de la pura razón está en acción, el mundo exterior entra de nuevo en juego, nos obliga a nuevas preguntas procedentes de la experiencia real, abre nuevas ramas de las matemáticas; y mientras tratamos de conquistar estos nuevos  campos de conocimiento para el dominio del pensamiento puro, con frecuencia encontramos las respuestas a viejos problemas no resueltos y así  avanzamos también con más acierto en las viejas teorías. Y me parece que las numerosas y sorprendentes analogías, y esa armonía aparentemente preestablecida que el matemático percibe tan a menudo en las cuestiones, métodos e ideas de las diversas ramas de su ciencia, tienen su origen  en este intercambio siempre recurrente entre pensamiento y experiencia.
 
 
 
Queda por discutir brevemente qué requisitos generales pueden establecerse justamente para la solución de un problema matemático. Antes  de nada, debería decir esto: que debe ser posible establecer la corrección de la solución por medio de un número finito de pasos basados en un número finito de hipótesis que están implicadas en el enunciado del problema 
y que deben ser exactamente formuladas. Este requisito de deducción lógica por medio de un número finito de procesos es simplemente el requisito de rigor en el razonamiento. En realidad, el requisito de rigor, que se ha convertido en un tópico en matemáticas, corresponde a una necesidad filosófica universal de nuestro entendimiento; por otra parte, solo satisfaciendo este requisito se contenta el pensamiento y alcanza su pleno  efecto la provocación del problema. Un problema nuevo, especialmente  cuando procede del mundo exterior de la experiencia, es como una rama joven, que crece y da fruto sólo cuando es injertada cuidadosamente y de acuerdo con las reglas estrictas de la horticultura en el tallo viejo, los logros ya establecidos de nuestra ciencia matemática. 

Es un error creer que el rigor en la demostración es enemigo de la simplicidad. Por el contrario, encontramos confirmado por numerosos ejemplos que el método riguroso es al mismo tiempo el más simple y el más fácilmente comprendido. El esfuerzo mismo por el rigor nos obliga a descubrir métodos de demostración más simples. También conduce frecuentemente a métodos que son más susceptibles de desarrollo que los viejos métodos de menos rigor. Así, la teoría de curvas algebraicas experimentó una notable simplificación y alcanzó una mayor unidad por medio de los métodos más rigurosos de la teoría de funciones y la introducción consistente de artificios trascendentes. Además, la demostración de que las cuatro operaciones aritméticas elementales, así como la diferenciación y la integración término a término, pueden aplicarse a series de potencias, y el reconocimiento de la utilidad de las series de potencias como resultado de esta demostración, contribuyeron materialmente a la simplificación de todo el análisis, particularmente la teoría de eliminación y la teoría de ecuaciones diferenciales, y también de las demostraciones de existencia exigidas en dichas teorías. Pero el ejemplo más sorprendente de mi afirmación es el cálculo de variaciones. El tratamiento de la primera y segunda variación de integrales definidas requería en parte cálculos extraordinariamente complicados, y los procesos aplicados por los viejos matemáticos carecían del rigor necesario. Weierstrass nos mostró el camino hacia un fundamento nuevo y seguro del cálculo de variaciones. Al final de mi conferencia mostraré brevemente, mediante los ejemplos de la integral simple y doble, cómo este camino conduce inmediatamente a una sorprendente simplificación del cálculo de variaciones. Pues en la demostración de los criterios necesarios y suficientes para la ocurrencia de un máximo y un mínimo, podemos prescindir por completo del cálculo de la segunda variación y, de hecho, de parte del aburrido razonamiento conectado con la primera variación ---por no hablar del avance que supone la eliminación de la restricción a variaciones para las que los coeficientes diferenciales de la función varían sólo ligeramente. 


Aun insistiendo en el rigor en la demostración como requisito para una solución perfecta de un problema, me gustaría, por otra parte, oponerme a la opinión de que sólo los conceptos del análisis, o incluso los de la aritmética sola, son susceptibles de un tratamiento completamente riguroso. Considero esta opinión, ocasionalmente defendida por hombres eminentes, completamente errónea. Tal interpretación unilateral del requisito de rigor nos llevaría pronto a la ignorancia de todos los conceptos que surgen de la geometría, la mecánica y la física, a una obstrucción del flujo de nuevo material procedente del mundo exterior, y en definitiva, de hecho, como última consecuencia, al rechazo de las ideas del continuo y de los números irracionales. ¡Qué nervio importante, vital para ciencia matemática, se seccionaría separando la geometría y la física matemática! Por el contrario, creo que donde quiera que surjan las ideas matemáticas, ya sea del lado de la teoría del conocimiento, o en geometría, o de las teorías de la ciencia natural o la ciencia física, se plantea para las matemáticas el problema de investigar los principios subyacentes a estas ideas y establecerlas sobre un sistema de axiomas simple y completo, de modo que la exactitud de las nuevas ideas y su aplicabilidad a la deducción no sea en ninguna medida inferior a la de los viejos conceptos aritméticos. 

A nuevos conceptos corresponden, necesariamente, nuevos símbolos. Escogemos éstos de tal forma que nos recuerdan los fenómenos que fueron la ocasión de la formación de los nuevos conceptos. Así, las figuras geométricas son signos o símbolos mnemotécnicos de intuición espacial, y como tales son utilizados por todos los matemáticos. ¿Quién no utiliza siempre junto con la doble desigualdad $a > b > c$ la imagen de tres puntos seguidos en una línea recta como imagen geométrica de la idea «entre»? ¿Quién no hace uso de dibujos de segmentos y rectángulos encerrados uno dentro de otro cuando se requiere demostrar con perfecto rigor un difícil teorema sobre la continuidad de funciones o la existencia de puntos de acumulación? ¿Quién prescindiría de la figura del triángulo, el círculo con su centro, o la intersección de tres ejes perpendiculares? ¿O quién abandonaría la representación del campo vectorial, o la imagen de una familia de curvas o superficies con su envolvente que juega un papel tan importante en geometría diferencial, en la teoría de ecuaciones diferenciales, en los fundamentos del cálculo de variaciones, y en otras ciencias puramente matemáticas? Los símbolos aritméticos son figuras escritas y las figuras geométricas son fórmulas dibujadas; y ningún matemático podría ahorrarse estas fórmulas dibujadas, como no podría prescindir en los cálculos de la introducción y eliminación de paréntesis o del uso de otros signos analíticos. 

El uso de símbolos geométricos como medio de demostración estricta presupone el conocimiento exacto y el dominio completo de los axiomas que yacen en el fundamento de dichas figuras; y para que estas figuras geométricas puedan ser incorporadas al tesoro general de los símbolos matemáticos es necesaria una rigurosa investigación axiomática de su contenido conceptual. Igual que al sumar dos números uno debe colocar los dígitos uno debajo de otro en el orden correcto de modo que sólo las reglas del cálculo, i.e., los axiomas de la aritmética, determinan el uso correcto de los dígitos, así también el uso de los símbolos geométricos está determinado por los axiomas de los conceptos geométricos y sus combinaciones.


 El acuerdo entre pensamiento geométrico y aritmético se manifiesta también en que nosotros no nos remontamos habitualmente hasta los axiomas en la cadena de razonamientos en las discusiones aritméticas, ni tampoco en las geométricas. Por el contrario, especialmente al atacar por primera vez un problema, aplicamos una combinación rápida, inconsciente, no absolutamente segura, confiando en una cierta sensación aritmética del comportamiento de los símbolos aritméticos, de los que podemos prescindir tan poco en aritmética como de la imaginación geométrica en geometría. Como ejemplo de una teoría aritmética que opera rigurosamente con ideas y símbolos geométricos, puedo mencionar el trabajo de Minkowski,  \textit{Geometrie der Zahlen}.
 
Éste puede ser el lugar para algunos comentarios sobre las dificultades que pueden ofrecer los problemas matemáticos y los medios de superarlas. 

Si no tenemos éxito en resolver un problema matemático, la razón suele consistir en que no hemos sabido reconocer el punto de vista más general desde el que el problema que tenemos ante nosotros aparece como tan solo un eslabón de una cadena de problemas afines. Una vez encontrado este punto de vista, no sólo es frecuente que el problema resulte más accesible a nuestra investigación sino que al mismo tiempo entramos en posesión de un método que también es aplicable a problemas afines. La introducción de caminos de integración complejos por parte de Cauchy y de la noción de ideales en la teoría de números por parte de Kummer pueden servir como ejemplos. Esta manera de encontrar métodos generales es ciertamente la más práctica y más segura; pues quien busca métodos sin tener en mente un problema definido busca casi siempre en vano. 

Al trabajar con problemas matemáticos, la especialización juega, creo yo, un papel aún más importante que la generalización. Quizá en la mayoría de los casos en que buscamos infructuosamente la respuesta a una pregunta, la causa del fracaso reside en el hecho de que problemas más sencillos y más fáciles que el que tenemos entre manos han sido o bien resueltos de manera incompleta o no resueltos en absoluto. Todo depende, entonces, de encontrar esos problemas más fáciles y de resolverlos por medio de artificios tan perfectos como sea posible y de conceptos suceptibles de generalización. Esta regla es una de las palancas más importantes para superar las dificultades matemáticas; y me parece que se utiliza casi siempre, aunque quizá de forma inconsciente.
 
En ocasiones sucede que buscamos la solución con hipótesis insuficientes o en un sentido incorrecto, y por esta razón no tenemos éxito. Entonces surge el problema de demostrar la imposibilidad de encontrar la solución bajo las hipótesis dadas, o en el sentido contemplado. Tales demostraciones de imposibilidad fueron efectuadas por los antiguos, por ejemplo, cuando demostraron que la razón entre la hipotenusa y un lado de un triángulo rectángulo isósceles es irracional. En las matemáticas posteriores, la cuestión de la imposibilidad de ciertas soluciones desempeña una parte destacada; y de este modo percibimos que los problemas viejos y difíciles, tales como la demostración del axioma de las paralelas, la cuadratura del círculo, o la solución por radicales de las ecuaciones de quinto grado, han encontrado finalmente soluciones completamente satisfactorias y rigurosas, aunque en un sentido diferente al originalmente pretendido. Probablemente es este hecho notable, junto con otras razones filosóficas, lo que da lugar a la convicción (que comparten todos los matemáticos, pero que nadie ha sustentado todavía con una demostración) en que todo problema matemático definido debe ser necesariamente susceptible de un acuerdo exacto, ya sea en forma de una respuesta real a la cuestión preguntada, ya sea por la demostración de la imposibilidad de su solución y con ello el fracaso necesario de todos los intentos. Tomemos cualquier problema definido no resuelto, tal como la irracionalidad de la constante $C$ de Euler-Mascheroni o la existencia de un número infinito de números primos de la forma $2^n + 1$. Por inabordables que estos problemas nos puedan parecer, y por impotentes que nos sintamos ante ellos, tenemos de todas formas la firme convicción de que sus soluciones deben seguirse por un número finito de procesos puramente lógicos. 

¿Es este axioma de la resolubilidad de cualquier problema una característica propia del pensamiento matemático, o es posible que sea una ley general inherente en la naturaleza de la mente, una creencia en que todas las preguntas que plantea deben ser susceptibles de respuesta? Pues en otras ciencias también encontramos viejos problemas que han sido zanjados de la manera más satisfactoria y más útil para la ciencia por la demostración de su imposibilidad. Cito el problema del movimiento perpetuo. Después de buscar infructuosamente la construcción de una máquina de movimiento perpetuo, los científicos investigaron las relaciones que deben existir entre las fuerzas de la naturaleza para que tal máquina sea imposible; y esta pregunta invertida llevó al descubrimiento de la ley de la conservación de la energía, que, una vez más, explicó la imposibilidad del movimiento perpetuo en el sentido originalmente pretendido. 

Esta convicción en la resolubilidad de todo problema matemático es un poderoso incentivo para el trabajador. Oímos dentro de nosotros la llamada perpetua: Existe el problema. Busca su solución. Puedes encontrarla por la pura razón, pues en matemáticas no hay \textit{ignorabimus}. 

El suministro de problemas en matemáticas es inagotable, y en cuanto un problema es resuelto otros muchos vienen a ocupar su lugar. Permítanme en lo que sigue, por tentativo que pueda ser, mencionar problemas concretos y definidos, extraídos de varias ramas de las matemáticas, a partir de cuya discusión puede esperarse un avance de la ciencia. 

Consideremos los principios del análisis y la geometría. Los logros más sugerentes y notables de la última centuria en este campo son, así me parece, la formulación aritmética del concepto del continuo en los trabajos de Cauchy, Bolzano y Cantor, y el descubrimiento de la geometría no euclídea por Gauss, Bolyai y Lobachesky. Por consiguiente, dirigiré primero su atención a algunos problemas pertenecientes a estos campos. 



\subsection*{1. El problema de Cantor del número cardinal del continuo }

Se dice que dos sistemas, i.e. dos conjuntos de números reales ordinarios o de puntos, son equivalentes (según Cantor) o del mismo \textit{número cardinal}, si pueden ponerse en una relación mutua tal que a todo número del primer conjunto corresponde uno y sólo un número definido del segundo. Las investigaciones de Cantor sobre tales «conjuntos de  puntos» sugieren un teorema muy plausible, que de todas formas, a pesar de los mayores esfuerzos, nadie ha conseguido demostrar. Éste es el teorema. 

\begin{quote} \small

Todo sistema de infinitos números reales, i.e., todo conjunto infinito de números (o puntos), es o bien equivalente al conjunto de los números naturales, $1, 2, 3, \dots$ o bien equivalente al conjunto de todos los números reales y por lo tanto al continuo, es decir, a los puntos de una recta; \textit{con respecto a la equivalencia existen, por consiguiente, sólo dos conjuntos de números, el conjunto numerable y el continuo. }

\end{quote}

De este teorema se seguiría inmediatamente que el continuo tiene el siguiente número cardinal más allá del cardinal del conjunto numerable; la demostración de este teorema constituiría, por lo tanto, un nuevo puente entre el conjunto numerable y el continuo. 

Permítanme mencionar otro enunciado muy notable de Cantor que está\linebreak en la más íntima conexión con el teorema mencionado y que, quizá, ofrezca la clave para su demostración. Se dice que un sistema de números reales es ordenado si para cada dos números del sistema está determinado cuál es anterior y cuál es posterior, y si al mismo tiempo esta determinación es de un tipo tal que, si $a$ es anterior a $b$ y $b$ es anterior a $c$, entonces $a$ es siempre anterior a $c$. La disposición natural de los números de un sistema se define como aquella en la que el más pequeño precede al más grande. Pero existen, como es fácil ver, otras infinitas maneras de disponer un sistema. 

Si pensamos en una disposición definida de números y seleccionamos de entre ellos un sistema concreto de dichos números, un denominado subsistema o subconjunto, también se demostrará que este subsistema está ordenado. Ahora Cantor considera un tipo particular de conjunto ordenado que él designa como conjunto bien ordenado y que está caracterizado de esta manera: que no sólo en el propio conjunto sino también en todo subconjunto existe un primer número. El sistema de enteros $1,2, 3,\dots$ en su orden natural es evidentemente un conjunto bien ordenado. Por el contrario, el sistema de todos los números reales, i.e., el continuo en su orden natural, no es evidentemente bien ordenado. En efecto, si consideramos los puntos de un segmento de una línea recta, con su punto inicial excluido, como nuestro conjunto parcial, éste no tendrá primer elemento. 

Surge ahora la pregunta de si la totalidad de los números puede disponerse de alguna otra manera tal que todo subconjunto pueda tener un primer elemento, i.e., si puede considerarse el continuo como un conjunto bien ordenado ---una pregunta que Cantor piensa que debe responderse afirmativamente---. Creo que es muy deseable obtener una demostración directa de este notable enunciado de Cantor, quizá dando realmente una disposición de números tal que en todo sistema parcial pueda señalarse un primer número. 


\subsection*{2. La compatibilidad de los axiomas de la aritmética}
 
Cuando nos ponemos a investigar los fundamentos de una ciencia, debemos establecer un sistema de axiomas que contengan una descripción exacta y completa de las relaciones que existen entre las nociones elementales de dicha ciencia. Los axiomas así establecidos son al mismo tiempo las definiciones de dichas nociones elementales; y ningún enunciado dentro del ámbito de la ciencia cuyo fundamento estamos poniendo a prueba debe considerarse correcto a menos que pueda derivarse de aquellos axiomas por medio de un número finito de pasos lógicos. En una consideración más próxima surge la cuestión de \textit{si es posible que, de alguna manera, ciertos enunciados de axiomas individuales dependan unos de otros y, por consiguiente, los axiomas tengan ciertas partes en común, que deberían aislarse si se quiere llegar a un sistema de axiomas que sean completamente independientes uno de otro}. 
 
Pero por encima de todo deseo señalar la siguiente como la más importante entre las numerosas preguntas que pueden plantearse con respecto a los axiomas: \textit{Demostrar que no son contradictorios, es decir, que un número finito de pasos lógicos basados en ellos nunca pueden llevar  a resultados contradictorios}. 
 
En geometría, la demostración de la compatibilidad de los axiomas puede efectuarse construyendo un campo de números adecuado, tal que relaciones análogas entre los números de este campo corresponden a los axiomas geométricos. Cualquier contradicción en las deducciones hechas a partir de los axiomas geométricos debe por lo tanto ser reconocible en la aritmética de este campo de números. De esta manera, la deseada demostración de la compatibilidad de los axiomas geométricos se hace depender del teorema de la compatibilidad de los axiomas aritméticos. 

Por el contrario, es necesario un método directo para la demostración de la compatibilidad de los axiomas aritméticos. Los axiomas de la aritmética no son en esencia otra cosa que las reglas del cálculo conocidas, con la adición del axioma de continuidad. Yo los recogí recientemente y al hacerlo reemplacé el axioma de continuidad por dos axiomas más simples, a saber, el bien conocido axioma de Arquímedes y un nuevo axioma que en esencia es como sigue: que los números forman un sistema de objetos que no es susceptible de posterior extensión, en tanto que sean válidos todos los demás axiomas (axioma de completitud). Estoy convencido de que debe ser posible encontrar una demostración directa de la compatibilidad de los axiomas aritméticos por medio de un estudio cuidadoso y una modificación adecuada de los métodos de razonamiento conocidos en la teoría de los números irracionales. 

Para mostrar la importancia del problema desde otro punto de vista, añado la siguiente observación: Si a un concepto se le asignan atributos contradictorios, yo digo que \textit{matemáticamente el concepto no puede existir}. Así, por ejemplo, un número real cuyo cuadrado es $-1$ no existe matemáticamente. Pero si puede demostrarse que los atributos asignados al concepto nunca pueden llevar a una contradicción por la aplicación de un número finito de pasos lógicos, entonces yo digo que la existencia matemática del concepto (por ejemplo, un número o una función que satisface determinadas propiedades) está probada con ello. En el caso presente, donde nos interesamos en los axiomas de los números reales en aritmética, la demostración de la compatibilidad de los axiomas es al mismo tiempo la demostración de la existencia del sistema completo de los números reales o del continuo. En realidad, cuando se haya conseguido la demostración completa de la compatibilidad de los axiomas, las dudas que en ocasiones se han expresado sobre la existencia del sistema completo de los números reales se convertirán en algo totalmente carente de fundamento. La totalidad de los números reales, i.e., el continuo según el punto de vista recién indicado, no es la totalidad de las posibles series de fracciones decimales, o de todas las posibles leyes de acuerdo con las cuales pueden proceder los elementos de una secuencia fundamental. Es más bien un sistema de objetos cuyas relaciones mutuas están gobernadas por los axiomas establecidos y para los que son verdaderas todas las proposiciones, y solo ésas, que pueden derivarse de los axiomas por un número finito de procesos lógicos. En mi opinión, el concepto del continuo es estricta y lógicamente sostenible sólo en este sentido. Me parece, de hecho, que esto también corresponde mejor a lo que la experiencia y la intuición nos dicen. El concepto del continuo, o incluso el del sistema de todas las funciones, existe, entonces, exactamente en el mismo sentido que el sistema de números racionales, por ejemplo, o que las clases más altas de números y números cardinales de Cantor. Pues estoy convencido de que la existencia de los últimos, igual que la del continuo, puede demostrarse en el sentido que he descrito; a diferencia del sistema de todos los números cardinales o de todos los \textit{alefs} de Cantor, para los que, como puede demostrarse, no puede establecerse un sistema de axiomas compatible en mi sentido. Cada uno de estos dos sistemas es, por consiguiente, matemáticamente no existente, de acuerdo con mi terminología. 


Del campo de los fundamentos de la geometría me gustaría mencionar el siguiente problema


\subsection*{3. La igualdad de los volúmenes de dos tetraedros de la misma base y la misma altura}

 
En dos cartas a Gerling, Gauss expresa su pesar porque ciertos teoremas de la geometría de sólidos dependen del método de \textit{exhausción}, i.e., en terminología moderna, del axioma de continuidad (o del axioma de Arquímedes). Gauss menciona en concreto el teorema de Arquímedes, según el cual la relación entre volúmenes de pirámides triangulares de la misma altura es igual a la relación entre sus bases. Ahora bien, el problema análogo en el plano ha sido resuelto. Gerling también consiguió demostrar la igualdad del volumen de poliedros simétricos dividiéndolos en partes congruentes. De todas formas, me parece probable que una demostración general de este tipo para el problema de Euclides recién mencionado sea imposible, y nuestra tarea debería consistir en dar una demostración rigurosa de su imposibilidad. Ésta se obtendría en cuanto consiguiéramos \textit{especificar dos tetraedros de iguales bases e iguales alturas que no puedan dividirse de ningún modo en tetraedros congruentes, y que no puedan ser combinados con tetraedros congruentes para formar dos poliedros que pudieran dividirse en tetraedros congruentes.}

\subsection*{4. El problema de la línea recta como la distancia más corta entre dos puntos}
 
Otro problema relativo a los fundamentos de la geometría es éste: si de entre los axiomas necesarios para establecer la geometría euclídea ordinaria excluimos el axioma de las paralelas, o suponemos que no se satisface, pero retenemos todos los demás axiomas, obtenemos, como es bien sabido, la geometría de Lobachevsky (geometría hiperbólica). Podemos decir por consiguiente que ésta es una geometría que está próxima a la geometría euclídea. Si además exigimos que no se satisfaga aquel axioma por el que, dados tres puntos en una línea recta, uno y sólo uno yace entre los otros dos, obtenemos la geometría de Riemann (elíptica), de modo que esta geometría parece ser la más próxima después de la de Lobachevsky. Si queremos llevar a cabo una investigación similar con respecto al axioma de Arquímedes, debemos considerar que éste no se satisface, y llegamos con ello a las geometrías «no arquimedianas» que han sido investigadas por Veronese y yo mismo. Ahora surge la cuestión más general de si a partir de otros puntos de vista sugestivos no pueden imaginarse geometrías tales que, con igual derecho, permanezcan próximas a la geometría euclídea. Aquí me gustaría dirigir su atención a un teorema que de hecho ha sido empleado por muchos autores como una definición de línea recta, viz., que la línea recta es la distancia más corta entre dos puntos. El contenido esencial de este enunciado se reduce al teorema de Euclides, según el cual en un triángulo la suma de dos lados es siempre mayor que el tercer lado ---un teorema que, como se ve fácilmente, trabaja solamente con conceptos elementales, i.e., con conceptos que se derivan directamente de los axiomas, y es por consiguiente más accesible a la investigación lógica---. Euclides demostró este teorema, con la ayuda del teorema del ángulo externo, sobre la base de los teoremas de congruencia. Ahora se demuestra inmediatamente que este teorema de Euclides no puede demostrarse solamente sobre la base de aquellos teoremas de congruencia que se relacionan con la aplicación de segmentos y ángulos, sino que es necesario uno de los teoremas sobre la congruencia de triángulos. Estamos preguntando entonces por una geometría en la que son válidos todos los axiomas de la geometría euclídea ordinaria, y en particular todos los axiomas de congruencia excepto el de la congruencia de triángulos (o todos excepto el teorema de la igualdad de los ángulos de la base en el triángulo isósceles), y en la que, además, se supone como axioma particular la proposición de que en todo triángulo la suma de dos lados es mayor que el tercero.

Se encuentra que tal geometría existe realmente y no es otra que la que Minkowski construyó en su libro, \textit{Geometrie der Zahlen}, y sobre la que basó sus investigaciones aritméticas. La de Minkowski es por consiguiente también una geometría que está próxima a la geometría euclídea ordinaria; está caracterizada en esencia por las siguientes estipulaciones:

\begin{enumerate}

\item  Los puntos que están a iguales distancias de un punto fijo $O$ yacen en una superficie cerrada convexa del espacio euclídeo ordinario que tiene $O$ como su centro.

\item Se dice que dos segmentos son iguales cuando uno puede ser superpuesto al otro por una traslación del espacio euclídeo ordinario.

\end{enumerate}

En la geometría de Minkowski el axioma de las paralelas es también válido. Estudiando el teorema de la línea recta como la distancia más corta entre dos puntos, yo llegué a una geometría en el que el axioma de las paralelas no es válido, mientras que se satisfacen todos los demás axiomas de la geometría de Minkowski. El teorema de la línea recta como la distancia más corta entre dos puntos y el teorema esencialmente equivalente de Euclides sobre los lados de un triángulo juegan un papel importante no sólo en la teoría de números, sino también en la teoría de superficies y en el cálculo de variaciones. Por esta razón, y porque creo que la investigación completa de las condiciones de validez de este teorema arrojarán nueva luz sobre la idea de distancia, así como sobre otras ideas elementales, por ejemplo, sobre la idea del plano, y la posibilidad de su definición por medio de la idea de la línea recta, \textit{me parece deseable la construcción y tratamiento sistemático de las geometrías posibles}. 


\subsection*{5. El concepto de Lie de grupo continuo de transformaciones sin la hipótesis de la diferenciabilidad de las funciones que definen el grupo} 

Es bien sabido que Lie, con la ayuda del concepto de grupo continuo de transformaciones, ha establecido un sistema de axiomas geométricos y, desde el punto de vista de su teoría de grupos, ha demostrado que este sistema de axiomas basta para la geometría. Pero puesto que Lie supone, en el mismo fundamento de su teoría, que las funciones que definen su grupo pueden ser diferenciadas, queda por decidir en el desarrollo de Lie si la hipótesis de la diferenciabilidad en conexión con la cuestión relativa a los axiomas de la geometría es realmente inevitable, o si quizá no aparece más bien como una consecuencia del concepto de grupo y los demás axiomas geométricos. Esta consideración, así como algunos otros problemas en conexión con los axiomas aritméticos, nos pone ante la pregunta más general: hasta qué punto podemos aproximamos en nuestras investigaciones al concepto de Lie de grupo continuo de transformaciones sin la hipótesis de la diferenciabilidad de las funciones. 


Lie define un grupo continuo finito de transformaciones como un sistema de transformaciones 
$$
x_i'=f_i(x_1, \dots,x_n;a_1, \dots, a_r) \quad (i=1, \dots, n)
$$
que tiene la propiedad de que cualesquiera dos transformaciones del sistema arbitrariamente escogidas, como 
\begin{equation}\nonumber
\begin{split}
x_i'=f_i(x_1, \dots,x_n;a_1, \dots, a_r) \\
x_i''=f_i(x'_1, \dots,x'_n;b_1, \dots, b_r) 
\end{split}
\end{equation}
aplicadas sucesivamente dan como resultado una transformación que también pertenece al sistema, y que es por lo tanto expresable en la forma 
$$
x_i''=f_i\{f_1(x,a), \dots, f_n(x,a);b_1, \dots, b_r\}=f_i(x_1, \dots, x_n;c_1, \dots, c_r)
$$
donde  $c_1, \dots, c_r$ son ciertas funciones de $a_1, \dots, a_r$ y $b_1, \dots, b_r$.

 La propiedad de grupo encuentra así su plena expresión en un sistema de ecuaciones funcionales y por sí misma no impone restricciones adicionales sobre las funciones $f_1, \dots, f_n; c_1, \dots, c_r$.
 Pero el tratamiento posterior de Lie de dichas ecuaciones funcionales,   viz., 
   la derivación de las bien conocidas ecuaciones diferenciales fundamentales, supone necesariamente la continuidad y diferenciabilidad de las funciones que definen el grupo. 
   
Con respecto a la continuidad: este postulado será ciertamente retenido por el momento \mbox{---aunque} sólo sea con vistas a las aplicaciones geométricas y aritméticas, en las que la continuidad de las funciones en cuestión aparece como una consecuencia del axioma de continuidad---. Por otra parte, la diferenciabilidad de las funciones que definen al grupo contiene un postulado que, en los axiomas geométricos, sólo puede expresarse de una manera más bien forzada y complicada. De aquí surge la cuestión de si, a través de la introducción de nuevas variables y parámetros apropiados, el grupo puede transformarse siempre en otro cuyas funciones definitorias son diferenciables; o si, al menos con la ayuda de ciertas hipótesis simples, es posible una transformación en grupos que admiten métodos de Lie. Una reducción a grupos analíticos es siempre posible, según un teorema anunciado por Lie pero demostrado por primera vez por Schur, cuando el grupo es transitivo y se supone la existencia de la primera y algunas segundas derivadas de las funciones que definen el grupo.


 
En el caso de grupos infinitos, la investigación de la cuestión correspondiente es, creo yo, también de interés. Más aún, nos vemos llevados así al vasto e interesante campo de las ecuaciones funcionales que hasta ahora ha sido investigadas normalmente sólo bajo la hipótesis de la diferenciabilidad de las funciones implicadas. En particular, las ecuaciones funcionales tratadas por Abel con tanto ingenio, las ecuaciones en diferencias, y otras ecuaciones que aparecen en la literatura matemática, no implican directamente nada que necesite el requisito de la diferenciabilidad de las funciones acompañantes. En la búsqueda de ciertas demostraciones de existencia en el cálculo de variaciones yo llegué directamente a este problema: demostrar la diferenciabilidad de la función en consideración a partir de la existencia de una ecuación en diferencias. En todos estos casos surge el problema: \textit{¿hasta qué punto son verdaderas las afirmaciones que podemos hacer en el caso de funciones diferenciables bajo modificaciones adecuadas sin esta hipótesis?}  

Puede comentarse además que H. Minkowski, en su anteriormente mencionada \textit{Geometrie der Zahlen} parte de la ecuación funcional 
$$
f(x_1+y_1, \dots, x_n+y_n) \leq f(x_1, \dots, x_n)+f(y_1, \dots, y_n)
$$
y a partir de ésta consigue realmente demostrar la existencia de ciertos cocientes diferenciales para la función en cuestión. 

Por otra parte deseo resaltar el hecho de que existen ciertamente ecuaciones funcionales analíticas cuyas únicas soluciones son funciones no diferenciables. Por ejemplo, puede construirse una función no diferenciable continua y uniforme $\varphi(x)$   que representa la única solución de las dos ecuaciones funcionales 
$$
\varphi(x+\alpha)- \varphi(x)=f(x), \qquad \varphi(x+\beta)- \varphi(x)=0
$$ 
 donde $\alpha$ y $\beta$ son dos números reales, y $f( x)$  denota, para todos los valores reales de $x$, una función  uniforme  analítica y regular. Tales funciones se obtienen de la manera más simple por medio de series trigonométricas por un proceso similar al utilizado por Borel (según un anuncio reciente de Picard) para la construcción de una solución no analítica, y doblemente periódica, de una cierta ecuación en derivadas parciales analítica. 
   
   
\subsection*{6. Tratamiento matemático de los axiomas de la física}


Las investigaciones sobre los fundamentos de la geometría sugieren el siguiente problema: tratar de la misma manera, por medio de axiomas, aquellas ciencias físicas en las que las matemáticas juegan un papel importante; en primera fila están la teoría de probabilidades y la mecánica. 

Con respecto a los axiomas de la teoría de probabilidades me parece deseable que su investigación lógica fuera acompañada por un desarrollo riguroso y satisfactorio del método de los valores medios en física matemática, y en particular en la teoría cinética de los gases. 

Se dispone de importantes investigaciones hechas por físicos sobre los fundamentos de la mecánica; remito a los escritos de Mach, Hertz, Boltzmann y Volkmann. Por consiguiente es muy deseable que la discusión de los fundamentos de la mecánica sea asumida también por matemáticos. Así, el trabajo de Boltzmann sobre los principios de la mecánica sugiere el problema de desarrollar matemáticamente los procesos de paso al límite, allí meramente indicados, que llevan desde la visión atomística a las leyes de movimiento de los medios continuos. Recíprocamente se podría tratar de derivar las leyes del movimiento de los cuerpos rígidos por un proceso de paso al límite a partir de un sistema de axiomas que dependen de la idea de condiciones continuamente variables de un  material que llena todo el espacio de forma continua, estando definidas  dichas condiciones por parámetros. Pues la cuestión de la equivalencia  de diferentes sistemas de axiomas es siempre de gran interés teórico. 

Si la geometría va a servir como modelo para el tratamiento de sistemas físicos, trataremos primero de incluir, mediante un pequeño número de axiomas, una clase lo más grande posible de fenómenos físicos, y de llegar luego poco a poco, añadiendo nuevos axiomas, a las teorías más especiales ---por donde quizá pueda derivarse un principio de subdivisión a partir de la profunda teoría de Lie de grupos de transformación infinitos---. Igual que hace en geometría, el matemático no tendrá meramente que tener en cuenta las teorías que se acercan a la realidad, sino  también todas las teorías lógicamente posibles. Siempre debe estar alerta para obtener una visión completa de todas las conclusiones derivables a partir del sistema de axiomas supuesto.

 Además, el matemático tiene el deber de verificar exactamente en  cada caso si los nuevos axiomas son compatibles con los anteriores. El físico, cuando desarrolla sus teorías, suele  encontrarse   obligado por los resultados de sus experimentos a hacer nuevas hipótesis, aunque en lo relativo a la compatibilidad entre las nuevas hipótesis y los viejos axiomas depende solamente de dichos experimentos o de una cierta intuición  física, una práctica que no es admisible en la construcción rigurosamente lógica de una teoría. La deseada demostración de la compatibilidad de todas las hipótesis también me parece de importancia, porque el esfuerzo por obtener tal demostración siempre nos obliga de manera más efectiva a una formulación exacta de los axiomas. 
  
Hasta aquí hemos considerado solamente cuestiones concernientes a los fundamentos de las ciencias matemáticas. En realidad, el estudio de los fundamentos de una ciencia es siempre particularmente atractivo, y la puesta a prueba de dichos fundamentos estará siempre entre los problemas más importantes del investigador. Weierstrass dijo en cierta ocasión que «el objetivo final a tener siempre en mente es llegar a una correcta comprensión de los fundamentos de la ciencia  ... 
  pero para hacer cualquier progreso en las ciencias es, por supuesto, indispensable el estudio de problemas concretos». De hecho, una comprensión completa de  sus teorías especiales es necesaria para el tratamiento exitoso de los fundamentos de la ciencia. Sólo el arquitecto que conoce su objetivo completamente y en detalle está en posición de establecer una base segura para una estructura. De modo que ahora nos dirigimos a los problemas especiales de las ramas separadas de las matemáticas y consideraremos  primero la aritmética y el álgebra. 
  
  
\subsection*{7. Irracionalidad y trascendencia de ciertos números }


Los teoremas aritméticos de Hermite sobre la función exponencial y su extensión por Lindemann merecen ciertamente la admiración de todas las generaciones de matemáticos. Pero inmediatamente se presenta la tarea de adentrarse más a lo largo del camino aquí iniciado, como ya ha hecho A. Hurwitz en dos artículos interesantes «\textit{Ueber arithmetische Eigenschaften gewisser transzendenter Funktionen}». Por consiguiente, me  gustaría esbozar una clase de problemas que, en mi opinión, deberían  abordarse a continuación. El hecho de que ciertas funciones trascendentes especiales, importantes para el análisis, tomen valores algebraicos para ciertos argumentos algebraicos nos parece particularmente notable y digno de una investigación completa. De hecho, esperamos que las funciones trascendentes tomen, en general, valores trascendentes incluso para argumentos algebraicos; y, aunque es bien sabido que existen funciones trascendentes enteras que incluso tienen valores racionales para  todos los argumentos algebraicos, seguiremos considerando altamente 
probable que la función exponencial $e^{i\pi z}$,
 por ejemplo, que evidentemente tiene valores algebraicos para todos los argumentos racionales $z$, tomará por otra parte siempre valores trascendentes para valores algebraicos irracionales del argumento $z$. Podemos dar también a este enunciado una forma geométrica, tal como sigue: 
 
 \begin{quote}\small
 
\textit{Si, en un triángulo isósceles, la razón del ángulo de la base al ángulo del vértice es algebraica pero no racional, entonces la razón entre base y lado es siempre trascendente}. 

\end{quote}

A pesar de la simplicidad de este enunciado y de su similaridad con los problemas resueltos por Hermite y Lindemann, considero muy difícil la demostración de este teorema; como también la demostración de que

\begin{quote} \small

\textit{La expresión $\alpha^\beta$,   para una base algebraica y un exponente algebraico irracional $\beta$, por ejemplo, el número $2^{\sqrt{i}}$  o $e^\pi=i^{-2i}$,  
    representa siempre un número trascendente o al menos un número irracional.} 

\end{quote}    
    
Es cierto que la solución de éstos y similares problemas debe conducimos a métodos enteramente nuevos y a una nueva idea sobre la naturaleza de números irracionales y trascendentes especiales. 

\subsection*{8. Problemas de números primos }

Progresos esenciales en la teoría de la distribución de números primos han sido realizados últimamente por Hadamard, de la Vallée-Pousin, Von Mangoldt y otros. Sin embargo, en el caso de la solución completa de los problemas planteados por el artículo de Riemann «\textit{Ueber die Anzahl der Primezahlen unter einer gegebenen Grosse}» aún queda por demostrar la corrección de un enunciado extraordinariamente importante de Riemann, viz., \textit{que los ceros de la función $\zeta(s)$  definida por la serie
$$
\zeta(s)= 1+ \frac{1}{2^s}+\frac{1}{3^s}+\frac{1}{4^s}+ \cdots
$$ 
tienen todos parte real $1/2$}, excepto los bien conocidos ceros reales enteros negativos. En cuanto esta demostración se haya establecido satisfactoriamente, el siguiente problema consistirá en comprobar con más exactitud la serie infinita de Riemann para el número de primos por debajo de un número dado y, especialmente, \textit{decidir si la diferencia entre el número de primos por debajo de un número $z$ y el logaritmo entero de $z$ se hacen de hecho infinitos de un orden no mayor que $1/2$ en $x$.}   Además, deberíamos determinar si la condensación ocasional de números primos que se ha advertido al contar primos es realmente debida a aquellos términos de la fórmula de Riemann que dependen de los primeros ceros complejos de la función $\zeta(s)$. 
   
   
Tras una exhaustiva discusión de la fórmula de números primos de Riemann, quizá podamos alguna vez estar en posición de alcanzar la solución rigurosa del problema de Goldbach, viz., si todo entero es expresable como la suma de dos números primos positivos; y además atacar la bien conocida cuestión de si existe un número infinito de pares de números primos que difieren en dos, o incluso el problema más general de si la ecuación diofántica lineal
$$
ax+by+c=0
$$
(con coeficientes enteros dados mutuamente coprimos) es siempre resoluble en números primos $x$ e $y$.

Pero el siguiente problema no me parece de menor interés y quizá sea de alcance aún más amplio: \textit{aplicar los resultados obtenidos para la distribución de números primos racionales a la teoría de la distribución de primos ideales en un campo de números k dado}, un problema que mira hacia el estudio de la función $\zeta_k(s)$ perteneciente el campo de números y definida por la serie
$$
\zeta_k(s)= \sum \frac{1}{n(j)^s}
$$
donde la suma se extiende a todos los ideales $j$ del dominio dado $k$, y $n(j)$ denota la norma del ideal $j$.

Puedo mencionar otros tres problemas especiales en teoría de números: uno sobre la leyes de reciprocidad, otro sobre ecuaciones diofánticas y un tercero tomado del dominio de las formas cuadráticas.


\subsection*{9. Demostración de la ley de reciprocidad más general en cualquier campo de números}


\textit{Para cualquier campo de números debe demostrarse la ley de reciprocidad para los residuos de la $l$-ésima potencia}, cuando $l$ denota un primo impar, y además cuando $l$ es una potencia de dos o una potencia de un primo impar.

La ley, así como los medios esenciales para su demostración, resultará, creo yo, de generalizar adecuadamente la teoría del campo de las $l$-ésimas raíces de la unidad, por mí desarrollada, y mi teoría de los campos cuadráticos relativos.

\subsection*{10. Determinación de la resolubilidad de la ecuación diofántica}

Dada una ecuación diofántica con cualquier número de incógnitas y con coeficientes numéricos enteros racionales: \textit{idear un proceso de acuerdo con el cual pueda determinarse en un número finito de operaciones si la ecuación es resoluble en enteros racionales}.

\subsection*{11. Formas cuadráticas con coeficientes numéricos
algebraicos cualesquiera}

Nuestro conocimiento actual de la teoría de los campos de números cua\-drá\-ticos nos sitúa en posición de atacar satisfactoriamente la teoría de formas cuadráticas con cualquier número de variables y con cualesquiera coeficientes numéricos algebraicos. Esto lleva en particular al interesante problema: resolver una ecuación cuadrática dada con coeficientes numéricos algebraicos en cualquier número de variables por números enteros o fraccionarios pertenecientes al dominio de racionalidad algebraico determinado por los coeficientes.
El siguiente problema importante puede constituir una transición al
álgebra y la teoría de funciones:


\subsection*{12. Extensión del teorema de Kronecker sobre campos abelianos
a cualquier dominio de racionalidad algebraico}

El teorema según el cual todo campo de números abelianos aparece a partir del dominio de los números racionales por la composición de campos de raíces de la unidad se debe a Kronecker. Este teorema fundamental en la teoría de ecuaciones integrales contiene dos enunciados, a saber:

Primero. Responde a la pregunta acerca del número y existencia de aquellas ecuaciones que tienen un grado dado, un grupo abeliano dado y un discriminante dado con respecto al dominio de los números racionales.

Segundo. Afirma que las raíces de tales ecuaciones forman un dominio de números algebraico que coincide con el dominio obtenido al asignar al argumento $z$ en la función exponencial $e^{i \pi z}$ todos los valores numéricos racionales en sucesión.

El primer enunciado está relacionado con la cuestión de la determinación de ciertos números algebraicos por sus grupos y su ramificación. Esta cuestión corresponde, por consiguiente, al conocido problema de la determinación de funciones algebraicas correspondientes a superficies de Riemann dadas. El segundo enunciado proporciona los números requeridos por medios trascendentes, a saber, por la función exponencial $e^{i\pi z}$.

Puesto que el dominio de los campos de números cuadráticos imaginarios es el más simple después del dominio de los números racionales, surge el problema de extender el teorema de Kronecker a este caso. El propio Kronecker ha hecho la afirmación de que las ecuaciones abelianas en el dominio de un campo cuadrático están dadas por las ecuaciones de transformación de funciones elípticas con módulos singulares, de modo que la función elíptica asume aquí el mismo papel que la función exponencial en el primer caso. Todavía no se ha ofrecido una demostración de la conjetura de Kronecker; pero creo que debe poderse obtener sin mucha dificultad sobre la base de la teoría de multiplicación compleja desarrollada por H. Weber, con la ayuda de los teoremas puramente aritméticos sobre campos de clase que yo he establecido.


Finalmente, la extensión del teorema de Kronecker al caso en el que, \textit{en lugar del dominio de los números racionales o del campo cuadrático imaginario, se establece un campo algebraico cualquiera como dominio de racionalidad}, me parece de la máxima importancia. Considero este problema como uno de los más profundos y de mayor alcance en la teoría de números y de funciones.

Se encuentra que el problema es accesible desde muchos puntos de vista. Considero que la clave más importante para la parte aritmética de este problema es la ley general de reciprocidad para residuos de potencias $l$-ésimas dentro de cualquier campo de números dado.

En cuanto a la parte de teoría de funciones del problema, el investigador en esta atractiva región estará guiado por las notables analogías que se advierten entre la teoría de funciones algebraicas de una variable y la teoría de números algebraicos. Hensel ha propuesto e investigado el análogo en la teoría de números algebraicos al desarrollo en serie de potencias de una función algebraica; y Landsberg ha tratado el análogo del teorema de Riemann-Roch. La analogía entre el género de una superficie de Riemann y el del número de clase de un campo de números también es evidente. Consideremos una superficie de Riemann de género $p = 1$ (por tocar sólo el caso más simple) y por otra parte un campo de números de clase $h = 2$. A la demostración de la existencia de una integral finita en todas partes sobre la superficie de Riemann, corresponde la demostración de la existencia de un entero $\alpha$ en el campo $\sqrt{\alpha}$ de números tal que el número representa un campo cuadrático, relativamente poco ramificado con respecto al campo fundamental. En la teoría de funciones algebraicas, el método de valores de contorno sirve como es bien sabido para la demostración del teorema de existencia de Riemann. También en la teoría de campos de números, la demostración de la existencia de tan sólo dicho número $\alpha$ ofrece la mayor dificultad. Esta demostración se logra con la asistencia indispensable del teorema según el cual en el campo de números hay siempre ideales primos con caracteres de residuo dado. Este último hecho es por consiguiente el análogo en teoría de números al problema de valores de contorno.

La ecuación del teorema de Abel en la teoría de funciones algebraicas expresa, como es bien sabido, la condición necesaria y suficiente de que los puntos en cuestión sobre la superficie de Riemann sean los ceros de una función algebraica perteneciente a la superficie. El análogo exacto del teorema de Abel, en la teoría del campo de números de clase $h = 2$, es la ecuación de la ley de reciprocidad cuadrática
$$\bigg(\frac{a}{j}\bigg)=+1
$$
que afirma que el ideal $j$ es un ideal principal del campo de números cuando y sólo cuando el residuo cuadrático del número $a$ con respecto al ideal $j$ es positivo.

Se verá que en el problema recién esbozado las tres disciplinas fundamentales de las matemáticas, la teoría de números, el álgebra y la teoría de funciones, entran en su más íntimo contacto, y estoy seguro de que la teoría de funciones analíticas de varias variables en particular se vería notablemente enriquecida si uno tuviera \textit{éxito en encontrar y discutir aquellas funciones que juegan para cualquier campo de números algebraico el papel correspondiente al de la función exponencial en el campo de los números racionales y de las funciones modulares elípticas en el campo de números cuadráticos imaginarios.}

Pasando al álgebra, mencionaré un problema de la teoría de ecuaciones y uno al que me ha llevado la teoría de invariantes algebraicos.


\subsection*{13. Imposibilidad de la solución de la ecuación general
de 7º grado por medio de funciones de sólo dos argumentos}

La nomografía trata del problema de resolver ecuaciones por medio del trazado de familias de curvas dependientes de un parámetro arbitrario. Se ve inmediatamente que toda raíz de una ecuación cuyos coeficientes dependen de sólo dos parámetros, es decir, toda función de dos variables independientes, puede representarse de muchas maneras de acuerdo con el principio que yace en la base de la nomografía. Además, es evidente que una gran clase de funciones de tres o más variables puede representarse por este principio sólo sin el uso de elementos variables, a saber: todas aquellas que pueden generarse formando primero una función de dos argumentos, igualando luego cada uno de estos argumentos a una función de dos argumentos, reemplazando a continuación cada uno de estos argumentos a su vez por una función de dos argumentos, y así sucesivamente, considerando admisible cualquier número finito de inserciones de funciones de dos argumentos. Así, por ejemplo, toda función racional de cualquier número de argumentos pertenece a esta clase de funciones construidas por tablas nomográficas; pues puede ser generada por los procesos de adición, sustracción, multiplicación y división y cada uno de estos procesos produce una función de sólo dos argumentos. Se ve fácilmente que las raíces de todas las ecuaciones que son resolubles por radicales en el dominio de racionalidad natural pertenecen a esta clase de funciones; pues aquí la extracción de raíces se añade a las cuatro operaciones aritméticas y, en realidad, representa una función de sólo un argumento. Análogamente, las ecuaciones generales de quinto y sexto grados son resolubles por tablas nomográficas apropiadas; pues, por medio de transformaciones de Tschirnhausen, que requieren sólo extracción de raíces, pueden reducirse a una forma donde los coeficientes dependen sólo de dos parámetros.

Ahora es probable que la raíz de la ecuación de séptimo grado sea una función de sus coeficientes que no pertenece a esta clase de funciones susceptibles de construcción nomográfica, i.e., que no pueden construirse por un número finito de inserciones de funciones de dos argumentos. Para demostrarlo, sería necesaria la demostración de que \textit{la ecuación de séptimo grado $f^7+xf^3+yf^2+zf+ 1 = 0$ no es resoluble con la ayuda de cualesquiera funciones continuas de sólo dos argumentos}. Permítaseme añadir que yo mismo he verificado mediante un riguroso proceso que existen funciones analíticas de tres argumentos $x$, $y$, $z$ que no pueden obtenerse por una cadena de funciones de sólo dos argumentos.

Empleando elementos auxiliares, la nomografía consigue construir
funciones de más de dos argumentos, como D'Ocagne ha demostrado recientemente en el caso de la ecuación de 7º grado.


\subsection*{14. Demostración de la finitud de ciertos sistemas completos
de funciones}

En la teoría de invariantes algebraicos, las cuestiones sobre la finitud de los sistemas completos de formas merecen, a mi parecer, un interés especial. L. Maurer ha conseguido últimamente extender los teoremas sobre finitud en teoría de invariantes demostrados por P. Gordan y yo mismo al caso en donde, en lugar del grupo proyectivo general, se escoge cualquier subgrupo como base para la definición de invariantes.

Un paso importante en esta dirección ha sido ya dado por A. Hurwitz quien, por un ingenioso proceso, consiguió efectuar la demostración, en su entera generalidad, de la finitud del sistema de invariantes octogonales de una forma base arbitraria.

El estudio de la cuestión de la finitud de invariantes me ha llevado a un problema simple que incluye dicha cuestión como un caso particular y cuya solución requiere probablemente un estudio más minuciosamente detallado de la teoría de eliminación y de los sistemas modulares algebraicos de Kronecker que el que se ha hecho hasta ahora.

Sea un número $m$ de funciones racionales enteras $X_1, X_2, \dots, X_m$ de las $n$ variables $x_1, x_2,\dots, x_n$
\begin{eqnarray*}\nonumber
X_1 &=&f_1(x_1, x_2, \dots, x_n) \\
 X_2 &=&f_2(x_1,x_2,\dots, x_n) \\
  .....&...&..........................\\
X_m &=&f_m(x_1, x_2, \dots, x_n).
\end{eqnarray*}

Toda combinación entera racional de $X_1, X_2,\dots, X_m$ debe evidentemente convertirse siempre, tras la sustitución de las expresiones anteriores, en una función entera racional de las $x_1, x_2,\dots, x_n$. De todas formas, muy bien puede haber funciones fraccionarias racionales de $X_1, X_2, \dots,X_m$ que por la operación de sustitución $S$ se convierten en funciones enteras en $x_1, x_2, \dots, x_n$. Propongo llamar función relativamente entera de $X_1,\dots, X_m$ a toda función racional de $X_1, \dots, X_n$ tal que se hace entera en $x_1, x_2, \dots, x_n$ tras la aplicación $S$. Toda función entera de $X_1, X_2, \dots..X_m$ es evidentemente también función relativamente entera; ade\-más la suma, diferencia y producto de funciones relativamente enteras son también relativamente enteras.

El problema resultante ahora es el de decidir si es siempre posible \textit{encontrar un sistema finito de funciones relativamente enteras $X_1, X_2, \dots..X_m$ por las que toda otra función relativamente entera de $X_1, X_2, \dots..X_m$ pueda expresarse de forma racional y entera}.

Podemos formular el problema de forma aún más simple si introducimos la idea de un campo de integridad finito. Por un campo de integridad finito entiendo un sistema de funciones de entre las que puede escogerse un número finito de funciones en términos de las cuales son expresables de forma racional y entera todas las demás funciones del sistema. Nuestro problema equivale entonces a éste: demostrar que todas las funciones relativamente enteras de cualquier dominio de racionalidad dado constituyen siempre un campo de integridad finito.
También se nos ocurre naturalmente refinar el problema mediante restricciones sacadas de la teoría de números, suponiendo que los coeficientes de las funciones dadas $f_1,f_2, \dots, f_m$ son enteros e incluyendo entre las funciones relativamente enteras de $X_1, X_2, \dots..X_m$, sólo tales funciones relativamente enteras de estos argumentos que se convierten, por la aplicación de las sustituciones $S$, en funciones enteras racionales de $x_1, x_2, \dots, x_n$ con coeficientes enteros racionales.

El siguiente es un simple caso particular de este problema refinado: Sean $m$ funciones racionales enteras $X_1, X_2, \dots,X_m$ de una variable $x$ con coeficientes racionales enteros, y sea dado un número primo $p$. Consideremos el sistema de aquellas funciones racionales enteras de $x$ que pueden expresarse en la forma
$$
\frac{G(X_1, \dots, X_m)}{p^h}
$$
donde $G$ es una función racional entera de los argumentos $X_1, X_2, \dots,X_m$ y $p^h$ es cualquier potencia del número primo $p$. Mis investigaciones anteriores muestran inmediatamente que todas las expresiones semejantes para un exponente $h$ fijo forman un dominio de integridad finito. Pero la cuestión aquí es si lo mismo es cierto para todos los exponentes $h$, i.e., si puede escogerse un número finito de tales expresiones por medio de las cuales para cada exponente $h$ cualquier otra expresión de esta forma es expresable de forma racional y entera.

De la región fronteriza entre álgebra y geometría mencionaré dos problemas. El primero concierne a la geometría enumerativa y el segundo a la topología de curvas y superficies algebraicas.


\subsection*{15. Fundamentación rigurosa del cálculo enumerativo de Schubert}

El problema consiste en esto: \textit{establecer rigurosamente y con una determinación exacta de los límites de su validez aquellos números geométricos que Schubert en especial ha determinado sobre la base del denominado principio de posición especial, o conservación del número, por medio del cálculo enumerativo por él desarrollado}.

Aunque el álgebra de hoy garantiza, en principio, la posibilidad de realizar los procesos de eliminación, para la demostración de los teoremas de la geometría enumerativa se requiere decididamente algo más, a saber, la realización del proceso de eliminación en el caso de ecuaciones de forma especial de tal manera que pueda preverse el grado de la ecuación final y la multiplicidad de sus soluciones.	

\subsection*{16. Problema de la topología de curvas y superficies algebraicas}




El número máximo de ramas cerradas e independientes que puede tener una curva algebraica plana de $n$-ésimo orden ha sido determinado por Harnack. Allí surge la cuestión adicional respecto a la posición relativa de las ramas en el plano. En cuanto a curvas de sexto orden, yo mismo he verificado ---por un proceso complicado, cierto es--- que de las once ramas que pueden tener según Harnack, no todas ni mucho menos pueden yacer externas una a otra, sino que debe existir una rama en cuyo interior yace una rama y en cuyo exterior yacen nueve ramas, o a la inversa. \textit{Una completa investigación de la posición relativa de las ramas independientes cuando su número es el máximo me parece de muy gran interés, y no menos la correspondiente investigación respecto al número, forma y posición de las hojas de una superficie algebraica en el espacio}. Hasta ahora, de hecho, ni siquiera se conoce cuál es el número máximo de hojas que puede tener realmente una superficie de cuarto orden en el espacio tridimensional.

En conexión con este problema puramente algebraico, deseo adelantar una cuestión que me parece que puede atacarse por el mismo método de variación continua de coeficientes, y cuya respuesta es de valor correspondiente para la topología de familias de curvas definidas por ecuaciones diferenciales. Se trata de la cuestión del número máximo y la posición de los ciclos límite de Poincaré para una ecuación diferencial de primer orden y grado de la forma
$$
\frac{dy}{dx}=\frac{Y}{X}
$$
donde $X$ e $Y$ son funciones racionales de $n$-ésimo grado en $x$ e $y$. Escrita
en forma homogénea, ésta es
$$
X \left(y\,\frac{dz}{dt}-z\,\frac{dy}{dt}\right)+ Y\left(z\, \frac{dx}{dt}-x\,\frac{dz}{dt} \right)+ Z \left(x\, \frac{dy}{dt}-t\, \frac{dx}{dt} \right)=0
$$
donde $X$, $Y$ y $Z$ son funciones homogéneas racionales de $n$-ésimo grado en $x$, $y$, $z$, y las últimas deben determinarse como funciones del parámetro $t$.


\subsection*{17. Expresión de formas definidas por cuadrados}



Una función o forma entera racional en cualquier número de variables con coeficientes reales se dice definida si no se hace negativa para ningún valor real de dichas variables. El sistema de todas las formas definidas es invariante con respecto a las operaciones de adición y multiplicación, pero el cociente de dos formas definidas ---en el caso en que fuera una función entera de las variables--- es también una forma definida. El cuadrado de cualquier forma es siempre evidentemente una forma definida. Pero puesto que, como yo he demostrado, no toda forma definida puede componerse por adición de cuadrados de formas, surge la cuestión ---que yo he respondido de forma afirmativa para formas ternarias---  de si toda forma definida no puede expresarse como un cociente de sumas de cuadrados de formas. Al mismo tiempo es deseable, para ciertas cuestiones respecto a la posibilidad de ciertas construcciones geométricas, saber si los coeficientes de las formas a utilizar en la expresión pueden tomarse siempre del dominio de racionalidad dado por los coeficientes de la forma representada.

Menciono otro problema geométrico.

\subsection*{18. Construcción del espacio a partir de poliedros congruentes}

Si preguntamos por aquellos grupos de movimientos en el plano para los que existe una región fundamental, obtenemos respuestas diversas, según el plano considerado sea de Riemann (elíptico), de Euclides o de Lobachevsky (hiperbólico). En el caso del plano elíptico existe un número finito de tipos esencialmente diferentes de regiones fundamentales, y un número finito de regiones congruentes basta para un recubrimiento completo de todo el plano; el grupo consiste de hecho en un número finito de movimiento solamente. En el caso del plano hiperbólico existe un número infinito de tipos esencialmente diferentes de regiones fundamentales, a saber, los bien conocidos polígonos de Poincaré. Para el recubrimiento completo del plano es necesario un número infinito de regiones congruentes. El caso del plano de Euclides queda entre éstos; pues en este caso existe solamente un número finito de tipos esencialmente diferentes de grupos de movimientos con regiones fundamentales, pero para un recubrimiento completo de todo el plano es necesario un número infinito de regiones congruentes.

Exactamente los hechos correspondientes se encuentran en el espacio de tres dimensiones. El hecho de la finitud de los grupos de movimientos en el espacio elíptico es una consecuencia inmediata de un teorema fundamental de C. Jordan, por el que el número de tipos esencialmente diferentes de grupos finitos de sustituciones lineales en $n$ variables no supera un cierto límite finito dependiente de $n$. Los grupos de movimientos con regiones fundamentales en el espacio hiperbólico han sido investigados por Fricke y Klein en las lecciones sobre la teoría de funciones automorfas, y finalmente Fedorov, Schoenfies y últimamente Rohn han dado la demostración de que, en el espacio euclídeo, existe sólo un número finito de tipos esencialmente diferentes de grupos de movimientos con una región fundamental. Ahora bien, aunque los resultados y métodos de demostración aplicables a espacios elípticos e hiperbólicos son válidos directamente también para el espacio $n$-dimensional, la generalización del teorema para el espacio euclídeo parece ofrecer claras dificultades. Por consiguiente es deseable la investigación de la siguiente cuestión: \textit{¿existe en el espacio $n$-dimensional euclídeo también sólo un número finito de tipos esencialmente diferentes de grupos de movimientos con una región fundamental?}

Una región fundamental de cada grupo de movimientos, junto con las regiones congruentes que surgen del grupo, llena evidentemente el espacio por completo. Surge la pregunta de \textit{si existen también poliedros que no aparecen como regiones fundamentales de grupos de movimientos, por medio de los cuales y con una yuxtaposición adecuada de copias congruentes es posible un llenado completo de todo el espacio.} Apunto la siguiente cuestión relacionada con la precedente, e importante  para la teoría de números y quizá a veces útil para la física y la química: ¿cómo se puede disponer de forma más densa en el espacio un número infinito de sólidos iguales de forma dada, e.g., esferas con radios dados o tetraedros regulares con aristas dadas (o en posición prescrita)?; es decir, ¿cómo pueden encajarse de modo que la razón del espacio llenado al no llenado sea la mayor posible?



Si consideramos el desarrollo de la teoría de funciones en el último siglo, notamos por encima de todo la importancia fundamental de esa clase de funciones que ahora designamos como funciones analíticas ---una clase de funciones que con toda probabilidad estarán permanentemente en el centro del interés matemático.

Existen muchos puntos de vista diferentes a partir de los cuales podríamos escoger, de entre la totalidad de todas las funciones imaginables, clases extensas dignas de una investigación particularmente completa. Consideremos, por ejemplo, la clase de las funciones caracterizadas por ecuaciones diferenciales algebraicas ordinarias o en derivadas parciales. Debería observarse que esta clase no contiene las funciones que aparecen en teoría de números, y cuya investigación es de la máxima importancia. Por ejemplo, la antes mencionada función $\zeta(s)$ no satisface ninguna ecuación diferencial algebraica, como se ve fácilmente con la ayuda de la bien conocida relación entre $\zeta(s)$ y $\zeta(1-s)$, si uno se remite al teorema demostrado por Holder según el cual la función $\Gamma(z)$ no satisface ninguna ecuación diferencial algebraica. Una vez más, la función de las dos variables $s$ y $x$ definida por la serie infinita
$$
\zeta(s,x)=x+ \frac{x^2}{2^s}+\frac{x^3}{3^s}+\frac{x^4}{4^s}+ \cdots
$$
que está en íntima relación con la función $\zeta(s)$, probablemente no satisface ninguna ecuación en derivadas parciales. En la investigación de esta cuestión habrá que utilizar la ecuación funcional
$$
x \,\frac{\partial \zeta(s,x)}{\partial x}= \zeta(s-1,x)
$$
Si, por otra parte, nos viéramos llevados por razones aritméticas o geomé\-tricas a considerar la clase de todas aquellas funciones que son continuas e infinitamente diferenciales, estaríamos obligados a prescindir en su investigación de ese instrumento flexible, la serie de potencias, y de la circunstancia de que la función está completamente determinada por la asignación de valores en cualquier región, por pequeña que sea. Por ello, mientras que la primera limitación del campo de funciones era demasiado estrecha, la última me parece demasiado amplia.

La idea de la función analítica incluye, por otra parte, toda la riqueza de funciones más importantes para la ciencia, ya tengan su origen en la teoría de números, en la teoría de ecuaciones diferenciales o de ecuaciones funcionales algebraicas, ya aparezcan en geometría o en física matemática; y, por consiguiente, en todo  dominio de funciones la función analítica ostenta justamente la supremacía indiscutida.


{\subsection*{19. ¿Son siempre necesariamente analíticas las soluciones
de problemas regulares en el cálculo de variaciones?}


Uno de los hechos más notables en los elementos de la teoría de funciones analíticas es para mí éste: que existen ecuaciones en derivadas parciales cuyas integrales son por necesidad funciones analíticas de las variables independientes, es decir, en pocas palabras, ecuaciones susceptibles únicamente de soluciones analíticas. Las ecuaciones en derivadas parciales de este tipo mejor conocidas son la ecuación de potencial
$$
\frac{\partial^2 f}{\partial x^2}+ \frac{\partial^2 f}{\partial y^2}=0
$$
y ciertas ecuaciones diferenciales lineales estudiadas por Picard; también la ecuación
$$
\frac{\partial^2 f}{\partial x^2}+ \frac{\partial^2 f}{\partial y^2}=e^f
$$
la ecuación en derivadas parciales de las superficies mínimas, y otras. Muchas de estas ecuaciones en derivadas parciales tienen la característica común de ser las ecuaciones diferenciales lagrangianas de ciertos problemas de variación, viz., de problemas de variación
$$
\iint f(p,q,z;x,y) \, dx \, dy = \text{ mínimo }\\
\left[
p=\frac{\partial z}{\partial x}, q = \frac{\partial z}{\partial y} \right]
$$
tales que satisfacen, para todos los valores de los argumentos que caen dentro del intervalo de discusión, la desigualdad
$$
\frac{\partial^2 F}{\partial p^2} \cdot \frac{\partial^2 F}{\partial q^2} + \left(\frac{\partial^2 F}{\partial p \, \partial q}\right)^2 >0
$$
siendo la propia $F$ una función analítica. Llamaremos, a este tipo de problemas, problemas de variación regular. Son principalmente los problemas de variación regular los que juegan un papel en geometría, en mecánica y en física matemática; y naturalmente surge la cuestión de si todas las soluciones de problemas de variación regular deben ser necesariamente funciones analíticas. En otras palabras, \textit{¿tiene toda ecuación en derivadas parciales lagrangianas de un problema de variación regular la propiedad de admitir exclusivamente integrales analíticas?} ¿Y es así incluso cuando la función está limitada a tomar, como, por ejemplo, en el problema de Dirichlet sobre la función potencial, valores de contorno que son continuos pero no analíticos?

Puedo añadir que existen superficies de curvatura gaussiana negativa constante que son representables por funciones que son continuas y poseen realmente todas las derivadas, y pese a todo no son analíticas; mientras que por otro lado es probable que toda superficie cuya curvatura gaussiana es constante y positiva sea necesariamente una superficie analítica. Y sabemos que las superficies de curvatura constante positiva están muy estrechamente relacionadas con este problema de variación regular: hacer pasar por una curva cerrada en el espacio una superficie de área mínima que encierre, en combinación con una superficie fija que pasa por la misma curva cerrada, un volumen de magnitud dada.


\subsection*{20. El problema general de los valores de contorno}

Un problema importante íntimamente conectado con el anterior es la cuestión concerniente a la existencia de soluciones de ecuaciones en derivadas parciales cuando los valores en la frontera de la región están prescritos. Este problema está resuelto en lo principal por los ingeniosos métodos de H. A. Schwarz, C. Neumann y Poincaré para la ecuación del potencial. Sin embargo, estos métodos no parecen en general susceptibles de extensión directa al caso en donde a lo largo de la frontera están prescritos o bien los coeficientes diferenciales o cualesquiera relaciones entre estos y los valores de la función. Ni pueden extenderse inmediatamente al caso en donde no se buscan superficies potenciales sino, digamos, superficies de área mínima, o superficies de curvatura gaussiana positiva constante, que tienen que pasar por una curva retorcida prescrita o extenderse sobre una superficie anular dada. Tengo la convicción de que será posible demostrar estos teoremas de existencia por medio de un principio general cuya naturaleza está indicada por el principio de Dirichlet. Este principio general nos permitirá quizá entonces aproximarnos a la cuestión: \textit{¿Tiene solución todo problema de variación regular, siempre que se satisfagan ciertas hipótesis respecto a las condiciones de contorno dadas} (por ejemplo, que las funciones implicadas en estas condiciones de contorno sean continuas y tengan en secciones una o más derivadas), \textit{y siempre también, si es necesario, que se amplíe adecuadamente la noción de una solución?}


\subsection*{21. Demostración de la existencia de ecuaciones diferenciales
lineales que tienen prescrito un grupo monodrómico}

En la teoría de ecuaciones diferenciales lineales con una variable independiente $z$, quiero señalar un problema importante, un problema que muy probablemente el propio Riemann pudo haber tenido en mente. Este problema es el siguiente: \textit{Demostrar que siempre existe una ecuación diferencial lineal de la clase fuchsiana, con puntos singulares y grupo monodrómico dado.} El problema requiere la presentación de $n$ funciones de la variable $z$, regulares en todo el plano complejo $z$ excepto en los puntos singulares dados; en dichos puntos las funciones pueden hacerse infinitas de orden sólo finito, y cuando $z$ recorre circuitos alrededor de dichos puntos las funciones experimentarán las sustituciones lineales prescritas. La existencia de tales ecuaciones diferenciales se ha demostrado probable contando las constantes, pero la demostración rigurosa ha sido obtenida hasta ahora sólo en el caso particular en donde las ecuaciones fundamentales de las sustituciones dadas tienen todas las raíces de magnitud absoluta unidad. L. Schlesinger ha dado esta demostración basada en la teoría de Poincaré de las $z$-funciones fuchsianas. La teoría de ecuaciones diferenciales lineales tendría evidentemente una apariencia más acabada si el problema aquí esbozado pudiera eliminarse mediante algún método perfectamente general.


\subsection*{22. Uniformización de relaciones analíticas
por medio de funciones automorfas}

Como Poincaré fue el primero en demostrar, siempre es posible uniformizar cualquier relación algebraica entre dos variables mediante el uso de funciones automorfas de una variable. Es decir, si se da cualquier relación algebraica entre dos variables, siempre puede encontrarse para dichas variables dos de tales funciones automorfas univaluadas de una
sola variable tal que su sustitución convierte la ecuación algebraica dada en una identidad. La generalización de este teorema fundamental a relaciones no algebraicas analíticas cualesquiera entre dos variables ha sido alcanzada también con éxito por Poincaré, aunque de un modo completamente diferente del que se sirvió en el problema especial mencionado en primer lugar. No obstante, a partir de la demostración de Poincaré de la posibilidad de uniformizar una relación analítica arbitraria entre dos variables no resulta evidente si pueden determinarse las funciones para que satisfagan ciertas condiciones adicionales. Es decir, no está demostrado si las dos funciones univaluadas de la nueva variable pueden escogerse de modo que, mientras esta variable recorre el dominio regular de dichas funciones, se alcanzan y representan realmente todos los puntos regulares del campo analítico dado. Por el contrario, parece ser el caso, por las investigaciones de Poincaré, de que además de los puntos de ramificación existen otros, en general infinitos puntos excepcionales discretos del campo analítico, que sólo pueden alcanzarse haciendo que la nueva variable se aproxime a ciertos puntos límite de las funciones. \textit{En vista de la importancia fiundamental de la formulación de la cuestión por Poincaré, me parece extraordinariamente deseable la discusión y solución de esta dificultad.}

En unión con este problema surge el problema de reducir a uniformidad una relación algebraica o cualquier otra relación analítica entre tres o más variables complejas ---un problema que se sabe que es resoluble en muchos casos particulares---. Para su solución, las recientes investigaciones de Picard sobre funciones algebraicas de dos variables deben considerarse estudios preliminares importantes y bienvenidos.


\subsection*{23. Desarrollo adicional de los métodos del cálculo de variaciones}

Hasta aquí, he mencionado en general problemas lo más definidos y especiales posible, con el convencimiento de que son precisamente tales problemas definidos y especiales los que más nos atraen y los que suelen ejercer una influencia más duradera en la ciencia. De todas formas, me gustaría acabar con un problema general, a saber, con un indicio de una rama de las matemáticas repetidamente mencionada en esta conferencia y que, a pesar del considerable avance que le ha dado Weierstrass, no recibe el aprecio general que, en mi opinión, merece. Me refiero al cálculo de variaciones.

La falta de interés por esto se debe quizá en parte a la necesidad de libros de texto modernos y fiables. Por ello es tanto más digno de alabanza el que A. Kneser, en una obra publicada muy recientemente, haya tratado el cálculo de variaciones desde los modernos puntos de vista y considerando la moderna demanda de rigor.

El cálculo de variaciones es, en el más amplio sentido, la teoría de la variación de funciones, y como tal aparece como una extensión necesaria del cálculo diferencial e integral. En este sentido, las investigaciones de Poincaré sobre el problema de tres cuerpos, por ejemplo, constituyen un capítulo del cálculo de variaciones, en cuanto que a partir de órbitas conocidas Poincaré deriva por el principio de variación nuevas órbitas de carácter similar.

Añado aquí una breve justificación de los comentarios generales sobre el cálculo de variaciones hechos al comienzo de mi conferencia.

Es sabido que el problema más simple en el cálculo de variaciones propiamente dicho consiste en encontrar una función $y$ de una variable $x$ tal que la integral definida
$$
J=\int_a^b F(y_x,y;x)\, dx, \quad y_x= \frac{dy}{dx}
$$
toma un valor mínimo comparado con los valores que toma cuando $y$ se reemplaza por otras funciones de $x$ con los mismos valores inicial y final.

La anulación de la primera variación en el sentido usual
$$
\delta J=0
$$
da para la deseada función $y$ la bien conocida ecuación diferencial
\begin{equation}\label{1}
\frac{dF_{y_x}}{dx} \, f_y =0, \qquad \left[ F_{y_x}= \frac{\partial F}{\partial y_x}, \, F_y= \frac{\partial F}{\partial y}\right]
\end{equation}


Para investigar más de cerca los criterios necesarios y suficientes para la ocurrencia del mínimo requerido, consideremos la integral
$$
J=\int_a^b\{ F +(y_x-p)F_p\} \, dx, \quad  \left[F=F(p,y;x), \, F_p=\frac{\partial F(p,y;x)}{\partial p}\right]
$$



\textit{Ahora investigamos cómo hay que escoger $p$, como función de $x$, $y$,
para que el valor de esta integral $J^*$ sea independiente del camino de integración, i.e., de la elección de la función $y$ de la variable $x$.}

 La integral $J^*$ tiene la forma
$$
J^*=\int_a^b \{Ay_x-B\} \, dx
$$
donde $A$ y $B$ no contienen $y_x$, y la anulación de la primera variación
$$
\delta J^*=0
$$
en el sentido que requiere la nueva cuestión da la ecuación
$$
\frac{\partial A}{\partial x}+\frac{\partial B}{\partial y}=0
$$
i.e., obtenemos para la función $p$ de las dos variables $x$, $y$ la ecuación en derivadas parciales de primer orden
\begin{equation}\label{11}
\frac{\partial F_p}{\partial x}+ \frac{\partial (pF_p-F)}{\partial y}=0
\end{equation}


La ecuación diferencial ordinaria de segundo orden (\ref{1}) y la ecuación en derivadas parciales (\ref{11}) están en la más íntima relación mutua. Esta relación se nos hace inmediatamente clara por la siguiente transformación simple
\begin{eqnarray*}
\delta J^* &=& \int_a^b\{F_y\delta y+F_p\delta p+(\delta y_x-\delta p)F_y+(y_x-p)\delta F_p\} \, dx\\
 &= & \int_a^b\{F_y\delta y+\delta y_x F_p +(y_x-p)\delta F_p\}\, dx\\
  &= & \delta J+ \int_a^b (y_x-p) \delta F_p \, dx
\end{eqnarray*}

De esto derivamos los siguientes hechos: Si construimos cualquier familia simple de curvas integrales de la ecuación diferencial ordinaria (\ref{1}) de segundo orden y luego formamos una ecuación diferencial ordinaria de primer orden
\begin{equation}\label{2}
y_x=p(x,y)
\end{equation}
que también admite estas curvas integrales como soluciones, entonces la función $p(x,y)$ es siempre una integral de la ecuación en derivadas parciales (\ref{11}) de primer orden; y recíprocamente, si $p(x,y)$ denota cualquier solución de la ecuación en derivadas parciales (\ref{11}) de primer orden, todas las integrales no singulares de la ecuación diferencial ordinaria (\ref{2}) de primer orden son al mismo tiempo integrales de la ecuación diferencial (\ref{1}) de segundo orden, o en pocas palabras si $y_x = p(x,y)$ es una ecuación integral de primer orden de la ecuación diferencial (\ref{1}) de segundo orden, $p(x,y)$ representa una integral de la ecuación en derivadas parciales (\ref{11}) y recíprocamente; por consiguiente, las curvas integrales de la ecuación diferencial ordinaria de segundo orden son al mismo tiempo las características de la ecuación en derivadas parciales (\ref{11}) de primer orden.

En el caso presente podemos encontrar el mismo resultado por medio de un simple cálculo; pues éste nos da las ecuaciones diferenciales (\ref{1}) y (\ref{11}) en cuestión en la forma
$$
y_{xx}F_{y_xy_x}+y_x F_{y_x x}+ F_{y_x x}-F_y=0\\
$$
$$
(p_x+pp_y)F_{pp}+pF_{py}+F_{px}-F_y=0
$$
donde los subíndices indican las derivadas parciales con respecto a $x$, $y$, $p$, $y_x$. La corrección de la relación afirmada queda clara a partir de esto.

La estrecha relación derivada antes y recién demostrada entre la ecuación diferencial ordinaria (\ref{1}) de segundo orden y la ecuación en derivadas parciales (\ref{11}) de primer orden es, así me lo parece, de importancia fundamental para el cálculo de variaciones. Pues a partir del hecho de que la integral $J^*$ es independiente del camino de integración se sigue que
\begin{equation}\label{3}
\int_a^b \{ F(p)-(y_x-p)F_p(p)\} \, dx = \int_a^b F (\overline{y}_x) \, dx
\end{equation}
si consideramos que la integral del primer miembro se toma a lo largo de cualquier camino $y$ y la integral del segundo miembro a lo largo de una curva integral de la ecuación diferencial
$$
\overline{y}_x = p(x,\overline{y})
$$

Con la ayuda de la ecuación (\ref{3}) llegamos a la fórmula de Weierstrass
\begin{equation}\label{4}
\int_a^b F(y_x) \, dx -\int_a^b F(\overline{y}_x) \, dx = \int_z^b E(y_x,p)\, dx
\end{equation}
donde $E$ designa la expresión de Weierstrass, dependiente de $y_x$, $p$, $y$, $x$, 
$$
E(y_x, p) = F(y_x) - F(p) - (y_x - p) F_p(p)
$$

Puesto que, por consiguiente, la solución depende solamente de encontrar una integral $p(x,y)$ que es univaluada y continua en un cierto entorno de la curva integral $y$, que estamos considerando, los desarrollos recién indicados llevan inmediatamente ---sin la introducción de la segunda variación, sino sólo por la aplicación del proceso polar a la ecuación diferencial (\ref{1})--- a la expresión de la condición de Jacobi y a la respuesta a la pregunta: hasta qué punto esta condición de Jacobi junto con la condición de Weierstrass $E >$ 0 es necesaria y suficiente para la ocurrencia de un mínimo.

Los desarrollos indicados pueden transferirse sin necesidad de más  
cálculos al caso de dos o más funciones requeridas, y también al caso de una integral doble o múltiple. Así, por ejemplo, en el caso de una integral doble
$$
J= F(z_x,z_y,z;x,y) \, d\omega, \quad \left[ z_x = \frac{\partial z}{\partial x}, \, z_y = \frac{\partial z}{\partial y}\right]
$$
que debe extenderse sobre una región dada $w$, la anulación de la primera variación (que debe entenderse en el sentido usual)
$$
\partial J=0
$$
da la bien conocida ecuación diferencial de segundo orden
$$
 \frac{\partial F_x}{\partial x}+ \frac{\partial F_{z_y}}{\partial y}-Fx=0 \quad \left[ F_{z_x}= \frac{\partial F}{\partial  z_x} , \, F_{z_y}= \frac{\partial F}{\partial  z_y} ,\, F_{z}= \frac{\partial F}{\partial  z}\right]
$$
para la función requerida $z$ de $x$ e $y$.

Por otra parte consideramos la integral
$$
J^*=\int\{F+(z_x-p)F_p+(z_y-q)F_q\}\, d\omega
$$
\textit{y preguntamos cómo deben tomarse $p$ y $q$ como funciones de $x$, $y$, $z$ para que el valor de esta integral pueda ser independiente de la elección de la superficie que pasa por la curva retorcida dada, i.e., de la elección de la función $z$ de las variables $x$ e $y$.}

La integral $J^*$ tiene la forma
$$
J^*= \int \{ A z_x+B z_y-C\} \, d\omega
$$
y la anulación de la primera variación 
$$
\delta J^*=0
$$
en el sentido que demanda la nueva formulación de la cuestión, da la ecuación
$$
\frac{\partial A}{\partial x}+\frac{\partial B}{\partial y}+\frac{\partial C}{\partial z}=0
$$
i.e., encontramos para las funciones $p$ y $q$ de las tres variables $x$, $y$ y $z$ la ecuación diferencial de primer orden
\begin{equation}\label{I}
\frac{\partial F_p}{\partial x}+\frac{\partial F_q}{\partial y}+\frac{\partial C(pF_p+qF_q-F)}{\partial x}=0
\end{equation}



Si añadimos a esta ecuación diferencial la ecuación en derivadas parciales
\begin{equation}\label{II}
p_y+qp_z=qx+pq_x
\end{equation}
que resulta de las ecuaciones
$$
z_x= p(x, y, z),\,  z_y = q(x, y, z)
$$
la ecuación en derivadas parciales (\ref{I}) para la función $z$ de las dos variables $x$ e $y$ y el sistema simultáneo de las dos ecuaciones en derivadas parciales de primer orden (\ref{II}) para las dos funciones $p$ y $q$ de las tres variables $x$, $y$ y $z$ están entre sí en una relación exactamente análoga a la que estaban las ecuaciones diferenciales (\ref{1}) y (\ref{11}) en el caso de la integral simple.

Se sigue del hecho de que la integral $J^*$ es independiente de la elección de la superficie de integración $z$ que
$$
\int\{F(p,q)+(z_x-p)F_p(p,q)+(z_y-q)F_q(p,q)\} \, d\omega = \int F(\overline{z}_x,\overline{z}_y)\, d\omega
$$
si consideramos que la integral del segundo miembro se toma sobre una superficie integral de las ecuaciones en derivadas parciales
$$
\overline{z}_x=p(x,y,\overline{z}),\, \overline{z}_y = p(x,y,\overline{z})
$$
y con la ayuda de esta fórmula llegamos inmediatamente a la fórmula
\begin{equation*}
\begin{split}
F(z_x,z_y)\, d\omega-\int F(\overline{z}_x,\overline{z}_y)\, d\omega= \int E(z_x,z_y,p,q)\, d\omega,\\ [E(z_x,z_y,p,q)=F(z_x,z_y)-F(p,q)-(z_x-p)F_p(p,q)-(z_y-q)F_q(p,q)]
\end{split}
\end{equation*}
que desempeña el mismo papel para la variación de integrales dobles que la fórmula previamente dada (\ref{4}) para integrales simples. Con la ayuda de esta fórmula podemos responder ahora a la pregunta de hasta qué punto
la condición de Jacobi junto con la condición de Weierstrass $E > 0$ es necesaria y suficiente para la ocurrencia de un mínimo.

Conectada con estos desarrollos está la forma modificada en la que A. Kneser, partiendo de otros puntos de vista, ha presentado la teoría de Weierstrass. Mientras que Weierstrass empleaba aquellas curvas integrales de la ecuación (\ref{1}) que pasan por un punto fijo para derivar condiciones suficientes para el extremal, Kneser por el contrario hace uso de cualquier familia simple de tales curvas y construye para toda familia semejante una solución, característica de dicha familia, de la ecuación en derivadas parciales que debe considerarse como una generalización de la ecuación de Jacobi-Hamilton.

Los problemas mencionados son simplemente muestras de problemas, pero bastarán para demostrar cuán rico, cuán variado y cuán extensa es la ciencia matemática de hoy, y nos apremia la cuestión de si las matemáticas están condenadas al destino de esas otras ciencias que se han dividido en ramas separadas, cuyos representantes difícilmente se entienden unos a otros y cuya conexión se hace cada vez más vaga. Yo no lo creo ni lo deseo. Las ciencias matemáticas son en mi opinión un todo indivisible, un organismo cuya vitalidad es una condición para la conexión de sus partes. Pues, con toda la variedad del conocimiento matemático, seguimos siendo claramente conscientes de la similitud de las estrategias lógicas, la relación de las ideas en matemáticas como un todo y las numerosas analogías en sus diferentes departamentos. También advertimos que, cuanto más se desarrolla una teoría matemática, más armoniosa y uniformemente procede su construcción, y se desvelan relaciones insospechadas entre ramas hasta entonces separadas de la ciencia. Sucede así que, con la extensión de las matemáticas, su carácter orgánico no sólo no se pierde sino que se manifiesta más claramente.

Pero, preguntamos, con la extensión del conocimiento matemático ¿no se hará finalmente imposible para el investigador individual abarcar todos los departamentos de este conocimiento? Como respuesta déjenme señalar cuán meticulosamente arraigada está la ciencia matemática, pues todo avance real va acompañado de la invención de herramientas más precisas y métodos más simples que al mismo tiempo ayudan a comprender teorías anteriores y desechan desarrollos más viejos y más complicados. Es así posible para el investigador individual, cuando hace suyas estas herramientas más precisas y métodos más simples, encontrar su camino en las diversas ramas de las matemáticas mucho más fácilmente que lo que es posible en cualquier otra ciencia.

La unidad orgánica de las matemáticas es inherente a la naturaleza de esta ciencia, pues las matemáticas son la base de todo conocimiento exacto de los fenómenos naturales. Ojalá pueda cumplir totalmente esta alta misión, y el nuevo siglo traiga maestros dotados y muchos discípulos celosos y entusiastas.
 
\end{document}








