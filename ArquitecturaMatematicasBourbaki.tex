\documentclass[a4paper, 12pt, draft]{article}

%%%%%%%%%%%%%%%%%%%%%%Paquetes
\usepackage[spanish]{babel}  
\usepackage[utf8]{inputenc}
\usepackage{tcolorbox}
\usepackage{amsmath}
\usepackage{cmbright}  %%%%%%% El tipo de letra
\usepackage{setspace}
\usepackage{enumerate}
\onehalfspacing  %%%%%%%%%%% Espacio y medio de interlineado
\parskip=1em  %%%%%%%%%%%% Separacion entre parrafos
%%%%%%%%%%%%%%%%%%%%%%



%%%%%%%%%%%%%%%%%
\title{La Arquitectura de las Matemáticas}
\author{N. Bourbaki}
\date{}
%%%%%%%%%%%%%%%%%

\begin{document}

\begin{tcolorbox}[colback=blue!5!white,colframe=blue!75!black]

\vspace{-1.8cm}
\textbf \maketitle

\end{tcolorbox}

\bigskip


\subsection*{1. ¿La matemática o las matemáticas?} 

Dar en el momento actual, una idea de conjunto de la ciencia matemática, es una empresa que parece ofrecer, de entrada, dificultades casi insalvables, dada la extensión y la variedad del tema. Al igual que en el caso de otras ciencias, el número de matemáticos y de trabajos consagrados a las matemáticas ha aumentado considerablemente desde finales del siglo XIX. Las memorias de matemáticas puras publicadas en el mundo durante un año normal, abarcan varios miles de páginas. En ellas, por supuesto, no todo tiene el mismo valor, pero después de una decantación del desecho inevitable, no es menos cierto que cada año la ciencia matemática se enriquece con muchos resultados nuevos, se diversifica y se ramifica constantemente en teorías que, sin cesar, se modifican, se refunden, se confrontan y se combinan unas con otras. Ningún matemático, ni siquiera consagrando en ello toda su actividad, estaría hoy en condiciones de seguir este desarrollo en todos sus detalles. Buen número de ellos se ciñen a un dominio de las matemáticas del que no pretenden salir, y no sólo ignoran casi por completo todo lo que no tiene que ver con la materia que han escogido sino que incluso serían incapaces de comprender el lenguaje y la terminología empleados por los colegas que se adscriben a una especialidad alejada de la suya. Pocos hay, incluso entre aquellos cuya cultura es más vasta, que no se sientan desorientados en ciertas regiones del universo matemático. Aquellos que, como Poincaré o Hilbert, imprimen el sello de su genio en casi todos los dominios, constituyen, incluso entre los más grandes, una rarísima excepción.

No se trata aquí, pues, de dar al profano una imagen precisa de aquello que los propios matemáticos no pueden concebir en su totalidad. Sin embargo, podemos preguntarnos si esta proliferación exuberante es el desarrollo de un organismo sólidamente construido, que adquiere cada día más cohesión y unidad en su propio crecimiento, o si, por el contrario, no es más que el signo exterior de una tendencia a un fraccionamiento cada vez mayor, debido a la naturaleza misma de las matemáticas, y si éstas no se estarán convirtiendo en una torre de Babel de disciplinas autónomas, aisladas unas de otras, tanto en sus principios como en sus métodos, e incluso en su lenguaje. En una palabra ¿existe hoy \textit{una} matemática o \textit{varias} matemáticas?

Aunque más actual que nunca, no debería creerse que esta pregunta es nueva; está planteada desde los primeros pasos de la ciencia matemática.\linebreak Y es que, en efecto, incluso dejando aparte las matemáticas aplicadas, subsiste, entre la geometría y la aritmética (al menos bajo su forma elemental) una evidente dualidad de origen, siendo inicialmente la segunda ciencia de lo \textit{discreto,} y la primera de la extensión \textit{continua,} dos aspectos que se oponen radicalmente desde el descubrimiento de los irracionales. Por otra parte, fue precisamente este descubrimiento el que resultó fatal en la primera tentativa de unificación de la ciencia, el aritmeticismo de los pitagóricos («todas las cosas son números»).

Nos veríamos conducidos demasiado lejos si tuviéramos que seguir, desde el pitagorismo hasta nuestros días, las vicisitudes de la concepción unitaria de las matemáticas. Es ésta, además, una tarea para la que está mejor preparado un filósofo que un matemático, ya que es un rasgo común de los diversos intentos para integrar en un todo coherente el conjunto de las matemáticas ---ya se trate de Platón, de Descartes o de Leibniz, del aritmeticismo o de la logística del siglo XIX--- el estar ligados a un sistema filosófico más o menos ambicioso, partiendo siempre, sin embargo, de ideas \textit{a priori} sobre las relaciones de las matemáticas con el doble universo del mundo exterior y el mundo del pensamiento. Lo mejor que podemos hacer en relación a este punto es remitir al lector sobre este punto al estudio histórico y crítico de Leon Brunschvicg: \textit{Les étapes de la philosophie mathématique}. Nuestra tarea es más modesta y más circunscrita: no pretenderemos examinar las relaciones de las matemáticas con lo real o con las grandes categorías del pensamiento; es en el seno de la matemática en donde pensamos quedarnos para buscar, analizando sus propios vericuetos, una respuesta a la pregunta que nos hemos planteado.



\subsection*{2. Formalismo lógico y método axiomático} 

Después del fracaso, más o menos aparente, de los diversos sistemas a los que hemos hecho alusión, parecía, a principios del presente siglo, que casi se hubiera renunciado a ver en las matemáticas una ciencia caracterizada por un objeto y un método únicos. Más bien se tenía la tendencia a considerarlas como \textit{«una serie de disciplinas fundadas sobre nociones particulares, delimitadas con precisión»,} ligadas por \textit{«mil caminos de comunicación»} que permitiesen a los métodos propios de una de estas disciplinas poder fecundar una o más de ellas (Brunschvicg, \textit{Op. cit.}, p. 447). Hoy, por el contrario, creemos que la evolución interna de la ciencia matemática ha promovido, a pesar de las apariencias, más que nunca la unidad de sus diversas partes y ha creado una especie de núcleo central más coherente que nunca. Lo esencial de esta evolución ha consistido en una sistematización de las relaciones que existen entre las diversas teorías matemáticas, resumida en una tendencia conocida generalmente bajo el nombre de «método axiomático».

Se la denomina también a veces «formalismo» o «método formalista», pero debemos guardarnos, desde un principio, del peligro de confusión que provocan estas palabras mal definidas, explotadas no pocas veces, por los adversarios de la axiomática. Todo el mundo sabe que el carácter externo de las matemáticas consiste en presentarse con el aspecto de aquella «larga cadena de razones» de la que Descartes hablaba: 

\begin{quote}

Toda teoría matemática es un encadenamiento de proposiciones que se deducen unas de otras conforme a las reglas de una lógica que, en lo esencial, es la establecida desde Aristóteles con el nombre de «lógica formal», convenientemente adaptada a los fines particulares del matemático.

\end{quote}

 Es, pues, una banalidad decir que este «razonamiento deductivo» es un principio de unidad para la matemática. Una observación tan superficial no puede ciertamente dar cuenta de la aparente complejidad de las diversas teorías matemáticas, no más, por ejemplo, que la pretensión de reunir en una ciencia única a la física y a la biología bajo el pretexto de que ambas aplican el método experimental. El modo de razonamiento por encadenamiento de silogismos, sólo es un \textit{mecanismo} transformador, aplicable indiferentemente a toda suerte de premisas, y no podría caracterizar, pues, la naturaleza de éstas. En otras palabras, la \textit{forma} exterior que la matemática da a su pensamiento del vehículo que la convierte en asimilable a otros\footnote{Por otra parte, todo matemático sabe que una demostración no está verdaderamente «comprendida» por más que nos hemos ceñido a verificar, paso a paso, la corrección de las deducciones que figuran en ella, si no hemos intentado concebir claramente las ideas que condujeron a construir esta cadena de deducción con preferencia a cualquier otra.}, y, para decirlo todo, del \textit{lenguaje} propio de la matemática; y no debemos esperar más de él. Codificar este lenguaje, ordenar su vocabulario y clarificar su sintaxis es hacer una obra muy útil que constituye, efectivamente, un aspecto del método axiomático, aquel que podemos llamar propiamente \textit{formalismo lógico} (o, como se dice también, la «logística»). Pero ---e insistimos en este punto--- \textit{este solo es un aspecto,} y el menos interesante. 



Lo que la axiomática se propone como fin esencial es precisamente lo que el formalismo lógico es incapaz de ofrecer por sí solo: la inteligibilidad profunda de las matemáticas. Del mismo modo que el método experimental parte de la creencia \textit{a priori} en la permanencia de las leyes naturales, el método axiomático encuentra su punto de apoyo en la convicción de que, si las matemáticas no son un encadenamiento de silogismo que se desarrollan al azar, no son tampoco una colección de artificios más o menos «astutos», hechos de aproximaciones fortuitas en las que triunfa la pura habilidad técnica. Allí donde el observador superficial sólo ve dos o más teorías muy distintas en apariencia, que se prestan, por intermedio de un matemático genial, una «ayuda inesperada» (Brunschvicg, \textit{Op. cit}. p. 446), el método axiomático enseña a buscar las razones profundas de ese descubrimiento, a encontrar las ideas comunes camufladas bajo el aparato exterior de los detalles propios de cada una de las teorías consideradas, a descubrir estas ideas y a ponerlas de manifiesto.

\subsection*{3. La noción de «estructura»}

¿Cómo se realiza dicha operación? Ahí es donde la axiomática se aproxima más al método experimental. Bebiendo como él en la fuente cartesiana, «dividirá las dificultades para resolverlas mejor». En las demostraciones de una teoría, buscará \textit{disociar} los principales resortes de los razonamientos que figuran en ellas. Después, tomando cada una de ellas \textit{aisladamente} y planteándose como un principio abstracto, desarrollará las consecuencias que le son propias. Finalmente, volviendo a la teoría estudiada, \textit{combinará} de nuevo los elementos constitutivos previamente liberados y estudiará cómo reaccionan unos con otros. No hay, por supuesto, nada nuevo en esta clásica ida y vuelta entre el análisis y la síntesis. Toda la originalidad del método reside en la forma como se aplica.

Para ilustrar con un ejemplo el procedimiento del que acabamos de dar una descripción esquemática, tomaremos una de las teorías axiomáticas más antigua (y una de las más simples), la de los \textsf{grupos abstractos}. 

 Consideremos, por ejemplo, las tres operaciones siguientes: 


\begin{enumerate}
\vspace{- 1em}

\item La adición de los números reales, donde la suma de dos números reales (positivos, negativos o nulos) se define de la manera ordinaria. 

\item La multiplicación de los enteros «módulo un número primo $p$», en donde los elementos considerados son los enteros $1, 2,\dots, (p-1)$, siendo, por convención, el «producto» de dos de estos números el \textit{resto} de la división por $p$ de su producto en el sentido ordinario.

\item La «composición» de los desplazamientos en el espacio euclídeo de tres dimensiones, siendo por definición el «compuesto» (o «pro\-duc\-to») de dos desplazamientos $S$, $T$ (tomados en este orden) el desplazamiento obtenido al efectuar primero el desplazamiento $T$ y después el desplazamiento $S$. 

\end{enumerate}

En cada una de estas tres teorías, a dos elementos, $x$, $y$ (tomados en este orden) del conjunto de elementos considerado (en el primer caso el conjunto de los números reales, en el segundo caso de los números $1, 2,\dots, (p-1)$, en el tercero el conjunto de todos los desplazamientos) se les hace corresponder (por un procedimiento particular de la teoría) un tercer elemento bien determinado, que convendremos en designar simbólicamente en los tres casos por $$ x\star y $$ (esto es: la suma de $x$ y de $y$ si son números reales, su producto «módulo $p$» si son enteros menores o iguales que $(p-1)$, su «compuesto» si se trata de desplazamientos). Si examinamos ahora las propiedades de esta «operación» en cada una de las teorías, constatamos que presentan un paralelismo notable; pero en el interior de cada una de dichas teorías, estas propiedades dependen unas de las otras, y un análisis de sus conexiones lógicas lleva a desprender un número reducido de ellas que son independientes (es decir que ninguna es consecuencia lógica de las otras). Podemos, por ejemplo\footnote{Esta elección no tiene nada de absoluta y se conocen numerosos sistemas de axiomas «equivalentes» al que explicitamos, siendo los enunciados de los axiomas de cada uno de estos sistemas consecuencias lógicas de los axiomas de uno cualquiera de los otros sistemas.}, tomar las tres siguientes, que expresaremos en nuestra notación simbólica común a las tres teorías pero que es fácil traducir al lenguaje particular de cada una de ellas: 

\begin{enumerate} [\indent a) ]
\vspace{- 1em}

\item Para cualesquiera elementos $x, y, z,$ tenemos $$ x\star (y \star z)=(x \star y) \star z $$ («asociatividad» de la operación $x\star y$).

\item Existe un elemento $e$ tal que, para todo elemento $x$, tenemos $$ e\star x=x \star e=x $$ (para la adición de los números reales es el número $0$; para la multiplicación «módulo $p$» es el número $1$; para la composición de desplazamientos es el desplazamiento \textit{«identidad»} que deja fijo cada punto del espacio).

\item Para todo elemento $x$, existe un elemento $x'$ tal que $$ x\star x' = x'\star x=e $$ (para la adición de los números reales, $x'$ es el número \textit{opuesto} $-x$; para la composición de desplazamientos, $x'$ es el desplazamiento \textit{inverso} de $x$, es decir, el que vuelve a llevar cada punto desplazado por $x$ a su posición primitiva; para la multiplicación «módulo $p$», la existencia de $x'$ resulta de un razonamiento de aritmética muy simple\footnote{Señalamos que los restos de la división por $p$ de los números $x_{1}, x_{2},..., x_{n},\dots $ no pueden ser todos distintos. Expresando que dos de dichos restos son iguales, se muestra fácilmente que una potencia $(x_{1})^{m}$ de $x_{1}$ tiene un resto igual a $1$; si $x'$ es el resto de la división por $p$ de $x^{m-1}$, se concluye que el producto «módulo $p$» de $x$ y de $x'$ es igual a $1$.}).

\end{enumerate}

Se constata entonces que las propiedades que son susceptibles de expresarse de la misma manera en las tres teorías, con la ayuda de la notación común, son consecuencias de las tres precedentes. Por ejemplo, nos proponemos demostrar que la relación $$ x\star y=x\star z\quad \text{implica}\quad y=z $$ Podríamos hacerlo en cada una de las teorías por un razonamiento que le fuera particular, pero podemos proceder de la manera siguiente, aplicable a todos los casos: de la relación $$ x\star y=x\star z $$ se deduce (teniendo $x'$ el sentido definido más arriba) $$ x'\star (x\star y)=x'\star (x\star z) $$ después aplicando $a)$ $$ (x'\star x)\star y=(x'\star x)\star z $$ utilizando $c)$ esta relación se escribe $$ e\star y= e\star z $$ y finalmente, aplicando $b)$ $$ y =z $$ que es lo que había que demostrar. En este razonamiento hemos hecho total abstracción de la \textit{naturaleza} de los elementos $x$, $y$, $z$ considerados, es decir que no tenemos necesidad de saber si eran números reales, enteros menores o iguales que $(p-1)$, o desplazamientos. La única premisa que ha intervenido es que la operación $x\star y$ sobre estos elementos satisface las propiedades \mbox{$a)$, $b)$ y~$c)$.} Entendemos, aunque no sea más que para evitar repeticiones fastidiosas que es cómodo desarrollar de \textit{una vez por todas} las consecuencias lógicas de las tres \textit{únicas} propiedades $a)$, $b)$ y $c)$. Naturalmente, por comodidad de lenguaje, hay que adoptar una terminología común. Decimos así que un conjunto en el que se ha definido una operación $x\star y$ que satisface las tres propiedades $a)$, $b)$ y $c)$ está provisto de una estructura de \textit{grupo} (o más brevemente, que es un \textit{grupo}); las propiedades $a)$, $b)$ y $c)$ se denominan los \textit{axiomas}\footnote{No hace falta decir que no hay ningún punto común entre este sentido de la palabra \textit{«axioma»} y el sentido tradicional de \textit{«verdad evidente»}.} de las estructuras de \textit{grupo}, y desarrollar sus consecuencias es desarrollar la \textit{teoría axiomática de los grupos}. 

Ahora ya podemos comprender qué es lo que hay que entender, de manera general, por una \textit{estructura matemática}. El rasgo común de las diferentes nociones designadas con este nombre genérico es que se aplican a conjuntos de elementos cuya naturaleza\footnote{Nos situamos aquí en el punto de vista «ingenuo» y no abordamos las espinosas preguntas, semifilosóficas, semimatemáticas, surgidas del problema de la «naturaleza» de los «seres» u «objetos» matemáticos. Nos bastará con decir que, poco a poco, las investigaciones axiomáticas de los siglos XIX y XX han sustituido también el pluralismo inicial de la representación mental de estos «seres» ---imaginados al principio como «abstracciones» ideales de la experiencia sensible que conservan toda la heterogeneidad de ésta--- por una noción unitaria que progresivamente conduce a todas las nociones matemáticas, primero a la del número entero, después, en una segunda etapa, a la noción de \textit{conjunto}. Esta última, considerada durante mucho tiempo como «primitiva» e «indefinible», fue objeto de polémicas sin fin debidas a su carácter de extrema generalidad y a la naturaleza muy vaga de las representaciones mentales que evoca. Las dificultades sólo se han desvanecido cuando se ha desvanecido la noción misma de \textit{conjunto} (y con ella, todos los pseudoproblemas metafísicos sobre los «seres» matemáticos) a la luz de las recientes investigaciones sobre el formalismo lógico. En esta nueva concepción, las estructuras matemáticas se convierten, propiamente hablando, en los únicos «objetos» de la matemática.

 El lector encontrará desarrollos más amplios sobre este punto en los dos artículos siguientes: J. Dieudonné: \textit{Les méthodes axiomatiques modernes et les fondements des mathématiques }(\textit{Revue Scientifique,} LXXVII (1939) p. 224-232); H. Cartan: \textit{Sur le fondement logique des mathématiques. (Revue Scientifique,} LXXXI (1943), p. 3-11).} no está especificada; para definir una estructura, se dan una o más \textit{relaciones} en las que intervienen estos elementos\footnote{En realidad, esta definición de las estructuras no es suficientemente general para las necesidades de las matemáticas. Hay que considerar también el caso en que tendrían lugar las relaciones que definen una estructura, no entre \textit{elementos} del conjunto considerado sino también entre partes de dicho conjunto, e incluso, más generalmente, entre elementos de conjuntos de «grado» aún más elevado en lo que se llama la «escala de los tipos». Para más precisiones sobre este punto, ver nuestros \textit{Elements de Mathématique}, livre I.} (en el caso de los grupos, era la relación $z=x\star y $ entre tres elementos arbitrarios); se postula después que la o las relaciones dadas satisfacen ciertas condiciones (que se enumeran) y que son los axiomas de la estructura considerada\footnote{En el caso de los grupos, habría que considerar, en rigor, como axioma, además de las propiedades $a)$, $b)$, $c)$ enunciadas más arriba, el hecho de que la relación $z =x\star y$ determina un $z$ y sólo uno, para $x$ e $y$ dados. De ordinario, se considera que esta propiedad se halla tácitamente implícita en la escritura de esta relación.}. Elaborar la teoría axiomática de una estructura dada es deducir las consecuencias lógicas de los axiomas de las estructura prohibiéndose cualquier otra hipótesis sobre los elementos considerados (en particular, cualquier hipótesis sobre su «naturaleza» propia).

\subsection*{4. Los grandes tipos de estructura} 

 Las \textit{relaciones} que forman el punto de partida de la definición de una estructura pueden ser asimismo de naturaleza bastante variada. La que interviene en las estructuras de grupo es lo que se llama una «ley de composición», es decir una relación entre tres elementos que determina al tercero de manera única en función de los dos primeros. Cuando las relaciones de definición de una estructura son «leyes de composición», la estructura correspondiente se llama \textit{estructura algebraica} (por ejemplo, una estructura de \textit{cuerpo} se define mediante dos leyes de composición. Con axiomas convenientes la adición y la multiplicación de los números reales definen una estructura de \textit{cuerpo} en el conjunto de dichos números).

Otro tipo importante viene dado por las estructuras definidas por una relación de \textit{orden}. Esta vez se trata de una relación entre dos elementos $x$, $y$, que, a menudo se enuncia «$x$ es menor o igual a $y$», y que anotaremos, en general $ x \prec y$. Aquí no suponemos ya que la relación determine de forma única uno de los elementos $x$, $y$ en función del otro. Los axiomas a los que se somete son los siguientes: 

\begin{enumerate}[\indent a)]
\vspace{- 1em}

\item Para todo $x$, tenemos $x \prec x$.

\item Las relaciones $x \prec y$ e $y \prec x$ implican $x=y$.

\item Las relaciones $x\prec y$ e $y\prec z$, implican $x\prec z$. 

 \end{enumerate}

 Un ejemplo evidente de conjunto provisto de una tal estructura es el conjunto de los enteros (o el de los números reales), reemplazando el signo $\prec$ por el signo $\le $. Observemos, sin embargo, que no hemos incluido en los axiomas la propiedad siguiente, que parece inseparable de la noción vulgar de «orden»: «cualesquiera que sean $x$ e $y$, tenemos $ x \prec y$ o $y \prec x$. Dicho de otra manera, no se excluye el caso en el que dos elementos puedan ser \textit{incomparables}. Esto, a primera vista, puede parecer paradójico, pero es fácil dar ejemplos muy importantes de \textit{estructura de orden} en los que se presenta tal fenómeno. Es lo que ocurre cuando la relación $ X \prec Y$, siendo $X$ e $Y$ partes de un mismo conjunto, significa «$X$ está contenido en $Y$»; o también cuando siendo $x$ e $y$ enteros positivos, $x \prec y$ significa «$x$ divide a $y$»; o finalmente cuando, siendo $f(x)$ y $g(x)$ funciones reales definidas en un intervalo $(a,b)$, $f(x)\prec g(x)$ significa «para todo $x$ se tiene que $f(x) \le g(x)$». Estos ejemplos muestran al mismo tiempo la gran variedad de dominios en los que intervienen las estructuras de orden y dejan presentir el interés de su estudio.

Diremos aún algunas palabras sobre un tercer gran tipo de estructuras, las \textit{estructuras topológicas} o \textit{topologías}: ofrecen una formulación matemática abstracta de las nociones intuitivas de \textit{entorno}, de \textit{límite} y de \textit{continuidad}, a las que nos conduce nuestra concepción del espacio. El esfuerzo de abstracción que necesita el enunciado de los axiomas de tal estructura es aquí netamente superior al que corresponde en los ejemplos precedentes, y el marco de esta exposición nos obliga a remitir a los lectores deseosos de precisiones sobre este punto a los tratados especializados\footnote{Ver por ejemplo nuestros \textit{Elements de mathématique}, livre III (\textit{Topologie genérale}), introducción y capítulo I.\textit{ Actual Scient et Industr.}, nº 858.}.

\subsection*{5. La estandarización del instrumento matemático}

Pensamos haber dicho suficiente para permitir al lector hacerse una idea bastante precisa del \textit{método axiomático}. Su rasgo más sobresaliente, además de lo que precede, es que permite una economía de pensamiento considerable. Las «estructuras» son herramientas para el matemático y una vez que ha discernido, entre los elementos que estudia, relaciones que satisfagan los axiomas de una estructura de un tipo conocido, dispone asimismo de todo el arsenal de teoremas generales relativos a las estructuras de este tipo allí donde, hasta entonces, debía forjarse él mismo, laboriosamente medios de abordaje cuya potencia dependía de su talento personal y que se veían entorpecidos frecuentemente con hipótesis inútilmente restrictivas, provenientes de las particularidades del problema estudiado. Podríamos decir, pues, que el método axiomático no es sino el «sistema de Taylor» de las matemáticas. 

Esta comparación es, sin embargo, insuficiente. El matemático no trabaja mecánicamente, como el obrero en una cadena de montaje. Nunca se insistirá suficientemente en el papel fundamental que juega, en sus investigaciones, una \textit{intuición} particular\footnote{Intuición que, por lo demás, se equivoca con frecuencia, como cualquier otra intuición.} que no es la intuición sensible vulgar sino más bien una suerte de adivinación directa (anterior a todo razonamiento) del normal comportamiento que debe esperar con todo derecho por parte de unos seres matemáticos que un prolongado y frecuente trato ha convertido en seres casi tan familiares como los seres del mundo real. Así, cada estructura aporta su propio lenguaje, completamente cargado de resonancias intuitivas particulares, emanadas de las teorías de las que ha desprendido el análisis axiomático que hemos descrito con anterioridad. Y para el investigador que descubre bruscamente esta estructura en los fenómenos que estudia, es como una modulación súbita que orienta de golpe en una dirección inesperada la corriente intuitiva de su pensamiento, y que ilumina con una nueva luz el paisaje matemático en el que se mueve. Piénsese ---para tomar un ejemplo antiguo--- en el progreso realizado a principios del siglo XIX con la representación geométrica de los [números] imaginarios. Desde nuestro punto de vista consistía en descubrir en el conjunto de los números complejos una estructura topológica bien conocida, la del \textit{plano euclídeo,} con todas las posibilidades de aplicación que ello implicaba y que, en manos de Gauss, Abel, Cauchy y Riemann, renovarían el \textit{Análisis} en menos de un siglo. 



Tales ejemplos se han multiplicado en los últimos cincuenta años: 

\begin{itemize}
\vspace{- 1em}

\item \textit{Espacio de Hilbert}, y más generalmente espacios funcionales que introducen las estructuras topológicas en conjuntos de elementos que ya no son \textit{puntos,} sino \textit{funciones.} 

\item \textit{Números $p$-ádicos de Hensel} en los que, cosa más sorprendente aún, la topología invade lo que, hasta entonces, era el reino de lo \textit{discreto,} de lo \textit{discontinuo} por excelencia, el conjunto de los números enteros.

\item \textit{Medida de Haar,} que amplía enormemente el campo de aplicación de la noción de integral y permite un análisis muy profundo de las propiedades de los grupos continuos.

\item Y otros tantos momentos decisivos del progreso de las matemáticas, de vuelcos en los que un relámpago de genialidad decidió la orientación nueva de una teoría, revelando en ella una estructura que no parecía \textit{a priori} tener papel alguno.

\end{itemize}

Es decir; menos que nunca la matemática queda reducida a un juego puramente mecánico de fórmulas aisladas. Más que nunca, la intuición reina con autoridad en la génesis de los descubrimientos pero disponiendo, desde entonces, de potentes palancas que le ofrece la teoría de los grandes tipos de estructura y dominando de un solo vistazo inmensos dominios unificados por la axiomática, ahí donde otrora parecía reinar el más informe caos.

\subsection*{6. Una visión de conjunto}

Guiados por la concepción axiomática, intentemos pues representarnos el conjunto del universo matemático. Ciertamente, apenas reconoceremos ya el orden tradicional que, al igual que las primeras clasificaciones de las especies animales, se limitaba a colocar una junto a otra las teorías que presentaban mayores parecidos exteriores. En lugar de los comportamientos bien delimitados del \textit{Álgebra,} del \textit{Análisis,} de la \textit{Teoría de los Números} y de la \textit{Geometría,} hallaremos, por ejemplo, la \textit{teoría de los números primos} junto a las \textit{curvas algebraicas,} o la \textit{geometría euclídea} junto a las \textit{ecuaciones integrales}; y el principio ordenador será la concepción de una \textit{jerarquía de estructuras,} que va de lo simple a lo complejo, de lo general a lo particular.

En el centro están los grandes tipos de estructuras de las que hemos enumerado antes las principales, las \textit{estructuras-madre} podríamos decir. En cada uno de estos tipos reina una diversidad bastante grande, ya que hay que distinguir la estructura más general del tipo considerado, con el menor número de axiomas, de las que se obtienen enriqueciéndola con axiomas suplementarios, aportando cada uno de ellos su cosecha de nuevas consecuencias. Es así, como la teoría de grupos, más allá de las generalidades válidas para todos los grupos, y dependiendo sólo de los axiomas enunciados más arriba, comporta una teoría particular de los \textit{grupos finitos} (donde se añade el axioma de que el número de elementos del grupo es finito), una teoría particular de los \textit{grupos abelianos} (donde $x\star y=y\star x$ para cualquier \mbox{$x$, $y$}), así como una teoría de los \textit{grupos abelianos finitos} (donde se supone que ambos axiomas se verifican simultáneamente). Asimismo, en los \textit{conjuntos ordenados,} se distinguen aquellos en los que como en el orden de los enteros o de los números reales), dos elementos cualesquiera son comparables, y que se llaman \textit{totalmente ordenados;} entre estos últimos, se estudian más particularmente aún los conjuntos llamados \textit{bien ordenados} (en los que, como para los enteros positivos, todo subconjunto tiene un «elemento mínimo»). Hay una gradación análoga en las estructuras topológicas. 

 

 Más allá de este primer núcleo, aparecen \textit{estructuras} que podríamos llamar \textit{múltiples}, en las que intervienen a la vez dos o más de las grandes estructuras-madre, no simplemente yuxtapuestas (lo que no aportaría nada nuevo) sino combinadas orgánicamente por uno o más axiomas que las ligan. Es lo que se conoce como \textit{álgebra topológica}, estudio de estructuras en las que figuran a la vez una o más \textit{leyes de composición} y una \textit{topología}, ligadas por la condición de que las operaciones algebraicas sean \textit{funciones continuas} (para la topología considerada) de los elementos que implican. No menos importante es la \textit{topología algebraica}, donde ciertos conjuntos de puntos del espacio, definidos por propiedades topológicas (\textit{símplices}, \textit{ciclos}, etc.) se toman ellos mismos como elementos sobre los que operan \textit{leyes de composición.} La combinación de las \textit{estructuras de orden} y del \textit{álgebra} es, también, fértil en resultados, y conduce por un lado a la \textit{teoría de la divisibilidad} y de los \textit{ideales,} y por otro a la \textit{integración} y a la \textit{«teoría espectral» de los operadores,} en los que la topología viene también a jugar su papel.

Más lejos empiezan por fin, hablando propiamente, las teorías particulares en las que los elementos de los conjuntos se consideran, completamente determinados en las estructuras generales analizadas hasta ahora, reciben una individualidad más caracterizada. Aquí es donde se encuentran las teorías de la matemática clásica: \textit{análisis de las funciones de variable real} o \textit{compleja,} \textit{geometría diferencial,} \textit{geometría algebraica,} \textit{teoría de los números;} pero han perdido su antigua autonomía y se han convertido ahora en encrucijadas en las que se cruzan y actúan entre sí numerosas estructuras matemáticas más generales.

Para conservar una justa perspectiva, nos hace falta, después de este rápido esquema, añadir enseguida que sólo debemos considerarlo una aproximación muy grosera del estado actual de las matemáticas tal y como es en realidad. Es a la vez \textit{esquemático, idealizado} y \textit{coagulado}. 

\begin{itemize}
\vspace{- 1em}

\item \textit{Esquemático.} Porque en detalle las cosas no ocurren de manera tan simple ni tan regular como puede parecer que hemos dicho. Hay, entre otras cosas, inesperadas vueltas hacia atrás en las que una teoría muy particular como la de los números reales viene a prestar una ayuda indispensable para la construcción de una teoría general como la \textit{Topología} o la \textit{Integración}.

\item \textit{Idealizado.} Porque hace falta que en todas las partes de las matemá\-ticas, la parte exacta de cada una de las grandes estructuras esté perfectamente reconocida y delimitada. En ciertos dominios (por ejemplo en \textit{Teoría de Números}), subsisten numerosos resultados aislados que no se han sabido clasificar ni ligar hasta ahora de manera satisfactoria con estructuras conocidas.

\item \textit{Coagulado.} Porque no hay nada más alejado del método axiomático que una concepción estática de la ciencia, y no querríamos dejar creer al lector que hemos pretendido dibujar un estado definitivo de ésta. Las estructuras no son inmutables ni en su número ni en su esencia. Es muy posible que el desarrollo ulterior de las matemáticas aumente el número de las estructuras fundamentales, revelando la fecundidad de nuevos axiomas o de nuevas combinaciones de axiomas y podemos, de antemano, dar por seguro progresos decisivos de estas invenciones de estructuras, a juzgar por los que han aportado las estructuras actualmente conocidas. Por otra parte, estas últimas no son en modo alguno edificios acabados y sería muy sorprendente que todo el jugo de su principios estuviera ya agotado. 

\end{itemize}

Así, con estos correctivos indispensables podemos tener una mayor conciencia de la vida interna de la matemática, de lo que constituye, a la vez, su unidad y su diversidad, al igual que una gran ciudad, cuyas avenidas no dejan de progresar, de manera un poco caótica, sobre el terreno circundante mientras que el centro se reconstruye periódicamente, siguiendo cada vez un plano más claro y una ordenación más majestuosa, echando abajo los viejos barrios y sus dédalos de callejones, para extender hacia la periferia avenidas más directas, más anchas y más cómodas.

\subsection*{7. Retorno al pasado y conclusión}

La concepción que hemos intentado exponer aquí no se ha formado de una sola vez y constituye el final de una evolución que viene siguiendo desde hace más de medio siglo, no sin haber encontrados serias resistencias, tanto en los filósofos como en los propios matemáticos. Muchos de estos últimos sólo consintieron, durante largo tiempo, en ver en la axiomática vanas sutilezas de lógicos, incapaces de fecundar teoría alguna. Dicha crítica se explica sin duda por un mero accidentes histórico: las primeras axiomatizaciones, que tuvieron la mayor resonancia (las de la aritmética con Dedekind y Peano, de la geometría euclídea con Hilbert) se referían a teorías \textit{univalentes,} es decir a teorías completamente determinadas por el sistema global de sus axiomas, sistema que no era, por consiguiente, susceptible de ser aplicado a ninguna otra teoría distinta de la que había sido extraído (al revés de lo que hemos visto para la teoría de los grupos, por ejemplo). Si hubiera sido así para todas las estructuras, el reproche de esterilidad dirigido al método axiomático habría estado plenamente justificado\footnote{Se ha asistido también, sobre todo en los principios de la axiomática, a un florecimiento de estructuras teratológicas, totalmente privadas de aplicaciones y cuyo único mérito era mostrar el alcance exacto de cada axioma observando lo que ocurría cuando se suprimía o se modificaba. Evidentemente, se podía tener la tentación de concluir que esos eran ¡los únicos productos que se podían esperar del método!}. Sin embargo, este ha mostrado el movimiento andando y, los rechazos que se constatan aún aquí y allá sólo se explican por lo mucho que de forma natural, le cuesta al espíritu admitir que, ante un problema concreto, una forma de intuición distinta de la directamente sugerida por los datos (y que, con frecuencia, únicamente se obtiene por medio de una abstracción superior y a veces difícil) pueda resultar igualmente fecunda.

En cuanto a las objeciones de los filósofos, se dirigen sobre todo a un terreno en el que, por falta de competencia, tendremos muchísimo cuidado en aventurarnos seriamente: el gran problema de las relaciones del mundo experimental y del mundo matemático\footnote{No abordaremos aquí las objeciones suscitadas por la aplicación de las reglas de la lógica formal a los razonamientos de las teorías axiomáticas. Se relacionan con las dificultades lógicas encontradas en la Teoría de Conjuntos. Señalemos simplemente que dichas dificultades pueden vencerse de una forma que no deja subsistir malestar ni duda alguna sobre la corrección de los razonamientos. Puede consultarse sobre este tema los artículos de H. Cartan y J. Dieudonné citados anteriormente.}. Que existe una conexión estrecha entre los fenómenos experimentales y las estructuras matemáticas, es algo que parece confirmar, de la forma más inesperada, los recientes descubrimientos de la física contemporánea, pero ignoramos totalmente las razones profundas de ello (si es que puede darse un sentido a estos términos) y tal vez lo ignoraremos siempre. En cualquier caso, es una constatación que, en este punto, podría incitar en un futuro a los filósofos a una mayor prudencia: antes de los desarrollos revolucionarios de la física moderna se gastaron muchos esfuerzos en querer hacer surgir las matemáticas, a cualquier precio, de verdades experimentales, especialmente de intuiciones espaciales inmediatas. Pero, por una parte, la física de los \textit{quanta} mostró que dicha intuición «macroscópica» de lo real cubría fenómenos «microscópicos» de una naturaleza totalmente distinta que surgían de ramas de las matemáticas que ciertamente no se habían imaginado para aplicaciones a las ciencias experimentales. Y, por otra parte, el método axiomático mostró que las «verdades» de las que se quería hacer pivotar las matemáticas no eran más que aspectos muy especiales de concepciones generales que no limitaban en absoluto su alcance. Si bien, a fin de cuentas, esta íntima fusión, cuya armoniosa necesidad nos hacía admirar, sólo aparecía como un contacto fortuito de dos disciplinas cuyos lazos están mucho más escondidos de lo que se podía suponer \textit{a priori.}

En la concepción axiomática, la matemática aparecía en suma como un reservorio de \textit{formas} abstractas (las estructuras matemáticas). Y sucede ---sin saber muy bien por qué--- que ciertos aspectos de la realidad experimental llegan a amoldarse a algunas de estas formas, como por una suerte de preadaptación. No puede negarse, por supuesto, que la mayor parte de dichas formas tenían en su origen un contenido intuitivo bien determinado, pero es precisamente al vaciarlas voluntariamente de este contenido cuando se les ha sabido dar toda la eficacia que tenían en potencia y se las ha hecho susceptibles de recibir interpretaciones nuevas y cumplir plenamente su papel elaborador.

Únicamente en este sentido de la palabra «forma» puede decirse que el método axiomático es un «formalismo». La unidad que confiere a la matemática no es el armazón de la lógica formal, unidad de esqueleto sin vida. Es la savia nutritiva de un organismo en pleno desarrollo, el dócil y fecundo instrumento de investigación en las que han trabajado conscientemente, desde Gauss, todos los grandes pensadores de las matemáticas, todos aquellos que, siguiendo la fórmula de Lejeune-Dirichlet, han tendido siempre a «sustituir» el cálculo por las ideas.

\end{document} 


