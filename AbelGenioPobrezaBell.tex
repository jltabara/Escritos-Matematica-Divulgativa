\documentclass[a4paper, 12pt, draft]{article}

%%%%%%%%%%%%%%%%%%%%%%Paquetes
\usepackage[spanish]{babel}  
\usepackage[utf8]{inputenc}
\usepackage{amsmath}
\usepackage{tcolorbox}
\usepackage{cmbright}  %%%%%%% El tipo de letra
\usepackage{setspace}
\onehalfspacing  %%%%%%%%%%% Espacio y medio de interlineado
\parskip=1em  %%%%%%%%%%%% Separacion entre parrafos
%%%%%%%%%%%%%%%%%%%%%%



%%%%%%%%%%%%%%%%%
\title{Abel. Genio y Pobreza}
\author{E. T. Bell}
\date{}
%%%%%%%%%%%%%%%%%

\begin{document}

\begin{tcolorbox}[colback=blue!5!white,colframe=blue!75!black]

\vspace{-1.8cm}
\textbf \maketitle

\end{tcolorbox}

\bigskip


\begin{quote} \it

He terminado un monumento más duradero que el bronce, y más altivo que las pirámides erigidas por los reyes, que no corroerá la lluvia, ni será destruido por los vientos ingobernados del norte, ni por la infinita sucesión de los años en el correr del tiempo. No moriré completamente; una gran parte de mi escapará a la Muerte y creceré aun lozano entre las alabanzas de la posteridad.


\end{quote}


\hfill Horacio (Odas 3, XXX).

\bigskip

Un astrólogo del año 1801 podría haber leído en las estrellas que una nueva galaxia de genios matemáticos se estaba formando para inaugurar el siglo más importante de la historia de la Matemática. En toda esa galaxia de talentos no habría una estrella más brillante que Niels Henrik Abel, el hombre de quien Hermite dijo: «Ha legado a los matemáticos algo que les mantendrá activos durante 500 años».

El padre de Abel era pastor de la pequeña aldea de Findó, en la diócesis de Kristiansand, Noruega, donde su segundo hijo, Niels Henrik, nació el 5 de agosto de 1802. En la familia paterna varios antepasados se habían distinguido en las actividades eclesiásticas, y todos, incluyendo el padre de Abel, eran hombres cultos. Anne Marie Simonsen, la madre de Abel, se distinguió principalmente por su gran hermosura, el amor a los placeres y por su carácter caprichoso, una combinación muy notable para ser la compañera de un pastor. Abel heredó de ella su hermosa presencia y el deseo muy humano de gozar de algo que no fueran los duros trabajos cotidianos, deseo que rara vez pudo satisfacer.

El pastor fue bendecido con siete hijos en una época en que Noruega estaba extraordinariamente empobrecida, como consecuencia de las guerras con Inglaterra y Suecia. De todos modos la familia era muy feliz. A pesar de la pobreza, que no siempre les permitía llenar el estómago, se mantenían alegres. Existe un cuadro notable de Abel, siendo ya genio matemático, sentado ante el fuego; el resto de la familia habla y ríe en la habitación, mientras él sigue con un ojo su Matemática, y con el otro a sus hermanos y hermanas. El ruido jamás le distrajo y podía intervenir en la charla mientras escribía.

Como algunos otros de los matemáticos de primera fila, Abel mostró  pronto su talento. Un maestro brutal dio lugar involuntariamente a que se abriera el camino para Abel. La educación en las primeras décadas del siglo XIX, era viril, al menos en Noruega. Los castigos corporales, como el método más sencillo de endurecer el carácter de los discípulos y satisfacer las inclinaciones sadistas de los pedagogos, eran generosamente administrados por cualquier travesura. Abel no aprendió en su propia piel, como se dice que Newton aprendió después de los golpes aplicados por un compañero, sino por el sacrificio de otro estudiante, que fue castigado tan brutalmente que murió. Esto era ya demasiado, hasta para los mismos directores de la enseñanza, y el maestro fue relevado de su cargo. Un matemático competente, aunque en modo alguno brillante, llenó la vacante producida. Se trataba de Bernt Michael Holmboë (1795-1850), quien más tarde (1839) publicó la primera edición de las obras completas de Abel.

Abel tenía a la sazón 15 años. Hasta entonces no había mostrado ningún talento particular para nada, salvo el hecho de que tolerara sus disgustos con cierto sentido humorístico. Bajo la cariñosa y clara enseñanza de Holmboë, Abel repentinamente descubrió lo que era. Teniendo 16 años comenzó a leer y a digerir perfectamente las grandes obras de sus predecesores, incluyendo algunas de Newton, Euler y Lagrange. La lectura de estos grandes matemáticos no sólo constituía su ocupación fundamental, sino su mayor deleite. Preguntado algunos años más tarde acerca de cómo pudo colocarse tan rápidamente en primera fila, replicó: «Estudiando a los maestros, no a sus discípulos», una prescripción que algunos autores de libros debían mencionar en sus prefacios como un antídoto de la venenosa mediocridad de su pedagogía mal inspirada.

Holmboë y Abel pronto fueron íntimos amigos. Aunque el maestro no era un matemático creador, conocía y apreciaba las obras maestras de la Matemática, y gracias a sus sugestiones Abel pronto dominó las obras más difíciles de los clásicos, incluyendo las {\it Disquisitiones Arithmeticae} de Gauss.

En la actualidad es un lugar común decir que muchas de las cosas que los antiguos maestros creyeron haber demostrado no fueron realmente probadas. Esto es cierto particularmente en lo que se refiere a algunos de los trabajos de Euler sobre las series infinitas y a algunos de los de Lagrange, sobre el Análisis. La mente aguda de Abel fue una de las primeras en descubrir las lagunas del razonamiento de sus predecesores, y resolvió dedicar buena parte de su vida a calafatear grietas haciendo riguroso el razonamiento. Uno de sus trabajos en esta dirección es la primera demostración del teorema general del binomio. Aunque ya habían sido tratados por Newton y Euler algunos casos especiales, no es fácil dar una sólida demostración del caso general, de modo que quizá no sea asombroso encontrar supuestas pruebas en algunos manuales, como si Abel no hubiera existido. Dicha demostración, sin embargo, fue sólo un detalle en el programa más vasto de Abel de aclarar la teoría y aplicación de las series infinitas.

El padre de Abel murió en 1820, a la temprana edad de 48 años. Abel tenía entonces 18. El cuidado de su madre y de los seis hermanos cayó sobre sus hombros. Confiando en sí mismo, Abel aceptó tranquilo esta responsabilidad. Abel era un alma genial y optimista. Con estricta justicia preveía que llegaría a ser un matemático respetado y que gozaría de ciertas comodidades en una cátedra universitaria. Entonces podría atender a su familia con holgura. Mientras tanto tuvo discípulos privados, y trabajó en lo que pudo. De pasada haremos notar que Abel era un maestro excepcional. Podría haber ganado lo suficiente para sus modestas necesidades, en cualquier cosa y en cualquier momento, pero teniendo a siete personas a su cargo pocas probabilidades tenía de triunfar. Jamás se lamentó de su suerte, se entregó afanosamente a la enseñanza particular, pero dedicó a las investigaciones matemáticas todos los momentos disponibles.

Convencido de que tenía en sus manos a uno de los más grandes matemáticos de todos los tiempos, Holmboë hizo cuanto pudo para lograr un subsidio para el joven, y contribuyó tan generosamente como le fue posible con su pecunio particular, no muy abundante. Pero el país era pobre hasta el punto de pasar hambre, y casi nada podía hacerse. En aquellos años de privación y de incesante trabajo, Abel se inmortalizó, pero sembró las semillas de la enfermedad que habría de matarle antes de que realizara la mitad de su obra.

La primera aspiración ambiciosa de Abel fue estudiar la ecuación general de quinto grado («quíntica»). Todos sus grandes predecesores en álgebra habían agotado sus esfuerzos para obtener una solución sin conseguirlo. Podremos imaginar fácilmente la alegría de Abel cuando creyó erróneamente que había triunfado. A través de Holmboë la supuesta solución fue enviada al más docto matemático danés de la época, quien por fortuna para Abel pidió algunos detalles sin comprometer una opinión acerca de la exactitud de la solución. Mientras tanto Abel había encontrado el fallo en su razonamiento. La supuesta solución no era de modo alguno la solución. Este fracaso produjo en él una saludable conmoción, poniéndole en el camino exacto al hacerle dudar de si siempre era posible una solución algebraica. Demostró la imposibilidad. Por entonces tenía 19 años.

Como esta cuestión de la «quíntica» general desempeña en álgebra un papel análogo al del experimento crucial para decidir el destino de toda una teoría científica, merece un momento de atención. Citaremos ahora algunas de las cosas que el mismo Abel dice.

La naturaleza del problema se explica fácilmente. En los primeros cursos de Álgebra aprendemos a resolver las ecuaciones generales de primer y segundo grado con una incógnita $x$, o sea
\begin{equation*}
\begin{split}
ax+b=0\\
ax^2+bx+c=0
\end{split}
\end{equation*}
y algo más tarde las de tercero y cuarto grado, o sea
\begin{equation*}
\begin{split}
ax^3+bx^2+cx+d=0 \\
ax^4+bx^3+cx^2+dx+e=0
\end{split}
\end{equation*}
Esto es, establecemos fórmulas finitas (cerradas) para cada una de estas ecuaciones generales de los primeros cuatros grados, expresando la incógnita $x$ en función de los coeficientes dados $a$, $b$, $c$, $d$, $e$. La solución de una de esas cuatro ecuaciones que se pueden obtener por medio de un número finito de sumas, multiplicaciones, sustracciones, divisiones y extracción de raíces, de los coeficientes dados, se llama algebraica. La importante calificación en esta definición de una solución algebraica es «finita»; no hay dificultad para encontrar soluciones de cualquier ecuación algebraica que no contenga extracción de raíces, aunque implique una infinidad de las otras operaciones racionales.

Después de este triunfo con las ecuaciones algebraicas de los cuatro primeros grados, los algebristas lucharon durante casi tres siglos para obtener la solución algebraica de la ecuación general de quinto grado. Fracasaron: entonces intervino Abel.

Vamos a reproducir los siguientes párrafos en parte porque muestran su gran inventiva en el pensamiento matemático y en parte por su interés intrínseco. Corresponden a la memoria de Abel {\it Sobre la resolución algebraica de ecuaciones}.

\begin{quote}\small

Uno de los problemas más interesantes del Álgebra es el de la solución algebraica de las ecuaciones, y observamos que casi todos los matemáticos distinguidos se han ocupado de este tema. Llegamos sin dificultad a la expresión de las raíces de las ecuaciones de los cuatro primeros grados en función de sus coeficientes. Fue descubierto un método uniforme para resolver estas ecuaciones, y se creyó que sería aplicable a las ecuaciones de cualquier grado, pero, a pesar de todos los esfuerzos de Lagrange y de otros distinguidos matemáticos, el fin propuesto no fue alcanzado. Esto llevó a la creencia de que la solución de las ecuaciones generales era algebraicamente imposible; pero esta creencia no podía ser comprobada, dado que el método seguido sólo llevaba a conclusiones decisivas en los casos en que las ecuaciones eran solubles. En efecto, los matemáticos se proponían resolver ecuaciones sin saber si era posible. Así se podía llegar a una solución, pero si por desgracia la solución era imposible, podríamos buscarla durante una eternidad sin encontrarla. Para llegar infaliblemente a una conclusión debemos por tanto seguir otro camino. Podemos dar al problema tal forma que siempre sea posible resolverlo, cosa que podemos hacer con cualquier problema. En lugar de preguntarnos sí existe o no una solución de relación que no nos es conocida, debemos preguntarnos si tal relación es en efecto posible...  Cuando se plantea un problema de esta forma, el enunciado contiene el germen de la solución e indica el camino que debe seguirse, y yo creo que habrá pocos ejemplos donde seamos incapaces de llegar a proposiciones de más o menos importancia, hasta cuando la complicación de los cálculos impide una respuesta completa al problema.

\end{quote}

Abel sigue diciendo que debe seguirse el método científico, pero ha sido poco usado debido a la extraordinaria complicación de los cálculos algebraicos que supone. «Pero, añade Abel, en muchos ejemplos esta complicación es sólo aparente y se desvanece en cuanto se aborda», y Abel añade: «He tratado de esta forma diversas ramas del Análisis, y aunque muchas veces me he encontrado ante problemas más allá de mi capacidad, he llegado de todos modos a gran número de resultados generales que aclaran la naturaleza de esas cantidades cuya dilucidación es el objeto de las Matemáticas. En otra ocasión mencionaré los resultados a que he llegado en esas investigaciones y el procedimiento que me ha conducido a ellos. En la presente memoria trataré el problema de la solución algebraica de las ecuaciones en toda su generalidad.»

Luego presenta dos problemas generales relacionados entre sí que se propone discutir:

\begin{enumerate}
\vspace{-1em}

   \item  Encontrar todas las ecuaciones de cualquier grado que sean resolubles algebraicamente.
   
   \item Determinar si una ecuación es o no resoluble algebraicamente.

\end{enumerate}

En el fondo, dice Abel, estos dos problemas son uno mismo, y aunque no pretende una completa solución, indica métodos seguros  para tratarlos de un modo completo.

La capacidad inventiva de Abel se aplicó a problemas más vastos antes de que tuviera tiempo de volver sobre éste, y su solución completa, el enunciado explícito de las condiciones necesarias y suficientes para que una ecuación algebraica se pueda resolver algebraicamente, es reservada a Galois. Cuando esta memoria de Abel fue publicada en 1818, Galois tenía 16 años y había iniciado su carrera de descubrimientos fundamentales. Galois conoció y admiró más tarde la obra de Abel, pero es probable que Abel jamás llegase a oír el nombre de Galois aunque cuando Abel visitó París, él y su brillante sucesor tan sólo estaban separados escasos kilómetros.

Aunque la labor de Abel en álgebra marca una época, pasa a un segundo plano por su creación de una nueva rama del Análisis. Esta obra es, como dijo Legendre, el «monumento que resistirá al tiempo». Si la historia de su vida nada añade al esplendor de sus hazañas, al menos nos muestra lo que el mundo perdió cuando Abel murió. Es un relato algo desalentador. Sólo su jovialidad perenne y su valor indomable en la lucha contra la pobreza, así como la falta de aliento por parte de los príncipes de las Matemáticas de su época amenizan la historia. Abel, sin embargo, encontró un generoso amigo, además de Holmboë.

En junio de 1822, cuando Abel tenía 19 años, completó sus estudios en la Universidad de Cristianía. Holmboë hizo todo lo posible por aliviar la pobreza del joven, convenciendo a sus colegas que debían contribuir para hacer posible que Abel continuara sus investigaciones matemáticas. Estos colegas hubieran deseado hacerlo, pero también eran muy pobres. Abel quería salir pronto de Escandinavia, deseaba visitar Francia, la reina matemática del mundo de aquellos días, donde esperaba conocer a las más grandes figuras (Abel se encontraba en realidad por encima de algunas de ellas, pero no lo sabía). Soñaba también con viajar por Alemania y hablar con Gauss, el príncipe indiscutido de todos ellos.

Los matemáticos y astrónomos amigos de Abel persuadieron a la Universidad para que pidiera al gobierno noruego un subsidio con objeto de que el joven pudiera estudiar Matemáticas en Europa. Para impresionar a las autoridades, Abel presentó una extensa memoria, que, a juzgar por su título, estaba probablemente relacionada con las actividades que le dieron más fama. El autor tenía un alto concepto de su obra, y creía que su publicación por la Universidad sería un honor para Noruega. Por desgracia la Universidad luchaba con dificultades económicas y la memoria se perdió. Después de una larga deliberación, el gobierno llegó a un acuerdo, pero en lugar de hacer lo que era sensato, es decir, enviar a Abel inmediatamente a Francia y Alemania, le concedió una pensión para que continuara sus estudios universitarios en Cristianía, con objeto de que perfeccionara su francés y su alemán. Esta era la solución que podía esperarse del sentido común de aquellos importantes funcionarios, pero el sentido común no siempre se aviene con el genio.

Abel trabajó año y medio en Cristianía sin perder el tiempo, pues, durante esos meses, se dedicó a luchar, no siempre triunfalmente, con el alemán y se inició más favorablemente en el francés, pero al mismo tiempo trabajó incesantemente en su matemática. Con su incurable optimismo también se comprometió con una joven, Crelly Kemp. Al fin, el 27 de agosto de 1825, cuando Abel tenía 23 años, sus amigos vencieron la última objeción del gobierno y un real decreto le concedió los fondos suficientes para viajar y estudiar durante un año en Francia y Alemania. No le concedieron mucho, pero el hecho de que le dieran algo, a pesar de las malas condiciones financieras del país, dice más en favor del estado de civilización de Noruega en 1825 que toda una enciclopedia de artes e industrias. Abel estaba muy agradecido. Tardó cerca de un mes en arreglar sus asuntos antes de partir, pero trece meses antes, creyendo inocentemente que todos los matemáticos eran tan generosos como él, ganó un escalón antes de haber puesto los pies en él.

De su propio bolsillo, sólo Dios sabe cómo, Abel pagó la impresión de la memoria en que demostraba la imposibilidad de resolver algebraicamente la ecuación general de quinto grado. Era una impresión muy defectuosa, pero la mejor que podía obtenerse en Noruega en aquella época. Abel creyó ingenuamente que esta memoria sería su pasaporte científico para los grandes matemáticos del continente. Esperaba que particularmente Gauss reconocería los grandes méritos de la obra, concediéndole una larga entrevista. No podía sospechar que «el príncipe de los matemáticos» no siempre mostraba una generosidad principesca para los jóvenes matemáticos que luchaban para que sus méritos fueran reconocidos.

Gauss recibió el trabajo, y Abel supo cuál había sido el recibimiento que le dispensó. Sin dignarse leerlo lo arrojó a un lado exclamando: «He aquí otra de esas monstruosidades». Abel resolvió no visitar a Gauss. Después de este suceso sintió gran antipatía por él, antipatía que manifestaba siempre que encontraba ocasión. Abel afirma que Gauss escribía confusamente e insinúa que los alemanes le consideraban en más de lo que valía. No es posible decir quien de los dos, Gauss o Abel, perdió más por esta antipatía perfectamente comprensible.

Gauss ha sido muchas veces censurado por su «orgulloso desprecio» en esta ocasión, pero quizás sean palabras demasiado fuertes para calificar su conducta. El problema de la ecuación general de quinto grado era muy conocido. Y tanto los matemáticos reputados como los aficionados a la Matemática se habían ocupado de él. Si en la actualidad cualquier matemático recibiera una supuesta prueba de la cuadratura del círculo, podría o no escribir una cortés carta para acusar recibo, pero es casi seguro que el manuscrito sería arrojado al cesto de los papeles, pues todos los matemáticos saben que Lindemann, en 1882, demostró que es imposible cuadrar el círculo valiéndose tan sólo de la regla y el compás, los únicos instrumentos que manejan los aficionados y de los que también se valió Euclides. Se sabe también que la demostración de Lindemann es accesible a cualquiera. En 1824, el problema de la quíntica general estaba casi a la par del problema de la cuadratura del círculo. Esto explica la impaciencia de Gauss. Recordaremos, sin embargo, que la imposibilidad no había sido aún probada, y el trabajo de Abel proporcionaba la demostración. Si Gauss hubiera leído algunos párrafos seguramente que la memoria le habría interesado y habría sido capaz de refrenar su temperamento. Es una lástima que no lo hiciera. Una palabra de Gauss y los méritos de Abel habrían sido reconocidos. También es posible que su vida se hubiera prolongado, como veremos cuando hayamos expuesto toda su historia.

Después de dejar su hogar, en septiembre de 1825, Abel visitó primeramente a los más notables matemáticos y astrónomos de Noruega y Dinamarca, y luego, en lugar de apresurarse a ir a Göttingen para conocer a Gauss, como era su propósito, marchó a Berlín.

Allí tuvo la inmensa fortuna de encontrarse con un hombre, August Leopold Crelle (1780-1856) que iba a ser para él un segundo Holmboë y que tenía mucho más peso en el mundo matemático de lo que tenía el generoso noruego. Pero si Crelle ayudó a que Abel lograra una reputación, éste le pagó ayudándole para que aumentara Crelle la suya. Para los que actualmente cultivan la Matemática, el nombre de Crelle es familiar, pues esa palabra, más que el nombre de un individuo, significa el gran periódico que fundó, y cuyos tres primeros volúmenes contienen 22 trabajos de Abel. El periódico permitió que Abel fuera conocido, o al menos más ampliamente conocido por los matemáticos del continente que hubiera podido serlo sin él. La gran obra de Abel inició el periódico tan estrepitosamente, que este estrépito fue oído por todo el mundo matemático, y finalmente el periódico labró la reputación de Crelle. Este aficionado a las Matemáticas merece algo más que una simple mención. Su capacidad para los negocios y su seguro instinto para elegir colaboradores que fueran verdaderos matemáticos, hicieron más por el progreso de las Matemáticas en el siglo XIX que media docena de doctas academias.

Crelle era un autodidacta amante de la Matemática más que un matemático creador. Su profesión era ingeniero civil. Llevó a la cima su obra al construir el primer ferrocarril en Alemania, lo que le proporcionó abundantes ingresos. En sus horas de ocio se dedicaba a la Matemática, que fue para él más que una simple diversión. Contribuyó a la investigación matemática antes y después de haber fundado, en 1826, su {\it Journal für die reine und angewandte Mathematik} ({\it Periódico para la Matemática pura y aplicada}), que fue un gran estímulo para los matemáticos alemanes. Esta es la gran contribución de Crelle al progreso de la Matemática.

Esta revista fue el primer periódico del mundo dedicado exclusivamente a la investigación matemática. Las exposiciones de las obras antiguas no eran bien recibidas. Los trabajos eran aceptados cualquiera que  fuera su autor, siempre que la cuestión fuera nueva, verdadera y de «importancia» suficiente, una exigencia intangible, para merecer la publicación. Desde 1823 esta revista apareció regularmente cada tres meses, y la palabra «Crelle» sigue siendo familiar para todos los matemáticos. En el caos después de la primera guerra mundial el «Crelle» estuvo a punto de derrumbarse, pero fue sostenido por suscriptores de todo el mundo que no se resignaban a que se perdiera este gran monumento de una civilización más tranquila que la nuestra. Actualmente existen centenares de periódicos dedicados, totalmente o en considerable parte, al progreso de las Matemáticas puras y aplicadas.

Cuando Abel llegó a Berlín en 1825, Crelle estaba pensando en lanzarse a esta gran aventura con sus propios medios económicos y Abel tuvo una parte en que tomara la decisión. Existen dos relatos acerca de la primera visita de Abel a Crelle, ambos interesantes. Por aquella época Crelle desempeñaba un cargo del gobierno para el que tenía poca aptitud y menos gusto: el de examinador del Instituto de Industria (Gewerbe-Institut) en Berlín. El relato de Crelle, de tercera mano (Crelle a Weierstrass y éste a Mittag-Leffler), de esta visita histórica es el siguiente:

\begin{quote}\small

Un buen día, un joven muy desconcertado, con un rostro juvenil e inteligente, penetró en mi habitación. Creyendo que se trataba de un candidato para ingresar en el Instituto le expliqué que eran necesarios diversos exámenes. Al fin, el joven abrió su boca y dijo en muy mal alemán: «No exámenes, sólo Matemáticas».

\end{quote}

Crelle vio que Abel era extranjero e intentó hablarle en francés, que Abel comprendía con alguna dificultad. Crelle le preguntó entonces qué labor había hecho en la Matemática. Con diplomacia Abel replicó que había leído, entre otras cosas, el trabajo de Crelle de 1823, recientemente publicado, sobre «facultades analíticas» (ahora llamadas «factoriales»). Dijo que la obra le había parecido muy interesante, pero, ya no tan diplomáticamente, señaló aquellas partes de la obra que estaban equivocadas. Fue aquí donde Crelle mostró su grandeza. En lugar de enfurecerse por la osada presunción de aquel joven, aguzó su oído, y le preguntó nuevos detalles que siguió con la mayor atención. Durante largo rato hablaron de Matemática, aunque tan sólo algunas partes de ella eran inteligibles para Crelle. Pero entendiera o no todo lo que el visitante le dijo, Crelle vio claramente lo que Abel era. Crelle jamás pudo comprender una décima parte de lo que Abel sabía, pero su seguro instinto le afirmaba que Abel era un matemático de primera categoría e hizo todo lo que estaba en su mano para que su joven protegido fuera conocido. Antes de que terminara la entrevista, Crelle había pensado que Abel sería uno de los primeros colaboradores del proyectado {\it Journal}.

El relato de Abel difiere, aunque no esencialmente. Leyendo entre líneas podemos ver que las diferencias se deben a la modestia de Abel. Al principio Abel temió que su proyecto de interesar a Crelle estaba destinado a caer en el vacío. Crelle no comprendía lo que el joven deseaba, ni sabía quién era, pero en cuanto Crelle le preguntó qué había leído en cuestiones matemáticas, la situación se aclaró considerablemente. Cuando Abel mencionó las obras de los maestros que había estudiado, Crelle prestó inmediatamente atención. Tuvieron una larga charla sobre diversos problemas importantes, y Abel se aventuró a hablar de su demostración de la imposibilidad de resolver algebraicamente la quíntica general. Crelle no había oído hablar de tal demostración y debía haber en ella algo equivocado. Pero aceptó un ejemplar del trabajo, lo hojeó, y, admitiendo que los razonamientos estaban más allá de su capacidad, publicó la prueba ampliada de Abel en su {\it Journal}. Aunque era un matemático limitado, sin pretensiones de grandeza científica, Crelle era un hombre de mente amplia, un verdadero gran hombre.

Crelle llevó a Abel a todas partes, considerándolo como el mayor descubrimiento matemático que había hecho. El autodidacta suizo Steiner, «el más grande geómetra después de Apolonio», acompañaba algunas veces a Crelle y Abel en sus paseos. Cuando los amigos de Crelle le veían llegar con estos dos genios, exclamaban: «Ahí viene el padre Adán con Caín y Abel».

La generosa sociabilidad de Berlín comenzó a distraer a Abel de su trabajo, y entonces marchó a Friburgo donde pudo concentrarse. Fue allí donde tomó cuerpo su obra máxima, la creación de lo que ahora se llama el teorema de Abel, pero tenía que marchar a París para conocer a los más grandes matemáticos franceses de la época: Legendre, Cauchy, etc.

Puede decirse que la recepción dispensada a Abel por los matemáticos franceses fue tan cortés como podía esperarse de distinguidos representantes de un pueblo muy cortés, en una época extraordinariamente cortés. Todos ellos fueron muy corteses con él, y esto es todo lo que obtuvo Abel de la visita en que había puesto tan ardientes esperanzas. Como es natural no llegaron a conocerle, ni supieron quién era, pues tampoco hicieron verdaderos esfuerzos para descubrir su personalidad. Si Abel abría la boca acerca de su propia obra, ellos, manteniéndose a cierta distancia, comenzaban inmediatamente a platicar acerca de su propia grandeza. Si no hubiera sido por su indiferencia, el venerable Legendre hubiera sabido ciertas cosas acerca de la pasión de su vida (las integrales elípticas) que le hubieran interesado extraordinariamente. Pero fue en el preciso momento en que subía a su carruaje cuando Abel le encontró, y sólo tuvo tiempo para saludarle cortésmente. Más tarde le presentó rendidas excusas.

En julio de 1826, Abel se alojó en París con una pobre pero codiciosa familia que le proporcionaba dos malas comidas por día y un inmundo aposento a cambio de un alquiler bastante elevado. Transcurridos cuatro meses de permanencia en París, Abel escribía sus impresiones a Holmboë:

\begin{quote}\small

\hfill París, 24 de Octubre de 1826.

\bigskip

Te diré que esta ruidosa capital del continente me ha producido por el momento el efecto de un desierto. Prácticamente no conozco a nadie, a pesar de hallarnos en la más agradable estación cuando todos se hallan en la ciudad...  Hasta ahora he conocido a Mr. Legendre, a Mr. Cauchy y a Mr. Hachette y a algunos matemáticos menos célebres, pero muy capaces: Mr. Saigey, editor del {\it Bulletin des Sciences} y Mr. Lejeune-Dirichlet, un prusiano que vino a verme el otro día creyéndome compatriota suyo. Es un matemático de gran penetración. Con Mr. Legendre ha probado la imposibilidad de resolver la ecuación
$$
x^5+y^5=z^5
$$
en enteros, y otras cosas importantes. Legendre es extraordinariamente cortés, pero desgraciadamente muy viejo. Cauchy está loco...  lo que escribe es excelente, pero muy confuso. Al principio no comprendía prácticamente nada, pero ahora veo algunas cosas con mayor claridad...  Cauchy es el único que se preocupa de Matemáticas puras. Poisson, Fourier, Ampére, etc., trabajan exclusivamente en problemas de magnetismo y en otras cuestiones físicas. Mr. Laplace creo que ahora no escribe nada. Su último trabajo fue un complemento a su teoría de las probabilidades. Muchas veces le veo en el Instituto. Es un buen sujeto. Poisson es un agradable camarada; sabe como comportarse con gran dignidad; Mr. Fourier, lo mismo, Lacroix es muy viejo. Mr. Hachette va a presentarme a algunos de estos hombres.

Los franceses son mucho más reservados con los extranjeros que los alemanes. Es extraordinariamente difícil obtener su intimidad, y no me aventuro a presentar mis pretensiones. En fin, todo principiante tiene aquí grandes dificultades para hacerse notar. Acabo de terminar un extenso tratado sobre cierta clase de funciones transcendentes [su obra maestra] para presentarlo al Instituto [Academia de Ciencias], en la sesión del próximo lunes. Lo he mostrado a Mr. Cauchy pero apenas se ha dignado mirarlo. Me aventuro a decir sin jactancia que es una obra de importancia. Tengo curiosidad por oír la opinión del Instituto y no dejaré de comunicártela...

\end{quote}

Luego cuenta lo que está haciendo, y añade un resumen de sus proyectos no muy optimistas. «Lamento haber pedido dos años para mis viajes, pues año y medio habrían sido suficientes.»

Abel deseaba abandonar Europa Continental, pues quería dedicar su tiempo a trabajar en lo que había ideado.

\begin{quote}
\small
Muchas cosas me quedan por hacer, pero en tanto me halle en el extranjero todo lo que haga será bastante malo. ¡Si yo tuviera mi cátedra como el Sr. Kielhau tiene la suya! Mi posición no está asegurada, pero no me inquieto acerca de esto; si la fortuna no me acompaña en una ocasión, quizá me sonría en otra.


\end{quote}

De una carta de fecha anterior dirigida al astrónomo Hansteen, tomamos dos párrafos, el primero relacionado con el gran proyecto de Abel de colocar el Análisis matemático, tal como existía en su época, sobre un fundamento firme, y el segundo mostrando algo de su aspecto humano.

\begin{quote}\small

En el análisis superior pocas proposiciones han sido demostradas con un rigor suficiente. En todas partes encontramos el desgraciado procedimiento de razonar desde lo especial a lo general, y es un milagro que esta forma de razonar sólo rara vez nos haya llevado a la paradoja. Es en efecto extraordinariamente interesante buscar la razón de esto. Esta razón, en mi opinión, reside en el hecho de que las funciones que hasta ahora se presentan en el Análisis pueden ser expresadas en su mayor parte por potencias...  Cuando seguimos un método general ello no es muy difícil [para evitar trampas]; pero tengo que ser muy circunspecto, pues las proposiciones sin prueba rigurosa (es decir sin prueba alguna) se han apoderado de mí en tal grado que constantemente corro el riesgo de usarlas sin nuevo examen. Estas bagatelas aparecerán en el {\it Journal} publicado por el Sr. Crelle.

\end{quote}

Expresa luego su gratitud por la forma de ser tratado en Berlín.

\begin{quote}\small

Cierto es que pocas personas se interesaron por mí. Pero estas pocas han sido infinitamente cariñosas y amables. Quizá pueda responder en alguna forma a las esperanzas que han puesto en mí, pues es desagradable para un bienhechor ver perderse todos sus esfuerzos.

\end{quote}

Abel cuenta entonces cómo Crelle le pidió que fijara su residencia en Berlín. Crelle estaba utilizando toda su habilidad para colocar al noruego Abel en una cátedra de la Universidad de Berlín. Esta era la Alemania de 1826. Abel era ya una segura promesa, y se veía en él el sucesor matemático más legítimo de Gauss. Poco importaba que se tratase de un extranjero. Berlín en 1826 deseaba lo mejor que hubiera en matemática. Un siglo más tarde la figura más descollante en la física matemática, Einstein, fue forzada a abandonar Berlín. He aquí el progreso. Pero continuemos con el confiado Abel.

\begin{quote}\small

Pensé al principio marchar directamente desde Berlín a París, satisfecho con la promesa de que el Sr. Crelle me acompañaría. Pero el Sr. Crelle tuvo dificultades, y tendré que viajar solo. Estoy constituido de tal modo que no puedo tolerar la soledad. Cuando estoy solo me hallo deprimido, me siento pendenciero, y tengo poca inclinación para el trabajo. Por tanto me he dicho a mí mismo que sería mucho mejor ir con el Sr. Boeck a Viena, y este viaje me parece injustificado por el hecho de que en Viena hay hombres como Litrow, Burg, y otros, todos ellos excelentes matemáticos; añádase también que será la única ocasión en mi vida de hacer este viaje. ¿Hay algo que no sea razonable en este deseo mío de ver algo de la vida del Sur? Puedo trabajar activamente mientras viajo. Una vez en Viena, existe para ir a París, una vía directa por Suiza. ¿Por qué no ver un poco todas estas cosas? ¡Dios mío! también a mí me gustan las bellezas de la naturaleza como a cualquier otro. Este viaje me hará llegar a París dos meses más tarde, esto es todo. Podré rápidamente recuperar el tiempo perdido. ¿No le parece que este viaje me hará mucho bien?

\end{quote}

Abel marchó al Sur, dejando su obra maestra al cuidado de Cauchy para que la presentara al Instituto. El prolífico Cauchy estaba entonces muy atareado recogiendo sus propios frutos, y no tenía tiempo para examinar los mejores frutos que el modesto Abel había depositado en su cesta. Hachette, un simple ayudante de matemático, presentó la obra de Abel {\it Memoria sobre una propiedad general de una clase muy extensa de funciones trascendentes}, a la Academia de Ciencias de París, el 10 de octubre de 1826. Esta es la obra que Legendre calificó más tarde, empleando palabras de Horacio, de «{\it monumentum aere perennius}», y la labor de quinientos años que, según Hermite, había dejado Abel, a las futuras generaciones de matemáticos. Era una de las más grandes conquistas de la Matemática.

¿Qué sucedió? Legendre y Cauchy fueron nombrados jueces; Legendre tenía 74 años, Cauchy 39. Legendre se quejó, en carta dirigida a Jacobi (5 de abril de 1829), de que «percibimos que la memoria era apenas legible; estaba escrita con una tinta casi blanca y las letras defectuosamente formadas; estuvimos de acuerdo en que el autor debió proporcionarnos una copia más limpia para ser leída». Cauchy se llevó la memoria a su casa, la extravió y todo quedó olvidado.

Para encontrar un parangón con este fenomenal olvido tendríamos que imaginarnos a un egiptólogo que perdiera la Piedra Roseta. Tan sólo por un verdadero milagro pudo ser desenterrada la memoria después de la muerte de Abel. Jacobi oyó hablar de ella a Legendre, con quien Abel mantuvo correspondencia después de volver a Noruega, y en una carta fechada el 14 de marzo de 1829, Jacobi exclama: «¡Qué descubrimiento es este de Abel!...  ¿Cómo es posible que este descubrimiento, quizá el más importante descubrimiento matemático que ha sido hecho en nuestro siglo, se haya comunicado a su Academia hace dos años y haya escapado de la atención de sus colegas?» Esta pregunta llegó hasta Noruega. Resumiendo esta larga historia diremos que el cónsul noruego en París hizo una reclamación diplomática acerca del perdido manuscrito y Cauchy lo encontró en 1830. Pero hasta el año 1841 no fue impreso en las {\it Mémoires présentés par divers savants de l'Académie royale des sciences de l'Institut de France, vol. 7, pp. 176-264}. Para coronar esta epopeya {\it in parvo} de crasa incompetencia, el editor o el impresor o ambos perdieron el manuscrito antes de que fueran leídas las pruebas de imprenta. La Academia en 1830, quiso sincerarse con Abel concediéndole el gran premio de Matemática en unión con Jacobi, pero Abel había muerto. 
Los siguientes párrafos de la memoria muestran su objeto:

\begin{quote}\small

Las funciones transcendentes hasta ahora consideradas por los mate\-máticos son escasas en número. Prácticamente toda la teoría de funciones transcendentes se reduce a la de funciones logarítmicas, circulares y exponenciales, funciones que en el fondo forman una sola especie. Tan sólo recientemente se ha comenzado a considerar algunas otras funciones. Entre las últimas, las transcendentes elípticas, algunas de cuyas notables y elegantes propiedades han sido desarrolladas por Mr. Legendre, ocupan el primer lugar. El autor [Abel] considera, en la memoria que tiene el honor de representar a la Academia, una clase muy extensa de funciones, todas aquellas cuyas derivadas pueden expresarse por medio de ecuaciones algebraicas cuyos coeficientes sean funciones racionales de una variable, y ha demostrado para estas funciones propiedades análogas a la de las funciones logarítmicas y elípticas...  y ha llegado al siguiente teorema:

{\it Si tenemos varias funciones cuyas derivadas pueden ser raíces de una y la misma ecuación algebraica, cuyos coeficientes son funciones racionales de una variable, podemos siempre expresar la suma de cualquier número de tales funciones por una función algebraica y logarítmica, siempre que establezcamos cierto número de relaciones algebraicas entre las variables de las funciones en cuestión.}



El número de estas relaciones no depende en modo alguno del número de funciones, sino sólo de la naturaleza de las funciones particulares consideradas

\end{quote}

Este teorema se conoce hoy con el nombre de Teorema de Abel, cuya demostración no es otra cosa que «un maravilloso ejercicio de Cálculo integral». Lo mismo que en Álgebra, en Análisis Abel alcanzó su prueba con una soberbia parsimonia. La prueba puede decirse, sin exageración, que está dentro de los alcances de un muchacho de 17 años que haya seguido el primer curso de Cálculo. No hay nada ampuloso en la simplicidad clásica de la prueba de Abel, pero no puede decirse lo mismo de algunas de las ampliaciones y retoques geométricos de la demostración original realizados en el siglo XIX. La prueba de Abel es como una estatua de Fidias; algunas de las otras semejan una catedral gótica y hasta una construcción barroca.

Existen motivos para una posible confusión en el párrafo citado de Abel. Abel sin duda quiso ser amablemente cortés para un anciano que le había protegido, en el mal sentido, cuando le conoció, pero que de todos modos había empleado gran parte de su larga vida de trabajo en un importante problema sin ver lo que había dentro de él. No es cierto que Legendre haya estudiado las funciones elípticas, como las palabras de Abel parecen indicar; lo que ocupó a Legendre gran parte de su vida fueron las integrales elípticas, que son tan diferentes de las funciones elípticas, como lo es un caballo del carro del cual tira, y ahí se encuentra precisamente el germen de una de las más grandes contribuciones de Abel a la Matemática. La cuestión es muy sencilla para quien haya seguido un curso elemental de Trigonometría, y para evitar fatigosas explicaciones de cuestiones elementales, las omitiremos en nuestra exposición.

Para quienes han olvidado todo lo que supieron de Trigonometría podemos presentar la esencia, la metodología de los progresos de Abel, recurriendo a una analogía. Nos referimos al carro y al caballo. El conocido proverbio acerca de colocar el carro delante del caballo, explica lo que Legendre hizo. Abel vio que si el carro tiene que moverse hacia adelante, el caballo tendrá que precederle. Mencionaremos otro ejemplo. Francis Galton, en sus estudios estadísticos de la relación entre la pobreza y la embriaguez crónica, fue llevado por su mente imparcial a reconsiderar la forma en que los indignados moralistas valoraban tales fenómenos sociales. En lugar de aceptar que las gentes son depravadas porque beben en exceso, Galton invirtió su hipótesis y aceptó provisionalmente que las gentes beben en exceso porque han heredado malas condiciones morales de sus antepasados, en una palabra: beben porque son depravados. Dando de lado todos los consejos moralizadores de los reformadores, Galton se aferraba a una hipótesis científica, no sentimental, a la cual pudo aplicar el razonamiento imparcial de la Matemática. Su trabajo no ha sido aún registrado socialmente. Por el momento nos bastará hacer notar que Galton, como Abel, invirtió su problema, colocando lo de arriba abajo, lo de dentro afuera, lo de atrás adelante, y lo de adelante atrás. Como Hiawatha y sus fabulosos mitones, Galton colocó dentro el lado de la piel y lo de dentro afuera.

Todo esto dista mucho de ser una trivialidad. Era uno de los métodos más poderosos para el descubrimiento (o invención) matemático hasta entonces ideado, y Abel fue el primer ser humano que lo usó conscientemente en sus investigaciones. «Siempre debéis invertir», como Jacobi dijo cuando le preguntaban el secreto de su descubrimiento matemático. Jacobi recordaba lo que Abel y él habían hecho. Si la solución del problema se hace imposible, intentemos invertir el problema. Por tanto, si encontramos incomprensible el carácter de Cardano cuando lo examinamos considerándolo como un hijo de su padre, desplacemos la cuestión, invirtámosla, y veamos lo que resulta cuando analicemos al padre de Cardano como el progenitor y creador de su hijo. En lugar de estudiar la «herencia» concentrémonos en la «dotación». Dirijámonos ahora a quienes recuerdan las lecciones de Trigonometría.

Supongamos que los matemáticos han sido tan ciegos que no hayan visto que $\sen(x)$, $\cos(x)$, y las otras funciones trigonométricas directas son más sencillas de usar, en las fórmulas de sumas y en otros casos, que las funciones inversas $\sen^{-1}(x)$ y $\cos^{-1}(x)$. Recordemos la fórmula $\sen ( x + y )$ en función del seno y coseno de $x$ e $y$ y comparémosla con la fórmula $\sen^{-1} ( x + y)$ en función de $x$ e $y$ . ¿No es la primera mucho más sencilla, más elegante, más «natural» que la última? Ahora, en el cálculo integral, las funciones trigonométricas inversas se presentan naturalmente como integrales definidas de irracionales algebraicas simples (segundo grado); tales integrales aparecen cuando se trata de encontrar la longitud de un arco de círculo por medio del Cálculo integral. Supongamos que las funciones trigonométricas inversas se han presentado al principio de esta forma. ¿No habría sido «más natural» considerar las inversas de estas funciones, es decir las funciones trigonométricas familiares como las funciones dadas que han de ser estudiadas y analizadas? Indudablemente, pero en muchos de los problemas más complicados, el más sencillo de los cuales es el de hallar la longitud del arco de una elipse por una integración, las difíciles funciones «elípticas» inversas (no «circulares» como para el arco de un círculo) se presentan primeramente. Abel vio que estas funciones debían ser «invertidas» y estudiadas, precisamente como en el caso de $\sen(x)$ y $\cos(x)$ en lugar de $\sen^{-1}(x)$ y $\cos^{ -1}(x)$. ¿No es esto sencillo? Sin embargo, Legendre, que era un gran matemático, trabajó durante más de cuarenta años en sus «integrales elípticas» (las difíciles «funciones inversas» de su problema) sin siquiera sospechar que podría invertir los términos. Esta forma extraordinariamente sencilla de enfocar un problema al parecer sencillo pero profundamente complicado, fue uno de los grandes progresos matemáticos del siglo XIX.

Sin embargo, todo esto no fue más que el comienzo, un tremendo comienzo, como la aurora de Kipling, aparece como un trueno, de lo que Abel hizo con su magnífico teorema y con su obra sobre las funciones elípticas. Las funciones trigonométricas o circulares tienen un solo período real, así $\sen( x + 2 \pi ) = \sen (x)$, etc., Abel descubrió que las nuevas funciones que resultaban por la inversión de una integral elíptica tienen precisamente dos períodos, cuya razón es imaginaria. Más tarde, los continuadores de Abel en esta dirección, Jacobi, Rosenhaim, Weierstrass, Riemann, y muchos más, penetraron profundamente en el gran teorema de Abel, y extendieron sus ideas descubriendo funciones de $n$ variables que tienen $2n$ períodos. Abel mismo también explotó sus descubrimientos. Sus sucesores aplicaron toda su obra a la Geometría, a la mecánica, a ciertas partes de la física matemática y a otros campos de la Matemática, resolviendo importantes problemas que sin la obra iniciada por Abel habrían sido insolubles.

Estando aún en París, Abel consultó algunos médicos acerca de lo que él pensaba era un simple catarro persistente. Fue informado de que padecía tuberculosis de los pulmones. Se negó a creerlo y volvió a Berlín para una breve visita. Sus recursos eran muy escasos. Una carta urgente le trajo, después de algún retraso, un préstamo de Holmboë. No ha de pensarse que Abel fuera un pedigüeño sin intención de devolver lo prestado. Tenía buenas razones para creer que tendría un buen puesto cuando volviera a su patria. Además, todavía le debían dinero. Con el préstamo de Holmboë, Abel pudo seguir viviendo e investigando desde marzo hasta mayo de 1827. Entonces, agotados todos sus recursos, volvió a Cristianía.

Esperaba que todo fuera ya de color de rosa. Seguramente le concederían un cargo universitario. Su genio había comenzado a ser reconocido. Existía una vacante. Abel no había vuelto aún, y Holmboë, aunque con repugnancia, aceptó la cátedra vacante, que él creía debía ser destinada a Abel. Tan sólo después de que el gobierno le amenazó con traer un extranjero si Holmboë no la ocupaba. Holmboë no tuvo, pues, culpa alguna. Se supuso que Holmboë sería mejor maestro que Abel, aunque Abel había demostrado ampliamente su capacidad para enseñar. Los que están familiarizados con la corriente teoría pedagógica americana, alentada por las escuelas de educación profesionales, de que cuanto menos sabe un hombre de lo que tiene que enseñar, mejor lo enseñará, comprenderán la situación perfectamente.

De todos modos las cosas se aclararon. La Universidad pagó a Abel lo que aun le debía por su viaje, y Holmboë le envió discípulos. El profesor de Astronomía, que había obtenido una licencia, sugirió que Abel fuera empleado para realizar parte de obra. Un matrimonio acomodado, los Schjeldrups, le dio alojamiento, tratándole como si fuera su propio hijo. Sin embargo no podía liberarse de la carga de sus familiares. Hasta última hora dependieron de él, no dejándole prácticamente nada para sus necesidades, sin que, a pesar de ello, Abel pronunciara una palabra de queja.

A mediados de enero de 1829 Abel supo que no viviría mucho tiempo. Tuvo una hemorragia que no fue posible ocultarla. «Lucharé por mi vida», gritaba en su delirio. Pero en los momentos más tranquilos, agotado e intentando trabajar, decía: «Igual que un águila enferma que contempla el sol», sabiendo que sus días estaban contados.

Abel pasó sus últimos días en Froland, en el hogar de una familia inglesa donde su prometida (Crelly Kemp) era institutriz. Sus últimos pensamientos fueron para su futura, y refiriéndose a ella, escribía así a su amigo Kielhau. «No es bella; tiene el cabello rojo y es pecosa, pero se trata de una mujer admirable». Era deseo de Abel que Crelly y Kielhau se casaran después de su muerte, y aunque los dos no se habían conocido lo hicieron, según había propuesto Abel semijocosamente. En los últimos días Crelly insistió en cuidar a Abel «para poseer, por lo menos, estos últimos momentos». En la madrugada del 6 de abril de 1829 murió, teniendo 26 años y 8 meses.



Dos días después de la muerte de Abel, Crelle le escribió diciendo que sus negociaciones habían llegado finalmente a buen fin y que sería nombrado para la Cátedra de Matemática de la Universidad de Berlín. 



\end{document}