\documentclass[a4paper, 12pt, draft]{article}

%%%%%%%%%%%%%%%%%%%%%%Paquetes
\usepackage[spanish]{babel}  
\usepackage[utf8]{inputenc}
\usepackage{tcolorbox}
\usepackage{cmbright}  %%%%%%% El tipo de letra
\usepackage{setspace}
\onehalfspacing  %%%%%%%%%%% Espacio y medio de interlineado
\parskip=1em  %%%%%%%%%%%% Separacion entre parrafos
%%%%%%%%%%%%%%%%%%%%%%



%%%%%%%%%%%%%%%%%
\title{La Invención Matemática}
\author{H. Poincaré}
\date{}
%%%%%%%%%%%%%%%%%

\begin{document}

\begin{tcolorbox}[colback=blue!5!white,colframe=blue!75!black]

\vspace{-1.8cm}
\textbf \maketitle

\end{tcolorbox}

\bigskip


La génesis de la invención matemática es un problema que debe
inspirar mucho interés al psicólogo. Es el acto en el cual el espíritu humano prescinde más del mundo exterior, en el que no obra más que por él mismo y sobre él mismo, de manera que estudiando
los procesos del pensamiento geométrico, podemos tener esperanzas de
alcanzar lo más esencial de él.

Se ha comprendido desde hace mucho tiempo; no hace muchos meses una revista
titulada {\it La enseñanza matemática}, dirigida por los señores
Laisant y Fehr, emprendió una encuesta sobre las costumbres del espíritu y los métodos de trabajo de los diferentes matemáticos.

Había ya esbozado los principales trazos de este artículo cuando
los resultados de la encuesta fueron publicados; no pude por consiguiente
utilizarlos; me limitaré a decir que la mayoría de los testimonios
confirman mis conclusiones, no digo que la totalidad, puesto que cuando se
consulta el sufragio universal no puede uno vanagloriarse de reunirla.

Un primer hecho debe sorprendernos, o más bien debía sorprendernos
si no estuviésemos tan acostumbrados. ¿Cómo es que hay gentes que
no comprenden las matemáticas? Si las matemáticas no invocan más
que las leyes de la lógica aceptadas por todos los espíritu
centrados; si su evidencia está fundada sobre los principios comunes a
todos los hombres y que ninguno podría negar sin estar loco, ¿cómo
existen tantas personas refractarias a ellas?

Que no todo el mundo tenga capacidad de inventiva no tiene nada de
particular. Que no todos puedan retener una demostración que hayan aprendido anteriormente, pase aun, pero que no
todos puedan comprender un razonamiento matemático, cuando se expone, he
aquí lo que al reflexionar parece sorprendente. Sin embargo, la mayoría no puede seguir este razonamiento sino con gran trabajo; esto es
indudable y la experiencia de los maestros de enseñanza secundaria no me
contradecirá seguramente.

Hay más aún. ¿Cómo es posible el error en matemáticas? Una
inteligencia sana no debe cometer una falta de lógica y, sin embargo,
hay espíritus muy finos que no tropezarán en un razonamiento muy
corto, tal como los que deben efectuar en los actos comunes de la vida,
incapaces de seguir o de repetir sin equivocarse las demostraciones de matemáticas, que si bien son más largas, no son después de todo más que una acumulación de pequeños razonamientos, análogos a los
que se hacen con tanta facilidad todos los días. ¿Es necesario agregar
que los matemáticos tampoco son infalibles? La respuesta nos parece
necesaria. Imaginemos una larga serie de silogismos en los cuales las
conclusiones de los primeros sirvan de premisas a los siguientes: seremos
capaces de comprender algunos de estos silogismos; y no es al pasar de las
premisas a la conclusión donde correremos el riesgo de equivocarnos.
Pero entre el momento en que encontramos por primera vez una proposición
como conclusión de un silogismo y aquel en que la volvemos a encontrar
como premisa de otro silogismo, habrá transcurrido a veces mucho
tiempo, habremos desarrollado numerosos anillos de la cadena; puede ocurrir
también que se haya olvidado y lo que es peor aún, que nos hubiéramos olvidado del sentido. Puede, pues, entonces, acontecer que se la
reemplace por una proposición diferente, o que, conservando el mismo
enunciado, que se le atribuya otro sentido; así es como se está
expuesto al error.

A menudo el matemático debe servirse de una regla: naturalmente comienza
a demostrar esta regla; mientras la demostración está fresca en la
memoria, comprende perfectamente su sentido y su alcance y no corre el
riesgo de alterarla; pero en cuanto confía en su memoria y no la
aplica más que de una manera mecánica, entonces si la memoria le
llega a faltar puede aplicarla al revés. Es de esta manera,
presentando un ejemplo simple y casi vulgar, como a veces nos
equivocamos en operaciones de cálculo por haber olvidado nuestra tabla
de multiplicar.

De esta manera la aptitud especial hacia las matemáticas será debida
a una memoria muy fiel, o bien a una fuerza de atención prodigiosa. Sería una cualidad análoga a la del jugador de {\it whist}, que recuerda las
cartas tiradas, o bien, para poner otro ejemplo, a la del jugador de ajedrez
que puede encarar un número de combinaciones muy grande y conservarlas
en su memoria. Todo buen matemático debería ser al mismo tiempo
buen jugador de ajedrez y viceversa; todo buen jugador de ajedrez debería ser igualmente un buen calculador numérico. Cierto que esto
ocurre a veces: Gauss era a la vez un geómetra de genio y un calculador
muy rápido y seguro.

Pero hay excepciones, o más bien me equivoco: no puedo llamar a esto
excepciones, pues si no las excepciones serían más numerosas que
los casos ajustados a la regla. Es Gauss, por el contrario, quien era una
excepción. En cuanto a mí, estoy obligado a confesarlo, soy incapaz
de hacer una suma sin faltas. Sería igualmente muy mal jugador de
ajedrez; calcularía bien que jugando de tal manera me expongo a tal
peligro; pasaría revista a muchos otros golpes que rechazaría por
otras tantas razones, y acabaría por jugar la combinación examinada
al principio, habiendo olvidado en el intervalo el peligro previsto.

En una palabra, mi memoria no es mala, pero es insuficiente para hacer de mí un buen jugador de ajedrez. ¿Por qué no me falla en un
razonamiento matemático en el que la memoria de los jugadores de ajedrez
se perdería? Es evidente: porque está guiada por la marcha general
del razonamiento. Una demostración matemática no es una simple
yuxtaposición de silogismos; son silogismos colocados en un cierto
orden, y el orden en el cual están colocados estos elementos es mucho más importante que ellos mismos. Si tengo el sentimiento, la intuición de este orden, de manera que me pueda dar cuenta rápidamente del
conjunto del razonamiento, no debo temer más olvidarme de uno de los
elementos, pues cada uno vendrá a colocarse en el cuadro que le he
preparado, sin que haya hecho ningún esfuerzo de memoria.

Me parece entonces, repitiendo un razonamiento aprendido, que lo
hubiera podido inventar; esto no es con frecuencia más que una ilusión; pero, asimismo, aunque no soy bastante fuerte para crear por mí
mismo, lo vuelvo a inventar a medida que lo repito.

Se concibe que este sentimiento, esta intuición del orden matemático
que nos hace adivinar las armonías y las relaciones ocultas, no puede
pertenecer a todo el mundo. Los unos no poseerán ni este
sentimiento delicado difícil de definir, ni una fuerza de memoria y de
atención por encima de lo vulgar, y entonces serán incapaces de
comprender las matemáticas un poco elevadas; esto ocurre en la mayoría. Otros no tendrán este sentimiento más que en pequeño
grado, pero estarán dotados de una memoria poco común y de una gran.
capacidad de atención. Aprenderán de memoria los detalles, unos después de otros, podrán comprender las matemáticas y alguna vez
aplicarlas, pero serán incapaces de crear. Los otros, en fin, poseerán en un grado más o menos elevado la intuición especial de que acabo
de hablar, y entonces no solamente podrán comprender las matemáticas
aunque su memoria no tenga nada de extraordinario, sino que podrán
llegar a ser creadores y tratarán de inventar con más o menos éxito, según que esta intuición esté en ellos más o menos
desarrollada.

¿Qué es, en efecto, la invención matemática? No consiste en
hacer nuevas combinaciones con otros seres matemáticos ya conocidos.
Esto cualquiera podría hacerlo, pero las combinaciones que se podrían formar así serían infinitas y la mayor parte estaría
totalmente desprovista de interés.

Inventar consiste precisamente en no construir combinaciones inútiles,
sino en construir sólo las que pueden ser útiles, que no son más
que una ínfima minoría. Inventar es discernir, es elegir.

Ya expliqué antes cómo debe hacerse esta elección; los hechos
materiales dignos de ser estudiados son los que por su analogía con
otros resultan capaces de conducirnos al conocimiento de una ley matemática, de la misma manera que los hechos experimentales nos conducen al
conocimiento de una ley física. Son los que nos revelan parentescos
insospechados
entre otros hechos conocidos desde hace mucho tiempo, pero que erróneamente se creyeron extraños entre sí.

Entre las combinaciones que se escogerán, las más fecundas serán
frecuentemente las que están formadas por elementos procedentes de
sitios muy alejados; y no quiero decir que sea suficiente para inventar el
acercar los objetos más dispares posibles; la mayor parte de las
combinaciones que se formarían de este modo serían totalmente estériles, pero algunas de ellas, muy pocas veces, serían las más
fecundas de todas.

Inventar, lo he dicho en otra oportunidad, es elegir, pero la palabra no es
del todo justa, hace pensar en un comprador al que se le presentan un gran número de muestras, que examina una después de la otra a fin de hacer
su elección. En este caso las muestras serían tan numerosas que una
vida entera no bastaría para examinarlas. No es así como suceden
las cosas. Las combinaciones estériles ni siquiera se presentarán al
espíritu del inventor. En el campo de su conciencia no aparecerán más que las combinaciones realmente útiles y algunas que rechazará, pero que participan un poco de los caracteres de las combinaciones útiles. Todo sucede como si el inventor fuera un examinador de segundo grado,
que no tuviera que examinar más que los candidatos declarados admisibles
después de una primera prueba.

Lo que he dicho hasta aquí es lo que se puede observar e inferir
leyendo los escritos de los geómetras, con la única condición de
reflexionar sobre esta lectura.

Pero ya es hora de entrar más en el tema, a fin de ver qué es lo que
pasa en el alma misma del matemático. Para esto creo que lo mejor que
puedo hacer es apelar a recuerdos personales.

Voy a limitarme a contaros cómo escribí mi primer trabajo sobre las
funciones fuchsianas. Os pido perdón, voy a emplear algunas expresiones técnicas, pero no os debéis asustar, pues no hay necesidad alguna de
que las comprendáis. Diré, por ejemplo, que encontré la
demostración de tal teorema en tales circunstancias, este teorema tendrá un nombre bárbaro que muchos de entre ustedes desconocerán.
Esto no tiene imporcia; lo que le interesa al psicólogo no es el
teorema, son las circunstancias.


Desde hacía quince días me esforzaba en demostrar que no pedía existir ninguna función análoga a lo que yo más tarde llamé
funciones fuchsianas; en aquella época era muy ignorante; todos los días me sentaba en mi mesa de trabajo, pasaba una hora o dos, ensayaba
un gran número de combinaciones y no llegaba a ningún resultado. Una
noche tomé café, contrariando mis costumbres, y no me pude dormir;
las ideas surgían en masa, las sentía cómo chocaban, hasta que
dos de ellas se engarzaron, por así decir, para formar una combinación estable. A la mañana siguiente ya había establecido la
existencia de una clase de funciones fuchsianas: las que derivan de la serie
hipergeométrica; no hice más que redactar los resultados; no tardé más que algunas horas.

Quise a continuación representar estas funciones por el cociente de dos
series; esta idea fue perfectamente consciente y reflexionada: la analogía con las funciones elípticas me guiaba. Me pregunté cuáles debían ser las propiedades de estas series si existiesen, y llegué sin dificultad a formar las series que he llamado thetafuchsianas.

En ese momento me fui de Caen, donde vivía, para tomar parte en un
concurso geológico emprendido por la Escuela de Minas. Las peripecias
del viaje me hicieron olvidar mis trabajos matemáticos; al llegar a
Coutances, subimos en un ómnibus para dar no sé qué paseo;
en el momento en que ponía el pie en el estribo la idea me vino sin que
nada en mis pensamientos anteriores me hubiera podido preparar para
ella, que las transformaciones de que había hecho uso para definir 
las funciones fuchsianas eran idénticas a las de la geometría 
no-euclidiana. No hice la verificación; no hubiera tenido tiempo,
puesto que apenas sentado en el ómnibus proseguí la conversación comenzada, pero tuve en seguida la absoluta certidumbre. De regreso a Caen
verifiqué el resultado más reposadamente para tranquilidad de mi espíritu.

Me puse entonces a estudiar las cuestiones aritméticas sin gran
resultado aparente y sin sospechar que pudiera tener la más mínima
relación con mis anteriores descubrimientos. Disgustado por mi fracaso,
me fui a pasar algunos días al mar y pensé en otras cosas. Un día paseándome sobre el tajamar, la idea me vino, siempre con los mismos caracteres de brevedad, instantaneidad y certeza inmediata, que las transformaciones aritméticas de formas cuadradas ternarias indefinidas eran idénticas a las de
la geometría no-euclidiana.

Ya de vuelta, en Caen, reflexioné sobre este resultado y saqué
las consecuencias; el ejemplo de las formas cuadráticas me enseñaba
que existían más grupos fuchsianos que los que correspondían a la serie hipergeométrica; vi que podía aplicarlos a la teoría de la serie thetafuchsiana y que, por consiguiente, existían
otras funciones fuchsianas además de las que se derivaban de la serie
hipergeométrica, las únicas que conocía hasta entonces. Me
propuse, naturalmente, formar todas estas funciones; realicé un asedio
sistemático y fui sacando una tras otra todas las obras avanzadas; sin
embargo había entre ellas algunas que se me resistían y cuya caída
debía acarrear la de los cuerpos de plaza. Pero todos mis esfuerzos no
sirvieron primero nada más que para hacerme conocer mejor la dificultad,
lo cual ya era algo. Todo este trabajo fue perfectamente consciente.

Después de esto partí para Mont-Valerien, donde tenía que
hacer mi servicio militar; tuve por lo tanto preocupaciones muy diferentes.
Un día, atravesando el bulevar, la solución de la dificultad que me
había detenido se me apareció de repente. No traté de
profundizarla inmediatamente y fue sólo después de mi
servicio cuando proseguí la cuestión. Tenía todos los
elementos, no tenía más que juntarlos y ordenarlos. Redacté
entonces mi memoria definitiva de un tirón y sin ninguna dificultad.

Me limitaré a este único ejemplo, pues es inútil multiplicarlos; en lo que concierne a mis otros descubrimientos, tendría que contar
hechos análogos y las objeciones aportadas por otros matemáticos en
la encuesta de {\it La enseñanza matemática} no podrían más que
confirmarlos.

%\end{document}

Lo que sorprenderá, primero, son estas apariencias de iluminación súbita, signo manifiesto de un largo trabajo inconsciente anterior; el
papel de ese trabajo inconsciente en la invención matemática me
parece indudable y se hallarán huellas en otros casos donde es menos
evidente. A menudo cuando se trabaja en una cuestión no se hace nada
bueno la primera vez que se pone uno a trabajar; tras esto se toma un
reposo más o menos largo y vuelve de nuevo a sentarse a trabajar delante
de su mesa. Durante la primera media hora se continúa no encontrando
nada, y después, de golpe la idea decisiva se presenta a la mente. Se
podría decir que el trabajo consciente ha sido más fructífero,
puesto que ha sido interrumpido y el reposo ha devuelto al espíritu
su fuerza y su frescor. Pero es más probable que este reposo haya sido
reemplazado por un trabajo inconsciente y que el resultado de este trabajo
se haya revelado en seguida al geómetra, lo mismo que en los casos
citados; solamente que la revelación, en vez de efectuarse en un paseo
o en un viaje, se produce durante un período de trabajo consciente, mas
con independencia de este trabajo, que desempeña además un papel
de desprendimiento, como si fuera el aguijón que hubiera excitado los resultados, ya adquiridos durante el reposo, que subsistían inconscientes, a tomar la forma consciente.

Hay otra observación que hacer respecto a las condiciones de trabajo de
la tarea inconsciente: que no es posible, o en todo caso que solamente es
fecundo, si es precedido, por una parte, y seguido por otra, de un período de trabajo consciente. Jamás (y los ejemplos que ya he citado lo
prueban bastante) estas inspiraciones repentinas se producen sino al cabo de
varios días de esfuerzos voluntarios, que han parecido absolutamente
infructuosos y donde se ha creído no hacer nada bueno, en los que da la
impresión de haber hecho una ruta totalmente falsa. Estos esfuerzos no
han sido tan estériles como se piensa, han puesto en marcha la máquina inconsciente; sin ellos no habría marchado ni, por lo tanto,
producido nada.

La necesidad del segundo período de trabajo consciente después de
la inspiración se comprende mejor aún. Hace falta poner en orden los
resultados de esta inspiración, deducir las consecuencias inmediatas,
ordenarlas, redactar las demostraciones, pero sobre todo verificarlas. Hablo
del sentimiento de certeza absoluta que acompaña a la inspiración;
en los casos citados este sentimiento no era engañoso, y frecuentemente
se presenta así: no hay que creer que esto sea una regla sin excepción; con frecuencia este sentimiento nos engaña sin que por ello sea
menos vivo, y uno no se da cuenta de ello
sino cuando se ha establecido la demostración. Observé sobre todo
este hecho, por las ideas que me han venido por la mañana o por la
noche en mi lecho, en un estado semihipnagógico.

Tales son los hechos. He aquí ahora las reflexiones que nos inspiran.
El yo inconsciente, o como hemos dicho, el yo subconsciente desempeña un
papel capital en la invención matemática. Pero se considera
vulgarmente el yo subconsciente como puramente automático. Además
hemos visto que el trabajo matemático no es más que un simple
trabajo mecánico, que no podríamos confiar a una máquina por más perfeccionada que se la supusiera. No se trata solamente de aplicar
las reglas, de fabricar todas las combinaciones posibles, de acuerdo con
ciertas leyes fijas. Las combinaciones así obtenidas serían demasiado
numerosas, inútiles y embarazosas. El verdadero trabajo del inventor
consiste en elegir entre estas combinaciones a fin de eliminar las que son inútiles, o más bien para no tomarse el trabajo de hacerlas. Las
reglas que deben guiar esta elección son tan finas y delicadas que es
poco más o menos que imposible enunciarlas en un lenguaje preciso; más bien se sienten que se formulan. ¿Cómo, en estas condiciones,
pensar en una criba capaz de aplicarlas mecánicamente?

Entonces se nos presenta una primera hipótesis: el yo inconsciente no es
inferior al yo consciente; no es sólo automático, es capaz de
discernir, tiene tacto, delicadeza, sabe elegir y adivinar. Qué digo, más aún, sabe adivinar mejor que el yo consciente, puesto que ha
tenido éxito allí donde el otro ha fracasado. En una palabra, el yo
inconsciente, ¿no es superior al yo consciente? Ustedes comprenden toda la
importancia de esta pregunta. El señor Boutroux, en una conferencia
reciente, ha mostrado cómo se ha presentado en diferentes ocasiones y qué consecuencias entrañará una respuesta afirmativa. 

¿Esta contestación afirmativa nos es impuesta por los hechos que acabo
de exponer? Confieso que por mi parte no la aceptaría sin repugnancia.
Revisemos los hechos y busquemos si no tienen otra explicación.



Es cierto que las combinaciones que se presentan al espiritu como una
especie de iluminación súbita, después de un trabajo
inconsciente un poco prolongado, son generalmente combinaciones útiles y
fecundas que parecen el resultado de la primera tría. Se deduce que el
yo subconsciente, habiendo adivinado por una intuición delicada que
estas combinaciones podrían ser inútiles, no ha formado más que 
éstas, o más bien, ha formado muchas otras que estaban desprovistas
de interés y que han permanecido en el inconsciente.

En este segundo punto de vista todas las combinaciones se formarían a
consecuencia del automatismo del yo subconsciente, pero solamente las que
fueran interesantes penetrarían el campo de la consciencia. Y esto aún es muy misteriosa ¿Cuál es la causa que hace que entre los mil
productos de 
nuestra actividad inconsciente haya algunos que puedan franquear su atrio,
mientras que otros permanecen dentro? ¿Es un simple azar el que les confiere
este privilegio? Evidentemente, no; entre todas las excitaciones de
nuestros sentidos, por ejemplo, sólo las más intensas lograrán
retener nuestra atención, a menos que esta atención no haya sido atraída hacia ellas por otras causas. Pero generalmente los fenómenos
subconscientes privilegiados, aquellos susceptibles de tornarse conscientes,
son los que directa o indirectamente afectan más profundamente nuestra
sensibilidad.

Podemos sorprendernos de ver invocar la sensibilidad con motivo de
demostraciones matemáticas que aparentemente no podrían interesar más que a la inteligencia. Esto sería olvidar el sentimiento de la
belleza matemática, de la armonía de los números y de las
formas, de la elegancia geométrica. Es un auténtico sentimiento estético que todos los verdaderos matemáticos conocen. He aquí una
verdadera sensibilidad.

Según esto, ¿cuáles son los seres matemáticos a los que
atribuimos este carácter de belleza y elegancia y que son susceptibles
de desarrollar en nosotros una especie de emoción estética? Aquellos
cuyos elementos están armoniosamente dispuestos, de manera que el espíritu pueda sin esfuerzo abarcar todo el conjunto penetrando en los
detalles. Esta armonía es a la vez una satisfacción para nuestras
necesidades estéticas y una ayuda para el espíritu que sostiene y guíe.

 Al mismo tiempo poniendo ante nuestros ojos un todo bien ordenado, nos hace presentir
una ley matemática. Puesto que ya antes lo hemos dicho, los solos hechos
matemáticos dignos de retener nuestra atención y susceptibles de ser 
útiles son los que pueden hacernos conocer una ley matemática. De
tal manera que llegamos a la conclusión siguiente: las combinaciones 
útiles son precisamente las más bellas, quiero decir, las que pueden
encantar más a esa sensibilidad especial que todos los matemáticos
conocen, pero que los profanos ignoran hasta el punto de sonreírse.

¿Qué sucede entonces? Entre las numerosas combinaciones que el yo
subconsciente ciegamente ha formado, casi todas carecen de interés y de
utilidad; pero por eso mismo no excitan la sensibilidad estética; la
conciencia no las conocerá jamás; algunas solamente son armoniosas,
y por consiguiente, a la vez inútiles y bellas, serán capaces de
conmover esa sensibilidad especial del geómetra, que acabo de referirme
y que una vez excitada llamará sobre ella nuestra atención y le dará así ocasión de volverse consciente.

Esto no es más que una hipótesis. Mientras tanto he aquí una
observación que podría confirmarla: cuando una iluminación súbita invade el espíritu del matemático, sucede con frecuencia
que lo engaña; pero acaece también algunas veces, lo he dicho, que
no soporta la prueba de una verificación; y ¡bien!, se advierte casi
siempre que esta idea es falsa; si hubiera sido justa habría halagado
nuestro instinto natural de elegancia matemática.

De este modo es esta sensibilidad estética especial la que juega el
papel de la criba delicada a la que me refería antes, y esto hace
comprender, por otra parte, por qué el que esté desprovisto de ella
no será jamás un verdadero inventor.

Todas las dificultades no han desaparecido, sin embargo; el yo consciente
está estrechamente limitado; en cuanto al yo subconsciente no conocemos
sus límites, y es por eso lo mucho que nos repugna suponer que él
haya podido formar en tan poco tiempo más combinaciones que la vida
entera de un ser consciente podría abarcar. Estos límites existen,
sin embargo; ¿es verosímil que pueda formar todas las combinaciones
posibles cuyo número aterraría a la imaginación? Esto
parecerá necesario, no obstante, porque si no produce más que una
pequeña parte de estas combinaciones y si lo hace al azar, tendrá
pocas posibilidades para que la buena que deba escoger se encuentre entre
ellas.

Puede ser que sea necesario encontrar la explicación en este período de trabajo consciente preliminar que precede siempre a todo trabajo
inconsciente fructífero. Permítaseme una grosera comparación.
Representémonos los elementos futuros de nuestras combinaciones como
parecidos a los átomos ganchudos de Epicuro. Durante el reposo completo
del espíritu, estos átomos permanecen inmóviles, están,
por así decirlo, enganchados al muro; este reposo completo puede
entonces prolongarse indefinidamente sin que estos átomos se encuentren, y por consiguiente, sin que pueda producirse ninguna combinación
entre ellos.

Por el contrario, durante un período de reposo aparente y de trabajo
subconsciente, algunos se desenganchan y ponen en movimiento. Surcan en
todos los sentidos el espacio, iba a decir el reducto donde están
encerrados, como podría hacerlo, por ejemplo, una nube de moscardones
o, si prefieren una comparación más sabia, como lo hacen las moléculas gaseosas en la teoría cinética de los gases. Sus choques
mutuos, pueden entonces producir combinaciones nuevas.

¿Cuál va a ser el papel del trabajo consciente preliminar?
Evidentemente movilizar alguno de estos átomos, desengancharlos del muro
y ponerlos en movimiento. Se cree que no se ha hecho nada bueno, porque se
han movido estos elementos de mil maneras diferentes para tratar de
reunirlos y no se ha podido encontrar un conjunto satisfactorio. Pero después de esta agitación que les ha sido impuesta por nuestra voluntad,
estos átomos no vuelven a su reposo primitivo, continúan su danza
libremente.

Según esto, nuestra voluntad no los ha elegido al azar, sino que
persigue un objeto perfectamente determinado; los átomos movilizados no
son, por lo tanto, átomos cualesquiera, son aquellos en los cuales
podemos razonablemente encontrar la solución buscada. Los átomos
movilizados van entonces a sufrir choques, que los harán entrar en
combinación, sea ellos, 
o sea con otros átomos que han permanecido inmóviles. entre


%%%%%%%%%%%%%%%

y que han chocado en su curso.




Pido perdón una vez más. Mi comparación es grosera: pero cómo podría hacer comprender mi pensamiento de otra no sé
forma.


%%%%%%%%%%%

Sea lo que fuere, las solas combinaciones que tienen posibilidad de
formarse, son aquellas en que uno de los elementos por lo menos, es uno
de estos átomos escogidos por nuestra voluntad. Según esto, es
evidente que entre ellos se encuentra lo que llamé antes la buena
combinación. Puede ser que allí haya un medio de atenuar lo que tenía de paradójica la hipótesis primitiva.






%%%%%%%%%%

 Otra observación.
 
 %%%%%%%%
No sucede nunca que el trabajo subconsciente nos provea, todo hecho, el
resultado de un cálculo un poco largo, en el que tenemos que aplicar
reglas fijas. Se podría creer que el yo subconsciente, automático,
es particularmente apto para este género de trabajo, es en cierta manera
exclusivamente mecánico. Parece que pensando por la noche en los
factores de una multiplicación, se podría esperar el producto
hecho al despertar, o también un cálculo algebraico; una verificación, por ejemplo, podría hacerse inconscientemente. Esto es falso,
como prueba la observación. Todo lo más que se puede esperar de
estas inspiraciones, frutos del trabajo subconsciente, son los puntos de
partida para cálculos parecidos; en cuanto a los cálculos mismos
hace falta hacerlos en el segundo período de trabajo consciente, el que
sucede a la inspiración y en el que se verifican los resultados de esta
inspiración, del que se sacan las consecuencias. Las reglas de estos cálculos son estrictas y complicadas; exigen disciplina, atención,
voluntad y, por consiguiente, consciencia. En el yo subconsciente reina, por
el contrario, lo que yo llamaría la libertad, si se puede dar este
nombre a la simple ausencia de disciplina y al desorden nacido del azar.
Solamente este desorden permite los acoplamientos inesperados.

Haré una última observación. Cuando he expuesto antes algunas
observaciones personales, he hablado de una noche de excitación en que
trabajaba contra mi voluntad; los casos en que esto sucede son frecuentes y
no es necesario que la actividad cerebral anormal esté causada por un
excitante físico como el que he citado. Pues bien, me parece que en ese caso asiste uno
mismo a su propio trabajo inconsciente, que se ha vuelto perceptible a la
consciencia sobreexcitada y que no por esto cambia de naturaleza. Se da uno
cuenta entonces vagamente de lo que distingue a los dos mecanismos, o si se
quiere, los métodos de trabajo de los dos yos. Las observaciones psicológicas que he podido hacer de esta manera parecen confirmar en sus líneas generales los puntos de vista que acabo de presentar.

El interés de la cuestión es tan grande que no me arrepiento de
haberla sometido al lector.



\end{document}
